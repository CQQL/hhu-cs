\documentclass[10pt,a4paper]{article}
\usepackage[utf8]{inputenc}
\usepackage[german]{babel}
\usepackage{mathrsfs}
\usepackage{amsmath}
\usepackage{amsfonts}
\usepackage{amssymb}
\usepackage{amsthm}
\usepackage[left=2cm,right=2cm,top=2cm,bottom=2cm]{geometry}
\usepackage{listings}

\begin{document}

\section{Aufgabe 1.1}

\section{Aufgabe 1.2}

\section{Aufgabe 1.3}

\section{Aufgabe 1.4}

\section{Aufgabe 1.5}

\section{Aufgabe 1.6}

\section{Aufgabe 1.7}

\section{Aufgabe 1.8}

\subsection{Requests}

\subsubsection{VERIFY}

\begin{lstlisting}
  VERIFY <TOKEN> <PIN>
\end{lstlisting}

\begin{lstlisting}
  OK
\end{lstlisting}

\subsubsection{STATUS}

\begin{lstlisting}
  STATUS <TOKEN> <PIN>
\end{lstlisting}

\begin{lstlisting}
  OK BALANCE=<BALANCE>
\end{lstlisting}

\subsubsection{WITHDRAW}

\begin{lstlisting}
  WITHDRAW <TOKEN> <PIN> <AMOUNT>
\end{lstlisting}

Wenn die Abhebung autorisiert und serverseitig gespeichert wurde, sendet der ZC
\begin{lstlisting}
  OK
\end{lstlisting}

Wenn nicht ausreichend Geld vorhanden ist, sendet der ZC
\begin{lstlisting}
  ERROR MISSING_BALANCE=<AMOUNT - BALANCE>
\end{lstlisting}

\subsection{Error Responses}

\subsubsection{INVALID\_CREDENTIALS}

Wenn Kennung oder PIN falsch sind, sendet der ZC

\begin{lstlisting}
  ERROR INVALID_CREDENTIALS
\end{lstlisting}

\subsubsection{MALFORMED\_REQUEST}

Wenn die Anfrage nicht formatiert ist wie erwartet, sendet der ZC

\begin{lstlisting}
  ERROR MALFORMED_REQUEST
\end{lstlisting}

\subsection{Beispiel}

\begin{lstlisting}
  BA: VERIFY 1234567 1011
  ZC: OK
  BA: STATUS 1234567 1011
  ZC: OK BALANCE=2000
  BA: WITHDRAW 1234567 1011 3000
  ZC: ERROR MISSING_BALANCE=1000
  BA: WITHDRAW 1234567 1011 1000
  ZC: OK
\end{lstlisting}

\section{Aufgabe 1.9}

\subsection{Teil 1}

\begin{equation}
  n := 8
\end{equation}
Die Dauer für eine Mautstation plus Strecke beträgt
\begin{equation}
  t_{1} = 14s \cdot n + \frac{80km}{120 \frac{km}{h}} = n \cdot 14s + \frac{2}{3}h = n \cdot 14s + \frac{2}{3}h \cdot 60 \frac{min}{h} \cdot 60 \frac{s}{min} = n \cdot 14s + 2400s = 2512s
\end{equation}
Die Gesamtdauer beträgt dann
\begin{equation}
  t = 3 \cdot t_{1} = 7536s \simeq 2h
\end{equation}

\subsection{Teil 2}

\begin{equation}
  n := 5
\end{equation}
\begin{equation}
  t_{1} = 2470s
\end{equation}
\begin{equation}
  t = 3 \cdot t_{1} = 7410 \simeq 2h1
\end{equation}

\section{Aufgabe 1.10}

\subsection{Teil 1}

Die Übertragung eines Pakets von einem Link dauert
\begin{equation}
  t_{1} = \frac{L + h}{R}
\end{equation}
Die Dauer einer Übertragung eines Pakets über die gesamte Strecke ist dann
\begin{equation}
  t_{Q} = Q \cdot t_{1} = Q \cdot \frac{L + h}{R}
\end{equation}
Um die Gesamtdauer zu berechnen, müssen erst alle Pakete losgeschickt werden und
dann muss man noch addieren, wie lange das letzte Paket braucht, um anzukommen.
\begin{equation}
  t = (M - 1) \cdot t_{1} + t_{Q} = (Q + M - 1) \cdot \frac{L + h}{R}
\end{equation}

\subsection{Teil 2}

\begin{equation}
  M := 1
\end{equation}
\begin{equation}
  t = t_{Q}
\end{equation}

\subsection{Teil 3}

Da die Verbindung direkt von Ende zu Ende geht, ist die Anzahl der Links
dazwischen irrelevant. Somit ist die Übertragungsdauer
\begin{equation}
  t = t_{s} + \frac{F + 2h}{R}
\end{equation}

\section{Aufgabe 1.11}

\subsection{Teil 1}

\begin{equation}
  d_{aus} = \frac{l}{v}
\end{equation}

\subsection{Teil 2}

\begin{equation}
  d_{ueber} = \frac{L}{R}
\end{equation}

\subsection{Teil 3}

\begin{equation}
  \frac{L}{R} + \frac{l}{v}
\end{equation}

\subsection{Teil 4}

Es wurde gerade versendet und liegt ganz am Anfang auf der Leitung.

\subsection{Teil 5}

Irgendwo auf der Leitung.

\subsection{Teil 6}

Bei dem Zielhost.

\subsection{Teil 7}

\begin{align*}
  d_{aus} & = d_{ueber}\\
  \Leftrightarrow \frac{l}{v} & = \frac{L}{R}\\
  \Leftrightarrow l & = \frac{Lv}{R} = \frac{500 Bit \cdot 2.5 \cdot 10^{8} \frac{m}{s}}{10^{6} \frac{Bit}{s}} = 1.25 \cdot 10^{5}m
\end{align*}

\section{Aufgabe 1.12}

\subsection{Teil 1}

Die Verzögerung $t_{i}$ des $i$-ten Packets einer Ladung, das versandt wird, ist
\begin{equation}
  t_{i} = (i - 1) \cdot \frac{L}{R}
\end{equation}
Die durchschnittliche Verzögerung $\bar{t}$ ist dann
\begin{align*}
  \bar{t} & = \frac{1}{N} \cdot \sum_{i = 1}^{N} t_{i}\\
  & = \frac{1}{N} \cdot \frac{L}{R} \cdot \sum_{i = 1}^{N} (i - 1)\\
  & = \frac{1}{N} \cdot \frac{L}{R} \cdot \sum_{i = 0}^{N - 1} i\\
  & = \frac{1}{N} \cdot \frac{L}{R} \cdot \frac{N \cdot (N - 1)}{2}\\
  & = \frac{L(N - 1)}{2R} = \frac{L}{R} \cdot \frac{N - 1}{2}
\end{align*}

\subsection{Teil 2}

Dann bleibt am Ende jeder Phase immer ein Paket mehr in der Schlange, bis die
Schlange voll ist und es zu Paketverlust kommt.

\section{Aufgabe 1.13}

\subsection{Teil 1}

\begin{equation}
  R \cdot d_{aus} = 10^{6} \frac{Bit}{s} \cdot \frac{10^{7}m}{2.5 \cdot 10^{8} \frac{m}{s}} = 0.4 \cdot 10^{5} Bit
\end{equation}

\subsection{Teil 2}

Das erste Bit benötigt $t = \frac{10^{7}}{2.5 \cdot 10^{8}}s = 4 \cdot
10^{-2}s$. In dieser Zeit kann man insgesamt $R \cdot t = 4 \cdot 10^{4} =
40000$ Bits verschicken.

\subsection{Teil 3}

Das bandwidth-delay product sagt aus, wie viele Bits man versenden kann, bevor
das erste ankommt.

\subsection{Teil 4}

Wenn man $10^{6}$ Bits pro Sekunde verschicken kann und man annimmt, dass zum
Umschalten der Signale keine Zeit benötigt wird, sondern sie die ganze Zeit
anliegen, liegt ein Bit $10^{-6}$ Sekunden lang an. Bei einer
Ausbreitungsgeschwindigkeit von $2.5 \cdot 10^{8} \frac{m}{s}$ ergibt sich eine
Länge von $2.5 \cdot 10^{2} m = 250m$.

\subsection{Teil 5}

\begin{equation}
  \frac{v}{R}
\end{equation}

\subsection{Teil 6}

Es dauert $t_{1} = \frac{450000 Bit}{R} = \frac{450000 Bit}{10^{6}
  \frac{Bit}{s}} = 0.45s$ um alle Bits zu versenden. Das letzte Bit benötigt
dann noch weitere $4 \cdot 10^{-2}s$. Insgesamt dauert die Übertragung also
$0.49s$.

\subsection{Teil 7}

Das Übertragen eines Pakets dauert
\begin{equation}
  t_{Packet} = \frac{30000 Bit}{R} + 4 \cdot 10^{-2}s = \frac{30000 Bit}{10^{6} \frac{Bit}{s}} + 4 \cdot 10^{-2}s = 7 \cdot 10^{-2}s
\end{equation}
Das Senden einer Bestätigung dauert
\begin{equation}
  t_{Ack} = 4 \cdot 10^{-2}s
\end{equation}
Die gesamte Übertragung dauert somit
\begin{equation}
  15 \cdot (t_{Packet} + t_{Ack}) = 15 \cdot 11 \cdot 10^{-2}s = 1.65s
\end{equation}

\end{document}