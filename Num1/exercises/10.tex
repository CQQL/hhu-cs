\documentclass[10pt,a4paper]{article}
\usepackage[utf8]{inputenc}
\usepackage[german]{babel}
\usepackage{mathrsfs}
\usepackage{amsmath}
\usepackage{amsfonts}
\usepackage{amssymb}
\usepackage{amsthm}
\usepackage[left=2cm,right=2cm,top=2cm,bottom=2cm]{geometry}

\begin{document}

\section{Aufgabe 30}

\subsection{Teil a}

\begin{equation}
  L = \begin{pmatrix}
    1 & 0\\
    200 & 1\\
  \end{pmatrix}
\end{equation}
\begin{equation}
  R = \begin{pmatrix}
    0.005 & 1\\
    0 & -190
  \end{pmatrix}
\end{equation}
\begin{equation}
  Ly = \begin{pmatrix}
    0.5\\1
  \end{pmatrix}
  \Leftrightarrow
  y = \begin{pmatrix}
    0.5\\-99
  \end{pmatrix}
\end{equation}
\begin{equation}
  Rx = y \Leftrightarrow
  x = \begin{pmatrix}
    -4\\0.52
  \end{pmatrix}
\end{equation}

\subsection{Teil b}

\begin{equation}
  P = \begin{pmatrix}
    0 & 1\\
    1 & 0
  \end{pmatrix}
\end{equation}
\begin{equation}
  L = \begin{pmatrix}
    1 & 0\\
    0.005 & 1\\
  \end{pmatrix}
\end{equation}
\begin{equation}
  R = \begin{pmatrix}
    1 & 1\\
    0 & 1
  \end{pmatrix}
\end{equation}
\begin{equation}
  Ly = \begin{pmatrix}
    1\\0.5
  \end{pmatrix}
  \Leftrightarrow
  y = \begin{pmatrix}
    1\\0.5
  \end{pmatrix}
\end{equation}
\begin{equation}
  Rx = y
  \Leftrightarrow
  x = \begin{pmatrix}
    0.5\\0.5
  \end{pmatrix}
\end{equation}

Nein.
Betrachte die Lösung von
\begin{equation}
  \begin{pmatrix}
    1 & 200\\
    1 & 1
  \end{pmatrix}
  \cdot x = \begin{pmatrix}
    100\\1
  \end{pmatrix}
\end{equation}
mit pivotisierter LR-Zerlegung.
\begin{equation}
  P = \begin{pmatrix}
    1 & 0\\
    0 & 1
  \end{pmatrix}
\end{equation}
\begin{equation}
  L = \begin{pmatrix}
    1 & 0\\
    1 & 1
  \end{pmatrix}
\end{equation}
\begin{equation}
  R = \begin{pmatrix}
    1 & 200\\
    0 & -190
  \end{pmatrix}
\end{equation}
\begin{equation}
  Ly = \begin{pmatrix}
    100\\1
  \end{pmatrix}
  \Leftrightarrow
  y = \begin{pmatrix}
    100\\-99
  \end{pmatrix}
\end{equation}
\begin{equation}
  Rx = y
  \Leftrightarrow
  x = \begin{pmatrix}
    0\\0.52
  \end{pmatrix}
\end{equation}

\section{Aufgabe 31}

\begin{equation}
  A_{1} = \begin{pmatrix}
    1 & 1 & 4\\
    2 & -3 & 4\\
    -4 & 8 & -4
  \end{pmatrix}
\end{equation}
\begin{equation}
  P_{1} = \begin{pmatrix}
    0 & 0 & 1\\
    0 & 1 & 0\\
    1 & 0 & 0
  \end{pmatrix}
\end{equation}
\begin{equation}
  L_{1} = \begin{pmatrix}
    1 & 0 & 0\\
    \frac{1}{2} & 1 & 0\\
    \frac{1}{4} & 0 & 1\\
  \end{pmatrix}
\end{equation}
\begin{equation}
  A_{2} = \begin{pmatrix}
    -4 & 8 & -4\\
    0 & 1 & 2\\
    0 & 3 & 3
  \end{pmatrix}
\end{equation}
\begin{equation}
  P_{2} = \begin{pmatrix}
    1 & 0 & 0\\
    0 & 0 & 1\\
    0 & 1 & 0
  \end{pmatrix}
\end{equation}
\begin{equation}
  L_{2} = \begin{pmatrix}
    1 & 0 & 0\\
    0 & 1 & 0\\
    0 & -\frac{1}{3} & 1
  \end{pmatrix}
\end{equation}
\begin{equation}
  R = A_{3} = \begin{pmatrix}
    -4 & 8 & -4\\
    0 & 3 & 3\\
    0 & 0 & 1
  \end{pmatrix}
\end{equation}
\begin{equation}
  P = P_{2}P_{1} = \begin{pmatrix}
    0 & 0 & 1\\
    1 & 0 & 0\\
    0 & 1 & 0
  \end{pmatrix}
\end{equation}
\begin{equation}
  L = (P_{2} L_{1}^{-1} P_{2}) L_{2}^{-1} = \begin{pmatrix}
    1 & 0 & 0\\
    -\frac{1}{4} & 1 & 0\\
    -\frac{1}{2} & \frac{1}{3} & 1
  \end{pmatrix}
\end{equation}

\section{Aufgabe 32}



\end{document}