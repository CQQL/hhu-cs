\documentclass[10pt,a4paper]{article}
\usepackage[utf8]{inputenc}
\usepackage[german]{babel}
\usepackage{mathrsfs}
\usepackage{amsmath}
\usepackage{amsfonts}
\usepackage{amssymb}
\usepackage{amsthm}
\usepackage[left=2cm,right=2cm,top=2cm,bottom=2cm]{geometry}

\DeclareMathOperator{\id}{Id}

\begin{document}

\section{Aufgabe 33}

\begin{proof}
  Sei $A$ eine $n \times x$-Matrix mit Bandbreite $m$ und LR-Zerlegung $L, R$.

  Angenommen $R$ habe eine Bandbreite größer $m$.
  Sei also $r_{i,i + k} \ne 0$ mit $(i + k) - i = k \ge m > m - 1$ und $i$ minimal in dem Sinne, dass alle Werte in der $i$-ten Spalte über $r_{i, i + k}$ $0$ sind.
  Nun betrachte man $A$ als Produkt $LR$.
  Wir wissen, dass $L$ $1$-en auf der Diagonale hat.
  \begin{equation}
    a_{i, i + k} = \sum_{j = 1}^{n} l_{i, j} r_{j, i + k} = \sum_{j = 1}^{i} l_{i, j} r_{j, i + k} = r_{i, i + k} + \sum_{j = 1}^{i - 1} l_{i, j} r_{j, i + k}
  \end{equation}
  Da $A$ Bandbreite $m$ hat, muss das $0$ sein und $\sum_{j = 1}^{i - 1} l_{i, j} r_{j, i + k} = -r_{i, i + k} \ne 0$.
  Aber dann gibt es ein $r_{j, i + k} \ne 0$ mit $j < i$.
  Dies ist ein Widerspruch zur Minimalität von $i$ und $R$ muss auch eine Bandbreite von höchstens $m$ haben.

  Angenommen $L$ habe eine Bandbreite größer $m$.
  Sei also $l_{i, i - k} \ne 0$ mit $i - (i - k) = k \ge m > m - 1$ und $i$ minmal in dem Sinne, dass alle Werte in der $i$-ten Zeile vor $l_{i, i - k}$ $0$ sind.
  Nun betrachte man $A$ als Produkt $LR$.
  \begin{equation}
    a_{i, i - k} = \sum_{j = 1}^{n} l_{i, j} r_{j, i - k} = \sum_{j = i - k}^{i} l_{i, j} r_{j, i - k} = \sum_{j = i - k}^{i - k} l_{i, j} r_{j, i - k} = l_{i, i - k} r_{i - k, i - k}
  \end{equation}
  Da $A$ Bandbreite $m$ hat, muss das $0$ sein und $r_{i - k, i - k} = 0$.
  Die Diagonalwerte von $R$ sind jedoch per Konstruktion ungleich $0$.
\end{proof}

\section{Aufgabe 34}

\subsection{Teil 1}

\begin{equation}
  v_{1} = \begin{pmatrix}
    4\\3\\0
  \end{pmatrix}
  - ||\begin{pmatrix}
    4\\3\\0
  \end{pmatrix}||e_{1} = \begin{pmatrix}
    -1\\3\\0
  \end{pmatrix}
\end{equation}
\begin{equation}
  Q_{1} = \id - \frac{2}{v_{1}^{T}v_{1}}v_{1}v_{1}^{T} =
  \begin{pmatrix}
    1 & 0 & 0\\
    0 & 1 & 0\\
    0 & 0 & 1
  \end{pmatrix}
  - \frac{1}{5} \begin{pmatrix}
    1 & -3 & 0\\
    -3 & 9 & 0\\
    0 & 0 & 0
  \end{pmatrix}
  =
  \frac{1}{5}
  \begin{pmatrix}
    4 & 3 & 0\\
    3 & -4 & 0\\
    0 & 0 & 5
  \end{pmatrix}
\end{equation}
\begin{equation}
  A_{2} = Q_{1}A = \begin{pmatrix}
    5 & 5 & 3\\
    0 & 0 & -4\\
    0 & 4 & 1
  \end{pmatrix}
\end{equation}

\begin{equation}
  v_{2} = \begin{pmatrix}
    0\\0\\4
  \end{pmatrix}
  - ||\begin{pmatrix}
    0\\0\\4
  \end{pmatrix}|| e_{2}
  =
  \begin{pmatrix}
    0\\-4\\4
  \end{pmatrix}
\end{equation}
\begin{equation}
  Q_{2} = \id - \frac{2}{v_{2}^{T}v_{2}}v_{2}v_{2}^{T} =
  \begin{pmatrix}
    1 & 0 & 0\\
    0 & 1 & 0\\
    0 & 0 & 1
  \end{pmatrix}
  - \frac{1}{16}
  \begin{pmatrix}
    0 & 0 & 0\\
    0 & 16 & -16\\
    0 & -16 & 16
  \end{pmatrix}
  = \begin{pmatrix}
    1 & 0 & 0\\
    0 & 0 & 1\\
    0 & 1 & 0
  \end{pmatrix}
\end{equation}
\begin{equation}
  R = A_{3} = Q_{2}A_{2} = \begin{pmatrix}
    5 & 5 & 3\\
    0 & 4 & 1\\
    0 & 0 & -4
  \end{pmatrix}
\end{equation}

\begin{equation}
  Q = Q_{1}Q_{2} = \frac{1}{5}\begin{pmatrix}
    4 & 0 & 3\\
    3 & 0 & -4\\
    0 & 5 & 0
  \end{pmatrix}
\end{equation}

\subsection{Teil 2}

\begin{equation}
  b' = Q^{T}b = \begin{pmatrix}
    -1\\6\\8
  \end{pmatrix}
\end{equation}
\begin{equation}
  Rx = b \Leftrightarrow x = \begin{pmatrix}
    -1\\2\\-2
  \end{pmatrix}
\end{equation}

\section{Aufgabe 35}

\section{Aufgabe 36}

\subsection{Teil 1}

\subsection{Teil 2}

\subsection{Teil 3}

\end{document}