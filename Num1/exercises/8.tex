\documentclass[10pt,a4paper]{article}
\usepackage[utf8]{inputenc}
\usepackage[german]{babel}
\usepackage{mathrsfs}
\usepackage{amsmath}
\usepackage{amsfonts}
\usepackage{amssymb}
\usepackage{amsthm}
\usepackage[left=2cm,right=2cm,top=2cm,bottom=2cm]{geometry}

\begin{document}

\section{Aufgabe 24}

Man verwende die lineare Transformation $g$ definiert durch
\begin{equation}
  g(x) = \tilde{a} + (x - a)\frac{\tilde{b} - \tilde{a}}{b - a}
\end{equation}
Dann ist
\begin{equation}
  g([a, b]) = [\tilde{a}, \tilde{b}]
\end{equation}
und
\begin{equation}
  g'(x) = \frac{\tilde{b} - \tilde{a}}{b - a}
\end{equation}

Nun benutzt man die Transformationsformel und erhält
\begin{align*}
  \int_{\tilde_{a}}^{\tilde{b}} f(x)\ dx & = \int_{g([a, b])} f(x)\ dx\\
  & = \int_{[a, b]} f(g(x)) * g'(x)\ dx\\
  & = \int_{a}^{b} f(g(x)) * g'(x)\ dx\\
  & = \sum_{i = 0}^{n} a_{i} f(g(x_{i})) * g'(x_{i})\\
  & = \sum_{i = 0}^{n} a_{i}\frac{\tilde{b} - \tilde{a}}{b - a} f\left(\tilde{a} + (x_{i} - a)\frac{\tilde{b} - \tilde{a}}{b - a}\right)\\
  & = \sum_{i = 0}^{n} \tilde{a}_{i} f(\tilde{x}_{i}) = \tilde{\mathbf{Q}_{n}}(f)
\end{align*}

\section{Aufgabe 25}

Seien $L_{m}, L_{n}$ zwei Legendre-Polynome.
Zu zeigen ist, dass ihr Skalarprodukt $0$ ist.
\begin{align*}
  \langle L_{m}, L_{n} \rangle & = \int_{-1}^{1} \frac{n!m!}{(2n)!(2m)!}\frac{\mathrm{d^{n}}}{\mathrm{d}x^{n}}\left[ (x^{2} - 1)^{n} \right] \frac{\mathrm{d^{m}}}{\mathrm{d}x^{m}}\left[ (x^{2} - 1)^{m} \right]\,\mathrm{d}x\\
\end{align*}

\section{Aufgabe 26}

\begin{equation}
  \alpha_{1} = \frac{5}{6} \int_{-1}^{1} x(x - \sqrt{\frac{3}{5}})\ dx = \frac{5}{6} \left[ \frac{1}{3}x^{3} - \frac{\sqrt{\frac{3}{5}}}{2}x^{2} \right]_{-1}^{1} = \frac{5}{6}\left( \frac{1}{3} - \frac{1}{2}\sqrt{\frac{3}{5}} - \left(-\frac{1}{3} - \frac{1}{2}\sqrt{\frac{3}{5}} \right) \right) = \frac{5}{9}
\end{equation}
\begin{equation}
  \alpha_{2} = -\frac{5}{3} \int_{-1}^{1} (x + \sqrt{\frac{3}{5}})(x - \sqrt{\frac{3}{5}})\ dx = -\frac{5}{3} \left[ \frac{1}{3}x^{3} - \frac{3}{5}x \right]_{-1}^{1} = -\frac{5}{3}(-\frac{8}{15}) = \frac{8}{9}
\end{equation}
\begin{equation}
  \alpha_{3} = \frac{5}{6} \int_{-1}^{1} (x + \sqrt{\frac{3}{5}})x\ dx = \frac{5}{6} \left[ \frac{1}{3}x^{3} + \frac{1}{2}\sqrt{\frac{3}{5}}x^{2} \right]_{-1}^{1} = \frac{5}{6}\left( \frac{1}{3} + \frac{1}{2}\sqrt{\frac{3}{5}} - \left( -\frac{1}{3} + \frac{1}{2}\sqrt{\frac{3}{5}} \right) \right) = \frac{5}{9}
\end{equation}

Sei $Q$ die resultierende Quadraturformel.
Weil das Integrieren linear ist und die Integrationsformel aufgrund ihrer Ordnung alle Polynome in $\mathbb{P}_{3}$ exakt integriert, müssen wir nur noch $x^{4}$ und $x^{5}$ betrachten.
\begin{equation}
  \int_{-1}^{1} x^{4}\ dx = \frac{2}{5}
\end{equation}
\begin{equation}
  Q(x^{4}) = \frac{5}{9} \cdot \frac{9}{25} + \frac{5}{9} \cdot \frac{9}{25} = \frac{5}{9} \cdot \frac{18}{25} = \frac{2}{5}
\end{equation}
\begin{equation}
  \int_{-1}^{1} x^{5}\ dx = 0
\end{equation}
\begin{equation}
  Q(x^{5}) = -\frac{5}{9} \cdot \left( -\sqrt{\frac{3}{5}} \right)^{5} + \frac{5}{9} \cdot \left( -\sqrt{\frac{3}{5}} \right)^{5} = 0
\end{equation}

Damit ist es gezeigt.

\end{document}