\documentclass[10pt,a4paper]{article}
\usepackage[utf8]{inputenc}
\usepackage[german]{babel}
\usepackage{mathrsfs}
\usepackage{amsmath}
\usepackage{amsfonts}
\usepackage{amssymb}
\usepackage{amsthm}
\usepackage[left=2cm,right=2cm,top=2cm,bottom=2cm]{geometry}

\begin{document}

\section{Aufgabe 4}

\subsection{Teil 1}

\begin{equation}
  \lim_{h \rightarrow 0} |\frac{2h^{3}}{h^{2}}| = \lim_{h \rightarrow 0} 2h = 0 \Rightarrow 2h^{3} \in \hbox{o}(h^{2})
\end{equation}

\subsection{Teil 2}

\begin{equation}
  \lim_{h \rightarrow 0} \frac{\sin h}{h} \le \lim_{h \rightarrow 0} \frac{h}{h} = 1 < \infty \Rightarrow \sin h \in \mathcal{O}(h)
\end{equation}

\subsection{Teil 3}

\begin{equation}
  \lim_{h \rightarrow 0} \frac{\sin(h) - h}{h} = \lim_{h \rightarrow 0} \frac{\sin h}{h} - 1 = 1 - 1 = 0 \Rightarrow \sin(h) - h \in o(h) \Rightarrow \sin h = h + o(h)
\end{equation}

\subsection{Teil 4}

\begin{equation}
  \lim_{h \rightarrow 0} \frac{\sqrt{h}}{1} = \lim_{h \rightarrow 0} \sqrt{h} = \lim_{h \rightarrow 0} h^{\frac{1}{2}} = \lim_{h \rightarrow 0} \exp(\frac{1}{2} \cdot \log(h)) = 0 < \infty \Rightarrow \sqrt{h} \in \mathcal{O}(1) \land \sqrt{h} \in o(1)
\end{equation}

\subsection{Teil 5}

Anwendung von L'Hospital liefert
\begin{equation}
  \lim_{h \rightarrow 0} \frac{1 - \cos(h)}{h} = \lim_{h \rightarrow 0} \frac{\sin h}{1} = 0 \Rightarrow 1 - \cos(h) \in o(h)
\end{equation}
und
\begin{equation}
  \lim_{h \rightarrow 0} \frac{1 - \cos(h)}{h^{2}} = \lim_{h \rightarrow 0} \frac{\sin h}{2h} = \frac{1}{2} \cdot \lim_{h \rightarrow 0} \frac{\sin h}{h} = \frac{1}{2} < \infty \Rightarrow 1 - \cos(h) \in \mathcal{O}(h^{2})
\end{equation}

\subsection{Teil 6}

\begin{equation}
  \lim_{h \rightarrow 0} \frac{e^{-\frac{1}{h}}}{h^{q}} =
\end{equation}

\section{Aufgabe 5}

\begin{equation}
  y_{1} = -\frac{p}{2} + \sqrt{(\frac{p}{2})^{2} - q}
\end{equation}
\begin{equation}
  y_{2} = -\frac{p}{2} - \sqrt{(\frac{p}{2})^{2} - q}
\end{equation}
\begin{align*}
  \frac{\partial y_{1}}{\partial p} & = -\frac{1}{2} + \frac{1}{4}p\frac{1}{\sqrt{(\frac{p}{2})^{2} - q}}\\
  & = \frac{p}{4}\frac{1}{\sqrt{(\frac{p}{2})^{2} - q}} - \frac{1}{2}\\
  & = \frac{p}{4 \sqrt{(\frac{p}{2})^{2} - q}} - \frac{1}{2}\\
  & = \frac{\frac{p}{2} - \sqrt{(\frac{p}{2})^{2} - q}}{2 \sqrt{(\frac{p}{2})^{2} - q}}\\
  & = \frac{y_{1}}{y_{2} - y_{1}}
\end{align*}
\begin{align*}
  \frac{\partial y_{1}}{\partial q} & = -\frac{1}{2} \cdot \frac{1}{\sqrt{(\frac{p}{2})^{2} - q}}\\
  & = -\frac{1}{2 \sqrt{(\frac{p}{2})^{2} - q}}\\
  & = \frac{1}{y_{2} - y_{1}}
\end{align*}
\begin{align*}
  \frac{\partial y_{2}}{\partial p} & = -\frac{1}{2} - \frac{1}{4}p\frac{1}{\sqrt{(\frac{p}{2})^{2} - q}}\\
  & = -\frac{p}{4\sqrt{(\frac{p}{2})^{2} - q}} - \frac{1}{2}\\
  & = \frac{-\frac{p}{2} - \sqrt{(\frac{p}{2})^{2} - q}}{2 \sqrt{(\frac{p}{2})^{2} - q}}\\
  & = \frac{y_{2}}{y_{1} - y_{2}}
\end{align*}
\begin{align*}
  \frac{\partial y_{2}}{\partial q} & = \frac{1}{2 \sqrt{(\frac{p}{2})^{2} - q}}\\
  & = \frac{1}{y_{1} - y_{2}}
\end{align*}

\begin{equation}
  k_{1, 1} = |\frac{\partial y_{1}}{\partial p} \cdot \frac{p}{y_{1}}| = |\frac{p}{y_{2} - y_{1}}|
\end{equation}
\begin{equation}
  k_{1, 2} = |\frac{\partial y_{1}}{\partial q} \cdot \frac{q}{y_{1}}| = |\frac{q}{y_{1}(y_{2} - y_{1})}|
\end{equation}
\begin{equation}
  k_{2, 1} = |\frac{\partial y_{2}}{\partial p} \cdot \frac{p}{y_{2}}| = |\frac{p}{y_{1} - y_{2}}|
\end{equation}
\begin{equation}
  k_{2, 2} = |\frac{\partial y_{2}}{\partial q} \cdot \frac{q}{y_{2}}| = |\frac{q}{y_{2}(y_{1} - y_{2})}|
\end{equation}
Für $y_{1} \simeq y_{2}$ werden diese Werte sehr groß.

\section{Aufgabe 6}

Für die Konditionszahl $k_{i}$ für $a_{i}$ sei $x$ fest und ich betrachte $p$ als Funktion in $a_{0}, \dots, a_{n}$
\begin{equation}
  k_{i} = |\frac{\partial p(a_{i})}{\partial a_{i}} \cdot \frac{a_{i}}{p(x)}| = |\frac{a_{i} x^{i}}{p(x)}|
\end{equation}

Sei $k$ die Konditionszahl für $x$
\begin{equation}
  k = |\frac{\partial p(x)}{\partial x} \cdot \frac{x}{p(x)}| = |\frac{na_{n}x^{n} + (n - 1)a_{n - 1}x^{n - 1} + \dots + a_{1}x^{1}}{a_{n}x^{n} + a_{n - 1}x^{n - 1} + \dots + a_{0}x^{0}}|
\end{equation}

\end{document}