\documentclass[10pt,a4paper]{article}
\usepackage[utf8]{inputenc}
\usepackage[german]{babel}
\usepackage{mathrsfs}
\usepackage{amsmath}
\usepackage{amsfonts}
\usepackage{amssymb}
\usepackage{amsthm}
\usepackage[left=2cm,right=2cm,top=2cm,bottom=2cm]{geometry}

\begin{document}

\section{Aufgabe 1}

\section{Aufgabe 2}

\subsection{Teil 1}

\subsection{Teil 2}

\section{Aufgabe 3}

\subsection{Teil 1}

\begin{equation}
  \mathcal{O}(n \cdot \log(n))
\end{equation}

\subsection{Teil 2}

\begin{equation}
  \mathcal{O}(n)
\end{equation}

\subsection{Teil 3}

\begin{equation}
  \mathcal{O}(n)
\end{equation}

\subsection{Teil 4}

\begin{equation}
  \mathcal{O}(|V| + |E|)
\end{equation}

\subsection{Teil 5}

\begin{equation}
  \mathcal{O}(|V| + |E|)
\end{equation}

\subsection{Teil 6}

\begin{equation}
  \mathcal{O}(|V| + |E|)
\end{equation}

\subsection{Teil 7}

Wenn $E*$ die Anzahl der Kanten im transitiven Abschluss und $k$ die Anzahl der starken Zusammenhangskomponenten sind,
\begin{equation}
  \mathcal{O}(|V| + |E^{*}| + k^{3})
\end{equation}

\section{Aufgabe 4}

\begin{itemize}
\item Was ist ein Kalkül?
\item Ist mit Median in der Klausur der Unter- oder Obermedian gemeint?
\end{itemize}

\end{document}