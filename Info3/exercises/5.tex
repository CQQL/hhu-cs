\documentclass[10pt,a4paper]{article}
\usepackage[utf8]{inputenc}
\usepackage[german]{babel}
\usepackage{mathrsfs}
\usepackage{amsmath}
\usepackage{amsfonts}
\usepackage{amssymb}
\usepackage{amsthm}
\usepackage[left=2cm,right=2cm,top=2cm,bottom=2cm]{geometry}

\begin{document}

\section{Aufgabe 1}

\subsection{Teil 1}

Als Median einer Folge mit gerader Länge benutze ich, wie im Skript, den Obermedian.

Wenn wir Gruppen der Länge $k$ bilden, erhalten wir $\lfloor \frac{N}{k} \rfloor$ volle Gruppen und maximal eine Gruppe mit $1$ bis $k - 1$ Elementen.
Sei $v$ der Median der Mediane dieser Gruppen.
Dann gibt es genau $\lfloor \frac{1}{2} \lfloor \frac{N}{k} \rfloor \rfloor$ volle Gruppen, deren Median echt kleiner als $v$ ist.
Die Anzahl der Schlüssel, die echt kleiner als $v$ sind, beträgt also mindestens
\begin{equation}
  (\lfloor \frac{k}{2} \rfloor + 1) \cdot \lfloor \frac{1}{2} \lfloor \frac{N}{k} \rfloor \rfloor \ge \lceil \frac{k}{2} \rceil \cdot (\lceil \frac{1}{2} \lfloor \frac{N}{k} \rfloor \rceil - 1)
\end{equation}
Analog dazu gibt es genau $\lceil \frac{1}{2} \lfloor \frac{N}{k} \rfloor \rceil - 1$ volle Gruppen, deren Median echt größer als $v$ ist.
Die Anzahl der Schlüssel, die echt größer als $v$ sind, ist also mindestens
\begin{equation}
  \lceil \frac{k}{2} \rceil \cdot (\lceil \frac{1}{2} \lfloor \frac{N}{k} \rfloor \rceil - 1) \ge \lceil \frac{k}{2} \rceil \cdot (\lceil \frac{1}{2} \lceil \frac{N}{k} \rceil - \frac{1}{2} \rceil - 1) \ge \frac{k}{2} \cdot \left( \frac{N}{2k} - \frac{3}{2} \right) = \frac{N}{4} - \frac{3k}{4}
\end{equation}
Unabhängig davon, ob das gesuchte Element nun vor oder hinter der Position von $v$ liegt, müssen im nächsten Schritt also maximal
\begin{equation}
  N - (\frac{N}{4} - \frac{3k}{4}) = \frac{3}{4}N + \frac{3}{4}k = \frac{3}{4} \cdot (N + k)
\end{equation}
Elemente untersucht werden.

\subsubsection{Fall 1}
\begin{equation}
  T(N) \le T(\lceil \frac{N}{3} \rceil) + T(\lceil \frac{3}{4}N + \frac{9}{4} \rceil) + a \cdot N
\end{equation}

\subsubsection{Fall 2}
\begin{equation}
  T(N) \le T(\lceil \frac{N}{7} \rceil) + T(\lceil \frac{3}{4}N + \frac{21}{4} \rceil) + a \cdot N
\end{equation}

\subsection{Teil 2}
\begin{proof}
  Damit es terminiert müssen die Folgen immer kürzer werden, also
  \begin{equation}
    \lceil \frac{N}{7} \rceil \le \frac{N}{7} + 1 < N \Leftrightarrow N < 7N - 7 \Leftrightarrow N > \frac{7}{6}
  \end{equation}
  und
  \begin{equation}
    \lceil \frac{3}{4}N + \frac{21}{4} \rceil \le \frac{3}{4}N + \frac{25}{4} < N \Leftrightarrow \frac{1}{4}N > \frac{25}{4} \Leftrightarrow N > 25
  \end{equation}

  Nehmen wir also an, dass der Algorithmus für hinreichend kleine N, insbesondere $N < 25$, in linearer Zeit terminiert, z.B. indem man es naiv bestimmt.

  Nehmen wir dann weiterhin an, dass es bereits bewiesen sei für alle $M < N$ und $N > 25$.
  Dann ergibt sich aus der Rekursionsgleichung das Folgende:
  \begin{align*}
    T(N) & \le c \cdot \lceil \frac{N}{7} \rceil + c \cdot \lceil \frac{3}{4}N + \frac{21}{4} \rceil + a \cdot N\\
    & \le c \cdot \frac{N}{7} + c + c \cdot \left( \frac{3}{4}N + \frac{25}{4} \right) + a \cdot N\\
    & = c \cdot \left( \frac{25}{28}N + \frac{29}{4} \right) + a \cdot N\\
    & = \frac{25c}{28}N + \frac{29c}{4} + a \cdot N
  \end{align*}
  Nun untersuchen wir noch, wann dieser letzte Ausdruck wirklich durch $c \cdot N$ beschränkt ist.
  Setze $c = p \cdot a$.
  \begin{align*}
    & \frac{25c}{28}N + \frac{29}{4}c + aN < cN\\
    \Leftrightarrow & \frac{25}{28}paN + \frac{29}{4}pa + aN < paN \textit{ Man kann $a$ kürzen, weil $a \ge 1$}\\
    \Leftrightarrow & \frac{25}{28}pN + \frac{29}{4}p + N < pN\\
    \Leftrightarrow & \frac{29}{4}p < (p - 1 - \frac{25}{28}p)N\\
    \Leftrightarrow & \frac{29}{4}p < \frac{3p - 28}{28}N \textit{ Um hier teilen zu können, muss $\frac{3p - 28}{28} > 1 \Leftrightarrow p > \frac{56}{3}$}\\
    \Leftrightarrow & N > \frac{29 \cdot 28}{12p - 112}p
  \end{align*}
  Wählen wir für $p$ z.B. den Wert $19$, so gilt die obere Schranke für alle $N > 133$.
  Insgesamt ist es damit gezeigt, weil wir die Werte für die ersten 133 Werte in konstanter Zeit bestimmen können.
\end{proof}

\subsection{Teil 3}
In der Ungleichung, die man am Ende von Teil 2 zeigen musste, ergibt sich dann, dass $p$ kleiner als $-24$ sein müsste, was aber nicht stimmt.
Allgemein scheint der Grund zu sein, dass der Aufwand zur Bestimmung des Medians der Mediane durch die erhöhte Anzahl der Gruppen über einen kritischen Wert steigt.

\section{Aufgabe 2}

\begin{equation}
  MdM(48\ 47\ 24\ 35, 3) = 47
\end{equation}
\begin{equation}
  MdM(62\ 48\ 99\ 53\ 73\ 66\ 82\ 85\ 77, 2)
\end{equation}
\begin{equation}
  MdM(62\ 82, 2) = 82
\end{equation}
\begin{equation}
  MdM(62\ 48\ 53\ 73\ 66\ 77, 2)
\end{equation}
\begin{equation}
  MdM(62\ 77, 2) = 77
\end{equation}
\begin{equation}
  MdM(62\ 48\ 53\ 73\ 66, 2) = 53 \Rightarrow MdM(F, 13) = 53
\end{equation}

\section{Aufgabe 3}

\subsection{Teil a}
Es finden die folgenden Vergleiche statt
\begin{equation}
  (38, 22), (38, 42), (38, 30), (38, 38)
\end{equation}
Das Element ist in der Folge enthalten.

\subsection{Teil b}
Es finden die folgenden Vergleiche statt
\begin{equation}
  (38, 22), (38, 55), (38, 30), (38, 38)
\end{equation}
Das Element ist in der Folge enthalten.

\subsection{Teil c}
Ich durchsuche hier den letztendlichen Bereich mit binärer Suche.
Es finden die folgenden Vergleiche statt
\begin{equation}
  (19, 1), (19, 5), (19, 10), (19, 22), (19, 15), (19, 19)
\end{equation}
Das Element ist in der Folge enthalten.

\end{document}