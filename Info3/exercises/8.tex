\documentclass[10pt,a4paper]{article}
\usepackage[utf8]{inputenc}
\usepackage[german]{babel}
\usepackage{mathrsfs}
\usepackage{amsmath}
\usepackage{amsfonts}
\usepackage{amssymb}
\usepackage{amsthm}
\usepackage[left=2cm,right=2cm,top=2cm,bottom=2cm]{geometry}

\begin{document}

\section{Aufgabe 1}
Da wir nach der Wahrscheinlichkeit für 2 oder mehr übereinstimmende Geburtstage suchen, suchen wir stattdessen nach der Gegenwahrscheinlichkeit $Q(n)$, nämlich dass alle $n$ Personen an unterschiedlichen Tagen Geburtstag haben.
Dann ist die Wahrscheinlichkeit, dass 1 Person an unterschiedlichen Tagen Geburtstag hat, natürlich 1.
Wenn nun schon $Q(n - 1)$ bekannt ist, ist die Wahrscheinlichkeit, dass noch eine weiter Person nicht an den $n - 1$ unterschiedlichen Geburtstagen der ersten $n - 1$ Personen Geburtstag hat, sondern an einem der restlichen $365 - n + 1$ $Q(n) = Q(n - 1) \cdot \frac{365 - n + 1}{365}$.
Das führt insgesamt zu
\begin{equation}
  Q(n) = \frac{365!}{365^{n} \cdot (365 - n)!}
\end{equation}
Also
\begin{equation}
  P(n) = 1 - Q(n) = 1 - \frac{365!}{365^{n} \cdot (365 - n)!}
\end{equation}
Das ergibt
\begin{equation}
  P(10) \simeq 0.117, P(23) \simeq 0.507, P(50) \simeq 0.97
\end{equation}
Im Kontext von Hashverfahren haben wir gerade die Wahrscheinlichkeit berechnet, dass $n$ universelle Hashfunktionen, die auf die Menge $[0, 364]$ abbilden, für einen gegebenen Schlüssel 2 oder mehr Kollisionen erzeugen.

\section{Aufgabe 2}
Die Belegungen sind danach wie folgt, wobei die erste Zeile die Adresse des Feldes darunter und die zweite Zeile die enthaltenen Werte sind.

\subsection{Teil a}
\begin{tabular}{l|l|l|l|l|l|l|l|l|l|l|l|l}
  0 & 1 & 2 & 3 & 4 & 5 & 6 & 7 & 8 & 9 & 10 & 11 & 12\\
  \hline
  14 & 1 & 30 & 32 & 17 & 5 & 19 &   &   &   & 23 &    & 2
\end{tabular}

\subsection{Teil b}
\begin{tabular}{l|l|l|l|l|l|l|l|l|l|l|l|l}
  0 & 1 & 2 & 3 & 4 & 5 & 6 & 7 & 8 & 9 & 10 & 11 & 12\\
  \hline
    & 1 & 14 & 30 & 17 & 5 & 19 & 32 &   &   & 23 & 2 & 
\end{tabular}

\subsection{Teil c}
\begin{tabular}{l|l|l|l|l|l|l|l|l|l|l|l|l}
  0 & 1 & 2 & 3 & 4 & 5 & 6 & 7 & 8 & 9 & 10 & 11 & 12\\
  \hline
    & 1 & 14 &   & 17 & 5 & 19 &   & 32 & 2 & 23 &  & 30
\end{tabular}

\section{Aufgabe 3}
Im schlechtesten Fall werden alle Schlüssel auf denselben Hashwert abgebildet.
Wenn man neue Schlüssel immer am Ende der Liste anhängt, muss man für jeden vorher eingefügten Schlüssel ein Listenelement weitergehen, bis man es anhängen kann.
Dies verwende ich als einen Schritt.
Die Rechenoperation zur Bestimmung des Hashwertes sind konstant und ich ignoriere die mal.
Also ist der Aufwand um $n$ solche Schlüssel einzufügen $\sum_{i = 0}^{n - 1} i = \frac{n \cdot (n - 1)}{2} \in O(n^{2})$.
Der Aufwand, alle diese Schlüssel jeweils einmal zu suchen, ist derselbe, weil der Schlüssel, der als $i$-tes eingefügt wurde, $i - 1$ Schlüssel vor sich hat, mit denen der gesucht immer verglichen werden muss.

Für den mittleren Fall nehme ich an, dass alle $n$ Schlüssel zufällig, gleichverteilt und unabhängig voneinander gewählt wurden.
Da wir mit einer universellen Hashfunktion arbeiten, ist die Wahrscheinlichkeit, dass zwei Schlüssel auf denselben Hashwert abgebildet werden, $\frac{1}{M}$.
Wenn man den $i$-ten Schlüssel einfügt, ist die Wahrscheinlichkeit, dass bereits einer der $i - 1$ vorherigen Schlüssel auf denselben Hashwert abgebildet wurde, $\frac{1}{M}$.


\section{Aufgabe 4}
\begin{proof}
  Nein, es ist keine universelle Menge von Hashfunktionen.
  Wenn es eine wäre, müsste folgendes gelten für alle unterschiedlichen Schlüssel $x, y \in [0, 9]$.
  \begin{equation}
    |\{ h_{i,j} \in H \mid h_{i,j}(x) = h_{i,j}(y) \}| \le \lceil \frac{|H|}{M} \rceil
  \end{equation}
  In diesem konkreten Fall ist $|H| = 100$ und $M = 10$.
  Es darf also maximal 10 Kollisionen pro Schlüsselpaar geben.
  Jetzt betrachte man allerdings die Schlüssel $2$ und $8$.
  Da $2^{i} \mod 10 = 8^{i} \mod 10$ für $i \in \{ 2, 4, 6, 8, 10 \}$, gibt es hier 50 Kollisionen, nämlich für jedes $i$ 10 Stück.
  Genau eine für jedes mögliche $j$, weil $a = b \Leftrightarrow a + j = b + j$.
\end{proof}

\end{document}