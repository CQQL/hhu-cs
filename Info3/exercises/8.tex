\documentclass[10pt,a4paper]{article}
\usepackage[utf8]{inputenc}
\usepackage[german]{babel}
\usepackage{mathrsfs}
\usepackage{amsmath}
\usepackage{amsfonts}
\usepackage{amssymb}
\usepackage{amsthm}
\usepackage[left=2cm,right=2cm,top=2cm,bottom=2cm]{geometry}

\begin{document}

\section{Aufgabe 1}
Da wir nach der Wahrscheinlichkeit für 2 oder mehr übereinstimmende Geburtstage suchen, suchen wir stattdessen nach der Gegenwahrscheinlichkeit $Q(n)$, nämlich dass alle $n$ Personen an unterschiedlichen Tagen Geburtstag haben.
Dann ist die Wahrscheinlichkeit, dass 1 Person an unterschiedlichen Tagen Geburtstag hat, natürlich 1.
Wenn nun schon $Q(n - 1)$ bekannt ist, ist die Wahrscheinlichkeit, dass noch eine weiter Person nicht an den $n - 1$ unterschiedlichen Geburtstagen der ersten $n - 1$ Personen Geburtstag hat, sondern an einem der restlichen $365 - n + 1$ $Q(n) = Q(n - 1) \cdot \frac{365 - n + 1}{365}$.
Das führt insgesamt zu
\begin{equation}
  Q(n) = \frac{365!}{365^{n} \cdot (365 - n)!}
\end{equation}
Also
\begin{equation}
  P(n) = 1 - Q(n) = 1 - \frac{365!}{365^{n} \cdot (365 - n)!}
\end{equation}
Das ergibt
\begin{equation}
  P(10) \simeq 0.117, P(23) \simeq 0.507, P(50) \simeq 0.97
\end{equation}
Im Kontext von Hashverfahren haben wir gerade die Wahrscheinlichkeit berechnet, dass $n$ universelle Hashfunktionen, die auf die Menge $[0, 364]$ abbilden, für einen gegebenen Schlüssel 2 oder mehr Kollisionen erzeugen.

\section{Aufgabe 2}
Die Belegungen sind danach wie folgt, wobei die erste Zeile die Adresse des Feldes darunter und die zweite Zeile die enthaltenen Werte sind.

\subsection{Teil a}
\begin{tabular}{l|l|l|l|l|l|l|l|l|l|l|l|l}
  0 & 1 & 2 & 3 & 4 & 5 & 6 & 7 & 8 & 9 & 10 & 11 & 12\\
  \hline
  14 & 1 & 30 & 32 & 17 & 5 & 19 &   &   &   & 23 &    & 2
\end{tabular}

\subsection{Teil b}
\begin{tabular}{l|l|l|l|l|l|l|l|l|l|l|l|l}
  0 & 1 & 2 & 3 & 4 & 5 & 6 & 7 & 8 & 9 & 10 & 11 & 12\\
  \hline
    & 1 & 14 & 30 & 17 & 5 & 19 & 32 &   &   & 23 & 2 & 
\end{tabular}

\subsection{Teil c}
\begin{tabular}{l|l|l|l|l|l|l|l|l|l|l|l|l}
  0 & 1 & 2 & 3 & 4 & 5 & 6 & 7 & 8 & 9 & 10 & 11 & 12\\
  \hline
    & 1 & 14 &   & 17 & 5 & 19 &   & 32 & 2 & 23 &  & 30
\end{tabular}

\section{Aufgabe 3}

\section{Aufgabe 4}

\end{document}