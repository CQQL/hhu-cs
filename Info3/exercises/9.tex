\documentclass[10pt,a4paper]{article}
\usepackage[utf8]{inputenc}
\usepackage[german]{babel}
\usepackage{mathrsfs}
\usepackage{amsmath}
\usepackage{amsfonts}
\usepackage{amssymb}
\usepackage{amsthm}
\usepackage{listings}
\usepackage[left=2cm,right=2cm,top=2cm,bottom=2cm]{geometry}

\begin{document}

\section{Aufgabe 2}
Man führe eine normale Suche nach $x$ durch, wobei man jedoch stets den letzten Schlüssel mitführt, der kleiner als $x$ war.
Wenn die normale Suche an einem Knoten endet, dessen Schlüssel kleiner oder gleich $x$ ist, ist dieser Schlüssel der gesuchte.
Wenn man jedoch an einem Knoten endet, dessen Schlüssel größer als $x$ ist, ist der letzte Schlüssel, der kleiner als $x$ war, der gesuchte.
\begin{proof}
  Man betrachte die Situation, in der die normale Suche abgeschlossen ist, und man an einem Knoten $r$ angekommen ist, dessen Schlüssel größer als der von $x$ ist.
  Dann sei $p_{1}, \dots, p_{k} = r$ der Weg, den man von der Wurzel bis nach $r$ genommen hat, und $p_{i}$ der letzte Schlüssel auf diesem Weg, der kleiner als $x$ war.
  Dabei sind $p_{1}, \dots, p_{k}$ eindeutig bestimmt, weil wir in einem Baum sind.
  Angenommen es gäbe einen Schlüssel $q$ im Baum, der noch näher von unten an $x$ läge.
  Weiterhin angenommen $q$ läge nicht auf diesem Weg.
  Dann haben $r$ und $q$ einen gemeinsamen Vater $p_{j}$, für den entweder $x < p_{j} < q$ oder $q < p_{j} < x$, was aber ein Widerspruch dazu ist, dass $q < x$ bzw. dass $q$ der größte Knoten im Baum ist, der kleiner als $x$ ist.
  $q$ liegt also auf dem Weg.
  Sei $q = p_{l}$.
  Wäre $l > i$, so wäre $p_{i} < q < x$, aber $p_{i}$ ist der letzte Knoten auf dem Weg, der kleiner als $x$ ist.
  Wäre hingegen $l < i$, so wäre $p_{i}$ im linken Teilbaum von $q$ wegen der Suchbaumeigenschaft.
  Da aber $x > q$, hätte die Suche jedoch im rechten Teilbaum weitergesucht, was im Widerspruch dazu steht, dass $p_{i}$ im Pfad enthalten ist.
  Es folgt also, dass $i = j$ und somit $p_{i} = q$.
\end{proof}
Es folgt der Algorithmus in Pseudocode.
\begin{lstlisting}
  def find<= (p, x, max = NULL)
    if k(p) == x
      return k(p)
    else if k(p) < x
      if empty?(right(p))
        return k(p)
      else
        return find<=(right(p), x, k(p))
    else
      if empty?(left(p))
        return max
      else
        return find<=(left(p), x, max)
\end{lstlisting}
Dann rufe man $find<=$ mit der Wurzel und dem gesuchten $x$ als Argument auf.

Da man nur einmal den Baum hinabsteigt und an jedem Knoten nur einen konstanten Aufwand betreibt, ist die Laufzeit höchstens proportional zur Länge des längsten Weges im Baum, also zur Höhe.

\end{document}