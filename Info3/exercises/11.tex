\documentclass[10pt,a4paper]{article}
\usepackage[utf8]{inputenc}
\usepackage[german]{babel}
\usepackage{mathrsfs}
\usepackage{amsmath}
\usepackage{amsfonts}
\usepackage{amssymb}
\usepackage{amsthm}
\usepackage[left=2cm,right=2cm,top=2cm,bottom=2cm]{geometry}

\begin{document}

\section{Aufgabe 1}
Die starken Zusammenhangskomponenten sind
\begin{align*}
  & (9)\\
  & (3, 6, 8, 10, 5, 7)\\
  & (4, 2, 1, 11)
\end{align*}

\section{Aufgabe 2}
\begin{tabular}{c|c|c}
Knoten & DFS & DFE\\
\hline
1 & 9 & 9\\
2 & 10 & 8\\
3 & 2 & 6\\
4 & 8 & 10\\
5 & 5 & 2\\
6 & 3 & 5\\
7 & 6 & 1\\
8 & 4 & 4\\
9 & 1 & 11\\
10 & 7 & 3\\
11 & 11 & 7
\end{tabular}
\\
\begin{tabular}{c|c}
  Baumkanten & (9, 3), (3, 6), (6, 8), (8, 5), (5, 7), (8, 10), (9, 4), (4, 1), (1, 2), (2, 11)\\\hline
  Vorwärtskanten & (9, 5), (3, 7), (3, 8), (4, 2), (1, 11)\\\hline
  Rückwärtskanten & (7, 6), (7, 8), (11, 4)\\\hline
  Seitwärtskanten & (10, 5), (1, 5), (11, 5), (11, 10)
\end{tabular}

\section{Aufgabe 3}
\begin{equation}
  9, 3, 4, 5, 6, 7, 8, 1, 2, 10, 11
\end{equation}

\section{Aufgabe 4}
Die Artikulationspunkte sind $3$ und $7$.

\begin{tabular}{c|c|c}
  Knoten & DFS & P[Knoten]\\
  \hline
  1 & 9 & 7\\
  2 & 2 & 1\\
  3 & 3 & 1\\
  4 & 4 & 3\\
  5 & 5 & 3\\
  6 & 6 & 3\\
  7 & 7 & 1\\
  8 & 8 & 7\\
  9 & 10 & 7\\
  10 & 11 & 1\\
  11 & 1 & 1
\end{tabular}

Die zweifachen Zusammenhangskomponenten sind
\begin{align*}
  & \{ \{ 3, 6 \}, \{ 5, 6 \}, \{ 4, 5 \}, \{ 3, 4 \} \}\\
  & \{ \{ 7, 9 \}, \{ 8, 9 \}, \{ 1, 9 \}, \{ 1, 8 \}, \{ 7, 8 \} \}\\
  & \{ \{ 2, 3 \}, \{ 2, 7 \}, \{ 2, 10 \}, \{ 2, 11 \}, \{ 3, 7 \}, \{ 7, 10 \}, \{ 7, 11 \}, \{ 10, 11 \} \}\\
\end{align*}

\end{document}