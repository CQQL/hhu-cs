\documentclass[10pt,a4paper]{article}
\usepackage[utf8]{inputenc}
\usepackage[german]{babel}
\usepackage{mathrsfs}
\usepackage{amsmath}
\usepackage{amsfonts}
\usepackage{amssymb}
\usepackage{amsthm}
\usepackage[left=2cm,right=2cm,top=2cm,bottom=2cm]{geometry}

\begin{document}

\section{Aufgabe 1}

\subsection{$1 \Rightarrow 2$}
\begin{proof}
  Da $G$ ein Baum ist, ist $G$ kreisfrei.
  Gäbe es zwischen zwei Knoten $u$ und $v$ mehr als einen Weg, die nicht identisch sind, gäbe es zwei Knoten $i$ und $j$ auf diesen Wegen, sodass es zwei komplett knotendisjunkte Wege zwischen ihnen gäbe.
  Diese würden jedoch einen Kreis bilden.
\end{proof}

\subsection{$2 \Rightarrow 3$}
\begin{proof}
  Dass $G$ zusammenhängt, folgt direkt, weil es zwischen jedem Knotenpaar einen Weg gibt.
  Sei $e = \{u, v\}$.
  Dann ist $e$ nach Vorraussetzung der einzige einfache Weg zwischen $u$ und $v$.
  Entfernt man diesen gibt es keinen Weg in $G$ mehr von $u$ nach $v$ und $G$ ist nicht mehr zusammenhängend.
\end{proof}

\subsection{$3 \Rightarrow 1$}
\begin{proof}
  Es ist nur zu zeigen, dass $G$ kreisfrei ist.
  Sei $p_{1}, \dots, p_{k}$ ein Kreis in $G$ und $u, v$ zwei adjazente Knoten darauf mit $u = p_{i}$ und $v = p_{j}$, sodass $i < j$.
  Nun betrachte man $G$ ohne die Kante $\{u, v\}$.
  Dann gibt es immer noch einen Weg von $u$ nach $v$: $p_{i}, p_{i - 1}, \dots, p_{1}, p_{k}, \dots p_{j + 1}, p_{j}$.
  Alle Wege, die die entfernte Kante enthielten können umgeschrieben werden, indem $u$ und $v$ durch den Weg zwischen ihnen ersetzt werden.
  $G$ ist also weiterhin zusammenhängend, was jedoch der Annahme widerspricht.
  Demnach gibt es keinen solchen Kreis in $G$ und $G$ ist kreisfrei.
\end{proof}

\section{Aufgabe 2}

\section{Aufgabe 3}

\section{Aufgabe 4}
Die Adjazenzmatrix $A$ sieht folgermaßen aus
\setcounter{MaxMatrixCols}{20}
\begin{equation}
  A =
  \begin{pmatrix}
    0 & 1 & 0 & 0 & 1 & 0 & 0 & 0 & 0 & 0 & 1\\
    0 & 0 & 0 & 0 & 0 & 0 & 0 & 0 & 0 & 0 & 1\\
    0 & 0 & 0 & 0 & 0 & 1 & 1 & 1 & 0 & 0 & 0\\
    1 & 1 & 0 & 0 & 0 & 0 & 0 & 0 & 0 & 0 & 0\\
    0 & 0 & 0 & 0 & 0 & 0 & 1 & 0 & 0 & 0 & 0\\
    0 & 0 & 0 & 0 & 0 & 0 & 0 & 1 & 0 & 0 & 0\\
    0 & 0 & 0 & 0 & 0 & 1 & 0 & 1 & 0 & 0 & 0\\
    0 & 0 & 0 & 0 & 1 & 0 & 0 & 0 & 0 & 1 & 0\\
    0 & 0 & 1 & 1 & 1 & 0 & 0 & 0 & 0 & 0 & 0\\
    0 & 0 & 0 & 0 & 1 & 0 & 0 & 0 & 0 & 0 & 0\\
    0 & 0 & 0 & 1 & 1 & 0 & 0 & 0 & 0 & 1 & 0
  \end{pmatrix}
\end{equation}

Dann fügt der Algorithmus die folgenden Kanten ein:
\begin{equation}
  Trans(A) =
  \begin{pmatrix}
    1 & 1 & 0 & 0 & 1 & 0 & 0 & 0 & 0 & 0 & 1\\
    0 & 1 & 0 & 0 & 0 & 0 & 0 & 0 & 0 & 0 & 1\\
    0 & 0 & 1 & 0 & 0 & 1 & 1 & 1 & 0 & 0 & 0\\
    1 & 1 & 0 & 1 & 0 & 0 & 0 & 0 & 0 & 0 & 0\\
    0 & 0 & 0 & 0 & 1 & 0 & 1 & 0 & 0 & 0 & 0\\
    0 & 0 & 0 & 0 & 0 & 1 & 0 & 1 & 0 & 0 & 0\\
    0 & 0 & 0 & 0 & 0 & 1 & 1 & 1 & 0 & 0 & 0\\
    0 & 0 & 0 & 0 & 1 & 0 & 0 & 1 & 0 & 1 & 0\\
    0 & 0 & 1 & 1 & 1 & 0 & 0 & 0 & 1 & 0 & 0\\
    0 & 0 & 0 & 0 & 1 & 0 & 0 & 0 & 0 & 1 & 0\\
    0 & 0 & 0 & 1 & 1 & 0 & 0 & 0 & 0 & 1 & 1
  \end{pmatrix}
\end{equation}

\section{Aufgabe 5}

\end{document}