\documentclass[10pt,a4paper]{article}
\usepackage[utf8]{inputenc}
\usepackage[german]{babel}
\usepackage{mathrsfs}
\usepackage{amsmath}
\usepackage{amsfonts}
\usepackage{amssymb}
\usepackage{amsthm}
\usepackage[left=2cm,right=2cm,top=2cm,bottom=2cm]{geometry}

\begin{document}

\section{Aufgabe 1}

\subsection{$1 \Rightarrow 2$}
\begin{proof}
  Da $G$ ein Baum ist, ist $G$ kreisfrei.
  Gäbe es zwischen zwei Knoten $u$ und $v$ mehr als einen Weg, die nicht identisch sind, gäbe es zwei Knoten $i$ und $j$ auf diesen Wegen, sodass es zwei komplett knotendisjunkte Wege zwischen ihnen gäbe.
  Diese würden jedoch einen Kreis bilden.
\end{proof}

\subsection{$2 \Rightarrow 3$}
\begin{proof}
  Dass $G$ zusammenhängt, folgt direkt, weil es zwischen jedem Knotenpaar einen Weg gibt.
  Sei $e = \{u, v\}$.
  Dann ist $e$ nach Vorraussetzung der einzige einfache Weg zwischen $u$ und $v$.
  Entfernt man diesen gibt es keinen Weg in $G$ mehr von $u$ nach $v$ und $G$ ist nicht mehr zusammenhängend.
\end{proof}

\subsection{$3 \Rightarrow 1$}
\begin{proof}
  Es ist nur zu zeigen, dass $G$ kreisfrei ist.
  Sei $p_{1}, \dots, p_{k}$ ein Kreis in $G$ und $u, v$ zwei adjazente Knoten darauf mit $u = p_{i}$ und $v = p_{j}$, sodass $i < j$.
  Nun betrachte man $G$ ohne die Kante $\{u, v\}$.
  Dann gibt es immer noch einen Weg von $u$ nach $v$: $p_{i}, p_{i - 1}, \dots, p_{1}, p_{k}, \dots p_{j + 1}, p_{j}$.
  Alle Wege, die die entfernte Kante enthielten können umgeschrieben werden, indem $u$ und $v$ durch den Weg zwischen ihnen ersetzt werden.
  $G$ ist also weiterhin zusammenhängend, was jedoch der Annahme widerspricht.
  Demnach gibt es keinen solchen Kreis in $G$ und $G$ ist kreisfrei.
\end{proof}

\section{Aufgabe 2}
Ich nummeriere die Knoten von links nach rechts und oben nach unten mit den Nummern $1$ bis $7$ durch.
Dann ist die Reihenfolge der Knoten $1, 2, 6, 7$ bereits eindeutig vorgegeben, während die Knoten $3, 4, 5$ untereinander beliebig sortiert werden können.
Dies ergibt eine Anzahl von $3 \cdot 2 \cdot 1 = 6$ verschiedenen topologischen Sortierungen von $H$.

Eine dieser Sortierungen, wie der Algorithmus sie auch erzeugen könnte, ist
\begin{equation}
  1, 2, 3, 4, 5, 6, 7
\end{equation}

\section{Aufgabe 3}

\section{Aufgabe 4}
Die Adjazenzmatrix $A$ sieht folgermaßen aus
\setcounter{MaxMatrixCols}{20}
\begin{equation}
  A =
  \begin{pmatrix}
    0 & 1 & 0 & 0 & 1 & 0 & 0 & 0 & 0 & 0 & 1\\
    0 & 0 & 0 & 0 & 0 & 0 & 0 & 0 & 0 & 0 & 1\\
    0 & 0 & 0 & 0 & 0 & 1 & 1 & 1 & 0 & 0 & 0\\
    1 & 1 & 0 & 0 & 0 & 0 & 0 & 0 & 0 & 0 & 0\\
    0 & 0 & 0 & 0 & 0 & 0 & 1 & 0 & 0 & 0 & 0\\
    0 & 0 & 0 & 0 & 0 & 0 & 0 & 1 & 0 & 0 & 0\\
    0 & 0 & 0 & 0 & 0 & 1 & 0 & 1 & 0 & 0 & 0\\
    0 & 0 & 0 & 0 & 1 & 0 & 0 & 0 & 0 & 1 & 0\\
    0 & 0 & 1 & 1 & 1 & 0 & 0 & 0 & 0 & 0 & 0\\
    0 & 0 & 0 & 0 & 1 & 0 & 0 & 0 & 0 & 0 & 0\\
    0 & 0 & 0 & 1 & 1 & 0 & 0 & 0 & 0 & 1 & 0
  \end{pmatrix}
\end{equation}

Dann fügt der Algorithmus die folgenden Kanten ein (jede Zeile ist ein Durchlauf, wobei die erste Zeile der Durchlauf der allerersten Schleife ist):
\begin{align*}
  & (1, 1), (2, 2), (3, 3), (4, 4), (5, 5), (6, 6), (7, 7), (8, 8), (9, 9), (10, 10), (11, 11)\\
  k = 1\ & (4, 5), (4, 11)\\
  k = 2\ & \\
  k = 3\ & (9, 6), (9, 7), (9, 8)\\
  k = 4\ & (9, 1), (9, 2), (9, 11), (11, 1), (11, 2)\\
  k = 5\ & (1, 7), (4, 7), (8, 7), (10, 7), (11, 7)\\
  k = 6\ & \\
  k = 7\ & (1, 6), (1, 8), (4, 6), (4, 8), (5, 6), (5, 8), (8, 6), (10, 6), (10, 8), (11, 6), (11, 8)\\
  k = 8\ & (1, 10), (3, 5), (3, 10), (4, 10), (5, 10), (6, 5), (6, 7), (6, 10), (7, 5), (7, 10), (9, 10)\\
  k = 9\ &\\
  k = 10\ & \\
  k = 11\ & (1, 4), (2, 1), (2, 4), (2, 5), (2, 6), (2, 7), (2, 8), (2, 10)
\end{align*}
\begin{equation}
  Trans(A) =
  \begin{pmatrix}
    1 & 1 & 0 & 1 & 1 & 1 & 1 & 1 & 0 & 1 & 1\\
    1 & 1 & 0 & 1 & 1 & 1 & 1 & 1 & 0 & 1 & 1\\
    0 & 0 & 1 & 0 & 1 & 1 & 1 & 1 & 0 & 1 & 0\\
    1 & 1 & 0 & 1 & 1 & 1 & 1 & 1 & 0 & 1 & 1\\
    0 & 0 & 0 & 0 & 1 & 1 & 1 & 1 & 0 & 1 & 0\\
    0 & 0 & 0 & 0 & 1 & 1 & 1 & 1 & 0 & 1 & 0\\
    0 & 0 & 0 & 0 & 1 & 1 & 1 & 1 & 0 & 1 & 0\\
    0 & 0 & 0 & 0 & 1 & 1 & 1 & 1 & 0 & 1 & 0\\
    1 & 1 & 1 & 1 & 1 & 1 & 1 & 1 & 1 & 1 & 1\\
    0 & 0 & 0 & 0 & 1 & 1 & 1 & 1 & 0 & 1 & 0\\
    1 & 1 & 0 & 1 & 1 & 1 & 1 & 1 & 0 & 1 & 1
  \end{pmatrix}
\end{equation}

\section{Aufgabe 5}
In einem solchen Graphen gibt es zwischen jedem möglichen geordneten Paar Knoten einen gerichteten Weg.
Und aus einer Menge mit $|V|$ Knoten, kann man $\sum_{i = 1}^{|V|} |V| = |V| \cdot (|V| - 1)$ geordnete Paare bilden.
Also gibt es $|V| \cdot (|V| - 1)$ Kanten im transitiven Abschluss von $G$.

Die Formel für die Anzahl der Paare folgt daraus, dass man, wenn man das $j$-te Element als erstes wählt, noch $|V| - 1$ Möglichkeiten für das zweite Element des Paares hat.

\end{document}