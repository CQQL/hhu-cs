\documentclass[10pt,a4paper]{article}
\usepackage[utf8]{inputenc}
\usepackage{amsmath}
\usepackage{amsfonts}
\usepackage{amssymb}
\usepackage{color}
\usepackage{listings}
\usepackage[left=2cm,right=2cm,top=2cm,bottom=2cm]{geometry}

\begin{document}

\section*{Aufgabe 2.1}

\subsection*{Teil 1}

\begin{align*}
237 / 2 = 118 \quad \textit{Rest $1$}\\
118 / 2 = 59 \quad \textit{Rest $0$}\\
59 / 2 = 29 \quad \textit{Rest $1$}\\
29 / 2 = 14 \quad \textit{Rest $1$}\\
14 / 2 = 7 \quad \textit{Rest $0$}\\
7 / 2 = 3 \quad \textit{Rest $1$}\\
3 / 2 = 1 \quad \textit{Rest $1$}\\
1 / 2 = 0 \quad \textit{Rest $1$}\\
\end{align*}
\begin{equation}
237_{10} = 11101101_{2}
\end{equation}

\begin{align*}
237 / 16 = 14 \quad \textit{Rest $13$}\\
14 / 16 = 0 \quad \textit{Rest $14$}\\
\end{align*}
\begin{equation}
237_{10} = ED_{16}
\end{equation}

\subsection*{Teil 2}

\begin{equation}
111010_{2} = 2 + 8 + 16 + 32 = 58_{10}
\end{equation}

\begin{equation}
111010_{2} = 11_{2} * 2^4 + 1010_{2} = 3A_{16}
\end{equation}

\section*{Aufgabe 2.2}

\subsection*{Teil 1}

Eine $n$-Bit-Zahl kann $2^{n}$ verschiedene Werte darstellen, weil
\begin{equation}
\sum_{k = 0}^{n - 1} 2^{k} = 2^n - 1
\end{equation}
wie wir in Info1 gezeigt haben.

Daher folgt, dass $7$-Bit-Zahlen $128$ verschiedene Werte von $0$ bis $127$ darstellen können und $8$-Bit-Zahlen $256$ verschiedene Werte von $0$ bis $255$.

\subsection*{Teil 2}

Da in der Zweierkomplementdarstellung das erste Bit für positive Zahlen stets $0$ und für negative Zahlen stets $1$ ist, bleiben noch $7$ Bit, um die eigentliche Zahl darzustellen.
Es lassen sich also jeweils $128$ positive und negative Werte darstellen, wobei der positive Wertebereich von $0$ bis $127$ geht und der negative von $-128$ bis $-1$.

\subsection*{Teil 3}

Nach meiner Argumentation aus Teil $2$ folgt
\begin{equation}
Wertebereich(n) = [-2^{n - 1}, 2^{n - 1} - 1]
\end{equation}

\subsection*{Teil 4}

\begin{equation}
87_{10} = 01010111_{2}
\end{equation}
\begin{equation}
2_{10} = 00000010_{2} \Rightarrow -2_{10} = 11111110_{2}
\end{equation}
\begin{equation}
17_{10} = 00010001_{2} \Rightarrow -17_{10} = 11101111_{2}
\end{equation}

\subsection*{Teil 5}

\subsubsection*{(a)}

\begin{equation}
14_{10} + (-22)_{10} = 00001110_{2} + 11101010_{2} = 11111000_{2} = -6_{10}
\end{equation}

Hier wird das OF nicht gesetzt, weil kein Übertrag an die höchstwertige Stelle erfolgt ist.

\subsubsection*{(b)}

\begin{equation}
103_{10} + 28_{10} = 01100111_{2} + 00011100_{2} = 10000011_{2} = -123_{10}
\end{equation}

Hier wird das OF gesetzt, weil ein Übertrag an die höchstwertige Stelle erfolgt ist.
Die Aussage des gesetzten OF ist dabei, dass das Ergebnis der Addition nicht in den Wertebereich passt und unwillkürlich das Vorzeichen invertiert wurde.

\section*{Aufgabe 2.3}

\begin{lstlisting}
#### Teil 1 ####

TAG 16, LINE 0, WORD 0, BYTE 0

16m, 72m, 20m, 9m, 183m, 60m, 19m, 12m, 148m, 22m, 8m, 125m, 183h, 34m, 47m, 60h

------------------------------------------
| Line | Valid | Tag | Bereich | Adresse |
------------------------------------------
|    1 |     0 |     |         |         |
|    1 |     1 |  16 |  16- 16 |      16 |
------------------------------------------
|    2 |     0 |     |         |         |
|    2 |     1 |  72 |  72- 72 |      72 |
------------------------------------------
|    3 |     0 |     |         |         |
|    3 |     1 |  20 |  20- 20 |      20 |
------------------------------------------
|    4 |     0 |     |         |         |
|    4 |     1 |   9 |   9-  9 |       9 |
------------------------------------------
|    5 |     0 |     |         |         |
|    5 |     1 | 183 | 183-183 |     183 |
|    5 |     1 | 183 | 183-183 |     183 |
------------------------------------------
|    6 |     0 |     |         |         |
|    6 |     1 |  60 |  60- 60 |      60 |
|    6 |     1 |  60 |  60- 60 |      60 |
------------------------------------------
|    7 |     0 |     |         |         |
|    7 |     1 |  19 |  19- 19 |      19 |
------------------------------------------
|    8 |     0 |     |         |         |
|    8 |     1 |  12 |  12- 12 |      12 |
------------------------------------------
|    9 |     0 |     |         |         |
|    9 |     1 | 148 | 148-148 |     148 |
------------------------------------------
|   10 |     0 |     |         |         |
|   10 |     1 |  22 |  22- 22 |      22 |
------------------------------------------
|   11 |     0 |     |         |         |
|   11 |     1 |   8 |   8-  8 |       8 |
------------------------------------------
|   12 |     0 |     |         |         |
|   12 |     1 | 125 | 125-125 |     125 |
------------------------------------------
|   13 |     0 |     |         |         |
|   13 |     1 |  34 |  34- 34 |      34 |
------------------------------------------
|   14 |     0 |     |         |         |
|   14 |     1 |  47 |  47- 47 |      47 |
------------------------------------------
|   15 |     0 |     |         |         |
------------------------------------------
|   16 |     0 |     |         |         |
------------------------------------------



#### Teil 2 ####

TAG 13, LINE 0, WORD 1, BYTE 2

16m, 72m, 20h, 9m, 183m, 60m, 19h, 12h, 148m, 22h, 8h, 125m, 183h, 34m, 47m, 60h

------------------------------------------
| Line | Valid | Tag | Bereich | Adresse |
------------------------------------------
|    1 |     0 |     |         |         |
|    1 |     1 |   2 |  16- 23 |      16 |
|    1 |     1 |   2 |  16- 23 |      20 |
|    1 |     1 |   2 |  16- 23 |      19 |
|    1 |     1 |   2 |  16- 23 |      22 |
------------------------------------------
|    2 |     0 |     |         |         |
|    2 |     1 |   9 |  72- 79 |      72 |
|    2 |     1 |   5 |  40- 47 |      47 |
------------------------------------------
|    3 |     0 |     |         |         |
|    3 |     1 |   1 |   8- 15 |       9 |
|    3 |     1 |   1 |   8- 15 |      12 |
|    3 |     1 |   1 |   8- 15 |       8 |
------------------------------------------
|    4 |     0 |     |         |         |
|    4 |     1 |  22 | 176-183 |     183 |
|    4 |     1 |  22 | 176-183 |     183 |
------------------------------------------
|    5 |     0 |     |         |         |
|    5 |     1 |   7 |  56- 63 |      60 |
|    5 |     1 |   7 |  56- 63 |      60 |
------------------------------------------
|    6 |     0 |     |         |         |
|    6 |     1 |  18 | 144-151 |     148 |
------------------------------------------
|    7 |     0 |     |         |         |
|    7 |     1 |  15 | 120-127 |     125 |
------------------------------------------
|    8 |     0 |     |         |         |
|    8 |     1 |   4 |  32- 39 |      34 |
------------------------------------------



#### Teil 3 ####

TAG 7, LINE 3, WORD 2, BYTE 2

16m, 72m, 20h, 9m, 183m, 60m, 19h, 12h, 148m, 22m, 8h, 125m, 183m, 34m, 47h, 60m

------------------------------------------
| Line | Valid | Tag | Bereich | Adresse |
------------------------------------------
|    0 |     0 |     |         |         |
|    0 |     1 |   0 |   0- 15 |       9 |
|    0 |     1 |   0 |   0- 15 |      12 |
|    0 |     1 |   0 |   0- 15 |       8 |
------------------------------------------
|    1 |     0 |     |         |         |
|    1 |     1 |   0 |  16- 31 |      16 |
|    1 |     1 |   0 |  16- 31 |      20 |
|    1 |     1 |   0 |  16- 31 |      19 |
|    1 |     1 |   1 | 144-159 |     148 |
|    1 |     1 |   0 |  16- 31 |      22 |
------------------------------------------
|    2 |     0 |     |         |         |
|    2 |     1 |   0 |  32- 47 |      34 |
|    2 |     1 |   0 |  32- 47 |      47 |
------------------------------------------
|    3 |     0 |     |         |         |
|    3 |     1 |   1 | 176-191 |     183 |
|    3 |     1 |   0 |  48- 63 |      60 |
|    3 |     1 |   1 | 176-191 |     183 |
|    3 |     1 |   0 |  48- 63 |      60 |
------------------------------------------
|    4 |     0 |     |         |         |
|    4 |     1 |   0 |  64- 79 |      72 |
------------------------------------------
|    5 |     0 |     |         |         |
------------------------------------------
|    6 |     0 |     |         |         |
------------------------------------------
|    7 |     0 |     |         |         |
|    7 |     1 |   0 | 112-127 |     125 |
------------------------------------------



#### Teil 4 ####

TAG 9, LINE 2, WORD 1, BYTE 2

16m, 72m, 20h, 9m, 183m, 60m, 19h, 12h, 148m, 22h, 8h, 125m, 183m, 34m, 47m, 60h

----------------------------------------------------------------------------
| Line | Valid | Tag | Bereich | Adresse | Valid | Tag | Bereich | Adresse |
----------------------------------------------------------------------------
|    0 |     0 |     |         |         |     0 |     |         |         |
|    0 |     1 |   1 |  32- 39 |      34 |     0 |     |         |         |
----------------------------------------------------------------------------
|    1 |     0 |     |         |         |     0 |     |         |         |
|    1 |     1 |   2 |  72- 79 |      72 |     0 |     |         |         |
|    1 |     1 |   2 |  72- 79 |      72 |     1 |   0 |   8- 15 |       9 |
|    1 |     1 |   2 |  72- 79 |      72 |     1 |   0 |   8- 15 |      12 |
|    1 |     1 |   2 |  72- 79 |      72 |     1 |   0 |   8- 15 |       8 |
|    1 |     1 |   1 |  40- 47 |      47 |     1 |   0 |   8- 15 |       8 |
----------------------------------------------------------------------------
|    2 |     0 |     |         |         |     0 |     |         |         |
|    2 |     1 |   0 |  16- 23 |      16 |     0 |     |         |         |
|    2 |     1 |   0 |  16- 23 |      20 |     0 |     |         |         |
|    2 |     1 |   0 |  16- 23 |      20 |     1 |   5 | 176-183 |     183 |
|    2 |     1 |   0 |  16- 23 |      19 |     1 |   5 | 176-183 |     183 |
|    2 |     1 |   0 |  16- 23 |      19 |     1 |   4 | 144-151 |     148 |
|    2 |     1 |   0 |  16- 23 |      22 |     1 |   4 | 144-151 |     148 |
|    2 |     1 |   0 |  16- 23 |      22 |     1 |   5 | 176-183 |     183 |
----------------------------------------------------------------------------
|    3 |     0 |     |         |         |     0 |     |         |         |
|    3 |     1 |   1 |  56- 63 |      60 |     0 |     |         |         |
|    3 |     1 |   1 |  56- 63 |      60 |     1 |   3 | 120-127 |     125 |
|    3 |     1 |   1 |  56- 63 |      60 |     1 |   3 | 120-127 |     125 |
----------------------------------------------------------------------------
\end{lstlisting}

Erstellt habe ich die Tabellen mit folgendem Ruby-Programm
\lstset{
language=ruby,                % choose the language of the code
basicstyle=\footnotesize,       % the size of the fonts that are used for the code
numbers=left,                   % where to put the line-numbers
numberstyle=\footnotesize,      % the size of the fonts that are used for the line-numbers
stepnumber=1,                   % the step between two line-numbers. If it is 1 each line will be numbered
numbersep=5pt,                  % how far the line-numbers are from the code
backgroundcolor=\color{white},  % choose the background color. You must add \usepackage{color}
showspaces=false,               % show spaces adding particular underscores
showstringspaces=false,         % underline spaces within strings
showtabs=false,                 % show tabs within strings adding particular underscores
frame=single,           % adds a frame around the code
tabsize=2,          % sets default tabsize to 2 spaces
captionpos=b,           % sets the caption-position to bottom
breaklines=true,        % sets automatic line breaking
breakatwhitespace=false,    % sets if automatic breaks should only happen at whitespace
escapeinside={\%*}{*)}          % if you want to add a comment within your code
}
\begin{lstlisting}
class LRU
  def initialize
    @list = []
  end

  def front
    @list.first
  end

  def send_back value
    @list.delete value
    @list << value
  end
end

class Cache
  WORD_LENGTH = 4

  def initialize capacity, line_length, tag_width, line_width, word_width, byte_width
    @capacity = capacity
    @line_length = line_length
    @lines = capacity / line_length
    @tag_width = tag_width
    @line_width = line_width
    @word_width = word_width
    @byte_width = byte_width

    @hit_list = []

    @cache = (1..@lines).map { |line| [{ valid: 0, tag: nil, address: nil }] }
  end

  def line_info
    "TAG #{@tag_width}, LINE #{@line_width}, WORD #{@word_width}, BYTE #{@byte_width}"
  end

  def hit address, hit
    @hit_list << { address: address, hit: hit }
  end

  def tag address
    (address & (0xFFFF << tag_offset)) >> tag_offset
  end

  def word_offset
    @byte_width
  end

  def line_offset
    word_offset + @word_width
  end

  def tag_offset
    line_offset + @line_width
  end

  def hit_list
    @hit_list.map { |info| "#{info[:address]}#{info[:hit] ? "h" : "m"}" }.join ", "
  end

  def range line, number = 0
    if line[:tag]
      start = (line[:tag] << tag_offset) + (number << line_offset)

      range = (start..start + @line_length - 1)

      "#{range.min.to_s.rjust 3}-#{range.max.to_s.rjust 3}"
    else
      " " * 7
    end
  end
end

class FullyAssociativeCache < Cache
  def initialize capacity, line_length, tag_width, line_width, word_width, byte_width
    super

    @lru = LRU.new
  end

  def read address
    tag = tag address

    (0...@lines).map do |i|
      if @cache[i].last[:tag] == tag && @cache[i].last[:valid]
        @cache[i] << { valid: 1, tag: tag, address: address }

        hit address, true

        @lru.send_back i

        return
      elsif @cache[i].last[:valid] == 0
        @cache[i] << { valid: 1, tag: tag, address: address }

        hit address, false

        @lru.send_back i

        return
      end
    end

    lru_line = @lru.front

    @cache[lru_line] << { valid: 1, tag: tag, address: address }

    hit address, false

    @lru.send_back lru_line
  end

  def to_table
    header = "| Line | Valid | Tag | Bereich | Adresse |"
    delimiter = "-" * header.length
    table = []

    table << delimiter
    table << header

    i = 1

    @cache.each do |line_history|
      table << delimiter

      line_history.each do |line|
        table << "| #{i.to_s.rjust 4} | #{line[:valid].to_s.rjust 5} | #{line[:tag].to_s.rjust 3} | #{range(line)} | #{line[:address].to_s.rjust 7} |"
      end

      i = i + 1
    end

    table << delimiter

    table
  end
end

class OneWayCache < Cache
  def read address
    tag = tag address
    line = line address

    if @cache[line].last[:valid] && @cache[line].last[:tag] == tag
      hit address, true
    else
      hit address, false
    end

    @cache[line] << { valid: 1, tag: tag, address: address }
  end

  def line address
    mask = (0...@line_width).map { |i| 1 << i }.reduce(:+)

    (address & (mask << line_offset)) >> line_offset
  end

  def to_table
    header = "| Line | Valid | Tag | Bereich | Adresse |"
    delimiter = "-" * header.length
    table = []

    table << delimiter
    table << header

    i = 0

    @cache.each do |line_history|
      table << delimiter

      line_history.each do |line|
        table << "| #{i.to_s.rjust 4} | #{line[:valid].to_s.rjust 5} | #{line[:tag].to_s.rjust 3} | #{range(line, i)} | #{line[:address].to_s.rjust 7} |"
      end

      i = i + 1
    end

    table << delimiter

    table
  end
end

class TwoWayCache < Cache
  def initialize capacity, line_length, tag_width, line_width, word_width, byte_width
    super

    @lines /= 2

    @cache = (1..@lines).map { |line| {lru: LRU.new, data: [[{ valid: 0, tag: nil, address: nil }, { valid: 0, tag: nil, address: nil }]]} }
  end

  def read address
    tag = tag address
    line = line address

    data = @cache[line][:data].last
    lru = @cache[line][:lru]

    if data[0][:valid] == 1 && data[0][:tag] == tag
      new_data = [{ valid: 1, tag: tag, address: address }, data[1]]

      lru.send_back 0
      hit address, true
    elsif data[1][:valid] == 1 && data[1][:tag] == tag
      new_data = [data[0], { valid: 1, tag: tag, address: address }]

      lru.send_back 1
      hit address, true
    else
      hit address, false

      if data[0][:valid] == 0
        new_data = [{ valid: 1, tag: tag, address: address }, data[1]]

        lru.send_back 0
      elsif data[1][:valid] == 0
        new_data = [data[0], { valid: 1, tag: tag, address: address }]

        lru.send_back 1
      else
        if lru.front == 0
          new_data = [{ valid: 1, tag: tag, address: address }, data[1]]

          lru.send_back 0
        else
          new_data = [data[0], { valid: 1, tag: tag, address: address }]

          lru.send_back 1
        end
      end
    end

    @cache[line][:data] << new_data
  end

  def line address
    mask = (0...@line_width).map { |i| 1 << i }.reduce(:+)

    (address & (mask << line_offset)) >> line_offset
  end

  def to_table
    header = "| Line |" + " Valid | Tag | Bereich | Adresse |" * 2
    delimiter = "-" * header.length
    table = []

    table << delimiter
    table << header

    i = 0

    @cache.each do |line|
        table << delimiter

        line[:data].each do |data_history|
          row = "| #{i.to_s.rjust 4} |"

          data_history.each do |data|
            row << " #{data[:valid].to_s.rjust 5} | #{data[:tag].to_s.rjust 3} | #{range(data, i)} | #{data[:address].to_s.rjust 7} |"
          end

          table << row
        end

        i = i + 1
    end

    table << delimiter

    table
  end
end

addresses = [16, 72, 20, 9, 183, 60, 19, 12, 148, 22, 8, 125, 183, 34, 47, 60]
caches = []
caches << FullyAssociativeCache.new(16, 1, 16, 0, 0 ,0)
caches << FullyAssociativeCache.new(64, 8, 13, 0, 1, 2)
caches << OneWayCache.new(128, 16, 7, 3, 2, 2)
caches << TwoWayCache.new(64, 8, 9, 2, 1, 2)

caches.each_with_index do |cache, index|
  puts "#### Teil #{index + 1} ####"
  puts

  puts cache.line_info
  puts

  addresses.each do |address|
    cache.read address
  end

  puts cache.hit_list
  puts

  puts cache.to_table
  puts

  puts
  puts
end
\end{lstlisting}

\end{document}