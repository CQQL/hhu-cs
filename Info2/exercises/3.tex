\documentclass[10pt,a4paper]{article}
\usepackage[utf8]{inputenc}
\usepackage[german]{babel}
\usepackage{amsmath}
\usepackage{amsfonts}
\usepackage{amssymb}
\usepackage[left=2cm,right=2cm,top=2cm,bottom=2cm]{geometry}

\begin{document}

\section*{Aufgabe 3.1}

Ich wandele die Zahl mittels der zweiten Technik im Skript um.

\begin{equation}
2,6875 * 2^{6} = 172
\end{equation}
\begin{equation}
172_{10} = 10101100_{2} \Rightarrow 2,6875_{10} = 10101100_{2}
\end{equation}

\section*{Aufgabe 3.2}

\begin{equation}
148,625 = (-1)^0 \cdot 1,0010100101_{2} \cdot 2^{7}
\end{equation}
\begin{equation}
7 = 134 - 127 \Rightarrow \textit{Der Exponent ist $134$}
\end{equation}
Also ist die IEEE754-Darstellung
\begin{equation}
0.10000110.00101001010000000000000
\end{equation}
Die Punkte sind einfach nur Trennzeichen, um das Lesen zu erleichtern.

\section*{Aufgabe 3.3}

In dieser Situation heißt LRU wohl Least Recently Bought.

Die Vorteil von FIFO ist, dass man eine determinierte Reihenfolge hat, wo ich in dieser Situation aber wiederum keinen Vorteil drin sehe.

Bei LRU hat man wahrscheinlich die Dinge im Regal, die häufig angefragt werden.

Deshalb würde ich LRU wählen.
Da einige Artikel öfter gekauft werden als andere, haben diese dann eine geringere Wahrscheinlichkeit durch LRU ausgewählt zu werden, sind also öfter im Regal und der Laden macht mehr Umsatz.

\section*{Aufgabe 3.4}

\subsection*{Teil 1}

In der linken Spalte steht ein Seitenzugriff und rechts die Seitenrahmenbelegung nachdem dieser Zugriff ausgeführt wurde.

\begin{tabular}{|l|l|l|l|l|}
\hline
Zugriff & Seite 1 & Seite 2 & Seite 3 & Seite 4\\
\hline
0 & 0 & & &\\
1 & 0 & 1 & &\\
2 & 0 & 1 & 2 &\\
3 & 0 & 1 & 2 & 3\\
0 & 0 & 1 & 2 & 3\\
1 & 0 & 1 & 2 & 3\\
4 & 0 & 1 & 4 & 3\\
0 & 0 & 1 & 4 & 3\\
1 & 0 & 1 & 4 & 3\\
2 & 0 & 1 & 4 & 2\\
3 & 0 & 1 & 3 & 2\\
4 & 4 & 1 & 3 & 2
\end{tabular}

\subsection*{Teil 2}

Da die Seiten $\{1, 2, 3, 4\}$ am Ende geladen sind, entstehen Page Faults bei Zugriffen auf die Seiten $[0, 7] - [1, 4] = \{0, 5, 6, 7\}$.

Ich nehme an, dass die Seite mit $\lfloor adresse / 1024 \rfloor$ berechnet wird, also z.B. $1772$ in Seite $1$ landet.

Dann ergeben sich Page Faults ganz konkret bei den Adressen $0 - 1023$, $5120 - 6143$, $6144 - 7167$ und $7168 - 8191$.

\subsection*{Teil 3}

Seien $a$ die Anfangsadresse des zugewiesenen Seitenrahmens, $b$ die angefragte Adresse, $c$ die untere Grenze des Adressraums, der auf die zugehörige Seite abgebildet wird.
Dann berechnet sich die physikalische Adresse durch $a + b - c$.

Damit ergibt sich
\begin{align}
1111 & \Rightarrow 1111\\
2222 & \Rightarrow 3246\\
3333 & \Rightarrow 2309\\
4444 & \Rightarrow 444
\end{align}

\subsection*{Teil 4}

In der linken Spalte steht ein Seitenzugriff und rechts die Seitenrahmenbelegung nachdem dieser Zugriff ausgeführt wurde.

\begin{tabular}{|l|l|l|l|l|}
\hline
Zugriff & Seite 1 & Seite 2 & Seite 3 & Seite 4\\
\hline
0 & 0 &  &  & \\
1 & 0 & 1 &  & \\
2 & 0 & 1 & 2 & \\
3 & 0 & 1 & 2 & 3\\
0 & 0 & 1 & 2 & 3\\
1 & 0 & 1 & 2 & 3\\
4 & 4 & 1 & 2 & 3\\
0 & 4 & 0 & 2 & 3\\
1 & 4 & 0 & 1 & 3\\
2 & 4 & 0 & 1 & 2\\
3 & 3 & 0 & 1 & 2\\
4 & 3 & 4 & 1 & 2
\end{tabular}

\subsection*{Teil 5}

In der linken Spalte steht ein Seitenzugriff und rechts die Seitenrahmenbelegung nachdem dieser Zugriff ausgeführt wurde.

\begin{tabular}{|l|l|l|l|}
\hline
Zugriff & Seite 1 & Seite 2 & Seite 3\\
\hline
0 & 0 &  & \\
1 & 0 & 1 & \\
2 & 0 & 1 & 2\\
3 & 3 & 1 & 2\\
0 & 3 & 0 & 2\\
1 & 3 & 0 & 1\\
4 & 4 & 0 & 1\\
0 & 4 & 0 & 1\\
1 & 4 & 0 & 1\\
2 & 4 & 2 & 1\\
3 & 4 & 2 & 3\\
4 & 4 & 2 & 3
\end{tabular}

\subsection*{Teil 6}

In Teilaufgabe $4$ treten $10$ und in Teilaufgabe $5$ $9$ Seitenfehler auf.

Dies entspricht nicht meiner Erwartung.
Ich habe erwartet, dass in Teil 5 mehr Seitenfehler auftreten, weil weniger Seitenrahmen vorhanden sind und deshalb wahrscheinlich öfter Seiten ausgetauscht werden müssen.
Man kann aus diesem konkreten Fall auch nicht auf die Allgemeinheit zurückschließen, aber man sieht, dass es Fälle gibt, in denen verschiedene Strategien besonders gut oder besonders schlecht funktionieren.

\end{document}