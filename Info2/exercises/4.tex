\documentclass[10pt,a4paper]{article}
\usepackage[utf8]{inputenc}
\usepackage[german]{babel}
\usepackage{amsmath}
\usepackage{amsfonts}
\usepackage{amssymb}
\usepackage{color}
\usepackage{listings}
\usepackage[left=2cm,right=2cm,top=2cm,bottom=2cm]{geometry}

\begin{document}

\section*{Aufgabe 4.1}

Da eine Seitentabelle wie ein Direct Mapped Cache funktioniert insofern, dass die Stelle in der Seitentabelle, an der geguckt werden muss, durch Offsetbits bestimmt ist, muss die Seitentabelle für jedes mögliche Offset eine Zeile haben.
Wir haben 32-Bit-Adressen und $16$ KB = $2^{14}$ Byte große Seiten.
Das heißt, dass die letzten 14 Bit das Offset innerhalb der Seite darstellen, während die verbleibenden $18$ Bit die Zeile in der Seitentabelle bestimmen.
Wie im Eingangssatz dargelegt, muss die Seitentabelle deshalb $2^{18}$ Einträge haben, die bei $2$ Byte pro Eintrag insgesamt $2^{18} * 2 = 2^{19} = 524288$ Byte oder auch $512$ KB belegen.

Bei einer zweistufigen Seitentabelle ist die Situation insofern anders, dass es eine Primärtabelle mit $2^{9}$ Einträgen gibt, die auf Sekundärtabellen verweisen, die wiederum $2^{9}$ Einträge haben, aber nur existieren, wenn auf sie schonmal zugegriffen wurde.
Um die Frage zu beantworten, gehe ich davon aus, dass auf die Speicherbereiche, die gerade nicht alloziert sind auch nocht nicht zugegriffen wurde.
Man kann es also so betrachten, als ob die Primärtabelle $2^{14 + 9} = 2^{23}$ Byte große Seiten verwalten würde.
Dann werden am oberen Ende des Adressraums $128 * 1MB = 2^{7} * 2^{20} = 2^{27}$ Bytes belegt.
Wenn man dies auf die Seiten der Primärtabelle aufteilt, erhält man $2^{27} / 2^{23} = 2^{4}$ Seiten, also $2^{4}$ Sekundärtabellen, die angelegt worden sind und jeweils $2^{9} * 2 = 2^{10}$ Byte Speicher belegen.
Da man das ganze am unteren Ende des Adressraums nochmal hat, verdoppeln sich die Sekundärtabellen und belegen insgesamt $2 * 2^{4} * 2^{10} = 2^{15}$ Byte Speicher.
Dazu addiert man noch den Speicher, der von der Primärtabelle belegt wird.
Ich nehme nochmal an, dass die Sekundärtabellen geeignet platziert werden (vielleicht immer am Beginn der ''großen`` Pageframes), sodass man sie ebenfalls mit $2$ Byte adressieren kann.
Dann belegt die Primärtabelle nämlich auch $2^{10}$ Byte und man kommt insgesamt auf $2^{15} + 2^{10} = 33792$ Byte belegten Speicher.

\section*{Aufgabe 4.2}

Die physikalische Adresse zu $1:12$ ist $5012$.
Die physikalische Adresse zu $2:2048$ existiert nicht, weil Segment $2$ eine Länge von $2048$ hat und damit die höchste Adresse $2047$ ist.
Die physikalische Adresse zu $3:32$ existiert nicht, weil Segment $3$ aktuell nicht im physischen Speicher vorhanden ist.
Die physikalische Adresse zu $4:97$ ist $98$.

\section*{Aufgabe 4.3}

Ich schreibe den Speicher immer als Spalte, da es sonst nicht auf die Seite passen würde.

\subsection*{Teil 3}

\lstset{
basicstyle=\footnotesize,       % the size of the fonts that are used for the code
numbers=left,                   % where to put the line-numbers
}
\begin{lstlisting}
13
24













7

3E  <- q
7A
7C
FF
9A
B3
C0
6
0
FF  <- p
89
28
8D
42
E2
\end{lstlisting}

\subsection*{Teil 4}

\lstset{
basicstyle=\footnotesize,       % the size of the fonts that are used for the code
numbers=left,                   % where to put the line-numbers
}
\begin{lstlisting}
9   # Freien Speicher reduzieren
24









2

BE  <- r
EF
7

3E  <- q
7A
7C
FF
9A
B3
C0
6
0
FF  <- p
89
28
8D
42
E2
\end{lstlisting}

\subsection*{Teil 5}

\lstset{
basicstyle=\footnotesize,       % the size of the fonts that are used for the code
numbers=left,                   % where to put the line-numbers
}
\begin{lstlisting}
9
15  # Zeiger auf den naechsten freien Block anpassen









2

BE  <- r
EF
15  # Laenge des freien Speichers. Hier wird der letzte Block mit zusammengefasst
0   # Zeiger auf den naechsten freien Block
3E  <- q
7A
7C
FF
9A
B3
C0
6
0
FF  <- p
89
28
8D
42
E2
\end{lstlisting}

\subsection*{Teil 6}

\lstset{
basicstyle=\footnotesize,       % the size of the fonts that are used for the code
numbers=left,                   % where to put the line-numbers
}
\begin{lstlisting}
9
15









2

BE  <- r
EF
1   # Freien Speicher reduzieren
0
3E  <- q
12  # Header einfuegen und belegten Speicher eintragen
    # 
A7  <- s
AF
C4
7F
03
90
CD  <- p
13
A0
F5
49
00
\end{lstlisting}

\subsection*{Teil 7}

\lstset{
basicstyle=\footnotesize,       % the size of the fonts that are used for the code
numbers=left,                   % where to put the line-numbers
}
\begin{lstlisting}
9
15









2

BE  <- r
EF
1
24  # Adresse des neuen freien Blocks eintragen
3E  <- q
12

A7  <- s
AF
C4
7F
03  # 3 Bytes freigeben
0   # Adresse des naechsten freien Blocks setzen
CD  <- p
13
A0
F5
49
00
\end{lstlisting}

\subsection*{Teil 8}

\lstset{
basicstyle=\footnotesize,       % the size of the fonts that are used for the code
numbers=left,                   % where to put the line-numbers
}
\begin{lstlisting}
9
15









2

BE  <- r
EF
1
24
3E  <- q
12

8A  <- s # Hier die Daten hinschreiben
C3
DF
56
0F
05
FF  <- p
10
6E
B7
C4
00
\end{lstlisting}

\subsection*{Teil 9}

\lstset{
basicstyle=\footnotesize,       % the size of the fonts that are used for the code
numbers=left,                   % where to put the line-numbers
}
\begin{lstlisting}
9
15









2

BE  <- r
EF
1
24
3E  <- q
12

8A  <- s
C3
DF
56
0F
05
4B  <- p <- t # Hier zeigt t hin, weil es hier offensichtlich F=15 Byte freien Speicher gibt
61
74
61
73
74  # Der Rest geht wohl entweder verloren oder das Programm wird beendet
\end{lstlisting}

\section*{Aufgabe 4.4}

\subsection*{Teil 1}

Eine mögliche Ausgabe ist

\begin{lstlisting}
[cqql@arch linux-ex]$ ./first 
Enter a number: 50
Enter another number: 20
Register Dump # 1
EAX = 00000046 EBX = 00000046 ECX = 9837D8E6 EDX = FFDBC4D4
ESI = 00000000 EDI = 00000000 EBP = FFDBC488 ESP = FFDBC468
EIP = 080484C7 FLAGS = 0202                     
Memory Dump # 2 Address = 08049AF8
08049AF0 75 6D 62 65 72 3A 20 00 59 6F 75 20 65 6E 74 65 "umber: ?You ente"
08049B00 72 65 64 20 00 20 61 6E 64 20 00 2C 20 74 68 65 "red ? and ?, the"
You entered 50 and 20, the sum of these is 70
\end{lstlisting}

\subsection*{Teil 2}

\lstset{
language={[x86masm]Assembler},                % choose the language of the code
basicstyle=\footnotesize,       % the size of the fonts that are used for the code
numbers=left,                   % where to put the line-numbers
numberstyle=\footnotesize,      % the size of the fonts that are used for the line-numbers
stepnumber=1,                   % the step between two line-numbers. If it is 1 each line will be numbered
numbersep=5pt,                  % how far the line-numbers are from the code
backgroundcolor=\color{white},  % choose the background color. You must add \usepackage{color}
showspaces=false,               % show spaces adding particular underscores
showstringspaces=false,         % underline spaces within strings
showtabs=false,                 % show tabs within strings adding particular underscores
frame=single,           % adds a frame around the code
tabsize=2,          % sets default tabsize to 2 spaces
captionpos=b,           % sets the caption-position to bottom
breaklines=true,        % sets automatic line breaking
breakatwhitespace=false,    % sets if automatic breaks should only happen at whitespace
escapeinside={\%*}{*)}          % if you want to add a comment within your code
}
\begin{lstlisting}
%include "asm_io.inc"

segment .data
	name db "Marten Lienen", 0

segment .text
	global asm_main
asm_main:
	enter 0,0
	pusha

	mov eax, name
	call print_string

	popa
	mov eax, 0
	leave
	ret

\end{lstlisting}

\section*{Aufgabe 4.5}

Bei meinem Speicher sind niedrige Adressen links und nach rechts wird es immer höher.

Hexadezimaldarstellung des Speichers: $FFF67530$

Ausgabe des Big-Endian-Systems: $-10$ und $30000$.
Ausgabe des Little-Endian-Systems: $-2305$ und $12405$.

\end{document}