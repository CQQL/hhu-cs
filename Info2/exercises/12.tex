\documentclass[10pt,a4paper]{article}
\usepackage[utf8]{inputenc}
\usepackage[german]{babel}
\usepackage{amsmath}
\usepackage{amsfonts}
\usepackage{amssymb}
\usepackage{color}
\usepackage{listings}
\usepackage[left=2cm,right=2cm,top=2cm,bottom=2cm]{geometry}

\begin{document}

\section*{Aufgabe 12.1}

\subsection*{Teil 1}

\begin{equation}
F0 = 1, F1 = 0, ENA = 0, ENB = 1, INVA = 0
\end{equation}
\begin{equation}
F0 = 1, F1 = 0, ENA = 0, ENB = 1, INVA = 1
\end{equation}
\begin{equation}
F0 = 1, F1 = 0, ENA = 1, ENB = 1, INVA = 0
\end{equation}
\begin{equation}
F0 = 1, F1 = 0, ENA = 1, ENB = 1, INVA = 1
\end{equation}

\subsection*{Teil 2}

\begin{equation}
g(A, B) = \overline{A}B
\end{equation}

\section*{Aufgabe 12.2}

\begin{tabular}{l|l}
MAR = CPP; rd & Adresse setzen und lesen\\
MAR = SP & auf Ergebnis warten und Schreibziel setzen\\
H = MDR & Ergebnis in H speichern, um Rechnung zu ermöglichen\\
MDR = TOS = TOS - H; wr & Berechnung durchführen und an SP schreiben\\
 & Auf erfolgreiches Speichern warten\\
\end{tabular}

\section*{Aufgabe 12.3}

\subsection*{Teil 1}

Ein einfacher Ansatz wäre, 1x um 8 Stellen nach links zu verschieben und dann 7x nach rechts.
Dabei gehen allerdings die 7 höchstwertigen Bits verloren.
Die kann man wiederherstellen, indem man das Ergebnis in einem Hilfsregister speichert und dann ein Register mit -1 lädt, 3x8x nach links verschiebt, mit der Ursprungswert per AND verknüpft, 7x nach rechts, 1x nach links schiebt und mit dem ersten Ergebnis per OR verknüpft.
Der zweite, aufwendige Schritt filtert dabei mit einer Bitmaske die Bits 25-31, die beim ersten Ergebnis fehlen, und lädt sie an die richtige Stelle im Ergebnis.

\subsection*{Teil 2}

Man kann jede Bitmaske erzeugen, indem man ein Register mit 0 lädt und dann 31x nach links verschiebt und dabei an den entsprechenden Stellen noch eine 1 addiert, die dann in der letzten Stelle landet, und im weiteren Verlauf an die vorgesehene Position wandert.
Bei bestimmten Mustern kann man auch mit $-1$ und der Negation arbeiten, weil dies einem bereits ein Register voller Einsen liefert.

Die Bitmaske für das least significant bit ist die $1$ und für das most significant bit die $2^{31}$.

Least significant bit
\begin{lstlisting}
H = 1 ; Bitmaske erzeugen
\end{lstlisting}

Most significant bit
\begin{lstlisting}
H = 1 ; Ein Register mit 0 laden
H = H << 8 ; * 2^8
H = H << 8 ; * 2^8
H = H >> 1 ; * 2^-1
H = H << 8 ; * 2^8
H = H << 8 ; * 2^8 => 31x nach links verschieben; fertig
\end{lstlisting}

\subsection*{Teil 3}

Mit einem numerischen Zähler lädt man die Anzahl der Durchläufe in ein Register und verringert den Wert bei jedem Durchlauf um 1.
Wenn er bei 0 angekommen ist, springt man aus der Schleife.

Mit einer 1-Bitmaske kann man nur Schleifen umsetzen, die eine Potenz von 2 mal durchlaufen.
Wenn die Schleife $2^{n}$ mal durchlaufen soll, erstellt man die Bitmaske so, dass die 1 an der n-ten Stelle steht.
Wenn dann am Ende des Schleifendurchgangs die verANDung der beiden Werte nicht $0$ ist, ist die Schleife vorbei.

\section*{Aufgabe 12.4}

\subsection*{Teil 1}

Wenn der Wert oben auf dem Stack $\le 0$ ist, soll eine $1$ auf den Stack gepusht werden, sonst eine $0$.

\subsection*{Teil 2}

Es funktioniert nicht, weil L1 und L3 dieselbe Speicherstelle belegen müssten, weil Sprungziele von bedingten Sprüngen immer genau $0x100$ auseinander liegen müssen.
Da beide Bedingungen dasselbe Sprungziel im if-Zweig haben, müssen die unterschiedlichen Anweisungen L1 und L3 beide an Position $Pos(L2) - 0x100$ gespeichert werden.

\subsection*{Teil 3}

Man kann eine weitere Zeile L4 einführen, die genau dasselbe macht wie L2.
Dann kann man L1-L2 und L3-L4 jeweils $0x100$ auseinander legen, sodass eine Kollision vermieden wird.

\begin{tabular}{l|l}
 & MAR = SP = SP + 1;\\
 & N = TOS; if (N) goto L2; else goto L1;\\
L1 & Z = TOS; if (Z) goto L4; else goto L3;\\
L2 & TOS = MDR = 1; wr; goto Main1;\\
L3 & TOS = MDR = 0; wr; goto Main1;\\
L4 & TOS = MDR = 1; wr; goto Main1;
\end{tabular}

\end{document}