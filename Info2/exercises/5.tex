\documentclass[10pt,a4paper]{article}
\usepackage[utf8]{inputenc}
\usepackage[german]{babel}
\usepackage{amsmath}
\usepackage{amsfonts}
\usepackage{amssymb}
\usepackage{color}
\usepackage{listings}
\usepackage[left=2cm,right=2cm,top=2cm,bottom=2cm]{geometry}

\begin{document}

\section*{Aufgabe 5.1}

\subsection*{Teil 1}

60%

\subsection*{Teil 2}

Neue Übungsblätter werden immer am Mittwoch Morgen bereitgestellt, Abgabeschluss ist immer der darauffolgende Mittwoch um 8:30 Uhr

\subsection*{Teil 3}

Keine Ahnung. Ist mir auch egal.

\subsection*{Teil 4}

Es steht bei der Abgabe unter dem Dateiuploadfeld.

\subsection*{Teil 5}

Datei entspricht nicht den erlaubten Datentypen!

\subsection*{Teil 6}

Weil die Session auslaufen könnte und man dann unter Umständen die eingegebenen Texte verliert.

\subsection*{Teil 7}

Dann wende ich mich an Norbert Goebel.

\subsection*{Teil 8}

informatik2@uni-duesseldorf.de

\subsection*{Teil 9}

Mo.	13:30–14:30 Uhr		bei Norbert Goebel	 (Raum 25.12.02.41)\\
Fr.	13:30–14:30 Uhr		bei Matthias Radig	 (Raum 25.12.02.38)

\section*{Aufgabe 5.2}

\subsection*{Teil 1}

Hier hat man ein Speicherleck, weil 10 mal Speicher reserviert, aber am Ende nur der zuletzt reservierte Speicher freigegeben wird.
Zu lösen wäre es, indem man den Speicher am Ende jedes Schleifendurchlaufs wieder freigibt.

\subsection*{Teil 2}

Zuerst nicht die erfolgreiche Speicherreservierung in $d$ überprüft, sondern in einer unbekannten Variable $i$.
Außerdem schreibt die zweite Wertzuweisung an eine nicht reservierte Speicherstelle.
Wahrscheinlich waren hier $0$ und $1$ als Indizes gemeint anstatt $1$ und $2$.

\subsection*{Teil 3}

$i$ ist ein Zeiger auf einen int-Zeiger, aber es wird nur für die erste Stufe Speicher reserviert.
Ein besserer Ausdruck fällt mir dafür nicht ein und ich gebe einfach eine korrigierte Version an
\lstset{
language={C},                % choose the language of the code
basicstyle=\footnotesize,       % the size of the fonts that are used for the code
numbers=left,                   % where to put the line-numbers
numberstyle=\footnotesize,      % the size of the fonts that are used for the line-numbers
stepnumber=1,                   % the step between two line-numbers. If it is 1 each line will be numbered
numbersep=5pt,                  % how far the line-numbers are from the code
backgroundcolor=\color{white},  % choose the background color. You must add \usepackage{color}
showspaces=false,               % show spaces adding particular underscores
showstringspaces=false,         % underline spaces within strings
showtabs=false,                 % show tabs within strings adding particular underscores
frame=single,           % adds a frame around the code
tabsize=2,          % sets default tabsize to 2 spaces
captionpos=b,           % sets the caption-position to bottom
breaklines=true,        % sets automatic line breaking
breakatwhitespace=false,    % sets if automatic breaks should only happen at whitespace
escapeinside={\%*}{*)}          % if you want to add a comment within your code
}
\begin{lstlisting}
int** i;
i = malloc(sizeof(int*));

if (i == NULL) {
  /* hier ausgelassen: Fehlermeldung, Programm beenden */ 
}

*i = malloc(sizeof(int));

if (*i == NULL) {
  /* hier ausgelassen: Fehlermeldung, Programm beenden */ 
}

**i = 5;

free(*i);
free(i);
\end{lstlisting}

\section*{Aufgabe 5.3}

Da es systemabhängig zu sein scheint und ich keine Angabe finden konnte, gehe ich davon aus, dass ein gesetztes PF gerade Parität der unteren 8 Bit anzeigt.
Und das AF ist zwar in der Tabelle, aber nicht im Index des Buches zu finden.
Nach Wikipedia nehme ich an, dass das AF gesetzt wird, wenn es einen Übertrag vom 4t-niedrigsten zum 5t-niedrigsten Bit gibt.
Das DF wird außerdem immer seinen Standardwert behalten, da in keiner Aufgabe CLD oder STD verwendet wird.

\subsection*{Teil 1}

Hier wird zuerst die Hexadezimalzahl $ffffffff$ in das eax-Register geschrieben und dann $1$ addiert.
Es wird ZF gesetzt, weil das Ergebnis $0$ ist.
Es wird CF gesetzt, weil es einen Übertrag über das höchste Bit hinaus gibt.
Es wird PF gesetzt, weil die unteren 8 Bit $0$ sind und somit eine gerade Parität haben.
Es wird AF gesetzt, weil es einen Übertrag von Bit 4 zu 5 gibt.

\subsection*{Teil 2}

Hier wird zuerst die Hexadezimalzahl $fffffff0$ in das eax-Register geschrieben und dann $1$ addiert.
Es wird SF gesetzt, weil das höchste Bit $1$ ist.

\subsection*{Teil 3}

Hier wird die Hexadezimalzahl $12345678$ in das eax-Register geschrieben und von da aus nochmal in das ebx-Register kopiert.
Dann werden die beiden Register verglichen.
Es wird ZF gesetzt, weil eax und ebx gleich sind und somit die Differenz $0$ ist.
Es wird PF gesetzt, weil die niedrigsten $8$ Bit der Differenz von eax und ebx gerade Parität aufweisen.

\subsection*{Teil 4}

Hier wird die Hexadezimalzahl $ffffffff$ in das eax- und danach die Hexadezimalzahl $fffffff1$ in das ebx-Register geschrieben.
Dann wird $eax - ebx$ berechnet.
Es wird OF gesetzt, weil sich das oberste Bit ändert.

\section*{Aufgabe 5.4}

Ich nehme immer ein Byte, kopiere es nach ecx, verschiebe es an die richtige Stelle, isoliere es mit einem and und füge es dann ebx hinzu mit einem or.
\lstset{
language={[x86masm]Assembler},                % choose the language of the code
basicstyle=\footnotesize,       % the size of the fonts that are used for the code
numbers=left,                   % where to put the line-numbers
numberstyle=\footnotesize,      % the size of the fonts that are used for the line-numbers
stepnumber=1,                   % the step between two line-numbers. If it is 1 each line will be numbered
numbersep=5pt,                  % how far the line-numbers are from the code
backgroundcolor=\color{white},  % choose the background color. You must add \usepackage{color}
showspaces=false,               % show spaces adding particular underscores
showstringspaces=false,         % underline spaces within strings
showtabs=false,                 % show tabs within strings adding particular underscores
frame=single,           % adds a frame around the code
tabsize=2,          % sets default tabsize to 2 spaces
captionpos=b,           % sets the caption-position to bottom
breaklines=true,        % sets automatic line breaking
breakatwhitespace=false,    % sets if automatic breaks should only happen at whitespace
escapeinside={\%*}{*)}          % if you want to add a comment within your code
}
\begin{lstlisting}
%include "asm_io.inc"

segment .text
        global asm_main
asm_main:
        enter 0,0
        pusha

        mov eax, 0A0B70807h
        mov ebx, 0
        mov ebx, eax
        shl ebx, 24
        mov ecx, eax
        shl ecx, 8
        and ecx, 000ff0000h
        or ebx, ecx
        mov ecx, eax
        shr ecx, 8
        and ecx, 00000ff00h
        or ebx, ecx
        mov ecx, eax
        shr ecx, 24
        and ecx, 0000000ffh
        or ebx, ecx

        popa
        mov eax, 0
        leave
        ret

\end{lstlisting}

\end{document}