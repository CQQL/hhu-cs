\documentclass[10pt,a4paper]{article}
\usepackage[utf8]{inputenc}
\usepackage[german]{babel}
\usepackage{mathrsfs}
\usepackage{amsmath}
\usepackage{amsfonts}
\usepackage{amssymb}
\usepackage{amsthm}
\usepackage[left=2cm,right=2cm,top=2cm,bottom=2cm]{geometry}

\begin{document}

\section{Aufgabe 1.1}

Anfallende Daten sind zum Beispiel
\begin{itemize}
\item Personendaten der Patienten
\item Krankheitsgeschichte der Patienten
\item Krankenkassendaten der Patienten
\item Verschriebene Medikamente
\item Mitarbeiter
\item Arbeitsstunden der Mitarbeiter
\item Patiententermine
\end{itemize}

Dabei gäbe es die Benutzergruppen Mitarbeiter und Patient.

Neben den offensichtlichen Programmen, um diese Daten zu verwalten, könnte man
\begin{itemize}
\item die Patienten benachrichtigen, wenn sie schon lange nicht mehr bei
  Vorsorgeuntersuchungen waren
\item die Arbeitszeiten automatisch zum Steuerberater übertragen
\item automatisch überprüfen, ob verschriebene Medikamente sich gegenseitig in
  ihrere Wirkung beeinflussen
\item und bestimmt noch viel mehr
\end{itemize}

Ich nehme an, dass ``ohne ein Datenbanksystem'' bedeuten soll, dass man die
komplette Verwaltung der Daten selbst programmiert und auf soetwas wie
XML-Dateien zurückgreift. Einige Nachteile sind dann
\begin{itemize}
\item Wenn man mehr Code schreibt, hat man mehr Gelegenheiten, Dinge falsch zu
  machen
\item Man kann (sehr wahrscheinlich) kein SQL zur Abfrage benutzen
\item Keine Transaktionen
\item Keine Privilegien
\item Kein Netzwerkzugriff
\end{itemize}
Die Vorteile der Datenbanksysteme sind die Negationen von all den gerade
genannten Nachteilen.

\section{Aufgabe 1.2}

\subsection{Teil a}

Ein Königreich hat genau einen König.

\subsection{Teil b}

Ein Königreich hat viele Einwohner.

\subsection{Teil c}

Im Laufe der Zeit besitzt ein Einwohner beliebig viele Instrumente, die er aber
vielleicht auch weiterverkauft und die so von ebenfalls beliebig vielen Leuten
besessen werden können.

\subsection{Teil d}

Eine Musikkategorie wie Rock hat viele Unterkategorien wie z.B. Punkrock, die
wiederum viele Unterkategorien haben wie etwas Hardpunk und immer so weiter.

\subsection{Teil e}

Twitternutzer folgen anderen Twitternutzern, die wiederum anderen Twitternutzern
folgen und so weiter. Dabei können sich auch zwei gegenseitig folgen.

\section{Aufgabe 1.3}

\begin{tabular}{c|c|c}
  $F_{1} : F_{2}$ & $[min_{1}, max_{1}]$ & $[min_{2}, max_{2}]$\\\hline
  $1 : 1$ & $[1, 1]$ & $[1, 1]$\\\hline
  $1 : N$ & $[0, *]$ & $[0, 1]$\\\hline
  $N : M$ & $[0, *]$ & $[0, *]$
\end{tabular}

\end{document}