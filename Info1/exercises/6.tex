\documentclass[a4paper,10pt]{article}
\usepackage[utf8]{inputenc}
\usepackage[german]{babel}
\usepackage{amsmath}
\usepackage{amsthm}
\usepackage{listings}

\title{Info1, Übungsblatt 6}
\author{Marten Lienen (2126759)}

\newtheorem*{claim}{Behauptung}

\begin{document}

\maketitle

\section*{Aufgabe 1}

\subsection*{EBNF-Darstellung}

Sei $A$ ein Alphabet, $V = \{<wort>\}, S = <wort>$.
\begin{lstlisting}
 <wort> ::= {(a, b, c, ...)}
\end{lstlisting}
Die Variable kann durch eine Folge von Terminalsymbolen beliebiger Länge ersetzt werden.

\section*{Aufgabe 2}

Da ich eine kontextfreie Grammatik für die Sprache angeben kann, ist sie kontextfrei.

\subsection*{Grammatik}

\begin{align*}
 & A = \{a, b\}\\
 & V = \{S, Q\}\\
 & P = \{S \rightarrow aQb, S \rightarrow bQa, Q \rightarrow a, Q \rightarrow b, Q \rightarrow QQ\}\\
 \\
 & G = (A, V, P, S)
\end{align*}

\subsection*{BNF}

\begin{align*}
 & A = \{a, b\}\\
 & V = \{<Start>, <Wort>\}\\
 & S = <Start>
\end{align*}

Die Regeln sind:
\begin{lstlisting}
 <Start> ::= a<Wort>b|b<Wort>a
 <Wort> ::= a|b|<Wort><Wort>
\end{lstlisting}

\subsection*{EBNF}

\begin{align*}
 & A = \{a, b\}\\
 & V = \{<Start>, <Wort>\}\\
 & S = <Start>
\end{align*}

Die Regeln sind:
\begin{lstlisting}
 <Start> ::= a<Wort>b|b<Wort>a
 <Wort> ::= {(a|b)}
\end{lstlisting}

\section*{Aufgabe 3}

\subsection*{a}

\begin{align*}
 A = \{&0, 1, 2, 3, 4, 5, 6, 7, 8, 9\}\\
 V = \{&S, R, T\}\\
 P = \{&S \rightarrow 0, S \rightarrow R, S \rightarrow RT,\\
 & R \rightarrow 1, R \rightarrow 2, R \rightarrow 3, R \rightarrow 4, R \rightarrow 5, R \rightarrow 6, R \rightarrow 7, R \rightarrow 8, R \rightarrow 9,\\
 & T \rightarrow TT, T \rightarrow 0, T \rightarrow R\}\\
 \\
 & G = (A, V, P, S)
\end{align*}

\begin{proof}
 Ich zeige, dass die Dezimaldarstellung jeder nicht negativen, ganzen Zahl in $L(G)$ enthalten ist.
 Sei $x$ eine nicht negative, ganze Zahl.
 Wenn $x = 0$, greift die erste Regel, $S \rightarrow 0$.
 Wenn $x \ne 0$ eine einstellige Dezimalzahl ist, benutzt man die Regeln $S \rightarrow R$ und $R \rightarrow 1 \dots 9$.
 Wenn $x$ $n \ge 2$ Stellen hat, wendet man $S \rightarrow RT$ an, sodass die erste Stelle wie eine einstellige Dezimalzahl erzeugt wird und benutzt danach $n - 2$ mal $T \rightarrow TT$, sodass man eine Folge von $n - 1$ $T$ hat, die dann durch eine der Ziffern $0$ bis $9$ ersetzt werden.

 Ich zeige, dass jedes Wort in $L(G)$ die Dezimaldarstellung ohne führende Null einer nicht negativen ganzen Zahl ist.
 Durch Anwendung von $S \rightarrow 0$ erhält man direkt $0$.
 Wendet man zuerst $S \rightarrow R$ und danach eine Regel $R \rightarrow 1 \dots 9$ an, erhält man eine der restlichen 9 einstelligen, positiven Dezimalzahlen.
 Durch Anwendung von $S \rightarrow RT$, $R \rightarrow 1 \dots 9$ und der $n$-fachen Anwendung von $T \rightarrow TT$ kann man mittels der restlichen $T$-Regeln jede Zahl erzeugen, die sich darstellen lässt als
 \begin{equation}
  x = a * 10^{n + 2} + a_0 * 10^{n + 1} + a_1 * 10^{n} + \dots + a_n * 10^0
 \end{equation}
 mit $1 \le a \le 9$, $0 \le k \le n$, $0 \le a_k \le 9$.
 Dies sind alle nicht negativen, ganzen Zahlen ohne führende Null, die größer als $9$ sind.
\end{proof}

\subsection*{b}

\subsubsection*{Kontextfreie Grammatik}

\begin{align*}
 A = \{&0, 1, 2, 3, 4, 5, 6, 7, 8, 9, +, -\}\\
 V = \{&S, P, Q, R, T\}\\
 P = \{&S \rightarrow P, S \rightarrow QP,\\
 & Q \rightarrow +, Q \rightarrow -,\\
 & P \rightarrow 0, P \rightarrow R, P \rightarrow RT,\\
 & R \rightarrow 1, R \rightarrow 2, R \rightarrow 3, R \rightarrow 4, R \rightarrow 5, R \rightarrow 6, R \rightarrow 7, R \rightarrow 8, R \rightarrow 9,\\
 & T \rightarrow TT, T \rightarrow 0, T \rightarrow R\}\\
 \\
 & G = (A, V, P, S)
\end{align*}

\subsubsection*{EBNF}
\begin{lstlisting}
 A = {0, 1, 2, 3, 4, 5, 6, 7, 8, 9, +, -}
 V = {<Ganze-Zahl>, <Vorzeichen>, <Zahl> <Ziffer>, <Nichtnull-Ziffer>}
 S = <Ganze-Zahl>
\end{lstlisting}

Die Regeln sind:
\begin{lstlisting}
 <Ganze-Zahl> ::= [<Vorzeichen>]<Zahl>
 <Vorzeichen> ::= (+|-)
 <Zahl> ::= (0|(<Nichtnull-Ziffer>{<Ziffer>}))
 <Ziffer> ::= (0|<Nichtnull-Ziffer>)
 <Nichtnull-Ziffer> ::= (1|2|3|4|5|6|7|8|9)
\end{lstlisting}

\section*{Aufgabe 4}

Der Algorithmus heißt $graycode(n)$.

\begin{description}
 \item[Eingabe $n$] Die Länge der Wörter
 \item[Ausgabe] Die geordnete Menge der Wörter
\end{description}

\begin{lstlisting}
 wenn n = 1
   gib zurueck {0, 1}
 ansonsten
   erste_haelfte := graycode(n - 1)
   graycode := {}
   umgedreht := reverse(erste_haelfte)
   
   fuer jedes Wort w in erste_haelfte
     neues_wort := konkateniert(0, w)
     
     fuege neues_wort hinten an graycode an
   
   fuer jedes Wort w in umgedreht
     neues_wort := konkateniert(1, w)
     
     fuege neues_wort hinten an graycode an
   
   gib zurueck graycode
\end{lstlisting}

Die erste Schleife erweitert alle Wörter des nächstkleineren Graycodes auf die geforderte Länge und fügt sie dem Ergebnis hinzu.
Die zweite Schleife fügt den kleineren Graycode umgedreht an, mit 1 konkateniert, an den Graycode an.
Ich nenne die erste Hälfte, die mit 0 beginnt, 0-Code und die zweite, die mit 1 beginnt, 1-Code.
Am Übergang vom 0 zum 1-Code liegt eine Änderung von 1 vor, weil beide Elemente sich nur an der ersten Stelle unterscheiden.
Das gleiche gilt auch am Übergang von 1- zum 0-Code, wenn man die Menge als Ring betrachtet.
Innerhalb der Hälften unterscheiden sich die Elemente ebenfalls immer nur um eine Operation.
Dies ist wahr, weil die Elemente alle aus einem Graycode sind, sodass sie sich von ihren Nachbarn immer nur um 1 Operation unterscheiden.
Das gilt auch noch nach der Konkatenierung, weil alle Wörter innerhalb einer Hälfte mit demselben Symbol erweitert wurden.

\end{document}
