\documentclass[a4paper,10pt]{article}
\usepackage[utf8]{inputenc}
\usepackage{amsmath}
\usepackage{amssymb}
\usepackage{amsthm}
\usepackage[german]{babel}

\title{Übungsblatt 3, Info1}
\author{Marten Lienen (2126759)}

\begin{document}

\maketitle

\section*{Aufgabe 1}

\subsection*{a}

Man wähle für $I_1$ die $\frac{n - 1}{2}$ Indizes, deren zugeordnete Werte am kleinsten sind.
Für $I_2$ wähle man die $\frac{n - 1}{2}$ Indizes, deren zugeordnete Werte am größten sind.
$m$ weise man den verbliebenen Index zu.
Dann haben $I_1$ und $I_2$ gleich viele Elemente.
Nach Art der Aufteilung wurde jeder Index gewählt und die Anzahl der gewählten Indizes ist $\frac{n - 1}{2} * 2 + 1 = n$.
Punkt 1 gilt demnach auch.
Nach der Weise wie die Indizes gewählt wurden, gilt auch Punkt 2.

\subsection*{b}

Man nehme die Liste $1, 5, 5, 5, 6$ mit den Indizes beginnend bei 1.
Dann könnte man $m$ aus $\{2, 3, 4\}$, $I_1 = \{1, m \in \{2, 3, 4\}\}$ und $I_1 = \{m \in \{2, 3, 4\}, 5\}$ wählen.
Für jedes Element gibt es also eine Auswahl, aber trotzdem gelten immer die Bedingungen aus a.

Um den Median zu berechnen, entferne man die $\frac{n - 1}{2}$ kleinsten Elemente aus der Menge.
Das kleinste Element der Restmenge ist dann der Median.

\subsection*{c}

Ein Vorteil ist, dass der Median wirklich als Element in der Menge vorkommt.

Ein Nachteil ist, dass die Abweichung zum arithmetischen Mittel sehr groß sein kann.

\section*{Aufgabe 2}

\begin{proof}
 \begin{equation*}
  P(n): f(n) = 2^n
 \end{equation*}

 $P(0)$ ist wahr, weil $f(0) = 1$ und die leere Menge hat genau eine Teilmenge, die leere Menge selbst.
 Nun gilt es zu zeigen, dass $P(n + 1)$ gilt, wenn $P(n)$ gilt.
 Wenn wir eine Menge $M$ um 1 Element erweitern ($N$ bezeichnet die erweiterte Menge), sind alle Teilmengen von $M$ auch Teilmenge von $N$.
 Dazu kommt dann noch jede Teilmenge mit dem zusätzlichen Element erweitert.
 Die Anzahl der Teilmengen verdoppelt sich also für jedes Element, dass man hinzufügt.

 \begin{equation*}
  2^n * 2 = 2^{n + 1} = P(n + 1)
 \end{equation*}
\end{proof}

\section*{Aufgabe 3}

\subsection*{a}

\begin{equation*}
 p \le q \Rightarrow g \in O(f)
\end{equation*}

\begin{align*}
 & 0 \le g(n) \le c * f(n)
 \overset{c := 1}{\Rightarrow} 0 \le n^p \le n^q
 \Leftrightarrow -\infty \le p \le q
\end{align*}

\subsection*{b}

\begin{equation*}
 p \le q \Rightarrow g \in O(f)
\end{equation*}

\begin{align*}
 & 0 \le g(n) \le c * f(n)
 \Leftrightarrow 0 \le a * n^p \le c * n^q
 \overset{c := a}{\Rightarrow} 0 \le n^p \le n^q
 \Leftrightarrow -\infty \le p \le q
\end{align*}

\subsection*{c}

\begin{equation*}
 p \le q \Rightarrow g \in O(f)
\end{equation*}

\subsection*{d}

\begin{equation*}
 p < q \Rightarrow g \in O(f)
\end{equation*}

\end{document}
