\documentclass[a4paper,10pt]{article}
\usepackage[utf8]{inputenc}
\usepackage[german]{babel}
\usepackage{amsmath}
\usepackage{amsthm}
\usepackage{listings}

\title{Info1, Übungsblatt 9}
\author{Marten Lienen (2126759)}

\begin{document}

\maketitle

\section*{a}

Ist das gleiche wie auf Blatt 6 oder 7.
Die Vorteile von structs sind, dass man den Code nachher besser lesen kann, weil die einzelnen Komponenten Namen haben, während man über ein Array besser iterieren kann.

\section*{b}

\lstset{language=c}
\begin{lstlisting}
 #include <math.h>

 typedef struct {
   double x;
   double y;
 } point_t;

 typedef struct {
   point_t e1;
   point_t e2;
   point_t e3;
 } triangle_t;

 double distance (point_t a, point_t b) {
   return sqrt(pow(a.x - b.x, 2) + pow(a.y - b.y, 2));
 }

 double circumference (triangle_t triangle) {
   return
     distance(triangle.e1, triangle.e2)
     + distance(triangle.e2, triangle.e3)
     + distance(triangle.e3, triangle.e1);
 }

 double area (triangle_t triangle) {
   double s = circumference(triangle) / 2;
   double a = distance(triangle.e1, triangle.e2);
   double b = distance(triangle.e2, triangle.e3);
   double c = distance(triangle.e3, triangle.e1);
  
   return sqrt(s * (s - a) * (s - b) * (s - c));
 }
\end{lstlisting}

\section*{c}

Man definiert eine Gerade $g$, die durch $p$ geht und parallel zur X-Achse verläuft.
Dann betrachtet man die Schnittmenge der Geraden $g$ und der ursprünglichen Geraden.
Wenn die Menge mehr als 1 Element hat, sind die Geraden identisch und $p$ liegt darauf.
Wenn die Menge genau 1 Element hat und der Schnittpunkt identisch zu $p$ ist, liegt $p$ ebenfalls auf der Geraden.
Wenn die Menge genau 1 Element hat und der Schnittpunkt $k$ nicht identisch zu $p$ ist, betrachtet man $q = x(p) - x(k)$.
Ist $q < 0$, so liegt $p$ ``links'', andernfalls liegt $q$ ``rechts''.
Wenn die Menge leer ist, sind die Geraden parallel und nicht identisch.
In diesem Fall kann man das Verfahren analog mit der Y-Achse und den Y-Koordinaten wiederholen.

\section*{d}

Folgender Code funktioniert mit allen Dreiecken, die keinen vertikalen Seiten haben:

\begin{lstlisting}
 #include <stdio.h>
 #include <stdlib.h>
 #include <stdbool.h>
 #include <math.h>

 typedef struct {
   double x;
   double y;
 } point_t;

 typedef struct {
   point_t e1;
   point_t e2;
   point_t e3;
 } triangle_t;

 typedef struct {
   point_t base;
   double slope;
 } line_t;

 typedef struct {
   line_t line;
   point_t a;
   point_t b;
 } line_segment_t;

 line_t line_from_points (point_t a, point_t b) {
   line_t line = {
     .base = a,
     .slope = (b.y - a.y) / (b.x - a.x)
   };
  
   return line;
 }

 line_segment_t line_segment_from_points (point_t a, point_t b) {
   line_segment_t line_segment = {
     .line = line_from_points(a, b),
     .a = a,
     .b = b
   };
  
   return line_segment;
 }

 double max (double a, double b) {
   return a > b ? a : b;
 }

 double min (double a, double b) {
   return a > b ? b : a;
 }

 bool is_point_on_line (point_t point, line_t line) {
   return point.x == line.base.x + line.slope * (point.y - line.base.y);
 }

 bool lines_intersect (line_t a, line_t b) {
   if (a.slope == b.slope) {
     return is_point_on_line(a.base, b);
   } else {
     return true;
   }
 }

 point_t intersection_point (line_t a, line_t b) {
   double factor =
     (b.slope * (b.base.x - a.base.x) + a.base.y - b.base.y)
     / (b.slope - a.slope);
  
   point_t point = {
     .x = a.base.x + factor,
     .y = a.base.y + factor * a.slope
   };
  
   return point;
 }

 bool is_point_in_rectangle (point_t p, point_t a, point_t b) {
   return
     p.x <= max(a.x, b.x)
     && p.x >= min(a.x, b.x)
     && p.y <= max(a.y, b.y)
     && p.y >= min(a.y, b.y);
 }

 bool line_intersects_segment (line_t line, line_segment_t segment) {
   if (lines_intersect(line, segment.line)) {
     return is_point_in_rectangle(
       intersection_point(line, segment.line),
       segment.a,
       segment.b);
   } else {
     return false;
   }
 }

 bool is_point_on_line_segment (point_t point, line_segment_t line_segment) {
   return
     is_point_on_line(point, line_segment.line)
     && is_point_in_rectangle(point, line_segment.a, line_segment.b);
 }

 bool is_in_triangle (triangle_t triangle, point_t point) {
   point_t zero = { .x = 0, .y = 0 };
   line_segment_t line1 = line_segment_from_points(triangle.e1, triangle.e2);
   line_segment_t line2 = line_segment_from_points(triangle.e2, triangle.e3);
   line_segment_t line3 = line_segment_from_points(triangle.e3, triangle.e1);
   line_t line_to_point = line_from_points(zero, point);
  
   bool is_point_on_border = 
     is_point_on_line_segment(point, line1)
     || is_point_on_line_segment(point, line2)
     || is_point_on_line_segment(point, line3);
  
   if (is_point_on_border) {
     return true;
   } else {
     bool i1 = line_intersects_segment(line_to_point, line1);
     bool i2 = line_intersects_segment(line_to_point, line2);
     bool i3 = line_intersects_segment(line_to_point, line3);

     return
       (i1 && !i2 && !i3)
       || (!i1 && i2 && !i3)
       || (!i1 && !i2 && i3);
   }
 }

 int main(int argc, char** argv) {
   triangle_t t = {
     .e1 = {
       .x = 1.1,
       .y = 0
     },
     .e2 = {
       .x = 1,
       .y = 1
     },
     .e3 = {
       .x = 2,
       .y = 0
     }
   };
  
   point_t p = {
     .x = 1.5,
     .y = 0.5,
   };
  
   printf("%d\n", is_in_triangle(t, p));

   return (EXIT_SUCCESS);
 }
\end{lstlisting}

\end{document}
