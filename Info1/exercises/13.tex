\documentclass[a4paper,10pt]{article}
\usepackage[utf8]{inputenc}
\usepackage[german]{babel}
\usepackage{amsmath}
\usepackage{amsthm}
\usepackage{color}
\usepackage{listings}
\lstset{
 language=java,                % choose the language of the code
 basicstyle=\footnotesize,       % the size of the fonts that are used for the code
 numbers=left,                   % where to put the line-numbers
 numberstyle=\footnotesize,      % the size of the fonts that are used for the line-numbers
 stepnumber=1,                   % the step between two line-numbers. If it is 1 each line will be numbered
 numbersep=5pt,                  % how far the line-numbers are from the code
 backgroundcolor=\color{white},  % choose the background color. You must add \usepackage{color}
 showspaces=false,               % show spaces adding particular underscores
 showstringspaces=false,         % underline spaces within strings
 showtabs=false,                 % show tabs within strings adding particular underscores
 tabsize=2,          % sets default tabsize to 2 spaces
 captionpos=b,           % sets the caption-position to bottom
 breaklines=true,        % sets automatic line breaking
 breakatwhitespace=false,    % sets if automatic breaks should only happen at whitespace
 escapeinside={\%*}{*)}          % if you want to add a comment within your code
}

\newtheorem{lemma*}{Lemma}

\title{Info1, Übungsblatt 13}
\author{Marten Lienen (2126759)}

\begin{document}

\maketitle

\section*{Aufgabe 1}

\subsection*{Teil a}

\subsection*{Teil b}

\section*{Aufgabe 2}

\subsection*{Teil a}

\begin{proof}
 Da ein Baum nur höchtens eine Wurzel haben kann, gibt es höchtens $1 = 2^t$ Knoten mit Höhe $t = 0$.
 
 Nehmen wir an $(B, w)$ sei ein binärer Baum mit Höhe $t$, der höchtens $2^t$ Knoten der Höhe $t$ enthält.
 Ein binärer Baum $(B', w)$, in dem alle Knoten aus $B$ noch weitere Kindknoten enthalten, kann höchtens $2 \cdot 2^t = 2^{t + 1}$ Knoten der Höhe $t + 1$ haben, weil jeder der Knoten mit Höhe $2^t$ höchtens $2$ Kinder haben kann.
\end{proof}

\subsection*{Teil b}

\begin{lemma*}
 \begin{equation}
  \sum_{k = 0}^h 2^k = 2^{h + 1} - 1
 \end{equation}
\end{lemma*}

\begin{proof}
 Sei $h = 0$.
 \begin{equation}
  \sum_{k = 0}^0 2^k = 2^0 = 2^{h + 1} - 1
 \end{equation}
 
 Angenommen es gelte
 \begin{equation}
  \sum_{k = 0}^h 2^k = 2^{h + 1} - 1
 \end{equation}
 Dann gilt auch
 \begin{equation}
  \sum_{k = 0}^{h + 1} 2^k = 2^{h + 1} + \sum_{k = 0}^h 2^k = 2^{h + 1} + 2^{h + 1} - 1 = 2 \cdot 2^{h + 1} - 1 = 2^{h + 2} - 1
 \end{equation}
\end{proof}

\begin{proof}
 Aus Teil $a$ wissen wir, dass jede Ebene $t$ eines binären Baums $B$ der Höhe $h$ höchtens $2^t$ Knoten enthalten kann.
 Wir erhalten also die Kapazität eines Baums, wenn wir die Kapazitäten aller Ebenen aufaddieren.
 \begin{equation}
  Kapazitaet(B) = \sum_{k = 0}^h 2^k = 2^{h + 1} - 1
 \end{equation}
\end{proof}

\subsection*{Teil c}

Wenn $h = n - 1$ gilt, bedeutet das, dass der längste Weg im Baum die Länge $n - 1$ hat.
Da der Baum nur $n$ Knoten hat und der Weg zu jedem Knoten eindeutig ist, ist der Baum einfach eine lange Kette.

\begin{proof}
 Da Wege in einem Baum eindeutig bestimmt und kreisfrei sind, ist der längst mögliche Weg und damit die größtmögliche Tiefe zu erreichen, indem man alle $n$ Knoten mit $n - 1$ Kanten zu einer Kette verbindet.
 Folglich gilt
 \begin{equation}
  h \le n - 1
 \end{equation}

 Wenn man die Formel für die maximale Anzahl an Knoten in einem binären Baum aus Teil $b$ umformt, erhält man
 \begin{align*}
  & n = 2^{h + 1} - 1\\
  \Leftrightarrow & n + 1 = 2^{h + 1}\\
  \Leftrightarrow & \log_2 (n + 1) = h + 1\\
  \Leftrightarrow & \lfloor \log_2 n \rfloor = \log_2 (n + 1) - 1 = h\\
 \end{align*}
 Die letzte Gleichung gilt, weil $\log_2 (n + 1)$ immer eine ganze Zahl ergibt.
 Weil $n$ aber auch nicht maximal sein kann, gilt
 \begin{equation}
  \lfloor \log_2 n \rfloor \le h
 \end{equation}
 und wenn man beide Ergebnisse zusammensetzt, kommt heraus
 \begin{equation}
  \lfloor \log_2 n \rfloor \le h \le n - 1
 \end{equation}
\end{proof}

\end{document}
