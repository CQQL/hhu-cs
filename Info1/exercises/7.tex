\documentclass[a4paper,10pt]{article}
\usepackage[utf8]{inputenc}
\usepackage[german]{babel}
\usepackage{amsmath}
\usepackage{amsthm}
\usepackage{listings}

\title{Info1, Übungsblatt 7}
\author{Marten Lienen (2126759)}

\newtheorem*{claim}{Behauptung}

\begin{document}

\maketitle

\section*{Übung 1}

\subsection*{a}

\subsubsection*{Zufallszahl}

Eine Zahl, die auf eine nicht vorhersagbare Weise gewählt wurde.

\subsubsection*{Zufallszahlenfolge}

Eine Folge von Zahlen, in der man die aus einer Zahl nicht seinen Nachfolger errechnen kann.

\subsubsection*{Pseudozufallszahl}

Eine Zahl, die theoretisch vorhersagbar ist, was aber sehr aufwändig ist bzw. so aufwändig, dass es nur theoretisch durchführbar ist.

\subsection*{b}

Für mein menschliches Auge sehen die Zahlen, die von nachfolgendem Programm ausgegeben werden zufällig aus.
Nur wiederholen sie sich periodisch.

\lstset{language=c}
\begin{lstlisting}
 #include <stdio.h>
 #include <stdlib.h>

 typedef unsigned int uint;

 void print_values (uint a, uint c, uint m, uint N, uint x_0) {
   uint i;
   uint x[N];
   x[0] = x_0;
  
   printf("%d\n", x[0]);
  
   for (i = 1; i < N; i++) {
     x[i] = (a * x[i - 1] + c) % m;
    
     printf("%d\n", x[i]);
   }
 }

 int main(int argc, char** argv) {
   if (argc < 6) {
     return 1;
   }
  
   uint a, c, m, N, x_0;
  
   a = atoi(argv[1]);
   c = atoi(argv[2]);
   m = atoi(argv[3]);
   N = atoi(argv[4]);
   x_0 = atoi(argv[5]);
  
   print_values(a, c, m, N, x_0);

   return 0;
 }
\end{lstlisting}

\subsection*{c}

Wenn man die Parameter nach dem Satz von Greenberger wählt, ist die Periodenlänge $m$.

\lstset{language=c}
\begin{lstlisting}
 #include <stdio.h>
 #include <stdlib.h>

 typedef unsigned int uint;

 void print_period_length (uint a, uint c, uint m, uint N, uint x_0) {
   uint i;
   uint x[N];
   x[0] = x_0;
  
   for (i = 1; i < N; i++) {
     x[i] = (a * x[i - 1] + c) % m;
    
     if (x[i] == x_0) {
       printf("Periodenlaenge: %d\n", i);
      
       return;
     }
   }
 }

 int main(int argc, char** argv) {
   if (argc < 6) {
     return 1;
   }
  
   uint a, c, m, N, x_0;
  
   a = atoi(argv[1]);
   c = atoi(argv[2]);
   m = atoi(argv[3]);
   N = atoi(argv[4]);
   x_0 = atoi(argv[5]);
  
   print_period_length(a, c, m, N, x_0);

   return 0;
 }
\end{lstlisting}

\section*{Übung 2}

\subsection*{a}

\begin{proof}
 Zu zeigen ist, dass $G(x)$ sich von $G(x + 1)$ an genau einer Stelle unterscheidet.
 
 Seien $x$ und $y = x + 1$ zwei Zahlen in Binärdarstellung.
 Weil $G$ jede Stelle $l$ mit der vorhergehenden mit einem $XOR$ verknüpft, ist der Graycode an der Stelle $j$ dann $0$, wenn die Stellen $j$ und $j - 1$ in der Binärzahl gleich sind, ansonsten ist er $1$.
 Jede zwei aufeinanderfolgende binäre Zahlen gleicher Länge (links mit führenden Nullen) lassen sich von links nach rechts in 3 Teile aufteilen.
 \begin{itemize}
  \item In den ersten $n$ Stellen stimmen sie genau überein
  \item Es folgt eine Stelle $j$, wo die kleinere Zahl eine $0$ und die größere Zahl eine $1$ hat
  \item Alle weiteren $k$ Stellen sind bei der kleineren Zahl $1$ und bei der größeren $0$
 \end{itemize}
 (Beachte: $n$ und $k$ können auch $0$ sein.)
 Da die Vorschrift $G$ sich an jeder Stelle nur auf Stellen links von dieser bezieht, müssen auch die ersten $n$ Stellen des Graycodes übereinstimmen.
 Die $j$-te Stelle des Graycodes unterscheidet sich weil $x_{j - 1} = y_{j - 1}$, aber $x_j \ne y_j$.
 Die $(j + 1)$-te Stelle des Graycodes ist immer eine $1$, weil $x_j = 0, x_{j + 1} = 1, y_j = 1, y_{j + 1} = 0$.
 Die $k - 1$ verbleibenden Stellen sind alle $0$, weil die letzten $k$ Stellen einer Zahl alle denselben Wert haben.
 $G(x)$ und $G(y)$ unterscheiden sich also genau an der Stelle, wo sich $x$ und $y$ zum ersten Mal unterscheiden.
 An allen anderen Stellen sind sie gleich.
 
 Es bleibt noch zu zeigen, dass es auch für $G(0)$ und $G(2^n - 1)$ gilt.
 $G(0)$ ist immer $0$, weil alle Binärziffern von $0$ gleich sind.
 $G(2^n - 1)$ ist an allen Stellen $0$ außer der ersten, weil $G$ die erste Stelle ohne Änderung überträgt.
\end{proof}

\lstset{language=c}
\begin{lstlisting}
 #include <stdio.h>

 void byte_to_string (char* string, int length, int x) {
   int i;
  
   for (i = 0; i < length; i++) {
     if (x & (1 << i)) {
       string[i] = '1';
     } else {
       string[i] = '0';
     }
   }
  
   string[length] = '\0';
 }

 int graycode_encode (int x) {
   return x ^ (x >> 1);
 }

 void print_graycodes (int n) {
   int i;
   char string[n + 1];
   int upper_bound = 1 << n;
  
   for (i = 0; i < upper_bound; i++) {
     byte_to_string(string, n, graycode_encode(i));
    
     printf("%s\n", string);
   }
 }

 int main(int argc, char** argv) {
   int n;
  
   scanf("%d", &n);
  
   print_graycodes(n);

   return 0;
 }
\end{lstlisting}

\subsection*{b}

\begin{proof}
 Zuerst gucken wir, was die Funktion $decode$ eigentlich tut.
 Sei $x$ eine Binärzahl der Länge $n$ und $y = G(x)$ ihr Graycode.
 Die Funktion erzeugt eine Zahl $q$ sodass für jede Stelle von $q$ gilt
 \begin{equation}
  q_k = \oplus_{j = 0}^k y_{k - j} = y_k \oplus y_{k - 1} \oplus y_{k - 2} \oplus \dots \oplus y_0
 \end{equation}
 Aber
 \begin{equation}
  y_k = x_k \oplus x_{k - 1} \Leftrightarrow y_k \oplus x_{k - 1} = x_k \oplus x_{k - 1} \oplus x_{k - 1} = x_k
 \end{equation}
 Man muss sich vorstellen, dass die Binärzahlen nach links (zur Seite der kleineren Indexe) mit Nullen aufgefüllt sind.
 Die $0$ ist das neutrale Element von $\oplus$.
 Es gilt aber auch
 \begin{equation}
  y_{k - 1} = x_{k - 1} \oplus x_{k - 2} \Leftrightarrow y_{k - 1} \oplus x_{k - 2} = x_{k - 1} \oplus x_{k - 2} \oplus x_{k - 2} = x_{k - 1}
 \end{equation}
 Daraus folgt
 \begin{equation}
  x_k = y_k \oplus x_{k - 1} = y_k \oplus y_{k - 1} \oplus x_{k - 2} = \dots
 \end{equation}
 Da aber die Binärzahlen nach links mit $0$, dem neutralen Element von $\oplus$ aufgefüllt sind, gibt es ein ``linkestes'' relevantes Element, also mit kleinsten relevanten Index $p$.
 Dieser Index ist genau die linkeste Stelle von $y$, da danach alle $0$ sind, also $p = 0$.
 \begin{equation}
  x_k = y_k \oplus y_{k - 1} \oplus \dots \oplus y_0 \oplus 0
 \end{equation}
 Wenn wir das nun mit der Formel von oben für $q_k$ vergleichen sehen, wir dass $q = x$, weil $q$ und $x$ sich am Ende in jeder Stelle gleichen werden.
\end{proof}

\end{document}
