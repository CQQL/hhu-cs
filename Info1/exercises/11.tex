\documentclass[a4paper,10pt]{article}
\usepackage[utf8]{inputenc}
\usepackage[german]{babel}
\usepackage{amsmath}
\usepackage{amsthm}
\usepackage{listings}

\title{Info1, Übungsblatt 11}
\author{Marten Lienen (2126759)}

\begin{document}

\maketitle

\section*{a}

\begin{proof}
 Sei $a$ ein korrekter Klammerausdruck der Länge $n$, die nach der BNF-Beschreibung immer gerade ist.	
 
 Sei $a \in A^*$, sodass der K-Akzeptor ``korrekt'' ausgibt.
\end{proof}

\section*{b}

\begin{lstlisting}
 import java.util.Stack;

 public class Klammern {
   public static boolean linke_klammer (char c) {
     return c == '(' || c == '{' || c == '<' || c == '[';
   }

   public static boolean passen (char a, char b) {
     return
       (a == '(' && b == ')') ||
       (a == '{' && b == '}') ||
       (a == '<' && b == '>') ||
       (a == '[' && b == ']');
   }

   public static boolean k_akzeptor (String ausdruck) {
     Stack<Character> klammerstack = new Stack<Character>();

     for (char c : ausdruck.toCharArray()) {
       if (linke_klammer(c)) {
	 klammerstack.push(c);
       } else {
	 if (klammerstack.empty()) {
	   return false;
	 } else if (!passen(klammerstack.pop(), c)) {
	   return false;
	 }
       }
     }

     return klammerstack.empty();
   }

   public static void main (String[] args) {
     boolean gueltig = k_akzeptor(args[0]);

     if (gueltig) {
       System.out.println("korrekt");
     } else {
       System.out.println("inkorrekt");
     }
   }
 }
\end{lstlisting}

\section*{c}

\begin{proof}
 Sei $a$ ein korrekter Klammerausdruck.
 Nach Definition enthält $a$ genausoviele linke wie rechte Klammern.
 Da Klammern immer nur paarweise mit der linken Klammer auf der linken Seite hinzugefügt werden können, enthält jedes Anfangsstück $a_j$ höchstens so viele rechte Klammern wie linke.
 
 Sei $a \in A^*$ mit genausovielen linken wie rechten Klammern und jedes Anfangsstück $a_j$ enthalte höchstens so viele rechte wie linke Klammern. 
\end{proof}

\end{document}
