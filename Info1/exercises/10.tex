\documentclass[a4paper,10pt]{article}
\usepackage[utf8]{inputenc}
\usepackage[german]{babel}
\usepackage{amsmath}
\usepackage{amsthm}
\usepackage{listings}

\title{Info1, Übungsblatt 10}
\author{Marten Lienen (2126759)}

\begin{document}

\maketitle

\section*{Übung 1}

\subsection*{a}

Die Anzahl aller Waggons sei $n$, wobei die Waggons von hinten nach vorne aufsteigend mit den Zahlen $1..n$ identifiziert werden.
Wenn $i = j$ gilt, ist die Reihenfolge bereits richtig.
O.b.d.A. nehme ich an, dass $i < j$.
Dann führe man folgenden Algorithmus aus
\begin{lstlisting}
 Waggons [j + 1, n] -> Gleis 2
 Waggon j -> Gleis 3
 Waggons [i, j - 1] -> Gleis 2
 Waggon j -> Gleis 1
 Waggon i -> Gleis 3
 Waggons [i + 1, j - 1] -> Gleis 1
 Waggon i -> Gleis 1
 Waggons [j + 1, n] -> Gleis 1
\end{lstlisting}

\subsection*{b}

Da man 2 beliebige Waggons miteinander vertauschen kann, kann man jede Reihenfolge erzeugen.

\subsection*{c}

Die Gleise können als Stapel aufgefasst werden, deren ``oben'' zur Verbindung zeigt.
Sei $W$ die Menge aller Waggons und $S = W^*$ die Menge aller Gleise.
Man kann jedes $s \in S$ schreiben als $s = w_1w_2\dots w_n$.
Die Probleme lauten:
\begin{enumerate}
 \item Gibt es eine Methode $switch: S \times [1, n] \times [1, n] \rightarrow S$, sodass
  \begin{equation}
   switch(s, i, j) = w_1 \dots w_{i - 1} w_j w_{i + 1} \dots w_{j - 1} w_i w_{j + 1} \dots w_n
  \end{equation}
 \item Gibt es eine Methode $permutate: S \times P$, sodass
  \begin{equation}
   permutate(s, a_n) = w_{a_1}w_{a_2} \dots w_{a_n}
  \end{equation}
  wobei $P$ die Menge aller Folgen mit $n$ Gliedern ist, deren Elemente zwischen $1$ und $n$ liegen und die jede dieser Zahlen genau einmal enthalten.
\end{enumerate}
Da Waggons nicht einfach hinundher gehoben werden können, müsste man bei der Durchführung in der physikalischen Welt diese beiden Extragleise zur Verfügung haben.
Ich weiß nicht, welche Eigenschaft hier sonst gemeint sein sollte.

\section*{Übung 2}

In dieser Implementierung wird eine Einfach-verkettete Liste verwendet, wobei Elemente so eingefügt werden, dass sie nach Priotität abwärts sortiert in der Liste stehen.
\begin{lstlisting}
 #include <stdio.h>
 #include <stdlib.h>
 #include <stdbool.h>

 typedef struct queue_item {
   struct queue_item* next;
   int priority;
   int object;
 } queue_item_t;

 typedef struct {
   queue_item_t* top;
 } priority_queue_t;

 queue_item_t* create_queue_item (int object, int priority) {
   queue_item_t* queue_item = (queue_item_t*)malloc(sizeof(queue_item_t));
   queue_item->next = NULL;
   queue_item->priority = priority;
   queue_item->object = object;
   
   return queue_item;
 }

 void destroy_queue_item (queue_item_t* queue_item) {
   if (queue_item->next != NULL) {
     destroy_queue_item(queue_item->next);
   }
   
   free(queue_item);
 }

 priority_queue_t* create_queue () {
   priority_queue_t* queue =
     (priority_queue_t*)malloc(sizeof(priority_queue_t));
   queue->top = NULL;
   
   return queue;
 }

 void destroy_queue (priority_queue_t* queue) {
   destroy_queue_item(queue->top);
   
   free(queue);
 }

 void insert_into_queue (queue_item_t* queue_item_list,
			 queue_item_t* queue_item) {
   if (queue_item_list->next == NULL) {
     queue_item_list->next = queue_item;
   } else if (queue_item_list->next->priority < queue_item->priority) {
     queue_item->next = queue_item_list->next;
     queue_item_list->next = queue_item;
   } else {
     insert_into_queue(queue_item_list->next, queue_item);
   }
 }

 void enqueue (priority_queue_t* queue, int object, int priority) {
   queue_item_t* queue_item = create_queue_item(object, priority);
   
   if (queue->top == NULL) {
     queue->top = queue_item;
   } else if (queue->top->priority < priority) {
     queue_item->next = queue->top;
     queue->top = queue_item;
   } else {
     insert_into_queue(queue->top, queue_item);
   }
 }

 void dequeue (priority_queue_t* queue) {
   queue_item_t* top = queue->top;
   
   queue->top = top->next;
   
   top->next = NULL;
   
   destroy_queue_item(top);
 }

 int front (priority_queue_t* queue) {
   return queue->top->object;
 }

 bool is_empty (priority_queue_t* queue) {
   return queue->top == NULL;
 }

 void print_queue (priority_queue_t* queue) {
   while (!is_empty(queue)) {
     printf("%d\n", front(queue));
     dequeue(queue);
   }
 }

 int main(int argc, char** argv) {
   priority_queue_t* queue = create_queue();
   
   enqueue(queue, 5, 1);
   enqueue(queue, 5, 3);
   enqueue(queue, 1, 2);
   enqueue(queue, 1, 2);
   enqueue(queue, 4, 10);
   
   print_queue(queue);

   return (EXIT_SUCCESS);
 }
\end{lstlisting}
Es gilt
\begin{align*}
 enqueue & \in \mathcal{O}(n)\\
 dequeue & \in \mathcal{O}(1)\\
 front & \in \mathcal{O}(1)\\
\end{align*}
weil $front$ und $dequeue$ immer nur ein einzelnes Element betrachten und $enqueue$ maximal $n$ Elemente, falls das neue Element die niedrigste Priotität haben sollte.

\end{document}
