\documentclass[a4paper,10pt]{article}
\usepackage[utf8]{inputenc}
\usepackage{amssymb}
\usepackage{amsmath}
\usepackage{amsthm}
\usepackage[ngerman]{babel}

\title{Übungsblatt Nr. 1}
\author{Marten Lienen}

\newtheorem{theorem}{Satz}

\begin{document}

\maketitle

\section{Aufgabe 1}

\subsection{a}

$f$ kann nicht surjektiv sein, weil die Zielmenge $n \in \mathbb{N}$ Zahlen enthält, die sich nicht als Quadrat von Zahlen der Quellmenge darstellen lassen, z.B. kann man $5$ nicht als
Produkt Quadrat einer natürlichen Zahl darstellen. $f$ ist jedoch injektiv, weil das Quadat von $n$ stets kleiner ist als das Quadat von $n + 1$, sodass keine zwei $n \in \mathbb{N}$
auf die selbe Zahl abgebildet werden können.

\subsection{b}

$f$ ist nicht surjektiv, weil $f^{-1}$ mindestens ein $n \in \mathbb{R}$ auf zwei Quellobjekte abbildet: $\sqrt{4} = \{2, -2\}$. $f$ ist auch nicht injektiv, weil mindestens zwei
$n \in \mathbb{R}$ auf das selbe Zielobjekt abgebildet werden: $2^2 = (-2)^2 = 4$

\subsection{c}

$f$ ist bijektiv, weil jedes $n \in \mathbb{R}_{> 0}$ genau ein Urbild in $\mathbb{R}_{> 0}$ hat.

\subsection{d}

$f$ ist nicht surjektiv, weil Elemente von $\mathbb{R}_{< 0}$ kein Urbild in $\mathbb{R}_{> 0}$ haben: $\sqrt{-4} = \{\}$. $f$ ist jedoch injektiv, weil alle $n \in \mathbb{R}_{> 0}$
eine eindeutige Abbildung haben. Dies muss stimmen, da es keine Elemente gibt, die Gleichung (1) erfüllen.

\begin{equation}
\begin{aligned}
 &n, y \in \mathbb{R}_{> 0}\\
 &n \ne y \land n^2 = y^2
\end{aligned}
\end{equation}

\section{Aufgabe 2}

\begin{theorem}
 Eine Abbildung $f: A \rightarrow A$ auf einer endlichen Menge $A$ ist genau dann injektiv, wenn sie surjektiv ist.
\end{theorem}

\begin{proof}
 ``$\Rightarrow$'': Wenn $f$ injektiv ist, ist jedem Element der Quellmenge ein Objekt der Zielmenge eindeutig zugeordnet. Da die Größe der Quell- und Zielmenge gleich ist, ist jedes Element
 der Zielmenge Bild genau eines Objekts der Quellmenge. $f$ ist also auch surjektiv.
 
 ``$\Leftarrow$'': Wenn $f$ surjektiv ist, ist jedes Element der Zielmenge Bild mindestens eines Elements der Quellmenge. Da die Größe der Quell- und Zielmenge gleich ist, bildet f also jedes
 Element von $A$ auf genau ein anderes Element von $A$ ab. $f$ ist somit auch injektiv.
\end{proof}

Daraus folgt dann offensichtlich, dass Abbildungen wie $f$, die surjektiv/injektiv sind, auch immer bijektiv sind.

\end{document}
