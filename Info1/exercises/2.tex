\documentclass[a4paper,10pt]{article}
\usepackage[utf8]{inputenc}
\usepackage{amsmath}
\usepackage{amssymb}
\usepackage{amsthm}
\usepackage[german]{babel}

\title{Übungsblatt 2}
\author{Marten Lienen (2126759)}

\newtheorem*{claim}{Behauptung}

\begin{document}

\maketitle

\section*{Aufgabe 1}

\subsection*{a}

\begin{claim}
 \begin{equation*}
  a, b, n, p , q, x \in \mathbb{N} \Rightarrow (a = nx \Rightarrow (b = px \Leftrightarrow b \bmod a = qx))
 \end{equation*}
\end{claim}

\begin{proof}
 \begin{description}
  \item[``$\Rightarrow$'']
  Entweder ist $a > b$. Dann ist $b \bmod a = b = px$.
  Ansonsten muss $a \le b$ gelten.
  Dann wäre $b \bmod a = 0 = 0 * x$.
  
  \item[``$\Leftarrow$'']
  Wenn $b \bmod a = qx$, dann $b = ka + qx = knx + qx = (kn + q)x = px$.
 \end{description}
\end{proof}

\subsection*{b}

\begin{description}
 \item[Eingabe] Zwei natürliche Zahlen $a$, $b$ mit $a \le b$
 \item[Ausgabe] Eine natürliche Zahl $x$
\end{description}

\begin{description}
 \item[Schritt 1] Wenn $b = 0$, setze $x := a$ und beende die Rechnung, sonst führe Schritt 2 aus.
 \item[Schritt 2] Führe Schritt 1 aus mit $a := b \bmod a$ und $b := a$.
\end{description}


\section*{Aufgabe 2}

\subsection*{a}

\begin{claim}
 \begin{equation}
  P(n): f(n) := \sum_{j = 1}^{n} j = \frac{n(n + 1)}{2}
 \end{equation}

\end{claim}

\begin{proof}
 \begin{equation*}
  P(0): f(0) = \frac{0(0 + 1)}{2} = 0
 \end{equation*}
 
 Wenn $P(n)$ gilt, dann ist die Formel für $n + 1$:
 \begin{equation*}
  \sum_{j = 1}^{n + 1} j = \frac{n(n + 1)}{2} + (n + 1)
 \end{equation*}
 
 \begin{align*}
  \frac{n(n + 1)}{2} + (n + 1) & = \frac{n(n + 1) + 2(n + 1)}{2}
  & = \frac{(2 + n)(n + 1)}{2}
  & = \frac{(n + 1)((n + 1) + 1)}{2} = f(n + 1)
 \end{align*}

 Es gilt also auch $P(n + 1)$.
\end{proof}

\subsection*{b}

Man stelle sich vor, die Zahlen $1\dots n$ wären an einer Kette aufgereiht.
Nun knickt man die Kette so, dass immer zwei Zahlen nebeneinander liegen.
Diese Zahlen zählt man dann paarweise zusammen, was jedes mal $n + 1$ ergibt.
Dann addiert man alle Zwischenwerte auf.
Da die Kette halbiert wurde, gibt es nur $\frac{n}{2}$ Zwischenwerte.
Die Summe aller Zahlen ist also $\frac{n}{2} * (n + 1)$.

\subsection*{c}

Die Anzahl der Verbindungen ist die Summe der Zahlen von $0$ bis $n - 1$, weil der erste Rechner $n - 1$ Verbindungen braucht, nämlich zu jedem Rechner außer sich selbst $1$.
Der nächste Rechner benötigt $n - 2$, weil er zum ersten Rechner bereits eine Verbindung hat.
Der letzte Rechner benötigt schließlich keine weitere Verbindung mehr, weil er bereits mit jedem verbunden ist.

\end{document}
