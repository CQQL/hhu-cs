\documentclass[a4paper,10pt]{article}
\usepackage[utf8]{inputenc}
\usepackage[german]{babel}
\usepackage{amsmath}
\usepackage{amsthm}
\usepackage{color}
\usepackage{listings}
\lstset{
 language=java,                % choose the language of the code
 basicstyle=\footnotesize,       % the size of the fonts that are used for the code
 numbers=left,                   % where to put the line-numbers
 numberstyle=\footnotesize,      % the size of the fonts that are used for the line-numbers
 stepnumber=1,                   % the step between two line-numbers. If it is 1 each line will be numbered
 numbersep=5pt,                  % how far the line-numbers are from the code
 backgroundcolor=\color{white},  % choose the background color. You must add \usepackage{color}
 showspaces=false,               % show spaces adding particular underscores
 showstringspaces=false,         % underline spaces within strings
 showtabs=false,                 % show tabs within strings adding particular underscores
 tabsize=2,          % sets default tabsize to 2 spaces
 captionpos=b,           % sets the caption-position to bottom
 breaklines=true,        % sets automatic line breaking
 breakatwhitespace=false,    % sets if automatic breaks should only happen at whitespace
 escapeinside={\%*}{*)}          % if you want to add a comment within your code
}

\title{Info1, Übungsblatt 12}
\author{Marten Lienen (2126759)}

\begin{document}

\maketitle

\section*{Aufgabe 1}

\subsection*{a}

\begin{proof}
 Da jede Kante $\{u, v\}$ in einem ungerichteten Graphen den sowohl den Grad von $u$ als auch von $v$ um eins erhöht, erhöht sie also die Summe aller Grade um $2$.
 Da es $|E|$ Kanten gibt, geschieht dies genau $|E|$ mal und es gilt
 \begin{equation}
  \sum_{v \in V} degree(v) = 2|E|
 \end{equation}
\end{proof}

\subsection*{b}

\begin{proof}
 Sei $G$ ein ungerichteter Graph und seine $n$ Knoten seien $v_j$ mit $1 \le j \le n$.
 Ich zeige es mit Induktion über $n$.
 Für $n = 1$, ist die Aussage offensichtlich wahr: $|E| = 0 \ge 0 = |V| - 1$.
 Nehmen wir an, dass $n \ge 2$ und $G$ sei zusammenhängend.
 Sei $G'$ ein neuer Graph mit $V' = V \cap \{ v_{n + 1} \}$ und $E' = E \cap \{ \{v_{n + 1}, v_j\} \}$.
 Dann ist $G'$ ebenfalls zusammenhängend und es gilt
 \begin{equation}
  |E'| = |E| + 1 \ge |V| = |V'| - 1
 \end{equation}
\end{proof}

\subsection*{c}

\begin{proof}
 Sei $u$ ein Knoten von ungeradem Grad.
 
\end{proof}

\subsection*{d}

\subsection*{e}

\section*{Aufgabe 2}

Das besondere an diesem Code ist, dass Autoboxing verwendet wird.
Der Vergleichsoperator vergleicht also die Instanzen der Integer-Objekte und nicht den Wert, den sie repräsentieren.
Um das Verhalten einer int-Variable zu erhalten, muss man hier mit equals vergleichen.
Schaut man sich den Bytecode an, sieht man, dass dieses Beispiel eigentlich folgendermaßen aussieht.
\begin{lstlisting}
 public class Autoboxing {
   public static void main (String[] args) {
     Integer a = Integer.valueOf(127);
     Integer b = Integer.valueOf(127);

     System.out.println(a == b);

     a = Integer.valueOf(a.intValue() + 1);
     b = Integer.valueOf(b.intValue() + 1);

     System.out.println(a == b);
   }
 }
\end{lstlisting}
Man sieht nun, dass die Ausgabe so ist wie sie ist, weil die Instanzen in Zeile 6 gleich und in Zeile 11 ungleich sind.
Jetzt ergibt sich aber noch die Frage, warum Zeile 6 und 11 nicht beide $false$ oder beide $true$ sind, wo $a$ und $b$ doch beides Mal den Rückgabewert von $Integer.valueOf$ mit dem jeweils gleichen Argument enthalten.
Die Antwort ist, dass $Integer.valueOf$ für Zahlen bis $127$ Instance-Caching anwendet, um die Performanz zu verbessern, für Zahlen über 127 jedoch jedes mal eine neue Instanz erzeugt.

Meine Aussage gilt nur für OpenJDK-6.

\end{document}
