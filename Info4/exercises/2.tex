\documentclass[10pt,a4paper]{article}
\usepackage[utf8]{inputenc}
\usepackage[german]{babel}
\usepackage{mathrsfs}
\usepackage{amsmath}
\usepackage{amsfonts}
\usepackage{amssymb}
\usepackage{amsthm}
\usepackage[left=2cm,right=2cm,top=2cm,bottom=2cm]{geometry}

\begin{document}

\section{Aufgabe 2.1}

\subsection{Teil a}

\subsubsection{(i)}

Wenn es Kühe gibt und nicht alle Kühe schwarze Flecken haben, gibt es mindestens ein Kuh, die keine schwarzen Flecken hat.
Sonst hätten ja alle Kühe schwarze Flecken.
Und wenn es eine Kuh gibt, die keine schwarzen Flecken hat, haben nicht alle Kühe schwarze Flecken.

\subsubsection{(ii)}

Wenn keine Kuh lila Flecken hat, haben alle Kühe keine lila Flecken.
Wenn alle Kühe keine lila Flecken haben, gibt es keine Kuh, die doch welche hat.

\subsection{Teil b}

\subsection{(i)}

\begin{align*}
  (\exists x \cdot f(x)) \Rightarrow g(a) & = \neg(\exists x \cdot f(x)) \lor g(a)\\
  & = \neg(\neg(\forall x \cdot \neg f(x))) \lor g(a)\\
  & = (\forall x \cdot \neg f(x)) \lor g(a)\\
  & = \forall x \cdot \neg f(x) \lor g(a)
\end{align*}

\subsection{(ii)}

\begin{align*}
(\forall y \cdot g(y, b) \Rightarrow f(y)) \land \neg(\exists x \cdot f(x)) & = (\forall y \cdot \neg g(y, b) \lor f(y)) \land (\forall x \cdot \neg f(x))
\end{align*}

\subsection{Teil c}

KNF mit negiertem Query
\begin{align*}
  \{ kuh(milka), lila(milka), (\forall x \cdot ((kuh(x) \land lila(x)) \Rightarrow werbung(x))), \neg(kuh(milka) \land werbung(milka)) \} & = \{ kuh(milka), lila(milka), (\forall x \cdot (\neg (kuh(x) \land lila(x)) \lor werbung(x))), \neg kuh(milka) \lor \neg werbung(milka) \}\\
  & = \{ kuh(milka), lila(milka), (\forall x \cdot (\neg kuh(x) \lor \neg lila(x) \lor werbung(x))), \neg kuh(milka) \lor \neg werbung(milka) \}
\end{align*}
Resolution von 1 und 4
\begin{equation}
  \{ kuh(milka), lila(milka), (\forall x \cdot (\neg kuh(x) \lor \neg lila(x) \lor werbung(x))), \neg kuh(milka) \lor \neg werbung(milka), \neg werbung(milka) \}
\end{equation}
Resolution von 3 und 5 mit Einsetzung $\{ x / milka \}$
\begin{equation}
  \{ kuh(milka), lila(milka), (\forall x \cdot (\neg kuh(x) \lor \neg lila(x) \lor werbung(x))), \neg kuh(milka) \lor \neg werbung(milka), \neg werbung(milka), \neg kuh(milka) \lor \neg lila(milka) \}
\end{equation}
Resolution von 1 und 6
\begin{equation}
  \{ kuh(milka), lila(milka), (\forall x \cdot (\neg kuh(x) \lor \neg lila(x) \lor werbung(x))), \neg kuh(milka) \lor \neg werbung(milka), \neg werbung(milka), \neg kuh(milka) \lor \neg lila(milka), \neg lila(milka) \}
\end{equation}
Resolution von 2 und 7
\begin{equation}
  \{ kuh(milka), lila(milka), (\forall x \cdot (\neg kuh(x) \lor \neg lila(x) \lor werbung(x))), \neg kuh(milka) \lor \neg werbung(milka), \neg werbung(milka), \neg kuh(milka) \lor \neg lila(milka), \neg lila(milka), \{  \} \}
\end{equation}
Widerspruch. Also ist der Query und somit die Folgerung wahr.

\section{Aufgabe 2.2}

\subsection{Teil a}

Eine Relation $R$ ist eine Teilmenge von $A_{1} \times \dots \times A_{n}$.
Dabei stehen $a_{1} \in A_{1}, \dots, a_{n} \in A_{n}$ in Relation, wenn $(a_{1}, \dots, a_{n}) \in R$.
Ein Beispielt für eine 3er Relation ist die Mutter-Vater-Kind-Relation.

Eine Funktion ist eine binäre Relation $R \subset A \times B$ für die darüber hinaus gilt, dass es für jedes $a \in A$ genau ein Tupel $(a, b) \in R$ gibt.

\subsection{Teil b}

\subsubsection{(i)}

\begin{equation}
  fun(r) = \{ 4 \mapsto \{ 7, 6 \}, 0 \mapsto \{ 1, 2 \}, 1 \mapsto \{ 1, 5 \}, 3, \mapsto \{ 3 \}, 2 \mapsto \{ 3 \} \}
\end{equation}

\subsubsection{(ii)}

$fun(r)$ ist eine totale Funktion, weil sie für jedes $s \in S$ definiert ist.

\subsection{Teil c}

\begin{equation}
  rel(f) = \cup_{x \in dom(f)} \{ x \mapsto y \mid y \in f(x) \}
\end{equation}

\section{Aufgabe 2.3}

\subsection{Teil a}

\begin{equation}
  F = \{ \neg A \lor B, \neg A \lor \neg C, \neg A \lor \neg B \lor C, A \lor B \lor \neg C \}, p = \{  \}
\end{equation}
Rekursion mit
\begin{equation}
  F = \{ B, \neg C, \neg B \lor C, A \lor B \lor \neg C \}, p = \{ A \}
\end{equation}
Unit-Propagation
\begin{equation}
  F = \{ \{  \} \}, p = \{ A, B, C \}
\end{equation}

Rekursion mit
\begin{equation}
  F = \{ \neg A \lor B, \neg A \lor \neg C, \neg A \lor \neg B \lor C, B \lor \neg C \}, p = \{ \neg A \}
\end{equation}
Rekursion mit
\begin{equation}
  F = \{ \neg A \lor \neg C, \neg A \lor C \}, p = \{ \neg A, B \}
\end{equation}
Rekursion mit
\begin{equation}
  F = \{ \neg A, \neg A \lor C \}, p = \{ \neg A, B, C \}
\end{equation}
Unit-Propagation
\begin{equation}
  F = \{ \}, p = \{ \neg A, B, C \}
\end{equation}
Es ist erfüllbar mit dem Modell $\{ \neg A, B, C \}$.

\subsection{Teil b}

Umwandeln in KNF ergibt
\begin{align*}
  (A \land B \land C) \Rightarrow (A \land B \land \neg C) & = \neg((A \land B \land C) \land \neg(A \land B \land \neg C))\\
  & = \neg(A \land B \land C) \lor (A \land B \land \neg C)\\
  & = \neg A \lor \neg B \lor \neg C \lor (A \land B \land \neg C)\\
  & = (A \lor \neg A \lor \neg B \lor \neg C) \land (B \lor \neg A \lor \neg B \lor \neg C) \land (\neg C \lor \neg A \lor \neg B \lor \neg C)\\
  & = \neg C \lor \neg A \lor \neg B
\end{align*}

\begin{equation}
  F = \{ \neg C \lor \neg A \lor \neg B \}, p = \{  \}
\end{equation}
Rekursion mit
\begin{equation}
  F = \{ \neg C \lor \neg B \}, p = \{ A \}
\end{equation}
Rekursion mit
\begin{equation}
  F = \{ \neg C \}, p = \{ A, B \}
\end{equation}
Rekursion mit
\begin{equation}
  F = \{ \{  \} \}, p = \{ A, B, C \}
\end{equation}
Widerspruch.
Rekursion mit
\begin{equation}
  F = \{ \}, p = \{ A, B, \neg C \}
\end{equation}
Es ist erfüllbar und ein Modell lautet $\{ A, B, \neg C \}$.

\end{document}