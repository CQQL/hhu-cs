\documentclass[10pt,a4paper]{article}
\usepackage[utf8]{inputenc}
\usepackage[german]{babel}
\usepackage{mathrsfs}
\usepackage{amsmath}
\usepackage{amsfonts}
\usepackage{amssymb}
\usepackage{amsthm}
\usepackage[left=2cm,right=2cm,top=2cm,bottom=2cm]{geometry}

\begin{document}

\section{Übung 8.1}

\subsection{Teil a}

\subsubsection{(i)}

\begin{proof}
  Angenommen $G_{1}$ ist $LL(k)$.
  Betrachte $a^{k}b$ und $a^{k}c$ mit den Ableitungen
  \begin{equation}
    S \vdash aA \vdash^{*} a^{k}b
  \end{equation}
  und
  \begin{equation}
    S \vdash aB \vdash^{*} a^{k}c
  \end{equation}
  mit
  \begin{equation}
    P_{1} = P_{1}' = \lambda, X = S, P_{2} = aA, P_{2}' = aB, P_{3} = P_{3}' = \lambda, P_{4} = a^{k}b, P_{4}' = a^{k}c
  \end{equation}
  Obwohl $FIRST_{k}(P_{4}) = FIRST_{k}(P_{4}')$, ist $P_{2} \ne P_{2}'$.
\end{proof}

\subsubsection{(ii)}

\begin{equation}
  G' = (\{ a, b, c \}, \{ S \}, S, P_{1}')
\end{equation}
\begin{align*}
  P_{1}' = \{ & S \rightarrow aS | b | c \}
\end{align*}
Dies ist $LL(1)$, weil wenn man 1 Terminal in das Suffix hineinschaut, ist klar, welche Regel angewandt wurde.

\subsubsection{(iii)}

Es kann keine $LL(0)$-Grammatik geben, weil man dann irgendwo entscheiden müsste, ob das Wort mit einem $b$ oder $c$ endet ohne ``hinzugucken''.

\subsection{Teil b}

\subsubsection{(i)}

Man betrachte die Worte $a = uwxxxxz$ und $b = uwxxz$.
Dann hat man die Ableitungen
\begin{equation}
  S \vdash^{*} uwExz \vdash uwxxz
\end{equation}
und
\begin{equation}
  S \vdash^{*} uwExz \vdash uwExxz \vdash^{*} uwxxxxz
\end{equation}
Dann sind die Bedingungen erfüllt mit
\begin{equation}
  P_{1} = uw, P_{2} = x, P_{2}' = Ex, P_{3} = P_{3}' = xz, P_{4} = xxz, P_{4}' = xxxxz
\end{equation}
aber $P_{2} \ne P_{2}'$.

\subsubsection{(ii)}

Für keins, weil man im Gegenbeweis von Teil (i) beliebig viele $x$ produzieren kann, sodass kein Lookahead ausreicht, um zu entscheiden, welche Regel angewandt wurde.

\section{Übung 8.2}

\subsection{Teil a}

\subsection{Teil b}

\section{Übung 8.3}

\subsection{Teil a}

\subsection{Teil b}

\subsection{Teil c}

\section{Übung 8.4}

\subsection{Teil a}

\subsection{Teil b}

\subsection{Teil c}

\end{document}