\documentclass[10pt,a4paper]{article}
\usepackage[utf8]{inputenc}
\usepackage[german]{babel}
\usepackage{mathrsfs}
\usepackage{amsmath}
\usepackage{amsfonts}
\usepackage{amssymb}
\usepackage{amsthm}
\usepackage[left=2cm,right=2cm,top=2cm,bottom=2cm]{geometry}

\begin{document}

\section{Übung 8.1}

\subsection{Teil a}

\subsubsection{(i)}

\begin{proof}
  Angenommen $G_{1}$ ist $LL(k)$.
  Betrachte $a^{k}b$ und $a^{k}c$ mit den Ableitungen
  \begin{equation}
    S \vdash aA \vdash^{*} a^{k}b
  \end{equation}
  und
  \begin{equation}
    S \vdash aB \vdash^{*} a^{k}c
  \end{equation}
  mit
  \begin{equation}
    P_{1} = P_{1}' = \lambda, X = S, P_{2} = aA, P_{2}' = aB, P_{3} = P_{3}' = \lambda, P_{4} = a^{k}b, P_{4}' = a^{k}c
  \end{equation}
  Obwohl $FIRST_{k}(P_{4}) = FIRST_{k}(P_{4}')$, ist $P_{2} \ne P_{2}'$.
\end{proof}

\subsubsection{(ii)}

\begin{equation}
  G' = (\{ a, b, c \}, \{ S \}, S, P_{1}')
\end{equation}
\begin{align*}
  P_{1}' = \{ & S \rightarrow aS | b | c \}
\end{align*}
Dies ist $LL(1)$, weil wenn man 1 Terminal in das Suffix hineinschaut, ist klar, welche Regel angewandt wurde.

\subsubsection{(iii)}

Es kann keine $LL(0)$-Grammatik geben, weil man dann irgendwo entscheiden müsste, ob das Wort mit einem $b$ oder $c$ endet ohne ``hinzugucken''.

\subsection{Teil b}

\subsubsection{(i)}

Man betrachte die Worte $a = uwxxxxz$ und $b = uwxxz$.
Dann hat man die Ableitungen
\begin{equation}
  S \vdash^{*} uwExz \vdash uwxxz
\end{equation}
und
\begin{equation}
  S \vdash^{*} uwExz \vdash uwExxz \vdash^{*} uwxxxxz
\end{equation}
Dann sind die Bedingungen erfüllt mit
\begin{equation}
  P_{1} = uw, P_{2} = x, P_{2}' = Ex, P_{3} = P_{3}' = xz, P_{4} = xxz, P_{4}' = xxxxz
\end{equation}
aber $P_{2} \ne P_{2}'$.

\subsubsection{(ii)}

Für keins, weil man im Gegenbeweis von Teil (i) beliebig viele $x$ produzieren kann, sodass kein Lookahead ausreicht, um zu entscheiden, welche Regel angewandt wurde.

\section{Übung 8.2}

\subsection{Teil a}

\subsection{Teil b}

\section{Übung 8.3}

\subsection{Teil a}

Ich geben die zwei Konfigurationenfolgen an.
\begin{align*}
  & z_{0}011\\
  \vdash & \square z_{1} 11\\
  \vdash & \square 1 z_{1} 1\\
  \vdash & \square 11 z_{1} \square\\
  \vdash & \square 1 z_{2} 1
\end{align*}
$011$ wird nicht akzeptiert, weil es keinen Turingbefehl für $(z_{2}, 1)$ gibt.

\begin{align*}
  & z_{0}01010\\
  \vdash & \square z_{1} 1010\\
  \vdash & \square 1 z_{1} 010\\
  \vdash & \square 10 z_{1} 10\\
  \vdash & \square 101 z_{1} 0\\
  \vdash & \square 1010 z_{1} \square\\
  \vdash & \square 101 z_{2} 0\\
  \vdash & \square 10 z_{3} 1 \square\\
  \vdash & \square 1 z_{3} 01 \square\\
  \vdash & \square z_{3} 101 \square\\
  \vdash & z_{3} \square 101 \square\\
  \vdash & \square z_{0} 101 \square\\
  \vdash & \square \square z_{4} 01 \square\\
  \vdash & \square \square 0 z_{4} 1 \square\\
  \vdash & \square \square 01 z_{4} \square\\
  \vdash & \square \square 0 z_{5} 1 \square\\
  \vdash & \square \square z_{3} 0 \square \square\\
  \vdash & \square z_{3} \square 0 \square \square\\
  \vdash & \square \square z_{0} 0 \square \square\\
  \vdash & \square \square \square z_{1} \square \square\\
  \vdash & \square \square z_{2} \square \square \square\\
  \vdash & \square \square \square z_{6} \square \square
\end{align*}
$01010$ wird akzeptiert, weil $z_{6} \in \{ z_{6} \}$.

\subsection{Teil b}

\begin{tabular}{c|c|c}
  Zustand & Bedeutung & Absicht\\\hline
  $z_{0}$ & Anfangszustand & Neuer Zyklus\\\hline
  $z_{1}$ & Zyklus begann mit $0$ & rechten Rand suchen\\\hline
  $z_{2}$ & rechter Rand erreicht in $0$-er Zyklus erreicht & Test ob Wort mit $0$ endet oder abgearbeitet wurde\\\hline
  $z_{3}$ & Wort endete mit gleichen Terminal wie es begann & linken Rand suchen und neustarten\\\hline
  $z_{4}$ & Zyklus begann mit $1$ & rechten Rand suchen\\\hline
  $z_{5}$ & rechter Rand in $1$-er Zyklus erreicht & Test ob Wort mit $1$ endet oder abgearbeitet wurde\\\hline
  $z_{6}$ & Wort abgearbeitet & Akzeptieren
\end{tabular}

\subsection{Teil c}

$.$ ist die Konkatenation und $sp$ die Spiegelung.
\begin{equation}
  L(M) = \{ w . x . sp(w) \mid w \in \{ 0, 1 \}^{*}, x \in \{ 0, 1, \lambda \} \}
\end{equation}

\section{Übung 8.4}

\subsection{Teil a}

Definiere $M = (\{ 0, 1 \}, \{ 0, 1, \square, A \}, \{ z_{0} \}, \delta, z_{0}, \square, \{ z_{5} \})$ mit $\delta$ gegeben durch
\begin{align*}
  (z_{0}, 0) & \rightarrow (z_{1}, A, R)\\
  (z_{0}, A) & \rightarrow (z_{0}, A, R)\\
  (z_{0}, \square) & \rightarrow (z_{5}, \square, N)\\
  (z_{1}, 0) & \rightarrow (z_{1}, 0, R)\\
  (z_{1}, 1) & \rightarrow (z_{2}, A, R)\\
  (z_{1}, A) & \rightarrow (z_{1}, A, R)\\
  (z_{2}, 1) & \rightarrow (z_{2}, 1, R)\\
  (z_{2}, 0) & \rightarrow (z_{3}, A, R)\\
  (z_{2}, A) & \rightarrow (z_{2}, A, R)\\
  (z_{3}, 0) & \rightarrow (z_{3}, 0, R)\\
  (z_{3}, \square) & \rightarrow (z_{4}, \square, L)\\
  (z_{4}, 0) & \rightarrow (z_{4}, 0, L)\\
  (z_{4}, 1) & \rightarrow (z_{4}, 1, L)\\
  (z_{4}, A) & \rightarrow (z_{4}, A, L)\\
  (z_{4}, \square) & \rightarrow (z_{0}, \square, R)\\
\end{align*}

\begin{tabular}{c|c|c}
  Zustand & Bedeutung & Absicht\\\hline
  $z_{0}$ & Anfangszustand & Neuer Zyklus\\\hline
  $z_{1}$ & $0$ gemerkt & $1$ finden\\\hline
  $z_{2}$ & $0$ und $1$ gemerkt & $0$ finden\\\hline
  $z_{3}$ & $0$, $1$ und $0$ gemerkt & rechten Rand finden\\\hline
  $z_{4}$ & rechter Rand gefunden & linken Rand finden\\\hline
  $z_{5}$ & gesamte Eingabe wurde verarbeitet & Akzeptieren
\end{tabular}

\subsection{Teil b}

\begin{align*}
  & z_{0} 010\\
  & A z_{1} 10\\
  & AA z_{2} 0\\
  & AAA z_{3} \square\\
  & AA z_{4} A\\
  & A z_{4} AA\\
  & z_{4} AAA\\
  & z_{4} \square AAA\\
  & z_{0} AAA\\
  & A z_{0} AA\\
  & AA z_{0} A\\
  & AAA z_{0} \square\\
  & AAA z_{5} \square\\
\end{align*}

\subsection{Teil c}

\begin{align*}
  & z_{0}00110\\
  \vdash & A z_{1} 0110\\
  \vdash & A0 z_{1} 110\\
  \vdash & A0A z_{2} 10\\
  \vdash & A0A1 z_{2} 0\\
  \vdash & A0A1A z_{3} \square\\
  \vdash & A0A1 z_{4} A\\
  \vdash & A0A z_{4} 1A\\
  \vdash & A0 z_{4} A1A\\
  \vdash & A z_{4} 0A1A\\
  \vdash & z_{4} A0A1A\\
  \vdash & z_{4} \square A0A1A\\
  \vdash & z_{0} A0A1A\\
  \vdash & A z_{0} 0A1A\\
  \vdash & AA z_{1} A1A\\
  \vdash & AAA z_{1} 1A\\
  \vdash & AAAA z_{2} A\\
  \vdash & AAAAA z_{2} \square\\
\end{align*}
Nun gibt es keinen weiteren Turingbefehl mehr und das Wort wird nicht akzeptiert.

\end{document}