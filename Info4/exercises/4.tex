\documentclass[10pt,a4paper]{article}
\usepackage[utf8]{inputenc}
\usepackage[german]{babel}
\usepackage{mathrsfs}
\usepackage{amsmath}
\usepackage{amsfonts}
\usepackage{amssymb}
\usepackage{amsthm}
\usepackage[left=2cm,right=2cm,top=2cm,bottom=2cm]{geometry}

\begin{document}

\section{Übung 4.1}

\subsection{Teil a}

\subsubsection{(i)}

\begin{equation}
  L(M_{1}) = \{ w \in \Sigma^{*} \mid \textit{$w$ erfüllt den regulären Ausdruck $(0^{*}(1(0(1)^{*}0)1)^{*})^{*}$} \}
\end{equation}

\subsubsection{(ii)}

\begin{equation}
  L(M_{2}) = \{ w \in \Sigma^{*} \mid \textit{$w$ erfüllt den regulären Ausdruck $0^{*}(1 (0 + ((01)^{*} 11)) (0 (1)^{*} 0 (1 + ((01)^{*} 00)))^{*} 1)^{*}$} \}
\end{equation}

\subsection{Teil b}

\subsubsection{(i)}

Worte dieser Sprache sind also beliebig lange Ketten von abwechselnden Nullen und Einsen.

Definiere den Automaten $M$ durch
\begin{equation}
  M = (\Sigma, \{ a, b, c, d \}, \delta, a, \{ a, b, c \})
\end{equation}
mit der Übergangsfunktion $\delta$
\\
\begin{tabular}{c|c|c|c|c}
  $\delta$ & a & b & c & d\\\hline
  0 & b & d & b & d\\\hline
  1 & c & c & d & d\\
\end{tabular}

\subsubsection{(ii)}

Definiere den Automaten $M$ durch
\begin{equation}
  M = (\Sigma, \{ a, b \}, \delta, a, \{ b \})
\end{equation}
mit der Übergangsfunktion $\delta$
\\
\begin{tabular}{c|c|c|c|c}
  $\delta$ & a & b\\\hline
  0 & b & a\\\hline
  1 & b & a\\
\end{tabular}

\section{Übung 4.2}

\subsection{Teil a}

\begin{proof}
  Sei $n$ die Länge von $w$.

  Sei $n = 0$.
  Dann ist nach der Definition der Übergangsfunktion $\hat{\delta}(z_{i}, w) = z_{i}$.

  Sei $n > 0$ und es für $n - 1$ bereits gezeigt.
  Dann hat man für $w = aw'$ mit $a \in \Sigma$
  \begin{equation}
    \hat{\delta}(z_{i}, aw') = \delta(\hat{\delta}(z_{i}, w'), a) = \delta(z_{i}, a) = z_{i}
  \end{equation}
\end{proof}

\subsection{Teil b}

\subsubsection{(i)}

\begin{proof}
  Sei $n$ die Länge von $w$.

  Sei $n = 1$.
  Dann ist es nach Vorraussetzung wahr.

  Sei $n > 1$ und es sei bereits gezeigt für $n - 1$.
  Sei $w = w'a$ mit $a \in \Sigma$.
  \begin{equation}
    \hat{\delta}(z_{0}, w) = \hat{\delta}(z_{0}, w'a) = \delta(\hat{\delta}(z_{0}, w'), a) = \delta(\hat{\delta}(z, w'), a) = \hat{\delta}(z, w'a) = \hat{\delta}(z, w)
  \end{equation}
\end{proof}

\subsubsection{(ii)}

\begin{proof}
  Sei $n = 1$.
  Da $w$ in der Sprache enthalten ist, gilt
  \begin{equation}
    \hat{\delta}(z_{0}, w^{1}) = z
  \end{equation}

  Sei $n > 1$ und es für $n - 1$ bereits gezeigt.
  \begin{equation}
    \hat{\delta}(z_{0}, w^{n}) = \hat{\delta}(\hat{\delta}(z_{0}, w^{n - 1}), w) = \hat{\delta}(z, w) = \hat{\delta}(z_{0}, w) = z
  \end{equation}
  Und da $z$ ein Finalzustand ist, ist $w^{n} \in L(M)$.
\end{proof}

\section{Übung 4.3}

\subsection{Teil a}

\subsection{Teil b}

\section{Übung 4.4}

\subsection{Teil a}

\subsection{Teil b}

\subsubsection{(i)}

\subsubsection{(ii)}

\end{document}