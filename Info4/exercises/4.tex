\documentclass[10pt,a4paper]{article}
\usepackage[utf8]{inputenc}
\usepackage[german]{babel}
\usepackage{mathrsfs}
\usepackage{amsmath}
\usepackage{amsfonts}
\usepackage{amssymb}
\usepackage{amsthm}
\usepackage[left=2cm,right=2cm,top=2cm,bottom=2cm]{geometry}
\usepackage{tikz}
\usetikzlibrary{automata,positioning}

\begin{document}

\section{Übung 4.1}

\subsection{Teil a}

\subsubsection{(i)}

\begin{equation}
  L(M_{1}) = \{ w \in \Sigma^{*} \mid \textit{$w$ erfüllt den regulären Ausdruck $(0^{*}(1(0(1)^{*}0)1)^{*})^{*}$} \}
\end{equation}

\subsubsection{(ii)}

\begin{equation}
  L(M_{2}) = \{ w \in \Sigma^{*} \mid \textit{$w$ erfüllt den regulären Ausdruck $0^{*}(1 (0 + ((01)^{*} 11)) (0 (1)^{*} 0 (1 + ((01)^{*} 00)))^{*} 1)^{*}$} \}
\end{equation}

\subsection{Teil b}

\subsubsection{(i)}

Worte dieser Sprache sind also beliebig lange Ketten von abwechselnden Nullen und Einsen.

Definiere den Automaten $M$ durch
\begin{equation}
  M = (\Sigma, \{ a, b, c, d \}, \delta, a, \{ a, b, c \})
\end{equation}
mit der Übergangsfunktion $\delta$
\\
\begin{tabular}{c|c|c|c|c}
  $\delta$ & a & b & c & d\\\hline
  0 & b & d & b & d\\\hline
  1 & c & c & d & d\\
\end{tabular}

\subsubsection{(ii)}

Definiere den Automaten $M$ durch
\begin{equation}
  M = (\Sigma, \{ a, b \}, \delta, a, \{ b \})
\end{equation}
mit der Übergangsfunktion $\delta$
\\
\begin{tabular}{c|c|c|c|c}
  $\delta$ & a & b\\\hline
  0 & b & a\\\hline
  1 & b & a\\
\end{tabular}

\section{Übung 4.2}

\subsection{Teil a}

\begin{proof}
  Sei $n$ die Länge von $w$.

  Sei $n = 0$.
  Dann ist nach der Definition der Übergangsfunktion $\hat{\delta}(z_{i}, w) = z_{i}$.

  Sei $n > 0$ und es für $n - 1$ bereits gezeigt.
  Dann hat man für $w = aw'$ mit $a \in \Sigma$
  \begin{equation}
    \hat{\delta}(z_{i}, aw') = \delta(\hat{\delta}(z_{i}, w'), a) = \delta(z_{i}, a) = z_{i}
  \end{equation}
\end{proof}

\subsection{Teil b}

\subsubsection{(i)}

\begin{proof}
  Sei $n$ die Länge von $w$.

  Sei $n = 1$.
  Dann ist es nach Vorraussetzung wahr.

  Sei $n > 1$ und es sei bereits gezeigt für $n - 1$.
  Sei $w = w'a$ mit $a \in \Sigma$.
  \begin{equation}
    \hat{\delta}(z_{0}, w) = \hat{\delta}(z_{0}, w'a) = \delta(\hat{\delta}(z_{0}, w'), a) = \delta(\hat{\delta}(z, w'), a) = \hat{\delta}(z, w'a) = \hat{\delta}(z, w)
  \end{equation}
\end{proof}

\subsubsection{(ii)}

\begin{proof}
  Sei $n = 1$.
  Da $w$ in der Sprache enthalten ist, gilt
  \begin{equation}
    \hat{\delta}(z_{0}, w^{1}) = z
  \end{equation}

  Sei $n > 1$ und es für $n - 1$ bereits gezeigt.
  \begin{equation}
    \hat{\delta}(z_{0}, w^{n}) = \hat{\delta}(\hat{\delta}(z_{0}, w^{n - 1}), w) = \hat{\delta}(z, w) = \hat{\delta}(z_{0}, w) = z
  \end{equation}
  Und da $z$ ein Finalzustand ist, ist $w^{n} \in L(M)$.
\end{proof}

\section{Übung 4.3}

\subsection{Teil a}

\begin{align*}
  \hat{\delta}(z_{0}, -123/45) & = \delta(\delta(\delta(\delta(\delta(\delta(\delta(\{z_{0}\}, -), 1), 2), 3), /), 4), 5)\\
  & = \delta(\delta(\delta(\delta(\delta(\delta(\{ z_{2} \}, 1), 2), 3), /), 4), 5)\\
  & = \delta(\delta(\delta(\delta(\delta(\{ z_{3} \}, 2), 3), /), 4), 5)\\
  & = \delta(\delta(\delta(\delta(\{ z_{3} \}, 3), /), 4), 5)\\
  & = \delta(\delta(\delta(\{ z_{3} \}, /), 4), 5)\\
  & = \delta(\delta(\{ z_{5} \}, 4), 5)\\
  & = \delta(\{ z_{4} \}, 5)\\
  & = \{ z_{4} \}\\
\end{align*}

\begin{equation}
  \{ z_{4} \} \cap \{ z_{4} \} \ne \emptyset
\end{equation}

Also ist $-123/45 \in L(G_{\mathbb{Q}})$.

\begin{align*}
  \hat{\delta}(z_{0}, -0/234) & = \delta(\delta(\delta(\delta(\delta(\delta(\{ z_{0} \}, -), 0), /), 2), 3), 4)\\
  & = \delta(\delta(\delta(\delta(\delta(\{ z_{2} \}, 0), /), 2), 3), 4)\\
  & = \delta(\delta(\delta(\delta(\{  \}, /), 2), 3), 4)\\
  & = \delta(\delta(\delta(\{  \}, 2), 3), 4)\\
  & = \delta(\delta(\{  \}, 3), 4)\\
  & = \delta(\{  \}, 4)\\
  & = \{  \}\\
\end{align*}

\begin{equation}
  \{  \} \cap \{ z_{4} \} = \emptyset
\end{equation}

Also ist $-0/234 \not\in L(G_{\mathbb{Q}})$.

\subsection{Teil b}

Ich werde die Automaten $M_{i}$ aus der Induktionsvoraussetzung als
\begin{tikzpicture}[shorten >= 1pt, node distance=3cm, on grid, auto]
  \node[state,initial] (Z) {$Z_{i}$};
  \node[state,accepting] (F) [right=of Z] {$F_{i}$};
  \path[->] (Z) edge node {} (F);
\end{tikzpicture}
darstellen, wobei $Z_{i}$ die Menge aller Startzustände und $F_{i}$ die Menge aller Endzustände sind und der einzelne Pfad die Menge aller Übergänge in dem Automaten darstellt.

\subsubsection{$\alpha = \emptyset$}

\begin{tikzpicture}[shorten >= 1pt, node distance=3cm, on grid, auto]
  \node[state,initial] (x) {$z_{0}$};
\end{tikzpicture}

\subsubsection{$\alpha = \lambda$}

\begin{tikzpicture}[shorten >= 1pt, node distance=3cm, on grid, auto]
  \node[state,initial,accepting] (x) {$z_{0}$};
\end{tikzpicture}

\subsubsection{$\alpha = a \in \Sigma$}

\begin{tikzpicture}[shorten >= 1pt, node distance=2cm, on grid, auto]
  \node[state,initial] (z0) {$z_{0}$};
  \node[state,accepting] (z1) [right=of z0] {$z_{1}$};
  \path[->]
  (z0) edge node {a} (z1)
  ;
\end{tikzpicture}

\subsubsection{$\alpha = \alpha_{1} \alpha_{2}$}

Sei $F'$ die Menge der Zustände im Automaten von $\alpha_{1}$, die einen direkten Übergang in einen finalen Zustand haben.

Kein Automat akzeptiert das leere Wort.\\
\begin{tikzpicture}[shorten >= 1pt, node distance=2cm, on grid, auto]
  \node[state,initial] (Z1) {$Z_{1}$};
  \node[state] (F) [right=of Z1] {$F'$};
  \node[state] (Z2) [right=of F] {$Z_{2}$};
  \node[state,accepting] (F2) [right=of Z2] {$F_{2}$};
  \path[->]
  (Z1) edge node {} (F)
  (F) edge node {} (Z2)
  (Z2) edge node {} (F2)
  ;
\end{tikzpicture}

Nur der erste Automat akzeptiert das leere Wort.\\
\begin{tikzpicture}[shorten >= 1pt, node distance=2cm, on grid, auto]
  \node[state,initial] (Z1) {$Z_{1}$};
  \node[state] (F) [below right=of Z1] {$F'$};
  \node[state,initial] (Z2) [above right=of F] {$Z_{2}$};
  \node[state,accepting] (F2) [right=of Z2] {$F_{2}$};
  \path[->]
  (Z1) edge node {} (F)
  (F) edge node {} (Z2)
  (Z2) edge node {} (F2)
  ;
\end{tikzpicture}

Nur der zweite Automat akzeptiert das leere Wort.\\
\begin{tikzpicture}[shorten >= 1pt, node distance=2cm, on grid, auto]
  \node[state,initial] (Z1) {$Z_{1}$};
  \node[state] (F) [right=of Z1] {$F'$};
  \node[state,accepting] (F1) [below right=of F] {$F_{1}$};
  \node[state] (Z2) [right=of F] {$Z_{2}$};
  \node[state,accepting] (F2) [right=of Z2] {$F_{2}$};
  \path[->]
  (Z1) edge node {} (F)
  (F) edge node {} (Z2)
  (F) edge node {} (F1)
  (Z2) edge node {} (F2)
  ;
\end{tikzpicture}

Beide Automaten akzeptieren das leere Wort.\\
\begin{tikzpicture}[shorten >= 1pt, node distance=2cm, on grid, auto]
  \node[state,initial] (Z1) {$Z_{1}$};
  \node[state] (F) [below right=of Z1] {$F'$};
  \node[state,accepting] (F1) [right=of F] {$F_{1}$};
  \node[state,initial] (Z2) [above right=of F] {$Z_{2}$};
  \node[state,accepting] (F2) [right=of Z2] {$F_{2}$};
  \path[->]
  (Z1) edge node {} (F)
  (F) edge node {} (Z2)
  (F) edge node {} (F1)
  (Z2) edge node {} (F2)
  ;
\end{tikzpicture}

\subsubsection{$\alpha = (\alpha_{1} + \alpha_{2})$}

\begin{tikzpicture}[shorten >= 1pt, node distance=2cm, on grid, auto]
  \node[state,initial] (Z1) {$Z_{1}$};
  \node[state,initial] (Z2) [below=of Z1] {$Z_{2}$};
  \node[state,accepting] (F1) [right=of Z1] {$F_{1}$};
  \node[state,accepting] (F2) [right=of Z2] {$F_{2}$};
  \path[->]
  (Z1) edge node {} (F1)
  (Z2) edge node {} (F2)
  ;
\end{tikzpicture}

\subsubsection{$\alpha = (\alpha_{1})^{*}$}

Sei $F'$ die Menge der Zustände im Automaten von $\alpha_{1}$, die einen direkten Übergang in einen finalen Zustand haben.

\begin{tikzpicture}[shorten >= 1pt, node distance=2cm, on grid, auto]
  \node[state,initial] (Z1) {$Z_{1}$};
  \node[state] (F) [right=of Z1] {$F'$};
  \node[state,accepting] (F1) [right=of F] {$F_{1}$};
  \path[->]
  (Z1) edge node {} (F)
  (F) edge node {} (F1)
  (F) edge [bend right] node {} (Z1)
  ;
\end{tikzpicture}

\section{Übung 4.4}

\subsection{Teil a}

Ich konstruiere den DFA nach Vorlesung.

\begin{tikzpicture}[shorten >= 1pt, node distance=3cm, on grid, auto]
  \node[state,initial] (z0) {$\{ z_{0} \}$};
  \node[state] (z2) [above right=of z0] {$\{ z_{2} \}$};
  \node[state] (em) [right=of z2] {$\{  \}$};
  \node[state,accepting] (z1z3) [below right=of z0] {$\{ z_{1}, z_{3} \}$};
  \node[state] (z1) [right=of z1z3] {$\{ z_{1} \}$};
  \path[->]
  (z0) edge [bend right] node {a} (z1z3)
  (z0) edge [bend left] node {b} (z2)
  (z2) edge node {a} (em)
  (z2) edge [bend left] node {b} (z1z3)
  (em) edge [loop above] node {a} (em)
  (em) edge [loop right] node {b} (em)
  (z1z3) edge [bend left] node {a} (z2)
  (z1z3) edge node {b} (z1)
  (z1) edge [bend right] node {a} (z2)
  (z1) edge node {b} (em);
\end{tikzpicture}

\subsection{Teil b}

\subsubsection{(i)}

\begin{proof}
  Definiere einen NFA $M = (\{ a, b \}, \{ x, y, z_{1}, \dots, z_{n - 1} \}, \delta, \{ x \}, \{ z_{n - 1} \})$ und seine Übergangsfunktion partiell durch
  \begin{align*}
    \delta(x, a) & = \{ x, y \}\\
    \delta(x, b) & = \{ x \}\\
    \delta(y, a) & = \{ z_{1} \}\\
    \delta(y, b) & = \{ z_{1} \}\\
    \delta(z_{i}, a) & = \{ z_{i + 1} \} \forall i \in [1, n - 2]\\
    \delta(z_{i}, b) & = \{ z_{i + 1} \} \forall i \in [1, n - 2]
  \end{align*}
  Wie direkt zu sehen, hat $M$ $n + 1$ Zustände.

  Sei $w \in L_{n}$.
  Sei $w = pq$, wobei $p = (a + b)^{*}a$ und $q = (a + b)^{n - 1}$.
  \begin{equation}
    \hat{\delta}(\{ x \}, w) = \hat{\delta}(\hat{\delta}(\{ x \}, p), q)
  \end{equation}
  Da alle Übergänge von $x$ auch wieder $x$ enthalten, wird $\hat{\delta}(\{ x \}, (a + b)^{*})$ auf jeden Fall $x$ enthalten.
  Da $p$ auf $a$ endet, enthält $\hat{\delta}(\{ x \}, p)$ $y$.
  Es bleibt also zu zeigen, dass $\hat{\delta}(\{ y \}, q)$ $z_{n - 1}$ enthält.
  Da $\hat{\delta}(\{ y \}, (a + b)) = \{ z_{1} \}$ und $\hat{\delta}(\{ z_{i} \}, (a + b)) = z_{i + 1}$ für $i \in [1, n - 2]$, ist es wahr, weil $|q| = n - 1$.
  Also ist $w \in L(M)$.

  Sei $w \in L(M)$.
  Sei $w = w_{1}w_{2}\dots w_{m}$ für ein $m \in \mathbb{N}$.
  Sei $W_{k} = w_{1}w_{2} \dots w_{k}$.
  Also ist $\hat{\delta}(x, w) = z_{n - 1}$.
  Also
  \begin{equation}
    \delta(\hat{\delta}(x, W_{m - 1}), w_{m}) = z_{n - 1}
  \end{equation}
  Also
  \begin{equation}
    \hat{\delta}(x, W_{m - 1}) = z_{n - 2}
  \end{equation}
  Nach dem gleich Schema erhält man
  \begin{equation}
    \hat{\delta}(x, W_{m - n + 1}) = z_{1}
  \end{equation}
  \begin{equation}
    \hat{\delta}(x, W_{m - n}) = y
  \end{equation}
  Deshalb muss $w_{m - n} = a$ sein.
  Der $n$-t letzte Buchstabe ist also ein $a$ und $w \in L_{n}$.

  Also $L_{n} = L(M)$.
\end{proof}

\subsubsection{(ii)}

\begin{proof}
  Ich werde es zeigen, indem ich den in (i) beschriebenen NFA konstruiere und dann diesen nach Vorlesung zu einem DFA umwandle.

  \begin{tikzpicture}[shorten >=1pt,node distance=2cm,on grid,auto]
    \node[state,initial] (x) {$x$};
    \node[state] (y) [right=of x] {$y$};
    \node[state] (z_1) [right=of y] {$z_{1}$};
    \node[state,accepting] (z_2) [right=of z_1] {$z_{2}$};
    \path[->]
    (x) edge [loop above] node {a} (x)
    (x) edge node {a} (y)
    (x) edge [loop below] node {b} (x)
    (y) edge [bend left] node {a} (z_1)
    (y) edge [bend right] node {b} (z_1)
    (z_1) edge [bend left] node {a} (z_2)
    (z_1) edge [bend right] node {b} (z_2);
  \end{tikzpicture}

  \begin{tikzpicture}[shorten >= 1pt, node distance=3cm, on grid, auto]
    \node[state,initial] (x) {$\{ x \}$};
    \node[state] (xy) [right=of x] {$\{ x, y \}$};
    \node[state] (xz1) [above right=of xy] {$\{ x, z_{1} \}$};
    \node[state,accepting] (xz2) [left=of xz1] {$\{ x, z_{2} \}$};
    \node[state] (xyz1) [below right=of xy] {$\{ x, y, z_{1} \}$};
    \node[state,accepting] (xyz2) [below right=of xz1] {$\{ x, y, z_{2} \}$};
    \node[state,accepting] (xyz1z2) [below right=of xyz1] {$\{ x, y, z_{1} z_{2} \}$};
    \node[state,accepting] (xz1z2) [below left=of xyz1] {$\{ x, z_{1} z_{2} \}$};
    \path[->]
    (x) edge [loop below] node {b} (x)
    (x) edge node {a} (xy)
    (xy) edge node {a} (xyz1)
    (xy) edge node {b} (xz1)
    (xz1) edge [bend left] node {a} (xyz2)
    (xz1) edge node {b} (xz2)
    (xyz1) edge node {a} (xyz1z2)
    (xyz1) edge node {b} (xz1z2)
    (xz2) edge node {a} (xy)
    (xz2) edge node {b} (x)
    (xyz2) edge node {a} (xyz1)
    (xyz2) edge [bend left] node {b} (xz1)
    (xyz1z2) edge [loop right] node {a} (xyz1z2)
    (xyz1z2) edge node {b} (xz1z2)
    (xz1z2) edge [bend left] node {a} (xyz2)
    (xz1z2) edge [bend left] node {b} (xz2);
  \end{tikzpicture}

  Dieser DFA ist nach Vorlesung äquivalent und hat 8 Zustände.
\end{proof}

\end{document}