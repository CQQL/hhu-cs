\documentclass[10pt,a4paper]{article}
\usepackage[utf8]{inputenc}
\usepackage[german]{babel}
\usepackage{mathrsfs}
\usepackage{amsmath}
\usepackage{amsfonts}
\usepackage{amssymb}
\usepackage{amsthm}
\usepackage[left=2cm,right=2cm,top=2cm,bottom=2cm]{geometry}

\begin{document}

Info4 Blatt 1\\
Marten Lienen (2126759)\\
Gruppe 1 (Di 10.30 - 12.00)\\

\section{Aufgabe 1.1}

\subsection{Teil a}

\subsubsection{i}

\begin{equation}
  \{A, B\}
\end{equation}

\subsubsection{ii}

\begin{equation}
  \{A, B, C\}, \{A, B, \neg C\}, \{A, \neg B, C\}, \{A, \neg B, \neg C\}, \{\neg A, \neg B, \neg C\}
\end{equation}

\subsubsection{iii}

\begin{equation}
  \{\neg A, B\}, \{ \neg A, \neg B \}
\end{equation}

\subsection{Teil b}

\subsubsection{i}

\begin{equation}
  (A \Rightarrow B) \land (C \land \neg A) \equiv (\neg A \lor B) \land C \land (\neg A)
\end{equation}

\subsubsection{ii}

\begin{align*}
  (A \Leftrightarrow (B \land \neg C)) \land (\neg B \Rightarrow A) & \equiv ((A \land (B \land \neg C)) \lor (\neg A \land \neg (B \land \neg C))) \land (B \lor A)\\
  & \equiv ((A \land B \land \neg C) \lor (\neg A \land (\neg B \lor C))) \land (B \lor A)\\
  & \equiv (((A \land B \land \neg C) \lor \neg A) \land ((A \land B \land \neg C) \lor (\neg B \lor C))) \land (B \lor A)\\
  & \equiv ((A \lor \neg A) \land (B \lor \neg A) \land (\neg C \lor \neg A) \land ((A \land B \land \neg C) \lor (\neg B \lor C))) \land (B \lor A)\\
  & \equiv ((A \lor \neg A) \land (B \lor \neg A) \land (\neg C \lor \neg A) \land (A \lor \neg B \lor C) \land (B \lor \neg B \lor C) \land (\neg C \lor \neg B \lor C)) \land (B \lor A)\\
  & \equiv (A \lor \neg A) \land (B \lor \neg A) \land (\neg C \lor \neg A) \land (A \lor \neg B \lor C) \land (B \lor \neg B \lor C) \land (\neg C \lor \neg B \lor C) \land (B \lor A)\\
  & \equiv (B \lor \neg A) \land (\neg C \lor \neg A) \land (A \lor \neg B \lor C) \land (B \lor \neg B \lor C) \land (\neg C \lor \neg B \lor C) \land (B \lor A)\\
\end{align*}

\subsection{Teil c}

\subsubsection{i}

\begin{align*}
  F & \equiv (A \Leftrightarrow \neg B) \lor (B \Rightarrow A)\\
  & \equiv (A \land \neg B) \lor (\neg A \land B) \lor \neg B \lor A\\
  & \equiv (A \land \neg B) \lor (\neg A \lor \neg B) \lor A\\
  & \equiv (A \land \neg B) \lor \neg B \lor \neg A \lor A\\
  & \equiv T \equiv A \lor \neg A \equiv G
\end{align*}

\subsubsection{ii}

$G$ ist weder äquivalent zu $F$ noch eine Schlussfolgerung von $F$, weil $F$ ein Modell ($\{A, \neg B\}$) hat, aber $G$ keins.

\subsubsection{iii}

\begin{align*}
  F & \equiv A \land \neg A \equiv F \textit{ (false)}\\
  & \equiv \neg A \land \neg B \land B\\
  & \equiv (\neg B \land \neg A) \land B\\
  & \equiv (\neg B \lor A) \land \neg A \land B\\
  & \equiv (\neg B \lor A) \land (\neg A \land B)\\
  & \equiv ((\neg B \lor A) \land (\neg A \land B)) \lor (B \land \neg B \land \neg A)\\
  & \equiv ((\neg B \lor A) \land (\neg A \land B)) \lor (B \land (\neg A \land \neg B))\\
  & \equiv ((\neg B \lor A) \land (\neg A \land B)) \lor (B \land \neg A \land (A \lor \neg B))\\
  & \equiv ((\neg B \lor A) \land (\neg A \land B)) \lor (\neg (\neg B \lor A) \land \neg (\neg A \land B))\\
  & \equiv ((B \Rightarrow A) \land (\neg A \land B)) \lor (\neg (B \Rightarrow A) \land \neg (\neg A \land B))\\
  & \equiv (B \Rightarrow A) \Leftrightarrow (\neg A \land B) \equiv G
\end{align*}

\section{Aufgabe 1.2}

Meine Vermutung ist, dass die Studenten schuldig sind, weshalb als Anfrage die Klausel $\neg S$ benutzt wird.
Die dritte Klausel ist eine abgeschwächte Version von (i), nämlich dass mindestens eine Gruppe schuldig ist.
Die Klauseln sind
\begin{equation}
  \{ \{ \neg T \Rightarrow \neg M \}, \{ \neg S \Rightarrow \neg T \}, \{ T \lor M \lor S \}, \{ \neg S \}\}
\end{equation}
In KNF
\begin{equation}
  \{ \{ T \lor \neg M \}, \{ S \lor \neg T \}, \{ T \lor M \lor S \}, \{ \neg S \}\}
\end{equation}
Resolution mit 2 und 4 ergibt
\begin{equation}
  \{ \{ T \lor \neg M \}, \{ S \lor \neg T \}, \{ T \lor M \lor S \}, \{ \neg S \}, \{ \neg T \}\}
\end{equation}
Resolution mit 1 und 5 ergibt
\begin{equation}
  \{ \{ T \lor \neg M \}, \{ S \lor \neg T \}, \{ T \lor M \lor S \}, \{ \neg S \}, \{ \neg T \}, \{ \neg M \}\}
\end{equation}
Resolution 3 mit 4
\begin{equation}
  \{ \{ T \lor \neg M \}, \{ S \lor \neg T \}, \{ T \lor M \lor S \}, \{ \neg S \}, \{ \neg T \}, \{ \neg M \}, \{ T \lor M \} \}
\end{equation}
Resolution 7 mit 5
\begin{equation}
  \{ \{ T \lor \neg M \}, \{ S \lor \neg T \}, \{ T \lor M \lor S \}, \{ \neg S \}, \{ \neg T \}, \{ \neg M \}, \{ T \lor M \}, \{ M \} \}
\end{equation}
Resolution 8 mit 6
\begin{equation}
  \{ \{ T \lor \neg M \}, \{ S \lor \neg T \}, \{ T \lor M \lor S \}, \{ \neg S \}, \{ \neg T \}, \{ \neg M \}, \{ T \lor M \}, \{ M \}, \{  \} \}
\end{equation}
Es wurde ein Widerspruch gefunden und $\neg S$ kann nicht wahr sein.
Also muss $S$ wahr sein und die Schüler sind schuldig.

\section{Aufgabe 1.3}

\subsection{Teil a}

\begin{equation}
  \forall r \cdot r \in S \leftrightarrow T \Rightarrow S <\| r = \{ x \mapsto y \mid (x \mapsto y) \in r \land x \in S \}
\end{equation}

\subsection{Teil b}

\begin{equation}
  \forall r \cdot r \in S \leftrightarrow T \Rightarrow S <<\| r = \{ x \mapsto y \mid (x \mapsto y) \in r \land x \not\in S \}
\end{equation}

\subsection{Teil c}

\begin{equation}
  \forall r \cdot r \in S \leftrightarrow T \Rightarrow r \|> T = \{ x \mapsto y \mid (x \mapsto y) \in r \land y \in T \}
\end{equation}

\subsection{Teil d}

\begin{equation}
  \forall r \cdot r \in S \leftrightarrow T \Rightarrow r \|>> T = \{ x \mapsto y \mid (x \mapsto y) \in r \land y \not\in T \}
\end{equation}

\subsection{Teil e}

\begin{equation}
  \forall r \cdot r \in S \leftrightarrow T \Rightarrow r[S] = \{ y \mid (x \mapsto y) \in r \land x \in S \}
\end{equation}

\subsection{Teil f}

\begin{equation}
  \forall r \cdot r \in S \leftrightarrow T \Rightarrow r\sim = \{ y \mapsto x \mid (x \mapsto y) \in r \}
\end{equation}

\subsection{Teil g}

\begin{equation}
  \forall r_{1}, r_{2} \cdot (r_{1}, r_{2}) \in (S \leftrightarrow T)^{2} \Rightarrow r_{1} <+ r_{2} = r_{2} \cup (dom(r_{2}) <<\| r_{1})
\end{equation}

\section{Aufgabe 1.4}

\subsection{Teil a}

\subsubsection{i}

\begin{equation}
  \{ ian \} <\| eats = \{ ian \mapsto eggs, ian \mapsto cheese, ian \mapsto pizza \}
\end{equation}

\subsubsection{ii}

\begin{equation}
  \{ jim \} <<\| eats = \{ ian \mapsto eggs, ian \mapsto cheese, ian \mapsto pizza, ken \mapsto pizza, lisa \mapsto cheese, lisa \mapsto salad, lisa \mapsto pizza \}
\end{equation}

\subsubsection{iii}

\begin{equation}
  dom(eats |> \{ eggs \}) = \{ ian, jim \}
\end{equation}

\subsubsection{iv}

\begin{equation}
  eats[\{ ian, lisa \}] = \{ eggs, cheese, pizza, salad \}
\end{equation}

\subsubsection{v}

\begin{equation}
  eats\sim[\{ cheese, eggs \}] = \{ ian, jim, lisa \}
\end{equation}

\subsubsection{vi}

\begin{equation}
  eats;cost = \{ ian \mapsto cheap, ian \mapsto expensive, jim \mapsto cheap, ken \mapsto expensive, lisa \mapsto cheap, lisa \mapsto expensive \}
\end{equation}

\subsubsection{vii}

\begin{equation}
  eats;(cost \|>> \{ expensive \}) = \{ ian \mapsto expensive, ken \mapsto expensive, lisa \mapsto expensive \}
\end{equation}

\subsubsection{viii}

\begin{equation}
  eats <+ \{ lisa \mapsto steak \} = \{ ian \mapsto eggs, ian \mapsto cheese, ian \mapsto pizza, jim \mapsto eggs, jim \mapsto salad, ken \mapsto pizza, lisa \mapsto steak \}
\end{equation}

\subsection{Teil b}

\subsubsection{entweder eggs oder pizza}

\begin{equation}
  ((eats \|> \{pizza\}) <+ (eats \|> \{eggs\}) \|> \{pizza\}) <+ ((eats \|> \{eggs\}) <+ (eats \|> \{pizza\}) \|> \{eggs\})
\end{equation}

\subsubsection{cheese und pizza}

\begin{equation}
  dom(dom(eats \|> \{pizza\}) <\| eats \|> \{cheese\})
\end{equation}

\end{document}