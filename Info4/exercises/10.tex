\documentclass[10pt,a4paper]{article}
\usepackage[utf8]{inputenc}
\usepackage[german]{babel}
\usepackage{mathrsfs}
\usepackage{amsmath}
\usepackage{amsfonts}
\usepackage{amssymb}
\usepackage{amsthm}
\usepackage[left=2cm,right=2cm,top=2cm,bottom=2cm]{geometry}
\usepackage{listings}

\DeclareMathOperator{\id}{id}

\begin{document}

\section{Aufgabe 10.1}

\subsection{Teil a}

\begin{proof}
  Sei $k = 1$.
  Dann ist
  \begin{lstlisting}
    x0 := 0; x1 := n; x2 := 1;
    LOOP x1 DO
      x3 := g(x2);

      LOOP x0 DO
        x4 := 0;

        LOOP x3 DO
          x4 := 1;
        END

        LOOP x4 DO
          x3 := x3 - 1;
        END
      END

      LOOP x3 DO
        x0 := x0 + 1;
      END

      x2 = x2 + 1;
    END
  \end{lstlisting}
  Also ist $f$ für $1$-stellige Funktionen LOOP-berechenbar.

  Sei $f$ für $k$-stellige Funktionen LOOP-berechenbar.
  Man betrachte ein $k + 1$-stelliges $g$.
  Sei $h_{x}$ die $k$-stellige Funktion $(x_{1}, \dots, x_{k}) \rightarrow g(x, x_{1}, \dots, x_{k})$.
  Sei $f_{x} = \max \{ h_{x}(n_{1}, \dots, n_{k}) \mid 1 \le n_{1}, dots, n_{k}, \le n \}$.
  $f_{x}$ ist also LOOP-berechenbar.
  \begin{lstlisting}
    x0 := 0; x1 := n; x2 := 1;
    LOOP x1 DO
      x3 := f_x2(n);

      LOOP x0 DO
        x4 := 0;

        LOOP x3 DO
          x4 := 1;
        END

        LOOP x4 DO
          x3 := x3 - 1;
        END
      END

      LOOP x3 DO
        x0 := x0 + 1;
      END

      x2 = x2 + 1;
    END
  \end{lstlisting}
  Also ist $f$ auch für ein $k + 1$-stelliges $g$ LOOP-berechenbar.
\end{proof}

\subsection{Teil b}

\begin{lstlisting}
  x1 := m; x2 := n;

  A: x3 := x1 - x2; // GOTO-berechenbar
  IF x3 = 0 THEN GOTO H;

  x3 := x1;
  x4 := x2;

  C: x3 := x3 - 1;
  x4 := x4 - 1;
  IF x3 = 0 THEN GOTO B;
  IF x4 = 0 THEN GOTO D;
  GOTO C;

  D: x2 := x2 - x1;
  GOTO A;
  B: x2 := x2 - x1;
  GOTO A;

  H: x0 := x1;
  HALT;
\end{lstlisting}

\section{Aufgabe 10.2}

\subsection{Teil a}

\subsubsection{$md(x, y)$}

\begin{equation}
  md(0, y) = 0(y)
\end{equation}
\begin{equation}
  md(n + 1, y) = h(n, md(n, y), y)
\end{equation}

\subsubsection{$md1(x, y)$}

\begin{equation}
  md1(0, y) = id_{1}^{1}(y)
\end{equation}
\begin{equation}
  md1(n + 1, y) = pred(id_{2}^{3}(n, md1(n, y), y))
\end{equation}

\subsubsection{$S(x)$}

\begin{equation}
  S(0) = 0
\end{equation}
\begin{equation}
  S(n + 1) = (1 \circ \id_{1}^{2})(n, S(n))
\end{equation}

\subsubsection{$eq(x, y)$}

\begin{equation}
  h(0) = 1
\end{equation}
\begin{equation}
  h(n + 1) = (0 \circ \id_{1}^{2})(n, h(n))
\end{equation}
\begin{equation}
  eq(0, y) = h(y)
\end{equation}
\begin{equation}
  eq(n + 1, y) = g(n, eq(n, y), y)
\end{equation}

\subsection{Teil b}

\begin{equation}
  div(0, y) = 1
\end{equation}
\begin{equation}
  div(n + 1, y) =
\end{equation}

\section{Aufgabe 10.3}

\subsection{Teil a}

\begin{equation}
  \min(x, y) = \begin{cases}
    x & \textit{wenn $x < y$}\\
    y & \textit{sonst}
  \end{cases}
\end{equation}
\begin{equation}
  \max(x, y) = \begin{cases}
    x & \textit{wenn $x > y$}\\
    y & \textit{sonst}
  \end{cases}
\end{equation}

\subsection{Teil b}

\subsection{Teil c}

\begin{equation}
  f(0) = 0
\end{equation}
\begin{equation}
  f(n + 1) = \id^{2}_{1}(n, f(n))
\end{equation}
Es ist die modifizierte Vorgängerfunktion.

\section{Aufgabe 10.4}

\subsection{Teil a}

\begin{equation}
  \mu \, eq (y) = \begin{cases}
    0 & \textit{wenn $y = 0$}\\
    undefiniert & \textit{sonst}
  \end{cases}
\end{equation}

\subsection{Teil b}

Potenzierung ist primitiv rekursiv.
\begin{equation}
  lq(x, y) = eq(2^{x}, y)
\end{equation}
\begin{equation}
  l(y) = (\mu lq)(y)
\end{equation}

\subsection{Teil c}

\begin{equation}
  \mu\, lt(y) = \begin{cases}
    undefiniert & \textit{wenn $y = 0$}\\
    0 & \textit{sonst}
  \end{cases}
\end{equation}
\begin{equation}
  \mu\, gt(y) = undefiniert
\end{equation}

\end{document}