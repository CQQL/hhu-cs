\documentclass[a4paper,10pt]{article}
\usepackage[utf8]{inputenc}
\usepackage{amsmath}
\usepackage{amssymb}
\usepackage{amsthm}

\title{Relationen}
\author{Marten Lienen}

\begin{document}

\maketitle

\section{Definition}

Eine Relation auf der Menge M ist eine Teilmenge R von MxM.

\subsection{Beispiel}

M ist eine Menge.

\begin{equation}
 R = \{ (x, y) \in MxN | x = y \}
\end{equation}

\subsection{Beispiel}

M = $\mathbb{N}$.

\begin{equation}
 R = \{ (x, y) \in \mathbb{N}x\mathbb{N} | x \le y \}
\end{equation}

\subsection{Beispiel}

M ist eine Menge, P(M) ist ihre Potenzmenge.

\begin{equation}
 R = \{ (A, B) \in P(M)xP(M) | A \cap B = \phi \}
\end{equation}

\section{Definition}

Sei M eine Menge, R eine Relation auf M.

0. (x, y) \in R, dann schreibt man x ~_R y
1. R heißt reflexiv, wenn $x ~_R x \forall x \in M$
2. R heißt symmetrisch, wenn $(x ~_R y \Rightarrow y ~_R x) \forall x, y \in M$
3. R heißt antisymmetrisch, wenn $x ~_R y \land y ~_R x) \Rightarrow x = y$
4. R heißt transitiv, wenn $\forall x, y, z \in M: (x ~_R y \land y ~_R  z) \Rightarrow x ~_R z$

\subsection{Beispiele}

M ist eine Menge. R ist die Gleichheitsrelation. R ist reflexiv, symmetrisch, antisymmetrisch und transitiv.

M = $\mathbb{N}$, $R = \text{Ordnungsrelation} auf \mathbb{N} = \{(x, y) \in \mathbb{N}^2 | x \le y\}$.
R ist reflexiv: $\x \le y \forall x$.
R ist nicht symmetrisch: $0 \le 1$, aber nicht $ 1 \le 0$.
R ist antisymmetrisch: 
R ist transitiv:

M ist eine Menge, $P(M) = \text{Potenzmenge}$.
$R = \{(A, B) \in P(M)^2 | A \cap B = \phi \}$

R ist nicht reflexiv.
R ist symmetrisch.
R ist nicht antisymmetrisch.
R ist nicht transitiv.

\section{Definition}

M ist eine Menge, R ist eine Relation über M.

R heißt Ordnungsrelation, wenn R reflexiv, antisymmetrisch und transitiv ist.
R heißt Äquivalenzrelation, wenn R reflexiv, symmetrisch und transitiv ist.

\section{Lemma}

Sei M = $\mathbb{Z}$.

$R = \{ (x, y) \in \mathbb{Z}^2 | x - y ist gerade \}$
ist eine Äquivalenzrelation.

\section{Satz}

$f: M \mapsto N$ ist eine Abbildung. $M_1, M_2 \subset M; N_1, N_2 \subset N$.
Dann gilt
1. $M_1 \subset M_2 \Rightarrow f(M_1) \subset f(M_2); N_1 \subset N_2 \Rightarrow f^{-1}(N_1) \subset f^{-1}(N_2)$
2. $f(M_1 \cup M_2) = f(M_1) \cup f(M_2)$
$f^{-1}(N_1 \cup N_2) = f^{-1}(N_1) \cup f^{-1}(N_2)$
3. $f(M_1 \cap M_2) \subset f(M_1) \cap f(M_2)$
$f^·{-1}(N_1 \cap N_2) = f^{-1}(N_1) \cap f^{-1}(N_2)$
4. $f(M_1) \ f(M_2) = \subset f(M_1 \ M_2)$
$f^{-1}(N_1 \ N_2) = f^{-1}(N_1) \ f^{-1}(N_2)$

Beweis siehe Übungsblatt.

\section{Satz}

Sei $f: M \mapsto N$ eine Abbildung.
Sei $R = \{ (x, y) \in M^2 | f(x) = f(y) \}$.
R ist eine Äquivalenzrelation.

Beweis:

\begin{equation}
 R \text{ist reflexiv} \Leftrightarrow \forall x \in M \text{gilt} x ~_R x \Leftrightarrow \forall x \in M \text{gilt} (x, y) \in R
 \Leftrightarrow \forall x \in M \text{gilt} f(x) = f(x)
\end{equation}

\begin{equation}
 R \text{ist symmetrisch} \Leftrightarrow
\end{equation}



\end{document}
