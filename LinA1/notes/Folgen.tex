\documentclass[a4paper,10pt]{article}
\usepackage[utf8]{inputenc}
\usepackage{amssymb}
\usepackage{amsmath}
\usepackage{amsthm}
\usepackage[german]{babel}

\title{Folgen und ihr Grenzwert}
\author{Marten Lienen}

\newtheorem{definition}{Definition}
\newtheorem{satz}{Satz}
\newtheorem{example}{Beispiel}

\begin{document}

\maketitle

\section*{Ausblick}

Wir werden sehen: Ist $n \in \mathbb{N}$, so gibt es ein $x \in \mathbb{R}$ mit $x \ge 0$ und $x^n = a$.
Dieses $x$ ist eindeutig bestimmt. Schreibe

\begin{equation}
 x := a^{\frac{1}{n}} := \sqrt[n]{a}
\end{equation}

\section{Folgen und ihr Grenzwert}

\begin{definition}
 Sind $X$ und $Y$ zwei Mengen, so ist eine Abbildung $f: X \mapsto Y$ eine Vorschrift, die jedem Element $x \in X$ ein Element $f(x)$ zuordnet.
\end{definition}

\begin{definition}
 Ist $Y$ eine Menge, so ist eine Folge in Y eine Abbildung $a: \mathbb{N} \mapsto Y$.
 Statt $a(n)$ schreibt man meist $a_n$ und spricht von der Folge $(a_n)$ statt von der Folge $a$.
 Statt ``Folge in $\mathbb{R}$'' sagen wir einfach Folge.
\end{definition}

\begin{definition}
 Sei $a_n$ eine Folge und $x_0 \in \mathbb{R}$.
 Die Folge $a_n$ konvergiert gegen $x_0$, falls gilt: Zu jedem $\epsilon > 0$ gibt es ein $N \in \mathbb{N}$, sodass $|a_n - x_0| < \epsilon$ für alle $n \ge N$, also dass $a_n \in ]x_0 - \epsilon, x_0 + \epsilon[$ für alle $n \ge \mathbb{N}$.
 Man nennt dann $_0$ Grenzwert oder Limes der Folge $a_n$ und schreibt dafür $\lim_{n \rightarrow \infty} a_n = x_0$. 
\end{definition}

\begin{definition}
 Die Folge $a_n$ heißt konvergent, wenn es ein $x_0 \in \mathbb{R}$ gibt, so dass $a_n$ gegen $x_0$ konvergiert.
 Andernfalls heißt $a_n$ divergent.
\end{definition}

\begin{satz}
 Eine Folge besitzt höchstens einen Grenzwert.
\end{satz}

\begin{example}
 Sei $a \in \mathbb{R}$. Betrachte die konstante Folge $a_n$ mit $a_m := a$ für alle $n \in \mathbb{N}$.
 Damit ist $a_n$ konvergent und $\lim_{n \rightarrow \infty} a_n = a$.
\end{example}

\begin{example}
 Sei $a_n := \frac{1}{n}$ für alle $n \in \mathbb{N}$. Dann ist $a_n$ konvergent und $lim_{n \rightarrow \infty} a_n = 0$.
 
 \begin{proof}
  Sei $\epsilon > 0$.
  Nach §1, Satz 2 gibt es ein $M \in \mathbb{N}$ mit $\frac{1}{M} \le \epsilon$.
  Sei $N := M + 1$.
  Ist $n \ge N$, so ist $0 < \frac{1}{n} \le \frac{1}{N} < \frac{1}{M} \le \epsilon$, also ist der Abstand $|\frac{1}{n} - 0| = |\frac{1}{n}| = \frac{1}{n} < \epsilon$.
 \end{proof}
\end{example}

\begin{example}
 Sei $a_n := (-1)^n$.
 Dann ist die Folge $a_n$ divergent.
 
 \begin{proof}
  Angenommen, es sei $\lim_{n \rightarrow \infty} a_n = x_0$.
  Dann gibt es ein $N \in \mathbb{N}$ mit der Eigenschaft $|a_n - x_0| < 1$ für alle $n \ge N$.
  Dann ist $2 = |a_{N + 1} - a_N| = |(a_{N + 1} - x_0) - (a_N - x_0)| \le |a_{N + 1} - x_0| + |a_N - x_0| < 1 + 1 = 2$ Widerspruch.
 \end{proof}
\end{example}

\begin{example}
 $\lim_{n \rightarrow \infty} \frac{n}{2^n} = 0$.
 
 \begin{proof}
  Zeige zunächst, dass $n^2 \le 2^n$ für alle $n \ge 4$ mit vollständiger Induktion.
  
  Für $n = 4$ gilt es.
  
  \begin{align*}
   & (n + 1)^2 = n^2 + 2n + 1
   & \le n^2 + 2n + n = n^2 + 3n \le n^2 + n*n = 2n^2 \le 2^{n + 1}
  \end{align*}

  Also gilt für $n \ge 4$
  \begin{align*}
   \frac{n}{2^n} = \frac{1}{n} * \frac{n^2}{2^n} \le \frac{1}{n} * 1 = \frac{1}{n}
  \end{align*}
 \end{proof}
\end{example}

\begin{satz}
 Eine Folge besitzt höchstens einen Grenzwert.
\end{satz}

\begin{proof}
 Die Folge $(a_n)$ besitze die beiden Grenzwerte $a$ und $b$ mit $a \ne b$.
 Sei $\epsilon := \frac{1}{2}|a - b| > 0$.
 Es gibt ein $N \in \mathbb{N}$ mit $|a_n - a| < \epsilon$ für alle $n \ge N$.
 Es gibt ein $M \in \mathbb{N}$ mit $|a_n - b| < \epsilon$ für alle $n \ge M$.
 Sei $P := max \{N, M\}$.
 $|a - b| \le |(a - a_n) - (b - a_n)| \le |a - a_n| + |b - a_n| < 2\epsilon = |a - b|$ Widerspruch.
\end{proof}

\end{document}
