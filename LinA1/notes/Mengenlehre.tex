\documentclass{article}
\usepackage[utf8]{inputenc}
\usepackage{amsmath}
\usepackage{amssymb}

\title{Mengenlehre}
\author{Marten Lienen}

\begin{document}

\maketitle

\section{Definition}

Eine Menge ist eine Zusammenfassung von Elementen zu einem neuen Objekt.

\section{Axiom 1}

$M = N$ gdw. $m \in M \Longrightarrow m \in N$ und $n \in N \Longrightarrow n \in M$

\section{Notation}
$\{\}$: Mengenklammern, z.B.: $\{0, 1, 2\}$\\
$\in$: ist Element von\\
$\notin$: ist nicht Element von\\
$\forall$: für jedes Element, z.B.: $\forall n \in N$ gilt $n > 5$\\
$\exists$: es gibt mindestens ein Element, so dass \dots \\
$\Longrightarrow$: daraus folgt, dass \dots \\

\section{Definition: Untermenge}

$N \subset M$ gdw. $x \in N \Longrightarrow x \in M$

\section{Axiom: Aussonderungsaxiom}

Zu jeder Menge M und zu jeder Eigenschaft P gibt es eine Teilmenge N, die gerade aus den Elementen von M
mit dieser Eigenschaft besteht.

\subsection{Beispiel}

Sei P eine Eigenschaft: $P(n) = n > 5$\\
Dann konstruieren folgende Mengenklammern die Menge aller Zahlen aus $\mathbb{N}$, die die Eigenschaft P
haben: $\{ n \in \mathbb{N} | P(n) \}$

\section{Leere Menge}

Es gibt eine leere Menge.

\subsection{Beweis}

Angenommen es gibt eine Menge M, dann kann man $\phi$ folgendermaßen als leere Menge definieren:
\begin{equation}
 \phi = \{ x \in M | x \ne x \}
\end{equation}

\subsection{Bemerkung}

Sei M eine Menge, dann gilt $\phi \subset M$, weil $x \in \phi \Longrightarrow x \in M$

\section{Vereinigungsaxiom}

Seien M und N Mengen, dann gibt es eine Menge $M \cup N$ (Die Vereinigung von M und N), sodass
\begin{equation}
 \forall m \in M \Longrightarrow m \in (M \cup N)
 \quad \text{und} \quad
 \forall n \in N \Longrightarrow n \in (M \cup N)
\end{equation}

Ist I eine Indexmenge und $(M_i)_{i \in I}$ eine Mengenfamilie, dann gibt es eine Menge $\cup_{i \in I} M_i$,
die alle Elemente aller Mengen der Mengenfamilie enthält.

\section{Satz}

Seien M undN zwei Mengen, dann gibt es eine Menge $M \cap N$, die genau die Elemente enthält, die in M und N enthalten sind.
\begin{equation}
 M \cap N = \{ x \in M \cup N | x \in M \land x \in N \}
\end{equation}

Ist I eine Indexmenge und $(M_i){i \in I}$ eine Mengenfamilie dann gibt es $\cap_{i \in I}M_i$, die genau die Elemente enthält,
die in allen Mengen der Mengenfamilie enthalten sind.

\section{Satz}

Seien M, N und O Mengen, dann gilt
\begin{align}
 M \cup M & = M & \text{(reflexiv)}\\
 M \cup N & = N \cup M & \text{(kommutativ)}\\
 (M \cup N) \cup O & = M \cup (N \cup O) & \text{(assoziativ)}
\end{align}

\section{Satz: Distributivgesetz}

Seien M, N und O Mengen
\begin{equation}
 M \cap (N \cup O) = (M \cap N) \cup (M \cap O)
\end{equation}

\section{Definition}

Das Komplement von N in M ist 
\begin{equation}
 M \setminus N = \{ x \in M | x \notin N \}
\end{equation}

\section{Satz}

\end{document}
