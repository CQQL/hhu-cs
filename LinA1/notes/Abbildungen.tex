\documentclass[a4paper,10pt]{article}
\usepackage[utf8]{inputenc}
\usepackage{amsmath}
\usepackage{amssymb}
\usepackage{amsthm}

\title{}
\author{Marten Lienen}

\begin{document}

\maketitle

\section{Abbildungen}

\begin{equation}
 C \subset A\\
 f(C) := \{ f(a): a \in C \}
\end{equation}

\section{Vollständige Induktion}

$\mathbb{N}$ ist mit der üblichen Ordnungsrelation wohlgeordnet, d.h. dass es in jeder Untermenge von $\mathbb{N}$
immer ein kleinstes Element gibt.

\section{Fibonacci-Zahlen}

\begin{equation}
 F_n = \frac{1}{\sqrt{5}} * \left(\left(\frac{1 + \sqrt{5}}{2}\right)^n - \left(\frac{1 - \sqrt{5}}{2}\right)^n\right)
\end{equation}

\section{Definition}

$f: M \Rightarrow N$\\
M heißt Defitionsbereich von f.\\
N heißt Wertebereich von f.

\section{Satz}

$f: M \Rightarrow N$\\
$f$ ist bijektiv $Leftrightarrow$ $\exists g: N \mapsto M f°g = id_N \land g°f = id_M$

Die Abbildung g ist eindeutig definiert und heißt die Umkehrabbildung von f: $f^{-1}.$

Es gibt die Umkehrabbildung $f^{-1}$ nur dann, wenn $f$ bijektiv ist.

\section{Satz}

Sei $f: M \mapsto N$.\\
1. $f$ ist injektiv $\Leftrightarrow$ $\exists g: N \mapsto M$ mit $g°f = id_M$\\
2. $f$ ist surjektiv $Leftrightarrow$ $\exists h: N \mapsto M$ mit $f°g = id_N$\\
3. Falls f bijektiv ist, dann sind g und h1 gleich die Umkehrabbildung $f^{-1}$.

\section{Satz}

Seien $f: M \mapsto N, g: N \mapsto O$ Abbildungen.\\
1. f injektiv, g injektiv \Rightarrow g°f injektiv\\
2. f surjektiv, g surjektiv \Rightarrow g°f surjektiv\\
3. f bijektiv, g bijektiv \Rightarrow g°f bijektiv

\begin{proof}
 Annahme f, g sind injektiv.\\
 Zu zeigen: g°f ist injektiv für $x, y \in M$\\
 
 \begin{equation}
  g°f: M \mapsto O
  g°f(x) = g°f(y) \Leftrightarrow g(f(n)) = g(f(y)) \Rightarrow^{\text{g injektiv}} f(x) = f(y) \Rightarrow^{\text{f injektiv}} x = y
  \Rightarrow g°f ist injektiv
 \end{equation}
\end{proof}

\section{Definition}

Der Graph von $f: M \mapsto N$ ist die Teilmenge von MxN:\\
$Graph(f) = \Gamma_f = \{ (x, y) \in MxN \with y = f(x) \}$ 

\section{Definition}

Seien M und N zwei Mengen, dann ist die Menge von allen Abbildungen von M nach N ist $N^M = \{ f: M \mapsto N \}$.

\end{document}
