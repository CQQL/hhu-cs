\documentclass[a4paper,10pt]{article}
\usepackage[utf8]{inputenc}
\usepackage{amsmath}
\usepackage{amssymb}
\usepackage{amsthm}
\usepackage{stmaryrd}

\title{LinA1, Übungsblatt 11}
\author{Marten Lienen (2126759), Gruppe 1; Fabian Schmittmann (2083559), Gruppe 5}

\begin{document}

\maketitle

\section*{Übung 1}

\subsection*{Teil 1}

\begin{equation}
 f \begin{pmatrix}1\\0\\0\end{pmatrix} = \begin{pmatrix}1\\0\\0\end{pmatrix}
 \qquad f \begin{pmatrix}0\\1\\0\end{pmatrix} = \begin{pmatrix}2\\4\\0\end{pmatrix}
 \qquad f \begin{pmatrix}0\\0\\1\end{pmatrix} = \begin{pmatrix}3\\5\\6\end{pmatrix}
\end{equation}

\begin{equation}
 A = 
  \begin{pmatrix}
   1 & 2 & 3\\
   0 & 4 & 5\\
   0 & 0 & 6
  \end{pmatrix}
\end{equation}

\subsection*{Teil 2}

\begin{equation}
 f\begin{pmatrix}
   x\\
   y\\
   z
  \end{pmatrix}
  =
  \begin{pmatrix}
   0\\
   0\\
   0
  \end{pmatrix}
 \Leftrightarrow
 x = y = z = 0 \Rightarrow Basis(Ker(f)) = \emptyset
\end{equation}
Weil die Basis des Kerns von Dimension $0$ ist, folgt nach dem Rangsatz, dass $Rg(f) = 3$.
Weil die gesamte Zielmenge $\mathbb{R}^3$ ebenfalls von Dimension $3$ ist, ist $Im(f) = \mathbb{R}^3$.
Deshalb ist die kanonische Basis von $\mathbb{R}^3$ eine Basis von $Im(f)$.

\subsection*{Teil 3}

Weil $\mathcal{B}$ eine Basis ist, ist $(e_2, e_3)$ linear unabhängig.
\begin{equation}
 \begin{pmatrix}
  1\\
  1\\
  0
 \end{pmatrix}
 =
 x
 \begin{pmatrix}
  0\\
  1\\
  0
 \end{pmatrix}
 + y
 \begin{pmatrix}
  0\\
  0\\
  1
 \end{pmatrix}
 \Rightarrow
 1 = 0 \lightning
\end{equation}
$e_1 + e_2$ ist keine lineare Kombination von $e_2$ und $e_3$.
$(e_2, e_1 + e_2, e_3)$ ist also linear unabhängig, mit $dim \mathbb{R}^3$ Vektoren und somit eine Basis von $\mathbb{R}^3$.

\subsection*{Teil 4}

\begin{align*}
 id_{\mathbb{R}^3}\begin{pmatrix}1\\0\\0\end{pmatrix} & = (-1) * e_2 + 1 * (e_1 + e_2) + 0 * e_3\\
 id_{\mathbb{R}^3}\begin{pmatrix}0\\1\\0\end{pmatrix} & = 1 * e_2 + 0 * (e_1 + e_2) + 0 * e_3\\
 id_{\mathbb{R}^3}\begin{pmatrix}0\\0\\1\end{pmatrix} & = 0 * e_2 + 0 * (e_1 + e_2) + 1 * e_3\\
\end{align*}

\begin{equation}
 Mat_{\mathcal{B},\mathcal{B}'}(id_{\mathbb{R}^3}) =
  \begin{pmatrix}
   -1 & 1 & 0\\
   1 & 0 & 0\\
   0 & 0 & 1
  \end{pmatrix}
\end{equation}

\subsection*{Teil 5}

\begin{align*}
 id_{\mathbb{R}^3}\begin{pmatrix}0\\1\\0\end{pmatrix} & = 0 * e_1 + 1 * e_2 + 0 * e_3\\
 id_{\mathbb{R}^3}\begin{pmatrix}1\\1\\0\end{pmatrix} & = 1 * e_1 + 1 * e_2 + 0 * e_3\\
 id_{\mathbb{R}^3}\begin{pmatrix}0\\0\\1\end{pmatrix} & = 0 * e_1 + 0 * e_2 + 1 * e_3\\
\end{align*}

\begin{equation}
 Mat_{\mathcal{B}',\mathcal{B}}(id_{\mathbb{R}^3}) =
  \begin{pmatrix}
   0 & 1 & 0\\
   1 & 1 & 0\\
   0 & 0 & 1
  \end{pmatrix}
\end{equation}

\subsection*{Teil 6}

\begin{equation}
 B = Mat_{\mathcal{B},\mathcal{B}'}(id_{\mathbb{R}^3})AMat_{\mathcal{B}',\mathcal{B}}(id_{\mathbb{R}^3}) = 
  \begin{pmatrix}
   2 & 1 & 2\\
   2 & 3 & 3\\
   0 & 0 & 6
  \end{pmatrix}
\end{equation}

\section*{Übung 2}

\subsection*{Teil 1}

\subsection*{Teil 2}

\subsection*{Teil 3}

\subsection*{Teil 4}

\subsection*{Teil 5}

\section*{Übung 3}

\subsection*{Teil 1}

\subsection*{Teil 2}

\subsection*{Teil 3}

\subsection*{Teil 4}

\section*{Übung 4}

\end{document}
