\documentclass[a4paper,10pt]{article}
\usepackage[utf8]{inputenc}
\usepackage[german]{babel}
\usepackage{amsmath}
\usepackage{amssymb}
\usepackage{amsthm}
\usepackage{stmaryrd}

\title{LinA1, Übungsblatt 11}
\author{Marten Lienen (2126759), Gruppe 1; Fabian Schmittmann (2083559), Gruppe 5}

\begin{document}

\maketitle

\section*{Übung 1}

\subsection*{Teil 1}

\begin{equation}
 f \begin{pmatrix}1\\0\\0\end{pmatrix} = \begin{pmatrix}1\\0\\0\end{pmatrix}
 \qquad f \begin{pmatrix}0\\1\\0\end{pmatrix} = \begin{pmatrix}2\\4\\0\end{pmatrix}
 \qquad f \begin{pmatrix}0\\0\\1\end{pmatrix} = \begin{pmatrix}3\\5\\6\end{pmatrix}
\end{equation}

\begin{equation}
 A = 
  \begin{pmatrix}
   1 & 2 & 3\\
   0 & 4 & 5\\
   0 & 0 & 6
  \end{pmatrix}
\end{equation}

\subsection*{Teil 2}

\begin{equation}
 f\begin{pmatrix}
   x\\
   y\\
   z
  \end{pmatrix}
  =
  \begin{pmatrix}
   0\\
   0\\
   0
  \end{pmatrix}
 \Leftrightarrow
 x = y = z = 0 \Rightarrow Basis(Ker(f)) = \emptyset
\end{equation}
Weil die Basis des Kerns von Dimension $0$ ist, folgt nach dem Rangsatz, dass $Rg(f) = 3$.
Weil die gesamte Zielmenge $\mathbb{R}^3$ ebenfalls von Dimension $3$ ist, ist $Im(f) = \mathbb{R}^3$.
Deshalb ist die kanonische Basis von $\mathbb{R}^3$ eine Basis von $Im(f)$.

\subsection*{Teil 3}

Weil $\mathcal{B}$ eine Basis ist, ist $(e_2, e_3)$ linear unabhängig.
\begin{equation}
 \begin{pmatrix}
  1\\
  1\\
  0
 \end{pmatrix}
 =
 x
 \begin{pmatrix}
  0\\
  1\\
  0
 \end{pmatrix}
 + y
 \begin{pmatrix}
  0\\
  0\\
  1
 \end{pmatrix}
 \Rightarrow
 1 = 0 \lightning
\end{equation}
$e_1 + e_2$ ist keine lineare Kombination von $e_2$ und $e_3$.
$(e_2, e_1 + e_2, e_3)$ ist also linear unabhängig, mit $dim \mathbb{R}^3$ Vektoren und somit eine Basis von $\mathbb{R}^3$.

\subsection*{Teil 4}

\begin{align*}
 id_{\mathbb{R}^3}\begin{pmatrix}1\\0\\0\end{pmatrix} & = (-1) * e_2 + 1 * (e_1 + e_2) + 0 * e_3\\
 id_{\mathbb{R}^3}\begin{pmatrix}0\\1\\0\end{pmatrix} & = 1 * e_2 + 0 * (e_1 + e_2) + 0 * e_3\\
 id_{\mathbb{R}^3}\begin{pmatrix}0\\0\\1\end{pmatrix} & = 0 * e_2 + 0 * (e_1 + e_2) + 1 * e_3\\
\end{align*}

\begin{equation}
 Mat_{\mathcal{B},\mathcal{B}'}(id_{\mathbb{R}^3}) =
  \begin{pmatrix}
   -1 & 1 & 0\\
   1 & 0 & 0\\
   0 & 0 & 1
  \end{pmatrix}
\end{equation}

\subsection*{Teil 5}

\begin{align*}
 id_{\mathbb{R}^3}\begin{pmatrix}0\\1\\0\end{pmatrix} & = 0 * e_1 + 1 * e_2 + 0 * e_3\\
 id_{\mathbb{R}^3}\begin{pmatrix}1\\1\\0\end{pmatrix} & = 1 * e_1 + 1 * e_2 + 0 * e_3\\
 id_{\mathbb{R}^3}\begin{pmatrix}0\\0\\1\end{pmatrix} & = 0 * e_1 + 0 * e_2 + 1 * e_3\\
\end{align*}

\begin{equation}
 Mat_{\mathcal{B}',\mathcal{B}}(id_{\mathbb{R}^3}) =
  \begin{pmatrix}
   0 & 1 & 0\\
   1 & 1 & 0\\
   0 & 0 & 1
  \end{pmatrix}
\end{equation}

\subsection*{Teil 6}

\begin{equation}
 B = Mat_{\mathcal{B},\mathcal{B}'}(id_{\mathbb{R}^3})AMat_{\mathcal{B}',\mathcal{B}}(id_{\mathbb{R}^3}) = 
  \begin{pmatrix}
   2 & 1 & 2\\
   2 & 3 & 3\\
   0 & 0 & 6
  \end{pmatrix}
\end{equation}

\section*{Übung 2}

\subsection*{Teil 1}

\subsection*{Teil 2}

Weil $\mathbb{R}^3$ von Dimension $3$ ist, reicht es zu zeigen, dass $(v_1, v_2, v_3)$ linear unabhängig ist.
Seien $x, y, z \in \mathbb{R}$.
\begin{align*}
 & xv_1 + yv_2 + zv_3 = 0\\
 \Leftrightarrow &
  \begin{cases}
   x + 4y + 2z & = 0\\
   x + 3y - 3z & = 0\\
   x - 2y + 2z & = 0
  \end{cases}\\
 \Leftrightarrow &
  \begin{cases}
   x + 4y + 2z & = 0\\
   y - 5z & = 0\\
   y & = 0
  \end{cases}\\
 \Leftrightarrow &
  \begin{cases}
   x + 4y + 2z & = 0\\
   z & = 0\\
   y & = 0
  \end{cases}\\
  \Leftrightarrow &
  \begin{cases}
   x & = 0\\
   z & = 0\\
   y & = 0
  \end{cases}\\
\end{align*}
$(v_1, v_2, v_3)$ ist also eine Basis von $\mathbb{R}^3$.

\subsection*{Teil 3}

\begin{align*}
 A v_1 =
  \begin{pmatrix}
   1\\
   1\\
   1
  \end{pmatrix} =
  1v_1 + 0v_2 + 0v_3\\
 A v_2 =
  \begin{pmatrix}
   8\\
   6\\
   -4
  \end{pmatrix} =
  0v_1 + 2v_2 + 0v_3\\
 A v_3 =
  \begin{pmatrix}
   -8\\
   12\\
   -8
  \end{pmatrix} =
  0v_1 + 0v_2 + -4v_3\\
\end{align*}

\begin{equation}
 B =
  \begin{pmatrix}
   1 & 0 & 0\\
   0 & 2 & 0\\
   0 & 0 & -4
  \end{pmatrix}
\end{equation}

\subsection*{Teil 4}

\begin{proof}
 Wir behaupten
 \begin{equation}
  B^n = 
   \begin{pmatrix}
    1^n & 0 & 0\\
    0 & 2^n & 0\\
    0 & 0 & -4^n
   \end{pmatrix}
 \end{equation}

 Wir beweisen es mit Induktion über den Exponent $n$.
 Für $n = 1$, wissen wir, dass es stimmt.
 
 \begin{align*}
  B^{n + 1} = B^n B & = 
   \begin{pmatrix}
    1^n * 1 & 0 & 0\\
    0 & 2^n * 2 & 0\\
    0 & 0 & -4^n * (-4)
   \end{pmatrix}\\
   & = 
   \begin{pmatrix}
    1^{n + 1} & 0 & 0\\
    0 & 2^{n + 1} & 0\\
    0 & 0 & -4^{n + 1}
   \end{pmatrix}
 \end{align*}
\end{proof}

\subsection*{Teil 5}

Wir bestimmen $A^n$, indem wir $A$ zur Basis $\mathcal{B}$ wechseln und dann wieder zurück.
\begin{equation}
 c_\mathcal{B} =
  \begin{pmatrix}
   1 & 4 & 2\\
   1 & 3 & -3\\
   1 & -2 & 2
  \end{pmatrix}
\end{equation}
\begin{equation}
 c_\mathcal{B}^{-1} =
  \begin{pmatrix}
   1 & 4 & 2\\
   1 & 3 & -3\\
   1 & -2 & 2
  \end{pmatrix}
\end{equation}
\begin{equation}
 A^n = c_\mathcal{B} B^n c_\mathcal{B}^{-1}
\end{equation}

\section*{Übung 3}

\subsection*{Teil 1}

\begin{equation}
 f(e_1) = 
  \begin{pmatrix}
   0\\
   0\\
   0\\
   0
  \end{pmatrix}
 \qquad
 f(e_2) = 
  \begin{pmatrix}
   1\\
   1\\
   0\\
   1
  \end{pmatrix}
 \qquad
 f(e_3) = 
  \begin{pmatrix}
   -1\\
   1\\
   -1\\
   0
  \end{pmatrix}
\end{equation}


\begin{equation}
 A = 
  \begin{pmatrix}
   0 & 1 & -1\\
   0 & 1 & 1\\
   0 & 0 & -1\\
   0 & 1 & 0
  \end{pmatrix}
\end{equation}

\subsection*{Teil 2}

\begin{equation}
 Basis(ker(f)) = Basis(Ker(A)) =
  \begin{pmatrix}
   1\\
   0\\
   0
  \end{pmatrix}
\end{equation}

\begin{equation}
 Basis(Im(A)) =
  \left\langle
   \begin{pmatrix}
    1\\
    1\\
    0\\
    1
   \end{pmatrix},
   \begin{pmatrix}
    -1\\
    1\\
    -1\\
    0
   \end{pmatrix}
  \right\rangle
\end{equation}

\subsection*{Teil 3}

Nach Satz 8.9.2 muss man hier die transponierte Matrix A bestimmen.

\begin{equation}
 A^T = 
  \begin{pmatrix}
   0 & 0 & 0 & 0\\
   1 & 1 & 0 & 1\\
   -1 & 1 & -1 & 0
  \end{pmatrix}
\end{equation}

\subsection*{Teil 4}

Sei $v \in B'^\vee$.
\begin{align}
 & A^T \cdot v = 
  \begin{pmatrix}
   0\\
   v_1 + v_2 + v_4\\
   -v_1 + v_2 - v_3
  \end{pmatrix} = 0\\
 \Leftrightarrow & 
  \begin{pmatrix}
   v_1 + v_2 + v_4 = 0\\
   -v_1 + v_2 - v_3 = 0
  \end{pmatrix}
 \Leftrightarrow
  \begin{pmatrix}
   v_4 = -v_1 - v_2\\
   v_3 = v_1 - v_2
  \end{pmatrix}
\end{align}
Alle Vektoren in $Ker(A^T)$ sind also von der folgenden Form
\begin{equation}
 \begin{pmatrix}
  v_1\\
  v_2\\
  v_1 - v_2\\
  -v_1 - v_2
 \end{pmatrix} =
 \begin{pmatrix}
  v_1\\
  0\\
  v_1\\
  -v_1
 \end{pmatrix} + 
 \begin{pmatrix}
  0\\
  v_2\\
  -v_2\\
  -v_2
 \end{pmatrix} = 
 v_1
 \begin{pmatrix}
  1\\
  0\\
  1\\
  -1
 \end{pmatrix} + 
 v_2
 \begin{pmatrix}
  0\\
  1\\
  -1\\
  -1
 \end{pmatrix}
\end{equation}
\begin{equation}
 Basis(Ker(A)) = \left(
  \begin{pmatrix}
   1\\
   0\\
   1\\
   -1
  \end{pmatrix},
  \begin{pmatrix}
   0\\
   1\\
   -1\\
   -1
  \end{pmatrix}
 \right)
\end{equation}

\begin{equation*}
 A^T e_1 =
  \begin{pmatrix}
   0\\
   1\\
   -1
  \end{pmatrix}
 \qquad
 A^T e_2 =
  \begin{pmatrix}
   0\\
   1\\
   1
  \end{pmatrix}
 \qquad
 A^T e_3 =
  \begin{pmatrix}
   0\\
   0\\
   -1
  \end{pmatrix}
 \qquad
 A^T e_4 =
  \begin{pmatrix}
   0\\
   1\\
   0
  \end{pmatrix}
\end{equation*}

Weil nach dem Rangsatz $dim(Im(A^T)) = 2$ sein muss und $A^T e_1$ und $A^T e_2$ Linearkombinationen von $A^T e_3$ und $A^T e_4$ sind, muss
\begin{equation}
 Basis(Im(A)) = \left(
  \begin{pmatrix}
   0\\
   1\\
   0
  \end{pmatrix},
  \begin{pmatrix}
   0\\
   0\\
   -1
  \end{pmatrix}
 \right)
\end{equation}
sein.

\section*{Übung 4}

Sei $v \in K^4$.
Dann berechnen wir den Kern von $A$.
\begin{align*}
 & Av = 0\\
 \Leftrightarrow &
  \begin{cases}
   (1 + a + a^2 + a^3)v_1 & = 0\\
   (1 + a + a^2 + a^3)v_2 & = 0\\
   (1 + a + a^2 + a^3)v_3 & = 0\\
   (1 + a + a^2 + a^3)v_4 & = 0\\
  \end{cases}
\end{align*}
\begin{align*}
 & a + a^2 + a^3 + 1 = 0\\
 \Leftrightarrow & (a + 1)(a^2 + 1) = 0 \Leftrightarrow a = -1
\end{align*}
Wenn $a = -1$, ist $A$ also eine Null-Matrix.
Der Kern von $A$ hat dann Dimension $4$ und $Rg(A)$ ist nach dem Rangsatz $0$.

Sei $a \ne -1$.
Dann wird obiges Gleichungssystem nur durch $v_1 = v_2 = v_3 = v_4 = 0$ gelöst.
Der Kern von $A$ hat dann Dimension $0$ und $Rg(A)$ ist nach dem Rangsatz $4$.

\end{document}
