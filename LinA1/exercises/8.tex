\documentclass[a4paper,10pt]{article}
\usepackage[utf8]{inputenc}
\usepackage[german]{babel}
\usepackage{amsmath}
\usepackage{amssymb}
\usepackage{amsthm}

\title{LinA1, Übungsblatt 8}
\author{Marten Lienen (2126759), Gruppe 1; Fabian Schmittmann (2083559), Gruppe 5}

\begin{document}

\maketitle

\section*{Übung 1}

\begin{proof}
 Elemente $m \in M_{n,m}(K)$ sind Tabellen mit $n$ Spalten und $m$ Zeilen, wobei $m_{i,j}$ die Zahl in der $i$-ten Spalte und $j$-ten Zeile bezeichnet.
 Wir definieren $f: M_{n,m}(K) \rightarrow K^{nm}$ mit
 \begin{equation}
  f(k) = \begin{pmatrix}
          k_{1,1}\\
          \vdots\\
          k_{1,m}\\
          k_{2,1}\\
          \vdots\\
          k_{2,m}\\
          \vdots\\
          k_{n,m}
         \end{pmatrix}
 \end{equation}
 und $f^{-1}$ mit
 \begin{equation}
  f^{-1}(v) = \begin{pmatrix}
               v_1 & v_{m + 1} & \dots & v_{(n - 1)m + 1}\\
               v_2 & v_{m + 2} & \dots & v_{(n - 1)m + 2}\\
               \vdots & \vdots & & \vdots\\
               v_m & v_{2m} & \dots & v_{nm}
              \end{pmatrix}
 \end{equation}
 Es gilt
 \begin{equation}
  f \circ f^{-1}(v) = f(\begin{pmatrix}
               v_1 & v_{m + 1} & \dots & v_{(n - 1)m + 1}\\
               v_2 & v_{m + 2} & \dots & v_{(n - 1)m + 2}\\
               \vdots & \vdots & & \vdots\\
               v_m & v_{2m} & \dots & v_{nm}
              \end{pmatrix})
              = \begin{pmatrix}
                 v_1\\
                 \vdots\\
                 v_m\\
                 v_{m + 1}\\
                 \vdots\\
                 v_{2m}\\
                 \vdots\\
                 v_{nm}
                \end{pmatrix} = v
 \end{equation}
 und
 \begin{equation}
  f^{-1} \circ f(k) = f^{-1}(\begin{pmatrix}
          k_{1,1}\\
          \vdots\\
          k_{1,m}\\
          k_{2,1}\\
          \vdots\\
          k_{2,m}\\
          \vdots\\
          k_{n,m}
         \end{pmatrix})
         = \begin{pmatrix}
            k_{1,1} & k_{2,1} & \dots & k_{n,1}\\
            k_{1,2} & k_{2,2} & \dots & k_{n,2}\\
            \vdots & \vdots & & \vdots\\
            k_{1,m} & k_{2,m} & \dots & k_{n,m}
           \end{pmatrix} = k
 \end{equation}
 $f$ ist also bijektiv.
 
 Seien $x, y \in K$ und $k, l \in M_{n,m}(K)$.
 \begin{align}
  f(xk + yl) & = f(\begin{pmatrix}
                  xk_{1,1} + yl_{1,1} & xk_{2,1} + yl_{2,1} & \dots & xk_{n,1} + yl_{n,1}\\
                  xk_{1,2} + yl_{1,2} & xk_{2,2} + yl_{2,2} & \dots & xk_{n,2} + yl_{n,2}\\
                  \vdots & \vdots & & \vdots\\
                  xk_{1,m} + yl_{1,m} & xk_{2,m} + yl_{2,m} & \dots & xk_{n,m} + yl_{n,m}
                 \end{pmatrix})\\
  & = \begin{pmatrix}
       xk_{1,1} + yl_{1,1}\\
       \vdots\\
       xk_{1,m} + yl_{1,m}\\
       xk_{2,1} + yl_{2,1}\\
       \vdots\\
       xk_{2,m} + yl_{2,m}\\
       \vdots\\
       xk_{n,m} + yl_{n,m}
      \end{pmatrix}
    = \begin{pmatrix}
       xk_{1,1}\\
       \vdots\\
       xk_{1,m}\\
       xk_{2,1}\\
       \vdots\\
       xk_{2,m}\\
       \vdots\\
       xk_{n,m}
      \end{pmatrix}
    + \begin{pmatrix}
       yl_{1,1}\\
       \vdots\\
       yl_{1,m}\\
       yl_{2,1}\\
       \vdots\\
       yl_{2,m}\\
       \vdots\\
       yl_{n,m}
      \end{pmatrix}\\
  & = x\begin{pmatrix}
       k_{1,1}\\
       \vdots\\
       k_{1,m}\\
       k_{2,1}\\
       \vdots\\
       k_{2,m}\\
       \vdots\\
       k_{n,m}
      \end{pmatrix}
    + y\begin{pmatrix}
       l_{1,1}\\
       \vdots\\
       l_{1,m}\\
       l_{2,1}\\
       \vdots\\
       l_{2,m}\\
       \vdots\\
       l_{n,m}
      \end{pmatrix} = xf(k) + yf(l)
 \end{align}
 $f$ ist linear und damit ein Isomorphismus.
 Das heißt, dass $M_{n,m}(K)$ isomorph zu $K^{nm}$ und deshalb ebenfalls ein Vektorraum, insbesondere ein $K^{nm}$-Vektorraum, ist.
\end{proof}

\section*{Übung 2}

\section*{Übung 3}

\begin{equation}
 AB = \begin{pmatrix}
       4 & 2\\
       5 & -1
      \end{pmatrix}
\end{equation}
\begin{equation}
 BA = \begin{pmatrix}
       2 & 4\\
       4 & 1
      \end{pmatrix}
\end{equation}

\section*{Übung 4}

\subsection*{Teil 1}

\subsection*{Teil 2}

\subsection*{Teil 3}

\section*{Übung 5}

\subsection*{Teil 1}

\begin{proof}
 Seien $x, y \in K$ mit $f(x) = f(y)$.
 \begin{align}
  & f(x) = \begin{pmatrix}
        1 & x\\
        0 & 1
       \end{pmatrix}
        = \begin{pmatrix}
        1 & y\\
        0 & 1
       \end{pmatrix} = f(y)\\
  \Leftrightarrow & (1 = 1) \land (x = y) \land (0 = 0) \land (1 = 1)\\
  \Rightarrow & x = y
 \end{align}
\end{proof}

\subsection*{Teil 2}

\begin{proof}
 \begin{align}
  f(x)f(y) = \begin{pmatrix}
        1 & x\\
        0 & 1
       \end{pmatrix}
       \begin{pmatrix}
        1 & y\\
        0 & 1
       \end{pmatrix}
       = \begin{pmatrix}
        1 & x + y\\
        0 & 1
       \end{pmatrix} = f(x + y) 
 \end{align}
\end{proof}

\end{document}
