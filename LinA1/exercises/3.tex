\documentclass[a4paper,10pt]{article}
\usepackage[utf8]{inputenc}
\usepackage{amsmath}
\usepackage{amssymb}
\usepackage{amsthm}
\usepackage[german]{babel}

\title{Übungsblatt 3, LinA1}
\author{Marten Lienen (2126759), Übungsgruppe 1}

\newtheorem*{claim}{Behauptung}

\begin{document}

\maketitle

\section*{Übungs 1}

\begin{itemize}
 \item Da $1 \in H$ und $\forall x, y \in H \Rightarrow xy \in H \Rightarrow 1y \in H \Rightarrow 1y = y$, hat $H$ $1$ als neutrales Element
 \item $\forall x \in H \Rightarrow x^{-1} \in H$ gilt per Definition von Untergruppen
 \item Assoziativität ist eine Eigenschaft der Verknüpfung, die von $(G, \cdot)$ übernommen wird.
 Da $\cdot$ für alle $x, y \in H$ definiert ist, gilt diese Eigenschaft auch in $(H, \cdot)$
\end{itemize}

Somit hat $H$ alle Eigenschaft einer Gruppe.

\section*{Übungs 2}

\begin{equation*}
 Bild(f) = \{f(g) \mid g \in G\}
\end{equation*}

\begin{claim}
 $Bild(f)$ ist eine Untergruppe von $G'$.
\end{claim}

\begin{proof}
 Zu zeigen ist, dass $1_{G'} \in Bild(f)$, $\forall x \in Bild(f) \Rightarrow x^{-1} \in Bild(f)$ und $\forall x, y \in Bild(f) \Rightarrow xy \in Bild(f)$.
 
 Nach Satz 2.1.12 gilt $f(1_G) = 1_{G'} \Rightarrow 1_{G'} \in Bild(f)$ für alle Gruppenhomomorphismen.
 $Bild(f)$ enthält also ein neutrales Element.
 
 Da $G$ eine Gruppe und $f$ ein Gruppenhomomorphismus ist, gilt nach Satz 2.1.12 $\forall g \in G \Rightarrow g^{-1} \in G \Rightarrow f(g) \in Bild(f) \land (f(g^{-1}) \in Bild(f) \Rightarrow f(g)^{-1} \in Bild(f))$.
 Also gilt $\forall x \in Bild(f) \Rightarrow x^{-1} \in Bild(f)$.
 
 Es gilt ebenfalls $\forall x, y \in Bild(f) \Rightarrow (\exists p, q \in G \Rightarrow xy = f(p) * f(q) = f(pq) \in Bild(f))$.
 Weil $p$ und $q$ aus der Gruppe $G$ sind, ist auch $pq$ aus $G$.
 Demnach muss der Gruppenhomomorphismus $f$ auch $pq$ abbilden.
 Also gilt $\forall x, y \in Bild(f) \Rightarrow xy \in Bild(f)$.
\end{proof}

\section*{Übungs 3}

\section*{Übungs 4}



\section*{Übungs 5}

\subsection*{1}

\begin{claim}
 $(\mathbb{Z}, +)$ ist eine kommutative Gruppe.
\end{claim}

\begin{proof}
 Zu zeigen ist, dass $\exists 0 \in \mathbb{Z} \Rightarrow 0 + x = x + 0 = x$, $\forall x \in \mathbb{Z} \Rightarrow \exists x^{-1} \in \mathbb{Z} \Rightarrow x + x^{-1} = 0$, $\forall x, y, z \in \mathbb{Z} \Rightarrow x + (y + z) = (x + y) + z$ und $\forall x, y \in \mathbb{Z} \Rightarrow x + y = y + x$.
 
 Als neutrales Element nehmen wir $0_{\mathbb{Z}}$, sodass gilt $\exists 0 \in \mathbb{Z} \Rightarrow 0 + x = x + 0 = x$.
\end{proof}

\subsection*{2}

\subsection*{3}

\subsection*{4}

\end{document}
