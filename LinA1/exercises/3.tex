\documentclass[a4paper,10pt]{article}
\usepackage[utf8]{inputenc}
\usepackage{amsmath}
\usepackage{amssymb}
\usepackage{amsthm}
\usepackage[german]{babel}

\title{Übungsblatt 3, LinA1}
\author{Marten Lienen (2126759), Übungsgruppe 1}

\newtheorem*{claim}{Behauptung}
\newtheorem*{definition}{Definition}
\newtheorem*{notice}{Bemerkung}
\newtheorem*{lemma}{Lemma}
\newtheorem*{example}{Beispiel}

\begin{document}

\maketitle

\section*{Übung 1}

\begin{itemize}
 \item Da $1 \in H$ und $\forall x, y \in H \Rightarrow xy \in H \Rightarrow 1y \in H \Rightarrow 1y = y$, hat $H$ $1$ als neutrales Element
 \item $\forall x \in H \Rightarrow x^{-1} \in H$ gilt per Definition von Untergruppen
 \item Assoziativität ist eine Eigenschaft der Verknüpfung, die von $(G, \cdot)$ übernommen wird.
 Da $\cdot$ für alle $x, y \in H$ definiert ist, gilt diese Eigenschaft auch in $(H, \cdot)$
\end{itemize}

Somit hat $H$ alle Eigenschaft einer Gruppe.

\section*{Übung 2}

\begin{equation*}
 Bild(f) = \{f(g) \mid g \in G\}
\end{equation*}

\begin{claim}
 $Bild(f)$ ist eine Untergruppe von $G'$.
\end{claim}

\begin{proof}
 Zu zeigen ist, dass $1_{G'} \in Bild(f)$, $\forall x \in Bild(f) \Rightarrow x^{-1} \in Bild(f)$ und $\forall x, y \in Bild(f) \Rightarrow xy \in Bild(f)$.
 
 Nach Satz 2.1.12 gilt $f(1_G) = 1_{G'} \Rightarrow 1_{G'} \in Bild(f)$ für alle Gruppenhomomorphismen.
 $Bild(f)$ enthält also ein neutrales Element.
 
 Da $G$ eine Gruppe und $f$ ein Gruppenhomomorphismus ist, gilt nach Satz 2.1.12 $\forall g \in G \Rightarrow g^{-1} \in G \Rightarrow f(g) \in Bild(f) \land (f(g^{-1}) \in Bild(f) \Rightarrow f(g)^{-1} \in Bild(f))$.
 Also gilt $\forall x \in Bild(f) \Rightarrow x^{-1} \in Bild(f)$.
 
 Es gilt ebenfalls $\forall x, y \in Bild(f) \Rightarrow (\exists p, q \in G \Rightarrow xy = f(p) * f(q) = f(pq) \in Bild(f))$.
 Weil $p$ und $q$ aus der Gruppe $G$ sind, ist auch $pq$ aus $G$.
 Demnach muss der Gruppenhomomorphismus $f$ auch $pq$ abbilden.
 Also gilt $\forall x, y \in Bild(f) \Rightarrow xy \in Bild(f)$.
\end{proof}

\section*{Übung 3}



Zu zeigen ist, dass $(K, +)$ eine Untergruppe von $(\mathbb{R}, +)$ und $(K\backslash\{0\}, \cdot)$ eine Untergruppe von $(\mathbb{R}\backslash\{0\}, \cdot)$ ist.

\begin{proof}
 Sei $x \in K$.
 Schreibe $x = a + b\sqrt{2}$.
 Es gilt $0 + 0\sqrt{0} \in K$, weil $0 \in \mathbb{Q}$.
 \begin{equation*}
  a + b\sqrt{2} + 0 + 0\sqrt{2} = (a + 0) + (b + 0)\sqrt{2} = a + b\sqrt{2}
 \end{equation*}
 $0 + 0\sqrt{0}$ ist also das neutrale Element von $K$.
 
 Zusätzlich bestimmen wir ein inverses Element $x^{-1}$, sodass $x + x^{-1} = 0$.
 \begin{equation*}
  x + x^{-1} = 0 \Leftrightarrow x^{-1} = -x = -(a + b\sqrt{2}) = -a - b\sqrt{2}
 \end{equation*}
 Da $\forall y \in \mathbb{Q} \Rightarrow -y \in \mathbb{Q}$, ist das auch $\forall x \in K \Rightarrow -x \in K$.
 
 Seien $x, y, z \in K$.
 \begin{align*}
  x + (y + z) & = x_1 + x_2\sqrt{2} + (y_1 + y_2\sqrt{2} + z_1 + z_2\sqrt{2})\\
  & = x_1 + x_2\sqrt{2} + ((y_1 + z_1) + (y_2 + z_2)\sqrt{2})\\
  & = (x_1 + y_1 + z_1) + (x_2 + y_2 + z_2)\sqrt{2}\\
  & = ((x_1 + y_1) + (x_2 + y_2)\sqrt{2}) + z_1 + z_2\sqrt{2}\\
  & = (x_1 + x_2\sqrt{2} + y_1 + y_2\sqrt{2}) + z_1 + z_2\sqrt{2} = (x + y) + z
 \end{align*}
 Also ist $+$ auf $K$ assoziativ.
 
 Für $x, y \in K$ schreiben wir $x = x_1 + x_2\sqrt{2}$ und $y = y_1 + y_2\sqrt{2}$.
 \begin{align*}
  x + y & = x_1 + x_2\sqrt{2} + y_1 + y_2\sqrt{2}\\
  & = (x_1 + y_1) + (x_2 + y_2)\sqrt{2}\\
  & = (y_1 + x_1) + (y_2 + x_2)\sqrt{2}\\
  & = y_1 + y_2\sqrt{2} + x_1 + x_2\sqrt{2} = y + x
 \end{align*}
 $+$ auf $K$ ist kommutativ.
 
 Wir definieren $*$.
 \begin{align*}
  *: K \times K \mapsto K = (x_1 + x_2\sqrt{2}, y_1 + y_2\sqrt{2}) \rightarrow (x_1 + y_1) + (x_2 + y_2)\sqrt{2}
 \end{align*}

 
 Sei $x \in K\backslash\{0\}$.
 Schreibe $x = x_1 + x_2\sqrt{2}$.
 Dann gibt es $1 = 1_1 + 1_2\sqrt{2} \in K$, sodass
 \begin{align*}
  x * 1 = (
 \end{align*}

\end{proof}

\section*{Übung 4}



\section*{Übung 5}

\subsection*{1}

\begin{claim}
 $(\mathbb{Z}, +)$ ist eine kommutative Gruppe.
\end{claim}

\begin{proof}
 Zu zeigen ist, dass $\exists 0 \in \mathbb{Z} \Rightarrow 0 + x = x + 0 = x$, $\forall x \in \mathbb{Z} \Rightarrow \exists x^{-1} \in \mathbb{Z} \Rightarrow x + x^{-1} = 0$, $\forall x, y, z \in \mathbb{Z} \Rightarrow x + (y + z) = (x + y) + z$ und $\forall x, y \in \mathbb{Z} \Rightarrow x + y = y + x$.
 
 Als neutrales Element nehmen wir $0_{\mathbb{Z}}$, sodass gilt $\exists 0 \in \mathbb{Z} \Rightarrow 0 + x = x + 0 = x$.
\end{proof}

\subsection*{2}

\subsection*{3}

\subsection*{4}

\end{document}
