\documentclass[a4paper,10pt]{article}
\usepackage[utf8]{inputenc}
\usepackage{amsmath}
\usepackage{amssymb}
\usepackage{amsthm}
\usepackage[german]{babel}

\title{Übungsblatt 3, LinA1}
\author{Marten Lienen (2126759), Übungsgruppe 1}

\newtheorem*{claim}{Behauptung}
\newtheorem*{definition}{Definition}
\newtheorem*{notice}{Bemerkung}
\newtheorem*{lemma}{Lemma}
\newtheorem*{example}{Beispiel}

\begin{document}

\maketitle

\section*{Übung 1}

\begin{itemize}
 \item Da $1 \in H$ und $\forall x, y \in H \Rightarrow xy \in H \Rightarrow 1y \in H \Rightarrow 1y = y$, hat $H$ $1$ als neutrales Element
 \item $\forall x \in H \Rightarrow x^{-1} \in H$ gilt per Definition von Untergruppen
 \item Assoziativität ist eine Eigenschaft der Verknüpfung, die von $(G, \cdot)$ übernommen wird.
 Da $\cdot$ für alle $x, y \in H$ definiert ist, gilt diese Eigenschaft auch in $(H, \cdot)$
\end{itemize}

Somit hat $H$ alle Eigenschaften einer Gruppe.

\section*{Übung 2}

\begin{equation*}
 Bild(f) = \{f(g) \mid g \in G\}
\end{equation*}

\begin{claim}
 $Bild(f)$ ist eine Untergruppe von $G'$.
\end{claim}

\begin{proof}
 Zu zeigen ist, dass $1_{G'} \in Bild(f)$, $\forall x \in Bild(f) \Rightarrow x^{-1} \in Bild(f)$ und $\forall x, y \in Bild(f) \Rightarrow xy \in Bild(f)$.
 
 Nach Satz 2.1.12 gilt $f(1_G) = 1_{G'} \Rightarrow 1_{G'} \in Bild(f)$ für alle Gruppenhomomorphismen.
 $Bild(f)$ enthält also ein neutrales Element.
 
 Da $G$ eine Gruppe und $f$ ein Gruppenhomomorphismus ist, gilt nach Satz 2.1.12 $\forall g \in G \Rightarrow g^{-1} \in G \Rightarrow f(g) \in Bild(f) \land (f(g^{-1}) \in Bild(f) \Rightarrow f(g)^{-1} \in Bild(f))$.
 Also gilt $\forall x \in Bild(f) \Rightarrow x^{-1} \in Bild(f)$.
 
 Es gilt ebenfalls $\forall x, y \in Bild(f) \Rightarrow (\exists p, q \in G \Rightarrow xy = f(p) * f(q) = f(pq) \in Bild(f))$.
 Weil $p$ und $q$ aus der Gruppe $G$ sind, ist auch $pq$ aus $G$.
 Demnach muss der Gruppenhomomorphismus $f$ auch $pq$ abbilden.
 Also gilt $\forall x, y \in Bild(f) \Rightarrow xy \in Bild(f)$.
\end{proof}

\section*{Übung 3}

Für jedes $x \in K$ werde ich seinen Wert auch als $x_1 + x_2\sqrt{2}$ schreiben.
Außerdem sind sie beiden Verknüpfungen auf $K$ definiert als
\begin{description}
 \item[$+: K \times K \mapsto K$] $(x, y) \rightarrow (x_1 + y_1) + (x_2 + y_2)\sqrt{2}$
 \item[$*: K \times K \mapsto K$] $(x, y) \rightarrow (x_1y_1 + 2x_2y_2) + (x_2y_1 + x_1y_2)\sqrt{2}$
\end{description}

Zu zeigen ist, dass $(K, +)$ eine Untergruppe von $(\mathbb{R}, +)$ und $(K^\times, *)$ eine Untergruppe von $(\mathbb{R}^\times, *)$ ist.

\begin{proof}
 Zuerst zeigen wir, dass $(K, +)$ eine Untergruppe von $(\mathbb{R}, +)$.
 Sei $x \in K$.
 Dann gibt es das neutrale Element $0_K \in K = 0 + 0\sqrt{2}$, weil $0 \in \mathbb{Q}$.
 \begin{equation*}
  x + 0_K = (x_1 + 0) + (x_2 + 0)\sqrt{2} = (0 + x_1) + (0 + x_2)\sqrt{2} = 0_K + x = x
 \end{equation*}
 
 Zusätzlich bestimmen wir ein inverses Element $x^{-1}$, sodass $x + x^{-1} = 0$.
 \begin{equation*}
  x + x^{-1} = 0 \Leftrightarrow x^{-1} = -x = -(x_1 + x_2\sqrt{2}) = (-x_1) + (-x_2)\sqrt{2}
 \end{equation*}
 Da $\forall y \in \mathbb{Q} \Rightarrow -y \in \mathbb{Q}$, ist das auch $\forall x \in K \Rightarrow -x \in K$.
 
 Seien $x, y, z \in K$.
 \begin{align*}
  x + (y + z) & = x_1 + x_2\sqrt{2} + (y_1 + y_2\sqrt{2} + z_1 + z_2\sqrt{2})\\
  & = x_1 + x_2\sqrt{2} + ((y_1 + z_1) + (y_2 + z_2)\sqrt{2})\\
  & = (x_1 + y_1 + z_1) + (x_2 + y_2 + z_2)\sqrt{2}\\
  & = ((x_1 + y_1) + (x_2 + y_2)\sqrt{2}) + z_1 + z_2\sqrt{2}\\
  & = (x_1 + x_2\sqrt{2} + y_1 + y_2\sqrt{2}) + z_1 + z_2\sqrt{2} = (x + y) + z
 \end{align*}
 Also ist $+$ auf $K$ assoziativ.
 
 Sei $x, y \in K$.
 \begin{align*}
  x + y & = x_1 + x_2\sqrt{2} + y_1 + y_2\sqrt{2}\\
  & = (x_1 + y_1) + (x_2 + y_2)\sqrt{2}\\
  & = (y_1 + x_1) + (y_2 + x_2)\sqrt{2}\\
  & = y_1 + y_2\sqrt{2} + x_1 + x_2\sqrt{2} = y + x
 \end{align*}
 $+$ auf $K$ ist kommutativ.
 
 Sei $x \in K^\times$.
 Dann gibt es das neutrale Element $1 = 1 + 0\sqrt{2} \in K^\times$, sodass
 \begin{align*}
  1x & = (1x_1 + 2 * 0 * x_2) + (0x_1 + 1x_2)\sqrt{2}\\
  & = (1x_1 + 2 * 0 * x_2) + (0x_1 + 1x_2)\sqrt{2}\\
  & = (x_11 + 2 * x_2 * 0) + (x_1 * 0 + x_2 * 1)\sqrt{2} = x1\\
  & = x_1 + x_2\sqrt{2} = x
 \end{align*}
 
 Dann bestimmen wir das inverse Element $k^{-1}$, sodass $xk^{-1} = 1$.
 \begin{align*}
  xk^{-1} = (x_1k^{-1}_1 + 2x_2k^{-1}_2) + (x_2k^{-1}_1 + x_1k^{-1}_2)\sqrt{2} = 1 + 0\sqrt{2} = 1
 \end{align*}
 Es muss also gelten
 \begin{align*}
  1 = x_1k^{-1}_1 + 2x_2k^{-1}_2 \land 0 = x_2k^{-1}_1 + x_1k^{-1}_2
 \end{align*}

 
 Es gilt auch die Assoziativität von $*$ auf $K$.
 Seien $x, y, z \in K$.
 \begin{align*}
  x(yz) & = x((y_1z_1 + 2y_2z_2) + (y_2z_1 + y_1z_2)\sqrt{2})\\
  & = (x_1(y_1z_1 + 2y_2z_2) + 2x_2(y_2z_1 + y_1z_2)) + (x_2(y_1z_1 + 2y_2z_2) + x_1(y_2z_1 + y_1z_2))\sqrt{2}\\
  & = ((x_1y_1z_1 + 2x_1y_2z_2) + 2(x_2y_2z_1 + x_2y_1z_2)) + ((x_2y_1z_1 + 2x_2y_2z_2) + (x_1y_2z_1 + x_1y_1z_2))\sqrt{2}\\
  & = ((x_1y_1 + 2x_2y_2)z_1 + 2(x_2y_1 + x_1y_2)z_2) + ((x_2y_1 + x_1y_2)z_1 + (x_1y_1 + 2x_2y_2)z_2)\sqrt{2}\\
  & = ((x_1y_1 + 2x_2y_2) + (x_2y_1 + x_1y_2)\sqrt{2})z = (xy)z
 \end{align*}
 
 Dazu ist $*$ auch kommutativ.
 Seien $x, y \in K^\times$.
 \begin{align*}
  xy & = (x_1y_1 + 2x_2y_2) + (x_2y_1 + x_1y_2)\sqrt{2}\\
  & = (y_1x_1 + 2y_2x_2) + (y_1x_2 + y_2x_1)\sqrt{2} = yx
 \end{align*}
\end{proof}

\section*{Übung 4}

\begin{proof}
 Zu zeigen ist, dass $f$ für $+$ und $*$ auf $\mathbb{C}$ jeweils ein Gruppenhomomorphismus ist.
 Seien $x, y \in \mathbb{C}$.
 
 \begin{align*}
  f(x) + f(y) & = \overline{x} + \overline{y}\\
  & = (x_1 - x_2i) + (y_1 - y_2i)\\
  & = (x_1 + y_1) - (x_2 + y_2)i = \overline{(x_1 + y_1) + (x_2 + y_2)i} = f(x + y)
 \end{align*}
 
 \begin{align*}
  f(x) * f(y) & = \overline{x} * \overline{y}\\
  & = (x_1 + (-x_2)i) * (y_1 + (-y_2)i)\\
  & = (x_1y_1 - x_2y_2) + ((-x_1y_2) + (-x_2y_1))i\\
  & = (x_1y_1 - x_2y_2) - (x_1y_2 + x_2y_1)i = \overline{(x_1y_1 - x_2y_2) + (x_1y_2 + x_2y_1)i} = f(xy)
 \end{align*}
\end{proof}

\section*{Übung 5}

\subsection*{1}

\begin{claim}
 $(\mathbb{Z}, +)$ ist eine kommutative Gruppe.
\end{claim}

\begin{proof}
 Zu zeigen ist, dass $\exists 0 \in \mathbb{Z} \Rightarrow 0 + x = x + 0 = x$, $\forall x \in \mathbb{Z} \Rightarrow \exists x^{-1} \in \mathbb{Z} \Rightarrow x + x^{-1} = 0$, $\forall x, y, z \in \mathbb{Z} \Rightarrow x + (y + z) = (x + y) + z$ und $\forall x, y \in \mathbb{Z} \Rightarrow x + y = y + x$.
 
 Als neutrales Element nehmen wir $0_{\mathbb{Z}}$, sodass gilt $\exists 0 \in \mathbb{Z} \Rightarrow 0 + x = x + 0 = x$.
 Nach Definition von $\mathbb{Z}$ enthält es $\forall x \in \mathbb{Z}$ ein $-x \in \mathbb{Z}$, sodass $x + (-x) = 0$.
 Für $+$ in $\mathbb{Z}$ gilt außerdem $x + y = y + x$ und $(x + y) + z = x + y + z = (x + y) + z$.
\end{proof}

\subsection*{2}

\begin{claim}
 $(\mathbb{Q}^\times, \cdot)$ ist eine kommutative Gruppe.
\end{claim}

\begin{proof}
 $\mathbb{Q^\times}$ enthält das neutrale Element $1$, sodass $x \cdot 1 = 1 \cdot x = x$.
 Es enthält auch das inverse Element $x^{-1} = \frac{1}{x}$ jedes $x \in \mathbb{Q}^\times$, sodass $xx^{-1} = 1$.
 Es gilt auch die Assoziativität.
 \begin{equation*}
  x \cdot (y \cdot z) = x(yz) = xyz = (xy)z = (x \cdot y) \cdot z
 \end{equation*}
 Und die Kommutativität.
 \begin{equation*}
  x \cdot y = xy = yx = y \cdot x
 \end{equation*}
\end{proof}

\subsection*{3}

\begin{claim}
 $K = (Bij(M), \circ)$ ist eine Gruppe.
\end{claim}

\begin{proof}
 Zu zeigen ist, dass $K$ ein neutrales Element und zu jedem $x \in K$ ein inverses Element hat und die Verknüpfung assoziativ ist.
 
 Das neutrale Element ist $id_M$.
 Sei $f \in K$.
 \begin{equation*}
  f \circ id_M = f(id_M(x)) = f(x) = id_M(f(x)) = id_M \circ f = f
 \end{equation*}

 Nach der Voraussetzung, dass $f$ eine Bijektion ist, gibt es $g \in K$ als inverses Element, sodass $f \circ g = id_M$.
 
 $K$ ist auch assoziativ.
 Seien $f, g, h \in K$.
 \begin{align*}
  f \circ (g \circ h) = f \circ g(h(x)) = f(g(h(x))) = f(g(x)) \circ h = (f \circ g) \circ h
 \end{align*}
\end{proof}

\subsection*{4}

\begin{claim}
 $(Bij(M), \circ)$ ist nicht kommutativ, wenn $M$ mindestens drei paarweise verschiedene Elemente enthält.
\end{claim}

\begin{proof}
 Nach der Definition muss $\forall f, g \in Bij(M) \Rightarrow f \circ g = g \circ f$ gelten, damit die Gruppe kommutativ ist.
 Demnach reicht es also ein Gegenbeispiel zu finden.
 Seien $a, b, c \in M$ paarweise verschieden und seien weiterhin $f, g \in Bij(M)$.
 $f$ sei definiert als $f(a) = b, f(b) = c, f(c) = a$ und $g$ sei definiert als $g(a) = a, g(b) = c, g(c) = b$.
 \begin{equation*}
  f(g(a)) = f(a) = a \ne c = g(b) = g(f(a))
 \end{equation*}
\end{proof}

\end{document}
