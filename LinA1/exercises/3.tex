\documentclass[a4paper,10pt]{article}
\usepackage[utf8]{inputenc}
\usepackage{amsmath}
\usepackage{amssymb}
\usepackage{amsthm}
\usepackage[german]{babel}

\title{}
\author{}

\newtheorem*{definition}{Definition}
\newtheorem*{notice}{Bemerkung}
\newtheorem*{lemma}{Lemma}
\newtheorem*{example}{Beispiel}

\begin{document}

\maketitle

\section*{Übung 3}



Zu zeigen ist, dass $(K, +)$ eine Untergruppe von $(\mathbb{R}, +)$ und $(K\backslash\{0\}, \cdot)$ eine Untergruppe von $(\mathbb{R}\backslash\{0\}, \cdot)$ ist.

\begin{proof}
 Sei $x \in K$.
 Schreibe $x = a + b\sqrt{2}$.
 Es gilt $0 + 0\sqrt{0} \in K$, weil $0 \in \mathbb{Q}$.
 \begin{equation*}
  a + b\sqrt{2} + 0 + 0\sqrt{2} = (a + 0) + (b + 0)\sqrt{2} = a + b\sqrt{2}
 \end{equation*}
 $0 + 0\sqrt{0}$ ist also das neutrale Element von $K$.
 
 Zusätzlich bestimmen wir ein inverses Element $x^{-1}$, sodass $x + x^{-1} = 0$.
 \begin{equation*}
  x + x^{-1} = 0 \Leftrightarrow x^{-1} = -x = -(a + b\sqrt{2}) = -a - b\sqrt{2}
 \end{equation*}
 Da $\forall y \in \mathbb{Q} \Rightarrow -y \in \mathbb{Q}$, ist das auch $\forall x \in K \Rightarrow -x \in K$.
 
 Seien $x, y, z \in K$.
 \begin{align*}
  x + (y + z) & = x_1 + x_2\sqrt{2} + (y_1 + y_2\sqrt{2} + z_1 + z_2\sqrt{2})\\
  & = x_1 + x_2\sqrt{2} + ((y_1 + z_1) + (y_2 + z_2)\sqrt{2})\\
  & = (x_1 + y_1 + z_1) + (x_2 + y_2 + z_2)\sqrt{2}\\
  & = ((x_1 + y_1) + (x_2 + y_2)\sqrt{2}) + z_1 + z_2\sqrt{2}\\
  & = (x_1 + x_2\sqrt{2} + y_1 + y_2\sqrt{2}) + z_1 + z_2\sqrt{2} = (x + y) + z
 \end{align*}
 Also ist $+$ auf $K$ assoziativ.
 
 Für $x, y \in K$ schreiben wir $x = x_1 + x_2\sqrt{2}$ und $y = y_1 + y_2\sqrt{2}$.
 \begin{align*}
  x + y & = x_1 + x_2\sqrt{2} + y_1 + y_2\sqrt{2}\\
  & = (x_1 + y_1) + (x_2 + y_2)\sqrt{2}\\
  & = (y_1 + x_1) + (y_2 + x_2)\sqrt{2}\\
  & = y_1 + y_2\sqrt{2} + x_1 + x_2\sqrt{2} = y + x
 \end{align*}
 $+$ auf $K$ ist kommutativ.
 
 Wir definieren $*$.
 \begin{align*}
  *: K \times K \mapsto K = (x_1 + x_2\sqrt{2}, y_1 + y_2\sqrt{2}) \rightarrow (x_1 + y_1) + (x_2 + y_2)\sqrt{2}
 \end{align*}

 
 Sei $x \in K\backslash\{0\}$.
 Schreibe $x = x_1 + x_2\sqrt{2}$.
 Dann gibt es $1 = 1_1 + 1_2\sqrt{2} \in K$, sodass
 \begin{align*}
  x * 1 = (
 \end{align*}

\end{proof}

\end{document}
