\documentclass[a4paper,10pt]{article}
\usepackage[utf8]{inputenc}
\usepackage[german]{babel}
\usepackage{amsmath}
\usepackage{amssymb}
\usepackage{amsthm}
\usepackage{stmaryrd}

\title{LinA1, Übungsblatt 5}
\author{Marten Lienen (2126759), Gruppe 1}

\newtheorem*{claim}{Behauptung}

\begin{document}

\maketitle

\section*{Übung 1}

\begin{proof}
 $\Rightarrow$: Angenommen $f^{-1}(w)$ sei ein Unterraum von $V$.
 Es gibt also $v_1, \dots, v_n \in V$ mit $f(v_k) = w \forall k \in [1, n]$.
 Da $f^{-1}(w)$ ein Unterraum ist, muss eines seiner Elemente das neutrale Element $0$ sein.
 Nach Satz 2.1.12 wird das neutrale Element auf das neutrale Element der Zielmenge abgebildet.
 Es folgt $w = 0$, weil es mindestens ein Element in $f^{-1}(w)$ gibt, dass auf $0$ abgebildet wird und nach Definition alle auf das gleiche abgebildet werden.
 
 $\Leftarrow$: Angenommen $w = 0$.
 $(f^{-1}(w), +)$ muss eine Untergruppe von $(V, +)$ sein.
 Dazu muss
 \begin{itemize}
  \item es ein neutrales Element haben\\
  Da das neutrale Element auf das neutrale Element abgebildet wird, enthält $f^{-1}(w)$ das neutrale Element $0 \in V$.
  \item es für jedes Element ein Inverses haben\\
  Sei $v \in f^{-1}(w)$ und $v^{-1}$ sein inverses Element.
  \begin{equation*}
   v + v^{-1} = 0 \Leftrightarrow v^{-1} = 0 - v \Leftrightarrow f(v^{-1}) = f(0 - v) = f(0) - f(v) = 0 - 0 = 0
  \end{equation*}
  $v^{-1}$ wird also ebenfalls auf $0$ abgebildet und ist somit auch in $f^{-1}(w)$ enthalten.
  \item $p + q \in f^{-1}(w)$ für alle $p, q \in f^{-1}(w)$ sein\\
  Seien $p, q \in f^{-1}(w)$.
  \begin{equation*}
   f(p + q) = f(p) + f(q) = 0 + 0 = 0
  \end{equation*}
  $p + q$ wird auch auf $0$ abgebildet und ist somit auch in $f^{-1}(w)$ enthalten.
 \end{itemize}
 Sei $x \in K$ und $v \in f^{-1}(w)$.
 \begin{equation*}
  f(xv) = xf(v) = x0 = 0
 \end{equation*}
 $xv$ ist auch in $f^{-1}(w)$ enthalten.
 
 $f^{-1}(w)$ ist ein Unterraum von $V$.
\end{proof}

\section*{Übung 2}

\begin{proof}
 $\Rightarrow$: Angenommen $W_x$ ist ein Unterraum von $V$.
 Dann muss $W_x$ zu jedem $v \in W_x$ ein Inverses haben.
 \begin{equation*}
  0 = v + v^{-1} = \begin{pmatrix}x + x\\v_2 + v^{-1}_2\end{pmatrix} = \begin{pmatrix}0\\0\end{pmatrix}
 \end{equation*}
 Es muss also gelten
 \begin{equation*}
  x + x = 2x = 0 \Rightarrow x = 0
 \end{equation*}

 $\Leftarrow$: Angenommen $x = 0$.
 Damit $(W_x, +)$ eine Untergruppe von $(K^2, +)$ ist, muss
 \begin{itemize}
  \item es ein neutrales Element haben\\
  Sei $k \in W_x$.
  \begin{equation*}
   k = k + 0_{W_x} \Leftrightarrow 0_{W_x} = k - k = \begin{pmatrix}0\\k_2 - k_2\end{pmatrix} = \begin{pmatrix}0\\0\end{pmatrix}
  \end{equation*}
  Da $K$ ein Körper ist, ist $0 \in K$ und somit auch $0_{W_x} \in W_x$.
  \item es für jedes Element ein Inverses haben\\
  Sei $k \in W_x$.
  \begin{equation*}
   0 = k + k^{-1} \Leftrightarrow k^{-1} = 0 - k = \begin{pmatrix}0\\0\end{pmatrix} - \begin{pmatrix}0\\k_2\end{pmatrix} = \begin{pmatrix}0\\-k_2\end{pmatrix}
  \end{equation*}
  Da $K$ ein Körper ist, ist auch $-k_2 \in K$ und $k^{-1} \in W_x$.
  \item $y + z \in W_x$ für alle $y, z \in W_x$ sein\\
  Seien $y, z \in W_x$.
  \begin{equation*}
   y + z = \begin{pmatrix}0\\y_2\end{pmatrix} + \begin{pmatrix}0\\z_2\end{pmatrix} = \begin{pmatrix}0\\y_2 + z_2\end{pmatrix}
  \end{equation*}
  $y_2 + z_2 \in K$, weil $K$ ein Körper ist, und somit auch $y + z \in W_x$.
 \end{itemize}

 Damit $W_x$ ein Unterraum von $K^2$ ist, muss noch jedes Produkt einer Skalarmultiplikation in $W_x$ ebenfalls in $W_x$ liegen.
 Sei $k \in K$ und $y \in W_x$.
 \begin{equation}
  ky = k\begin{pmatrix}0\\y_2\end{pmatrix} = \begin{pmatrix}0\\ky_2\end{pmatrix}
 \end{equation}
 $ky_2 \in K$, weil $K$ ein Körper ist, und somit ist $ky \in W_x$.
 
 $W_x$ ist ein Unterraum von $K^2$, wenn $x = 0$.
\end{proof}

\section*{Übung 3}

\begin{proof}
 Sei $d \in V$.
 $d_n$ bezeichne den $n$-ten Skalar in $d$.
 Es gibt Skalare $x_k = d_k$ für alle $1 \le k \le n$, sodass
 \begin{align*}
  \sum_{k = 1}^n x_ke_k & = \begin{pmatrix}1x_1\\0x_1\\\vdots\\0x_1\end{pmatrix} + \begin{pmatrix}0x_2\\1x_2\\\vdots\\0x_2\end{pmatrix} + \dots + \begin{pmatrix}0x_n\\0x_n\\\vdots\\1x_n\end{pmatrix}\\
  & = \begin{pmatrix}x_1\\0\\\vdots\\0\end{pmatrix} + \begin{pmatrix}0\\x_2\\\vdots\\0\end{pmatrix} + \dots + \begin{pmatrix}0\\0\\\vdots\\x_n\end{pmatrix} = \begin{pmatrix}x_1\\x_2\\\vdots\\x_n\end{pmatrix} = \begin{pmatrix}d_1\\d_2\\\vdots\\d_n\end{pmatrix} = d
 \end{align*}
 $d$ ist also eine Linearkombination des Systems $(e_1, \dots, e_n)$ und das System somit ein EZS.
 
 Seien $x_k$ mit $1 \le k \le n$ Skalare.
 Bildet man wie oben eine Linearkombination $d \in <e_1, \dots, e_n>$, so gilt
 \begin{equation*}
  d = \sum_{k = 1}^n x_ke_k = \begin{pmatrix}x_1\\x_2\\\vdots\\x_n\end{pmatrix}
 \end{equation*}
 Für die Darstellung von $0$ gilt also
 \begin{equation*}
  \begin{pmatrix}x_1\\x_2\\\vdots\\x_n\end{pmatrix} = \begin{pmatrix}0\\0\\\vdots\\0\end{pmatrix} \Rightarrow x_k = 0 \forall 1 \le k \le n
 \end{equation*}
 Das System $(e_1, \dots, e_n)$ ist linear unabhängig.
 
 Da es ein linear unabhängiges EZS von $V$ ist, ist es eine Basis von $V$.
\end{proof}

\section*{Übung 4}

\begin{proof}
 Angenommen $(1, \sqrt{2})$ sei linear abhängig.
 Es gäbe also $x, y \in \mathbb{Q}$ mit
 \begin{equation*}
  1x + y\sqrt{2} = 0
 \end{equation*}
 mit mindestens $x \ne 0$ oder $y \ne 0$.
 \begin{align*}
  & y = 0 \Rightarrow x = 0 \lightning\\
  & y \ne 0 \Rightarrow \sqrt{2} = -\frac{x}{y} \lightning \text{weil $\sqrt{2}$ nicht rational ist}
 \end{align*}
 $(1, \sqrt{2})$ ist also linear unabhängig.
 
 Sei $v = v_1 + v_2\sqrt{2} \in \mathbb{V}$ mit $v_1, v_2 \in \mathbb{Q}$.
 \begin{equation*}
  v_1\begin{pmatrix}1\\0\end{pmatrix} + v_1\begin{pmatrix}0\\\sqrt{2}\end{pmatrix} = (v_1 + 0\sqrt{2}) + (0 + v_2\sqrt{2}) = v_1 + v_2\sqrt{2} = v
 \end{equation*}
 $v$ ist also eine lineare Kombination von $(1, \sqrt{2})$.
 Somit ist $(1, \sqrt{2})$ ein EZS und letztendlich eine Basis von $V$.
\end{proof}

\section*{Übung 5}

\subsection*{a}

Angenommen das System sei linear abhängig, sodass es $x_1, x_2, x_3 \in \mathbb{R}$ gibt mit mindestens einem $x_n \ne 0$, sodass
\begin{equation*}
 x_1\begin{pmatrix}1\\0\\-1\end{pmatrix} + x_2\begin{pmatrix}1\\2\\1\end{pmatrix} + x_3\begin{pmatrix}0\\-3\\2\end{pmatrix} = 0 = \begin{pmatrix}x_1\\0\\-x_1\end{pmatrix} + \begin{pmatrix}x_2\\2x_2\\x_2\end{pmatrix} + \begin{pmatrix}0\\-3x_3\\2x_3\end{pmatrix}
\end{equation*}
Daraus ergibt sich folgendes Gleichungssystem
\begin{align*}
 x_1 + x_2 = 0 & \Rightarrow \text{1. } x_1 = -x_2\\
 x_2 - 3x_3 = 0 & \Rightarrow \text{3. } x_3 = 0\\
 -x_1 + x_2 + 2x_3 = 0 & \Rightarrow \text{2. } x_2 = -x_3\\
 & \Rightarrow x_1 = -x_2 = x_3 = 0
\end{align*}
Das ist ein Widerspruch und somit ist das System linear unabhängig.

\subsection*{b}

In Übung 3 wurde gezeigt, dass $\mathbb{R}^3$ eine Basis mit 3 Vektoren hat.
Da alle Basen eines Vektorraums gleich viele Vektoren haben und eine Basis ein maximales, linear unabhängiges System ist, ist dieses System mit 4 Vektoren linear abhängig.

\subsection*{c}

Da 0 eine Linearkombination von Vektoren dieses Systems mit Nicht-Null-Koeffizienten ist, ist es linear abhängig.
\begin{equation}
 0 = 2 * \begin{pmatrix}0\\0\\0\end{pmatrix} + 0 * \begin{pmatrix}1\\2\\3\end{pmatrix} + 0 * \begin{pmatrix}2\\3\\4\end{pmatrix}
\end{equation}

\subsection*{d}

Seien $x_1, x_2 \in \mathbb{R}$.
Angenommen $x_1 \ne 0$.
\begin{equation*}
 x_1\begin{pmatrix}1\\1\\1\end{pmatrix} + x_2\begin{pmatrix}2\\3\\4\end{pmatrix} \Rightarrow \begin{pmatrix}1\\1\\1\end{pmatrix} = -\frac{x_2}{x_1}\begin{pmatrix}2\\3\\4\end{pmatrix} \Rightarrow -\frac{2x_2}{x_1} = 1 = -\frac{3x_2}{x_1} \lightning
\end{equation*}

Angenommen $x_2 \ne 0$.
\begin{equation*}
 x_1\begin{pmatrix}1\\1\\1\end{pmatrix} + x_2\begin{pmatrix}2\\3\\4\end{pmatrix} \Rightarrow \begin{pmatrix}2\\3\\4\end{pmatrix} = -\frac{x_1}{x_2}\begin{pmatrix}1\\1\\1\end{pmatrix} \Rightarrow 2 = -\frac{x_1}{x_2}1 = 3 \lightning
\end{equation*}

Es gilt also $x_1 = x_2 = 0$ und somit ist das System linear unabhängig.

\subsection*{e}

Seien $x_1, x_2, x_3 \in \mathbb{R}$.
Aus
\begin{equation*}
 x_1 \begin{pmatrix}1\\x\\0\end{pmatrix} + x_2 \begin{pmatrix}x\\1\\0\end{pmatrix} + x_3 \begin{pmatrix}0\\x\\1\end{pmatrix} = 0
\end{equation*}
ergeben sich folgende Bedingungen
\begin{align*}
 x_1 + xx_2 = 0 & \Rightarrow x_1(1 - x^2) = 0\\
 xx_1 + x_2 + xx_3 = 0 & \Rightarrow x_2 = -xx_1\\
 x_3 = 0 & \Rightarrow x_3 = 0\\
\end{align*}
\begin{equation*}
 x_1 = \begin{cases}
        0 & \text{ wenn } x \ne 1\\
        \mathbb{R} & \text{ wenn } x = 1
       \end{cases}
\end{equation*}

Das System ist genau dann linear unabhängig, wenn $x \ne 1$.

\end{document}
