\documentclass[a4paper,10pt]{article}
\usepackage[utf8]{inputenc}
\usepackage[german]{babel}
\usepackage{amsmath}
\usepackage{amssymb}
\usepackage{amsthm}
\usepackage{stmaryrd}

\title{LinA1, Übungsblatt 13}
\author{Marten Lienen (2126759), Gruppe 1; Fabian Schmittmann (2083559), Gruppe 5}

\begin{document}

\maketitle

\section*{Übung 1}

\subsection*{Teil 1}

$A$ entsteht aus $I_n$ indem man zuerst mit $D_k(a_{k,k})$ linksmultipliziert für alle $k \in [1, n]$.
Danach multipliziert man noch einen ganzen Haufen Typ 1-Matrixen links an, die jeweils die $k - 1$-te Zeile aus der $k$-ten Zeile erzeugen, aber den Wert der Determinanten wegen Lemma 13.1.3.1 nicht beeinflussen.
Wegen Lemma 13.1.3.2 und $det(I_n) = 1$ gilt dann schon
\begin{equation}
 det(A) = a_{1,1}\dots a_{n,n}
\end{equation}

\subsection*{Teil 2}

$A$ entsteht aus $I_n$ indem man zuerst mit $D_k(a_{k,k})$ linksmultipliziert für alle $k \in [1, n]$.
Danach multipliziert man noch einen ganzen Haufen Typ 1-Matrixen links an, die jeweils die $k + 1$-te Zeile aus der $k$-ten Zeile erzeugen, aber den Wert der Determinanten wegen Lemma 13.1.3.1 nicht beeinflussen.
Wegen Lemma 13.1.3.2 und $det(I_n) = 1$ gilt dann schon
\begin{equation}
 det(A) = a_{1,1}\dots a_{n,n}
\end{equation}

\section*{Übung 2}

\begin{align*}
 & \begin{pmatrix}
  \alpha\\
  1\\
  0
 \end{pmatrix} = x
 \begin{pmatrix}
  1\\
  2\\
  -1
 \end{pmatrix} + y
 \begin{pmatrix}
  3\\
  -1\\
  1
 \end{pmatrix}\\
 \Leftrightarrow &
  \begin{cases}
   \alpha & = x + 3y\\
   1 & = 2x - y\\
   0 & = -x + y
  \end{cases}\\
 \Leftrightarrow &
  \begin{cases}
   \alpha & = x + 3y\\
   1 + y& = 2x\\
   x & = y
  \end{cases}\\
 \Leftrightarrow &
  \begin{cases}
   \alpha & = x + 3y\\
   y& = 1\\
   x & = 1
  \end{cases}\\
 \Rightarrow &\ \alpha = 4
\end{align*}


\section*{Übung 3}

\section*{Übung 4}

\subsection*{Teil 1}

\subsection*{Teil 2}

\section*{Übung 5}

\begin{proof}
 Wir wählen $\det$ mit $j = n$, sodass immer die letzte Spalte wegfällt.
 Die Matrizen $A_{i,n}$ haben auf den ersten $i - 1$ Zeilen ein $X$ auf der Diagonalen und $-1$ in der Zelle darunter.
 Die verbleibenden $n - i$ Zeilen haben $-1$ auf der Diagonalen und ein $X$ in der Zelle darüber.
 Die Matrizen $A_{i,n} = (a_{p,q})$ sind also folgendermaßen aufgebaut:
 \begin{equation}
  a_{p,q} =
   \begin{cases}
    X & \text{wenn $p = q < i$}\\
    X & \text{wenn $p + 1 = q$ und $p \ge i$}\\
    -1 & \text{wenn $p = q \ge i$}\\
    -1 & \text{wenn $p = q + 1$ und $q < i$}\\
    0 & \text{sonst}
   \end{cases}
 \end{equation}
 Man kann diese $A_{i,n}$ aus $I_n$ erzeugen:
 \begin{align*}
  A_{i,n} = \prod_{k = 1}^{n - 1} \left( D_k(X) \right) \cdot \prod_{k = n - 1}^{i + 1} \left( T_{k, k - 1}(-X) \right) \cdot \prod_{k = n - 1}^{i} \left( D_k(-\frac{1}{X}) \right) \cdot \prod_{k = i - 2}^{1} \left( T_{k,k + 1}(-\frac{1}{X}) \right) \cdot I_n
 \end{align*}
 Allerdings beeinflussen die Typ-1-Matrizen den Wert der Determinanten nicht.
 Folglich gilt:
 \begin{align*}
  \det(A_{i,n}) = X^{n - 1} \cdot \frac{1}{X^{n - i}} = X^{i - 1}
 \end{align*}
 Wenn wir das jetzt in die Definition von $\det$ einsetzten erhalten wir:
 \begin{align*}
  \det(A) = \sum_{i = 1}^n (-1)^{i + n} a_{i,n} X^{i - 1}
 \end{align*}
\end{proof}

\section*{Übung 6}

\begin{proof}
 Wenn $\lambda = 0$, ist $\lambda I_n - A$ invertierbar, weil $A$ invertierbar ist.
 
 
\end{proof}

\end{document}
