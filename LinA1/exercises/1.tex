\documentclass[a4paper,10pt]{article}
\usepackage[utf8]{inputenc}
\usepackage{amsmath}
\usepackage{amsthm}
\usepackage{amssymb}
\usepackage[german]{babel}

\title{Übungsblatt 1}
\author{Marten Lienen (2126759), Gruppe 1}

\begin{document}

\maketitle

\section*{Übung 1}

Zu zeigen
\begin{equation}
 M = N \Leftrightarrow M \cup N = M \cap N
\end{equation}

\begin{proof}
 \begin{align}
  M = N & \Rightarrow ((x \in M) \Leftrightarrow (x \in N))\\
  & \Rightarrow ((x \in M \lor x \in M) \Leftrightarrow (x \in N \land x \in N))\\
  & \Rightarrow ((x \in M \lor x \in N) \Leftrightarrow (x \in M \land x \in N))\\
  & \Rightarrow M \cup N = M \cap N
 \end{align}
 
 \begin{align}
  M \cup N = M \cap N & \Rightarrow M = N
 \end{align}
 
 Es gibt drei Arten wie $M$ und $N$ zueinander in Beziehung stehen können:
 \begin{itemize}
  \item $M \supset N$: $M = \{0, 1, 2\}, N = \{0, 1\}$.
   \begin{align}
    \{0, 1, 2\} \cup \{0, 1\} & = \{0, 1, 2\} \cap \{0, 1\}\\
    \{0, 1, 2\} & = \{0, 1\} \qquad \text{Widerspruch!}
   \end{align}
  \item $M \subset N$: Das gleiche wie der erste Fall, weil $\cup$ und $\cap$ symmetrisch sind.
  \item $M = N$: Als letzte verbleibende Möglichkeit muss diese zutreffen.
 \end{itemize}
\end{proof}

\section*{Übung 2}

Zu zeigen
\begin{equation}
 M \cup N = N \cap O \Rightarrow M \subset N \subset O
\end{equation}

\begin{proof}
 Alle Elemente aus $M$ und $N$ sind auch in $N$ und $O$ enthalten. $M$ muss also eine Untermenge von $N$ und $O$ sein, weil $M$ sonst Elemente enthalten könnte, die nicht in $N$ und $O$ zu finden sind.
 Aus dem gleichen Grund muss auch $N$ eine Untermenge von $N$ und $O$ sein.
\end{proof}

\section*{Übung 3}

\begin{align}
 M = \{x_0, \dots, x_n, y_0, \dots, y_k\}\\
 N = \{x_0, \dots, x_n, z_0, \dots, z_p\}
\end{align}

\subsection*{1}

\begin{align}
 \mathfrak{P}(M \cap N) = \mathfrak{P}(\{x_0, \dots, x_n\})
\end{align}

Die Potenzmengen von $M$ und $N$ teilen sich genau die Untermengen, die nur die Elemente von $x_0, \dots, x_n$ enthalten.
Es gilt also:
\begin{equation}
 \mathfrak{P}(M \cap N) = \mathfrak{P}(M) \cap \mathfrak{P}(N)
\end{equation}

\subsection*{2}

Die rechte Seite muss eine Untermenge der linken Seite sein, weil die Potenzmenge von $M \cup N$ auch Teilmengen enthält, die Elemente aus $\{y_0, \dots, y_k\}$ und $\{x_0, \dots, x_p\}$ enthalten, wobei das auf der rechten Seite nicht möglich ist, weil die Potenzmengen von $M$ und $N$ alleine keine solchen Untermengen enthalten.

Es gilt die Gleicheit, wenn $M \subset N$ oder $M \supset N$. 

\section*{Übung 4}

\subsection*{1}

Zu zeigen
\begin{equation}
 f(n) := 0^2 + 1^2 + 2^2 + \dots + n^2 = \frac{n(n + 1)(2n + 1)}{6}
\end{equation}

Ich werde es mit vollständiger Induktion beweisen, wobei Gleichung (1) die Aussage P darstellt.

$P(0)$ ist wahr, weil $f(0) = 0^2 = \frac{0 * (0 + 1)(2 * 0 + 1)}{6}$.

Wenn $P$ für $n$ gilt, dann gilt auch
\begin{align}
 0^2 + 1^2 + 2^2 + \dots + n^2 + (n + 1)^2 & = \frac{n(n + 1)(2n + 1)}{6} + (n + 1)^2\\
 & = \frac{n(n + 1)(2n + 1) + 6 * (n + 1)^2}{6}\\
 & = \frac{(n + 1)(n(2n + 1) + 6(n + 1))}{6}\\
 & = \frac{(n + 1)(2n^2 + 7n + 6)}{6}\\
 & = \frac{(n + 1)((n + 2)(2n + 3)}{6}\\
 & = \frac{(n + 1)((n + 1) + 1)(2(n + 1) + 1)}{6}\\
 & = f(n + 1)
\end{align}

\subsection*{2}

Zu zeigen
\begin{equation}
 f(n) := 0^3 + 1^3 + 2^3 + \dots + n^3 = \frac{n^2(n + 1)^2}{4}
\end{equation}

\begin{equation}
 P(n) := f(n) \text{ ist richtig}
\end{equation}

$P(0)$ ist wahr, weil $f(0) = 0^3 = \frac{0^2(0 + 1)^2}{4} = 0$.

Angenommen $P(n)$ ist wahr, dann ist auch auch $P(n + 1)$ wahr.

\begin{proof}
 \begin{align}
  0^3 + 1^3 + 2^3 + \dots + n^3 + (n + 1)^3 & = \frac{n^2(n + 1)^2}{4} + (n + 1)^3\\
  & = \frac{n^2(n + 1)^2 + 4(n + 1)^3}{4}\\
  & = \frac{(n^2 + 4(n + 1))(n + 1)^2}{4}\\
  & = \frac{(n^2 + 4n + 4)(n + 1)^2}{4}\\
  & = \frac{(n + 2)^2(n + 1)^2}{4}\\
  & = \frac{(n + 1)^2((n + 1) + 1)^2}{4}\\
  & = f(n + 1)
 \end{align}
\end{proof}

$f$ ist also die korrekte Funktionsvorschrift, bewiesen mit vollständiger Induktion.

\end{document}
