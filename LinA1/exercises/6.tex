\documentclass[a4paper,10pt]{article}
\usepackage[utf8]{inputenc}
\usepackage[german]{babel}
\usepackage{amsmath}
\usepackage{amssymb}
\usepackage{amsthm}

\title{LinA1, Übungsblatt 6}
\author{Marten Lienen (2126759), Gruppe 1}

\newtheorem{lemma}{Lemma}

\begin{document}

\maketitle

\section*{Übung 1}

\begin{lemma}
 Die Summe zweier Unterräume ist ebenfalls ein Unterraum.
\end{lemma}

\begin{proof}
 Seien $U, V$ Unterräume von $W$.
 Zu zeigen ist, dass $(U + V, +)$ eine Untergruppe von $(W, +)$ ist und $xp \in U + V$ für alle $x \in K, p \in U + V$.
 
 Sei $p \in U + V$, sodass $p = u + v$ mit $u \in U, v \in V$.
 $U + V$ enthält ein neutrales Element $0_{U + V}$, weil $U, V$ Unterräume sind.
 \begin{equation}
  0_{U + V} = 0_U + 0_V
 \end{equation}
 \begin{equation}
  p + 0_{U + V} = u + v + 0_U + 0_V = (u + 0_U) + (v + 0_V) = u + v = p
 \end{equation}
 
 Seien $p, p^{-1} \in U + V$, sodass $p = u + v$ und $p^{-1} = u' + v'$ mit $u, u' \in U, v, v' \in V$.
 \begin{align}
  & 0 = p + p^{-1}\\
  \Leftrightarrow p^{-1} = -p = -(u + v) = (-u) + (-v)
 \end{align}
 Da $U$ und $V$ Unterräume sind, sind auch $-u \in U$ und $-v \in V$ und somit ist auch das Inverse $p^{-1} \in U + V$.
 
 Seien $p, q \in U + V$, sodass $p = u + v$ und $q = u' + v'$ mit $u, u' \in U, v, v' \in V$.
 \begin{equation}
  p + q = u + v + u' + v' = (u + u') + (v + v')
 \end{equation}
 Da $U$ und $V$ Unterräume sind, sind auch $u + u' \in U$ und $v + v' \in V$ und somit $p + q \in U + V$.
 
 $(U + V, +)$ ist eine Untergruppe von $(W, +)$.
 
 Seien $p \in U + V$ und $x \in K$, sodass $p = u + v$ mit $u \in U, v \in V$.
 Da das Distributivgesetz in $W$ gilt und $u, v \in W$, gilt es auch in
 \begin{equation}
  xp = x(u + v) = xu + xv
 \end{equation}
 Da $U$ und $V$ Unterräume sind, sind $xu \in U$ und $xv \in V$ und deshalb $xp \in U + V$.
 
 Es gilt also $xp \in U + V$ für alle $x \in K, p \in U + V$.
\end{proof}

\begin{proof}
 Da $U_1, U_2, U_3$ Unterräume von $V$ sind, können wir mit ihnen beliebig die Dimensionsformel anwenden.
 \begin{align}
  & dim U_1 + dim U_2 = dim(U_1 + U_2) + dim(U_1 \cap U_2)\\
  \Leftrightarrow & dim(U_1 + U_2) = dim U_1 + dim U_2 - dim(U_1 \cap U_2)
 \end{align}
 $U_1 + U_2$ ist nach dem Lemma ebenfalls ein Unterraum von $V$, sodass wir wiederum die Dimensionsformel anwenden können.
 \begin{equation}
  dim(U_1 + U_2) + dim U_3 = dim(U_1 + U_2 + U_3) + dim((U_1 + U_2) \cap U_3)
 \end{equation}
 Durch Einsetzen von $dim(U_1 + U_2) = dim U_1 + dim U_2 - dim(U_1 \cap U_2)$ erhalten wir
 \begin{align}
  & dim U_1 + dim U_2 - dim(U_1 \cap U_2) + dim U_3 = dim(U_1 + U_2 + U_3) + dim((U_1 + U_2) \cap U_3)\\
  \Leftrightarrow & dim U_1 + dim U_2 + dim U_3 = dim(U_1 + U_2 + U_3) + dim((U_1 + U_2) \cap U_3) + dim(U_1 \cap U_2)
 \end{align}
\end{proof}

\section*{Übung 2}

\begin{proof}
 Mit dem Rangsatz können wir folgende Ungleichung aufstellen
 \begin{align*}
  & dim W = dim Ker(g) + Rg(g)\\
  \Leftrightarrow & Rg(g) = dim W - dim Ker(g)\\
  \Rightarrow & Rg(g) \le dim W
 \end{align*}
 Die letzte Umformung ist gültig, da $dim Ker(g) \ge 0$, weil eine Menge und somit eine Basis nicht weniger als $0$ Elemente haben kann.
 
 Wir wissen ebenfalls, dass
 \begin{equation}
  dim V = dim Ker(f) + Rg(f)
 \end{equation}
 und
 \begin{equation}
  dim V = dim Ker(g \circ f) + Rg(g \circ f)
 \end{equation}
 Durch Gleichsetzen erhalten wir
 \begin{align*}
  & dim Ker(f) + Rg(f) = dim Ker(g \circ f) + Rg(g \circ f)\\
  \Leftrightarrow & Rg(f) = Rg(g \circ f) + dim Ker(g \circ f) - dim Ker(f)
 \end{align*}
 Da $g$ linear ist, bildet es $0$ auf $0$ ab.
 Jedes Element, das in $Ker(f)$ ist, wird also auf ein Element in $Ker(g)$ abgebildet und somit in $Ker(g \circ f)$.
 Zusätzlich könnte $Im(f)$ noch weitere Elemente enthalten, die durch $g$ auf $0$ abgebildet werden und deshalb ebenfalls in $Ker(g \circ f)$ sind.
 Daher gilt
 \begin{align}
  & Ker(f) \subseteq Ker(g \circ f)\\
  \Rightarrow & dim Ker(f) \le dim Ker(g \circ f)
 \end{align}
 Wenden wir dies auf $Rg(f) = Rg(g \circ f) + dim Ker(g \circ f) - dim Ker(f)$ an, erhalten wir
 \begin{equation}
  Rg(f) \le Rg(g \circ f)
 \end{equation}

 Aus $Rg(f) \le Rg(g \circ f)$ und $Rg(g) \le dim W$ folgt
 \begin{equation}
  Rg(f) + Rg(g) \le Rg(g \circ f) + dim W
 \end{equation}
\end{proof}

\section*{Übung 3}

\begin{proof}
 Sei $(f(v_1), \dots, f(v_n))$ ein linear unabhängiges System.
 \begin{equation}
  x_1v_2 + x_2v_2 + \dots + x_nv_n = 0_V
 \end{equation}
 Zu zeigen ist, dass $x_k = 0, 1 \le k \le n$.
 Durch Anwendung von $f$ erhalten wir
 \begin{align*}
  & f(x_1v_2 + x_2v_2 + \dots + x_nv_n) = f(0_V)\\
  \Rightarrow & x_1f(v_2) + x_2f(v_2) + \dots + x_nf(v_n) = 0_W\\
  \Rightarrow & x_1 = x_2 = \dots = x_n = 0
 \end{align*}
\end{proof}

\section*{Übung 4}

\subsection*{1}

\begin{proof}
 ``$\Rightarrow$'': Sei $f$ injektiv.
 Wir setzen die Linearkombination von $(f(v_1), \dots, f(v_n)) = 0$.
 \begin{equation}
  x_1f(v_1) + \dots + x_nf(v_n) = f(x_1v_1 + \dots + x_nv_n) = 0
 \end{equation}
 $(v_1, \dots, v_n)$ ist linear unabhängig.
 In der Gleichung
 \begin{equation}
  y_1v_1 + \dots + y_nv_n = 0
 \end{equation}
 sind also alle $y_k = 0$.
 Da lineare Abbildungen immer $0$ auf $0$ abbilden, wird auch $y_1v_1 + \dots + y_nv_n$ durch $f$ auf $0$ abgebildet.
 Weil $f$ injektiv ist, gilt
 \begin{align*}
  & f(x_1v_1 + \dots + x_nv_n) = 0 = f(y_1v_1 + \dots + y_nv_n)\\
  \Rightarrow & x_1v_1 + \dots + x_nv_n = y_1v_1 + \dots + y_nv_n = 0
 \end{align*}
 Da $(v_1, \dots, v_n)$ linear unabhängig ist, müssen auch $x_1 = x_2 = \dots = x_n = 0$ sein.
 $(f(v_1), \dots, f(v_n))$ ist also ebenfalls linear unabhängig.
 
 ``$\Leftarrow$'': Sei $(f(v_1), \dots, f(v_n))$ linear unabhängig.
 Seien $v, w \in V$ mit $f(v) = f(w)$.
 Weil $(v_1, \dots, v_n)$ eine Basis von $V$ ist, kann man sie schreiben als
 \begin{align*}
  & v = x_1v_1 + \dots + x_nv_n\\
  & w = y_1v_1 + \dots + y_nv_n
 \end{align*}
 Daraus folgt
 \begin{align*}
  & f(v) = f(w)\\
  \Rightarrow & x_1f(v_1) + \dots + x_nf(v_n) = y_1f(v_1) + \dots + y_nf(v_n)\\
  \Rightarrow & (x_1 - y_1)f(v_1) + \dots + (x_n - y_n)f(v_n) = 0
 \end{align*}
 Da $(f(v_1), \dots, f(v_n))$ linear unabhängig ist, müssen in der letzten Gleichung alle Koeffizienten $0$ sein, d.h.
 \begin{equation}
  (x_k - y_k = 0 \Rightarrow x_k = y_k) \forall 1 \le k \le n
 \end{equation}
 Daraus folgt $v = w$ und $f$ ist injektiv.
\end{proof}

\subsection*{2}

\begin{proof}
 ``$\Rightarrow$'': Sei $f$ surjektiv.
 Zu zeigen ist, dass jedes $w \in W$ als Linearkombination von $(f(v_1), \dots, f(v_n))$ dargestellt werden kann.
 Sei $w \in W$.
 Weil $f$ surjektiv ist, gibt es ein $v \in V$ mit $f(v) = w$.
 Da $(v_1, \dots, v_n)$ eine Basis von $V$ ist, ist $v$ eine Linearkombination von $(v_1, \dots, v_n)$.
 \begin{equation}
  v = x_1v_1 + \dots + x_nv_n
 \end{equation}
 \begin{equation}
  f(v) = f(x_1v_1 + \dots + x_nv_n) = x_1f(v_1) + \dots + x_nf(v_n) = w
 \end{equation}
 $(f(v_1), \dots, f(v_n))$ ist also ein EZS von $W$.
 
 ``$\Leftarrow$'': Sei $(f(v_1), \dots, f(v_n))$ ein EZS von $W$.
 Sei $w \in W$.
 Weil $(f(v_1), \dots, f(v_n))$ ein EZS ist, kann man $w$ schreiben als
 \begin{equation}
  w = x_1f(v_1) + \dots + x_nf(v_n) = f(x_1v_1 + \dots + x_nv_n)
 \end{equation}
 Da $(v_1, \dots, v_n)$ eine Basis von $V$ ist, ist $x_1v_1 + \dots + x_nv_n$ ein Vektor $v$ in $V$ mit $f(v) = w$.
 Somit ist $f$ surjektiv.
\end{proof}

\subsection*{3}

\begin{proof}
 ``$\Rightarrow$'': Wenn $f$ bijektiv ist, ist surjektiv und injektiv.
 Nach (1) ist das System $(f(v_1), \dots, f(v_n))$ linear unabhängig und nach (2) ist es ein EZS.
 Somit ist es nach Definition eine Basis.
 
 ``$\Leftarrow$'': Wenn das System $(f(v_1), \dots, f(v_n))$ eine Basis ist, ist es nach Definition ein linear unabhängiges EZS.
 Nach (1) ist $f$ injektiv und nach (2) surjektiv.
 Somit ist $f$ nach Definition bijektiv.
\end{proof}

\section*{Übung 5}

\subsection*{1}

\subsubsection*{$(Hom(V, W), +)$ ist eine kommutative Gruppe}

\begin{proof}
 Seien $f, g \in Hom(V, W), v \in V$.
 \begin{equation}
  (f + g)(v) = f(v) + g(v) = f(v) \Rightarrow g(v) = 0
 \end{equation}
 Das neutrale Element ist die Nullabbildung, die linear ist und somit ein Homomorphismus.

 Seien $f, g \in Hom(V, W), v \in V$.
 \begin{equation}
  (f + g)(v) = f(v) + g(v) = 0 \Rightarrow g(v) = -f(v)
 \end{equation}
 Das inverse Element von $f$ ist die Abbildung, die $v$ auf $-f(v)$ abbildet.
 Dieser Wert existiert für alle $v \in V$, weil $V$ ein Vektorraum ist.
 \begin{align*}
  g(x_1v_1 + x_2v_2) & = -f(x_1v_1 + x_2v_2) = -(x_1f(v_1) + x_2f(v_2))\\
  & = -x_1f(v_1) - x_2f(v_2) = x_1f(-v_1) + x_2f(-v_2)\\
  & = x_1g(v_1) + x_2g(v_2)
 \end{align*}
 $g$, das Inverse von $f$, ist ein Homomorphismus.

 Seien $f, g, h \in Hom(V, W), v \in V$.
 \begin{equation}
  (f + (g + h))(v) = f(v) + g(v) + h(v) = ((f + g) + h)(v)
 \end{equation}
 $+$ ist assoziativ.

 Seien $f, g \in Hom(V, W), v \in V$.
 \begin{equation}
  (f + g)(v) = f(v) + g(v) = g(v) + f(v) = (g + f)(v)
 \end{equation}
 $+$ ist kommutativ.
 
 Seien $f, g, h \in Hom(V, W), v \in V$.
 \begin{equation}
  h(v) := (f + g)(v)
 \end{equation}
 \begin{align*}
  h(x_1v_1 + x_2v_2) & = f(x_1v_1 + x_2v_2) + g(x_1v_1 + x_2v_2)\\
  & = x_1f(v_1) + x_2f(v_2) + x_1g(v_1) + x_2g(v_2)\\
  & = x_1(f(v_1) + g(v_1)) + x_2(f(v_2) + g(v_2)) = x_1h(v_1) + x_2h(v_2)
 \end{align*}
 Die Verknüpfung von $f$ und $g$ ist auch ein Homomorphismus und somit ist $(Hom(V, W), +)$ abgeschlossen.
\end{proof}

\subsubsection*{$((xy)f)(v) = x((yf)(v))$ für alle $f \in Hom(V, W)$}

\begin{proof}
 Seien $x, y \in K, f \in Hom(V, W), v \in V$.
 \begin{equation}
  ((xy)f)(v) = (xy) * f(v) = x(y * f(v)) = x((yf)(v))
 \end{equation}
 Das Assoziativgesetz gilt, weil $f(v) \in W$ und $W$ ein Vektorraum ist.
\end{proof}

\subsubsection*{$1_K * f = f$ für alle $f \in Hom(V, W)$}

\begin{proof}
 Seien $1_K \in K, f \in Hom(V, W), v \in V$.
 \begin{equation}
  (1_K * f)(v) = 1_K * f(v) = f(v)
 \end{equation}
 $1_K * f(v) = f(v)$ gilt, weil $f(v) \in W$ und $W$ ein Vektorraum ist.
\end{proof}

\subsubsection*{$x * (f + g)(v) = (x * f)(v) + (x * g)(v)$ für alle $f, g \in Hom(V, W)$}

\begin{proof}
 Seien $x \in K, f, g \in Hom(V, W), v \in V$.
 \begin{equation}
  x * (f + g)(v) = x * (f(v) + g(v)) = x * f(v) + x * g(v) = (x * f)(v) + (x * g)(v)
 \end{equation}
 Das Distributivgesetz gilt, weil $f(v), g(v) \in W$ und $W$ ein Vektorraum ist.
\end{proof}

\subsubsection*{$((x + y) * f)(v) = (x * f)(v) + (y * f)(v)$ für alle $f \in Hom(V, W)$}

\begin{proof}
 Seien $x, y \in K, f \in Hom(V, W), v \in V$.
 \begin{equation}
  ((x + y) * f)(v) = (x + y) * f(v) = x * f(v) + y * f(v) = (x * f)(v) + (y * f)(v)
 \end{equation}
 Das Distributivgesetz gilt, weil $f(v) \in W$ und $W$ ein Vektorraum ist.
\end{proof}

\subsubsection*{$Hom(V, W)$ ist abgeschlossen bezüglich der Skalarmultiplikation}

\begin{proof}
 Seien $x, x_1, x_2 \in K, f, g \in Hom(V, W), v \in V$.
 \begin{equation}
  g(v) := (x * f)(v) = x * f(v)
 \end{equation}
 \begin{align*}
  g(x_1v_1 + x_2v_2) & = x * f(x_1v_1 + x_2v_2) = x * (x_1f(v_1) + x_2f(v_2))\\
  & = xx_1f(v_1) + xx_2f(v_2) = x_1g(v_1) + x_2g(v_2)
 \end{align*}

\end{proof}

\subsection*{2}

\begin{proof}
 $f(\langle v_1, \dots, v_n \rangle) \subseteq \langle f(v_1), \dots, f(v_n) \rangle$: Seien $v \in f(\langle v_1, \dots, v_n \rangle)$ und $x_1, \dots, x_n \in K$.
 Weil $f$ eine lineare Abbildung ist, gilt
 \begin{equation}
  v = f(x_1v_1 + \dots x_nv_n) = x_1f(v_1) + \dots + x_nf(v_n)
 \end{equation}
 Also ist $v \in \langle f(v_1), \dots, f(v_n) \rangle$.
 
 $f(\langle v_1, \dots, v_n \rangle) \supseteq \langle f(v_1), \dots, f(v_n) \rangle$: Seien $v \in \langle f(v_1), \dots, f(v_n) \rangle$ und $x_1, \dots, x_n \in K$.
 Weil $f$ eine lineare Abbildung ist, gilt
 \begin{equation}
  v = x_1f(v_1) + \dots + x_nf(v_n) = f(x_1v_1 + \dots x_nv_n)
 \end{equation}
 Also ist $v \in f(\langle v_1, \dots, v_n \rangle)$.
\end{proof}

\end{document}
