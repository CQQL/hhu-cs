\documentclass[a4paper,10pt]{article}
\usepackage[utf8]{inputenc}
\usepackage[german]{babel}
\usepackage{amsmath}
\usepackage{amssymb}
\usepackage{amsthm}
\usepackage{stmaryrd}

\title{LinA1, Übungsblatt 6}
\author{Marten Lienen (2126759), Gruppe 1}

\newtheorem*{claim}{Behauptung}

\begin{document}

\maketitle

\section*{Übung 1}

\section*{Übung 2}

\section*{Übung 3}

\section*{Übung 4}

\subsection*{1}

\subsection*{2}

\subsection*{3}

\begin{proof}
 ``$\Rightarrow$'': Wenn $f$ bijektiv ist, ist surjektiv und injektiv.
 Nach (1) ist das System $(f(v_1), \dots, f(v_n))$ linear unabhängig und nach (2) ist es ein EZS.
 Somit ist es nach Definition eine Basis.
 
 ``$\Leftarrow$'': Wenn das System $(f(v_1), \dots, f(v_n))$ eine Basis ist, ist es nach Definition ein linear unabhängiges EZS.
 Nach (1) ist $f$ injektiv und nach (2) surjektiv.
 Somit ist $f$ nach Definition bijektiv.
\end{proof}

\section*{Übung 5}

\end{document}
