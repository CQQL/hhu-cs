\documentclass[a4paper,10pt]{article}
\usepackage[utf8]{inputenc}
\usepackage{amsmath}
\usepackage{amssymb}
\usepackage{amsthm}
\usepackage{stmaryrd}

\title{LinA1, Übungsblatt 10}
\author{Marten Lienen (2126759), Gruppe 1; Fabian Schmittmann (2083559), Gruppe 5}

\begin{document}

\maketitle

\section*{Übung 1}

\subsection*{Teil 1}

\begin{align*}
 A = & \begin{pmatrix}
  1 & 2 & 1 & 2 & 1 & 2\\
  2 & 5 & 4 & 5 & 4 & 5\\
  1 & 4 & 6 & 6 & 6 & 6\\
  2 & 5 & 6 & 9 & 7 & 11\\
 \end{pmatrix}\\
 \longrightarrow T_{4,1}(-2)T_{3,1}(-1)T_{2,1}(-2)
 & \begin{pmatrix}
  1 & 2 & 1 & 2 & 1 & 2\\
  2 & 5 & 4 & 5 & 4 & 5\\
  1 & 4 & 6 & 6 & 6 & 6\\
  2 & 5 & 6 & 9 & 7 & 11\\
 \end{pmatrix}\\
 \longrightarrow T_{4,2}(-1)T_{3,2}(-2)T_{1,2}(-2)
 & \begin{pmatrix}
  1 & 2 & 1 & 2 & 1 & 2\\
  0 & 1 & 2 & 1 & 2 & 1\\
  0 & 2 & 5 & 4 & 5 & 4\\
  0 & 1 & 4 & 5 & 5 & 7\\
 \end{pmatrix}\\
 \longrightarrow T_{4,3}(-2)T_{2,3}(-2)T_{1,3}(3)
 & \begin{pmatrix}
  1 & 0 & -3 & 0 & -3 & 0\\
  0 & 1 & 2 & 1 & 2 & 1\\
  0 & 0 & 1 & 2 & 3 & 2\\
  0 & 0 & 2 & 4 & 3 & 6\\
 \end{pmatrix}\\
 \longrightarrow T_{4,4}(-\frac{1}{3})
 & \begin{pmatrix}
  1 & 0 & 0 & 6 & 6 & 6\\
  0 & 1 & 0 & -3 & -4 & -3\\
  0 & 0 & 1 & 2 & 3 & 2\\
  0 & 0 & 0 & 0 & -3 & 2\\
 \end{pmatrix}\\
 \longrightarrow T_{3,4}(-3)T_{2,4}(4)T_{1,4}(-6)
 & \begin{pmatrix}
  1 & 0 & 0 & 6 & 6 & 6\\
  0 & 1 & 0 & -3 & -4 & -3\\
  0 & 0 & 1 & 2 & 3 & 2\\
  0 & 0 & 0 & 0 & 1 & -\frac{2}{3}\\
 \end{pmatrix}\\
 \longrightarrow
 & \begin{pmatrix}
  1 & 0 & 0 & 6 & 0 & 10\\
  0 & 1 & 0 & -3 & 0 & -\frac{17}{3}\\
  0 & 0 & 1 & 2 & 0 & 4\\
  0 & 0 & 0 & 0 & 1 & -\frac{2}{3}\\
 \end{pmatrix} = ZSF(A)
\end{align*}

\subsection*{Teil 2}

\begin{equation}
 \begin{pmatrix}
  1 & 0 & 0 & 6 & 0 & 10\\
  0 & 1 & 0 & -3 & 0 & -\frac{17}{3}\\
  0 & 0 & 1 & 2 & 0 & 4\\
  0 & 0 & 0 & 0 & 1 & -\frac{2}{3}\\
 \end{pmatrix} \cdot \begin{pmatrix}a\\b\\c\\d\\e\\f\end{pmatrix} =
 \begin{pmatrix}
  a + 6d + 10f\\
  b - 3d - \frac{17}{3}f\\
  c + 2d + 4f\\
  e - \frac{2}{3}f
 \end{pmatrix}
\end{equation}
\begin{align*}
 & \begin{pmatrix}
  a + 6d + 10f\\
  b - 3d - \frac{17}{3}f\\
  c + 2d + 4f\\
  e - \frac{2}{3}f
 \end{pmatrix} = 0
 \Leftrightarrow
 \begin{pmatrix}
  a\\
  b\\
  c\\
  e
 \end{pmatrix} =
 \begin{pmatrix}
  -6d - 10f\\
  3d + \frac{17}{3}f\\
  -2d - 4f\\
  \frac{2}{3}f
 \end{pmatrix}\\
 \Rightarrow & Ker(ZSF(A)) = 
  \begin{pmatrix}
   -6d - 10f\\
   3d + \frac{17}{3}f\\
   -2d - 4f\\
   d\\
   \frac{2}{3}f\\
   f
  \end{pmatrix} = 
  d\begin{pmatrix}
   -6\\
   3\\
   -2\\
   1\\
   0\\
   0
  \end{pmatrix} +
  f\begin{pmatrix}
   -10\\
   \frac{17}{3}\\
   -4\\
   0\\
   \frac{2}{3}\\
   1
  \end{pmatrix}
  = \left\langle
  \begin{pmatrix}
   -6\\
   3\\
   -2\\
   1\\
   0\\
   0
  \end{pmatrix},
  \begin{pmatrix}
   -10\\
   \frac{17}{3}\\
   -4\\
   0\\
   \frac{2}{3}\\
   1
  \end{pmatrix}
  \right\rangle
\end{align*}
Nach der Vorlesung gilt $Ker(A) = Ker(ZSF(A))$.
Da man $A$ mit einem beliebigen Vektor aus $\mathbb{R}^6$ multiplizieren kann, ist die Dimension der Quellmenge von $A$ $6$.
Mit dem Rangsatz ergibt sich $Rg(A) = 6 - dim Ker(A) = 4$.

\section*{Übung 2}

\subsection*{Teil 1}

\begin{align*}
 ZSF(A) & = T_{1,3}(1)T_{2,3}(-2)T_{1,2}(-2)T_{3,2}(6)T_{2,2}(-\frac{4}{3})T_{3,1}(-7)T_{2,1}(-4)A\\
  & =
  \begin{pmatrix}
   1 & 0 & 0\\
   0 & 1 & 0\\
   0 & 0 & 1
  \end{pmatrix}
\end{align*}
Nach der Vorlesung ist $A$ genau dann invertierbar, wenn $ZSF(A) = I_3$.

\subsection*{Teil 2}

Folglich ist $T_{1,3}(1)T_{2,3}(-2)T_{1,2}(-2)T_{3,2}(6)T_{2,2}(-\frac{4}{3})T_{3,1}(-7)T_{2,1}(-4)$ das Inverse von $A$.
\begin{align*}
 A^{-1} & = T_{1,3}(1)T_{2,3}(-2)T_{1,2}(-2)T_{3,2}(6)T_{2,2}(-\frac{4}{3})T_{3,1}(-7)T_{2,1}(-4)I_3\\
 & =
  \begin{pmatrix}
   -\frac{2}{3} & -\frac{4}{3} & 1\\
   -\frac{2}{3} & \frac{11}{3} & -2\\
   1 & -2 & 1
  \end{pmatrix}
\end{align*}

\section*{Übung 3}

\subsection*{Teil 1}

\begin{align*}
 & \begin{cases}
  3x + 2y + 6z = a\\
  4x + 5y + 12z = b\\
  2x + 2y + 5z = c
 \end{cases}\\
 \Leftrightarrow & 
 \begin{cases}
  x = \frac{1}{3}a - \frac{2}{3}y - 2z\\
  y = \frac{1}{5}b - \frac{4}{5}x - \frac{12}{5}z\\
  z = \frac{1}{5}c - \frac{2}{5}x - \frac{2}{5}y
 \end{cases}\\
 \Leftrightarrow & 
 \begin{cases}
  x = \frac{1}{3}a - \frac{2}{3}y - 2z\\
  y = \frac{1}{5}b - \frac{4}{15}a + \frac{8}{15}y - \frac{4}{5}z \Leftrightarrow y = \frac{3}{7}b - \frac{4}{5}a - \frac{4}{5}z\\
  z = \frac{1}{5}c - \frac{2}{5}x - \frac{2}{5}y
 \end{cases}\\
 \Leftrightarrow & 
 \begin{cases}
  x = \frac{13}{15}a - \frac{2}{7}b - \frac{22}{15}z\\
  y = \frac{3}{7}b - \frac{4}{5}a - \frac{4}{5}z\\
  z = \frac{1}{5}c - \frac{2}{5}x - \frac{6}{35}b + \frac{8}{35}a \Leftrightarrow z = \frac{7}{27}c - \frac{14}{27}x - \frac{6}{27}b + \frac{8}{27}a
 \end{cases}\\
 \Leftrightarrow & 
 \begin{cases}
  x = \frac{329}{495}a - \frac{31}{231}b - \frac{35}{198}c + \frac{35}{99}x \Leftrightarrow \frac{64}{99}x = \frac{329}{495}a - \frac{31}{231}b - \frac{35}{198}c\\
  y = \frac{3}{7}b - \frac{4}{5}a - \frac{4}{5}z\\
  z = \frac{7}{27}c - \frac{14}{27}x - \frac{6}{27}b + \frac{8}{27}a
 \end{cases}\\
\end{align*}

\subsection*{Teil 2}

\begin{align*}
 & \begin{cases}
  x^3 y^2 z^6 = 1\\
  x^4 y^5 z^{12} = 2\\
  x^2 y^2 z^5 = 3
 \end{cases}\\
 \Leftrightarrow & 
 \begin{cases}
  x^6 y^4 z^{12} = 1\\
  x^4 y^5 z^{12} = 2\\
  x^2 = \frac{3}{y^2 z^5}
 \end{cases}\\
 \Leftrightarrow & 
 \begin{cases}
  \frac{3^3}{y^6 z^{15}} y^4 z^{12} = 3^3 y^{-2} z^{-3} = 1 \Leftrightarrow y^{-2} = \frac{1}{3^3 z^{-3}} \Leftrightarrow y^2 = \frac{3^3}{z^3}\\
  \frac{3^2}{y^4 z^{10}} y^5 z^{12} = 3^2 y z^2 = 2\\
  x^2 = \frac{3}{y^2 z^5}
 \end{cases}\\
 \Leftrightarrow & 
 \begin{cases}
  y^2 = \frac{3^3}{z^3}\\
  3^4 y^2 z^4 = 2^2\\
  x^2 = \frac{3}{y^2 z^5}
 \end{cases}\\
 \Leftrightarrow & 
 \begin{cases}
  y^2 = \frac{3^3}{z^3} = \frac{3^{24}}{2^6} \Rightarrow y = \frac{3^{12}}{2^3}\\
  3^4 \frac{3^3}{z^3} z^4 = 3^7 z = 2^2 \Rightarrow z = \frac{2^2}{3^7}\\
  x^2 = \frac{3}{\frac{3^3}{z^3} z^5} = \frac{1}{3^2 z^2} = \frac{1}{3^2 \frac{2^4}{3^{14}}} = \frac{1}{\frac{2^4}{3^{12}}} \Rightarrow x = \frac{3^{6}}{2^2}
 \end{cases}\\
\end{align*}

\section*{Übung 4}

\subsection*{Teil 1}

\begin{align*}
 R_\alpha R_\beta & =
 \begin{pmatrix}
  \cos(\alpha)\cos(\beta) + \sin(\alpha)\sin(\beta) & -\cos(\alpha)\sin(\beta) - \sin(\alpha)\cos(\beta)\\
  \sin(\alpha)\cos(\beta) + \cos(\alpha)\sin(\beta) & \sin(\alpha)\sin(\beta) + \cos(\alpha)\cos(\beta)
 \end{pmatrix}\\
 & =
 \begin{pmatrix}
  \cos(\alpha + \beta) & -\sin(\alpha + \beta)\\
  \sin(\alpha + \beta) & \cos(\alpha + \beta)
 \end{pmatrix}
 = R_{\alpha + \beta}
\end{align*}

\subsection*{Teil 2}

Wenn $\cos \alpha = 0$, gilt
\begin{equation}
 R_\alpha =
 \begin{pmatrix}
  0 & 1\\
  1 & 0
 \end{pmatrix} = R_\alpha^{-1}
\end{equation}
oder
\begin{equation}
 R_\alpha =
 \begin{pmatrix}
  0 & -1\\
  -1 & 0
 \end{pmatrix} = R_\alpha^{-1}
\end{equation}
Wenn $\sin \alpha = 0$, gilt
\begin{equation}
 R_\alpha =
 \begin{pmatrix}
  1 & 0\\
  0 & 1
 \end{pmatrix} = R_\alpha^{-1}
\end{equation}
oder
\begin{equation}
 R_\alpha =
 \begin{pmatrix}
  -1 & 0\\
  0 & -1
 \end{pmatrix} = R_\alpha^{-1}
\end{equation}

Wenn $\cos \alpha \ne 0$ und $\sin \alpha \ne 0$, betrachten wir
\begin{equation}
 \begin{pmatrix}
  \cos \alpha & -\sin \alpha\\
  \sin \alpha & \cos \alpha\\
 \end{pmatrix}
 \begin{pmatrix}
  a & b\\
  c & d\\
 \end{pmatrix} =
 \begin{pmatrix}
  1 & 0\\
  0 & 1\\
 \end{pmatrix}
\end{equation}
und es ergibt sich
\begin{align*}
 a & = \frac{\cos \alpha}{\cos^2 \alpha + \sin^2 \alpha}\\
 b & = \frac{\sin \alpha}{\cos^2 \alpha + \sin^2 \alpha}\\
 c & = -\frac{\sin \alpha}{\cos^2 \alpha + \sin^2 \alpha}\\
 d & = \frac{\sin \alpha - \sin^2 \alpha}{\cos^3 \alpha + \cos \alpha \sin^2 \alpha}
\end{align*}
Weil das Inverse existiert, ist $R_\alpha$ invertierbar.

\section*{Übung 5}

\subsection*{Teil 1}

Seien $A, B \in M_n(K)$ und $x, y \in K$.
\begin{align*}
 Tr(xA + yB) & = \sum_{k = 1}^n xa_{k,k} + yb_{k,k}\\
 & = \sum_{k = 1}^n xa_{k,k} + \sum_{k = 1}^n yb_{k,k}\\
 & = x\sum_{k = 1}^n a_{k,k} + y\sum_{k = 1}^n b_{k,k} = xTr(A) + yTr(B)
\end{align*}

\subsection*{Teil 2}

Sei $C = AB$ und $D = BA$.
Dann gilt
\begin{equation}
 c_{ii} = \sum_{k = 1}^n a_{ik} b_{ki}
\end{equation}
und
\begin{equation}
 d_{ii} = \sum_{k = 1}^n a_{ki} b_{ik}
\end{equation}
\begin{equation}
 Tr(C) = \sum_{q = 1}^n c_{qq} = \sum_{q = 1}^n \sum_{k = 1}^n a_{qk} b_{kq} = \sum_{q = 1}^n \sum_{k = 1}^n a_{kq} b_{qk} = \sum_{q = 1}^n d_{qq} = Tr(D)
\end{equation}
Dies gilt, weil jeder Summand $a_{qk}b_{kq}$ in der jeweils anderen Summe auftritt, wenn $q' = k$ und $k' = q$.

\end{document}
