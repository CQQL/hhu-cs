\documentclass[a4paper,10pt]{article}
\usepackage[utf8]{inputenc}
\usepackage[german]{babel}
\usepackage{amsmath}
\usepackage{amssymb}
\usepackage{amsthm}
\usepackage{stmaryrd}

\title{LinA1, Übungsblatt 12}
\author{Marten Lienen (2126759), Gruppe 1; Fabian Schmittmann (2083559), Gruppe 5}

\begin{document}

\maketitle

\section*{Übung 1}

\subsection*{Teil 1}

\subsection*{Teil 2}

\section*{Übung 2}

\subsection*{Teil 1}

Seien $q, p \in V$ und $x, y \in K$.
\begin{equation}
 f_v(x q + y p) = (v, xq + yp) = x(v, q) + y(v, p) = x f_v(q) + y f_v(p)
\end{equation}

\subsection*{Teil 2}

Seien $q, p, v \in V$ und $x, y \in K$.
\begin{align*}
 F(xq + yp)(v) & = f_{xq + yp}(v) = (xf_q + yf_p)(v) = xf_q(v) + yf_p(v)\\
 & = xF(q)(v) + yF(p)(v) = (xF(q) + yF(p))(v)
\end{align*}

\subsection*{Teil 3}

\begin{equation}
 F^{-1}(f_v) = v
\end{equation}
\begin{equation}
 F^{-1}(F(v)) = F^{-1}(f_v) = v \Rightarrow F^{-1} \circ F = id_V
\end{equation}
\begin{equation}
 F(F^{-1}(f_v)) = F(v) = f_v \Rightarrow F \circ F^{-1} = id_{V^\vee}
\end{equation}

\subsection*{Teil 4}

Sei $v \in M^\perp$.
Weil $(v, w) = 0 \quad \forall w \in M$ ist, gilt $F(v)(w) = 0 \quad \forall w \in M$ und $F(v) \in M^T$, also $F(M^\perp) \subset M^T$.

Sei $f \in M^T$.
Wir definieren $f$ mithilfe der kanonischen Dualbasis $(e_1^\vee, \dots, e_n^\vee)$, sodass
\begin{equation}
 f(v) = (\sum_{k = 1}^n x_k e_k^\vee)(v) =
  \left( \begin{pmatrix}
   x_1\\
   \vdots\\
   x_n
  \end{pmatrix}, v\right) = 0
\end{equation}
Also ist $f \in F(M^\perp)$ und $M^T \subset F(M^\perp)$ und letztendlich $F(M^\perp) = M^T$.

\section*{Übung 3}

\subsection*{Teil 1}

Da $V = \langle f, g, h \rangle$, ist $(f, g, h)$ ein EZS von $V$.
Es bleibt zu zeigen, dass $(f, g, h)$ linear unabhängig ist.
Sei $x \in [0, 1]$.
\begin{align}
 & af(x) + bg(x) + ch(x) = 0\\
 \Leftrightarrow &
 a + bx + cx^2 = 0
\end{align}
Wenn $c \ne 0$, hat die Gleichung maximal 2 Lösungen, sodass sie für maximal für 2 Werte aus $[0, 1]$ gelöst werden kann.
Folglich muss $a = b = c = 0$ sein.
$(f, g, h)$ ist also eine Basis von $V$.

\subsection*{Teil 2}

\begin{proof}
 Seien $f, g, h \in V$.
 
 \begin{equation}
  (f, g) = \int_0^1 f(t) g(t) dt = \int_0^1 g(t) f(t) dt = (g, f)
 \end{equation}
 
 \begin{equation}
  (f, f) = \int_0^1 f(t) f(t) dt = \int_0^1 f(t)^2 dt \ge 0
 \end{equation}
 
 Seien $x, y \in \mathbb{R}$.
 \begin{align*}
  (f, xg + yh) & = \int_0^1 f(t) (xg + yh)(t) dt\\
  & = \int_0^1 f(t)xg(t) + f(t)yh(t) dt\\
  & = \int_0^1 f(t)xg(t) + \int_0^1 f(t)yh(t) dt\\
  & = x\int_0^1 f(t)g(t) dt + y\int_0^1 f(t)h(t) dt\\
  & = x(f, g) + y(f, h)
 \end{align*}
 Aus der Symmetrie folgt, dass es auch für das erste Argument linear ist.

 $(,)$ ist also symmetrisch, positiv definit, bilinear und somit ein Skalarprodukt.
\end{proof}

\subsection*{Teil 3}



\section*{Übung 4}

\subsection*{Teil 1}

\subsection*{Teil 2}

\subsection*{Teil 3}

\end{document}
