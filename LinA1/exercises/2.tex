\documentclass[a4paper,10pt]{article}
\usepackage[utf8]{inputenc}
\usepackage{amssymb}
\usepackage{amsmath}
\usepackage{amsthm}
\usepackage[german]{babel}

\title{Übungsblatt 2, LinA1}
\author{Marten Lienen (2126759), Übungsgruppe 1}

\newtheorem*{claim}{Behauptung}
\newtheorem*{explanation}{Begründung}

\begin{document}

\maketitle

\section*{Übung 1}

\subsection*{1}

\begin{description}
 \item[``$\subset$'']
 \begin{align*}
  & \forall x \in f(A_1 \cup A_2) \exists y \in (A_1 \cup A_2) f(y) = x\\
  & \Rightarrow y \in A_1 \lor y \in A_2\\
  & \Rightarrow x \in f(A_1) \lor x \in f(A_2)\\
  & \Rightarrow x \in (f(A_1) \cup f(A_2))
 \end{align*}

 \item[``$\supset$'']
 \begin{align*}
  & \forall x \in (f(A_1) \cup f(A_2)) \exists y (y \in A_1 \lor y \in A_2) (f(y) = x)\\
  & \Rightarrow y \in (A_1 \cup A_2)\\
  & \Rightarrow x \in f(A_1 \cup A_2)
 \end{align*}
\end{description}

\subsection*{2}

\begin{description}
 \item[``$\subset$'']
 \begin{align*}
  & \forall x \in f(A_1 \cap A_2) \exists y (y \in A_1 \cap y \in A_2) f(y) = x\\
  & \Rightarrow y \in A_1 \land y \in A_2\\
  & \Rightarrow x \in f(A_1) \land f(A_2)\\
  & \Rightarrow x \in f(A_1) \cap f(A_2)
 \end{align*}
\end{description}

\subsection*{3}

\begin{description}
 \item[``$\subset$'']
 \begin{align*}
  & \forall x \in f^{-1}(B_1 \cup B_2) \exists y \in (B_1 \cup B_2) f^{-1}(y) = x\\
  & \Rightarrow y \in B_1 \lor y \in B_2\\
  & \Rightarrow x \in f^{-1}(B_1) \lor x \in f^{-1}(B_2)\\
  & \Rightarrow x \in (f^{-1}(B_1) \cup f^{-1}(B_2))
 \end{align*}
 
 \item[``$\supset$'']
 \begin{align*}
  & \forall x \in (f^{-1}(B_1) \cup f^{-1}(B_2)) \exists y (y \in B_1 \lor y \in B_2) (f^{-1}(y) = x)\\
  & \Rightarrow y \in (B_1 \cup B_2)\\
  & \Rightarrow x \in f^{-1}(B_1 \cup B_2)
 \end{align*}
\end{description}

\subsection*{4}

\begin{description}
 \item[``$\subset$'']
 \begin{align*}
  & \forall x \in f^{-1}(B_1 \cap B_2) \exists y (y \in B_1 \cap y \in B_2) f^{-1}(y) = x\\
  & \Rightarrow y \in B_1 \land y \in B_2\\
  & \Rightarrow x \in f^{-1}(B_1) \land f^{-1}(B_2)\\
  & \Rightarrow x \in f^{-1}(B_1) \cap f^{-1}(B_2)
 \end{align*}
 
 \item[``$\supset$'']
 \begin{align*}
  & \forall x \in (f^{-1}(B_1) \cap f^{-1}(B_2)) \exists y (y \in B_1 \land y \in B_2) f(y) = x\\
  & \Rightarrow y \in B_1 \cap B_2\\
  & \Rightarrow x \in f^{-1}(B_1 \cap B_2)
 \end{align*}
\end{description}

\subsection*{5}

\begin{description}
 \item[``$\subset$'']
 \begin{align*}
  & \forall x \in (f(A_1) \backslash f(A_2)) ((\exists y \in A_1 f(y) = x) \land (\forall z \in A_2 f(z) \ne x))\\
  & \Rightarrow (y \in A_1 \backslash A_2 \Rightarrow f(y) = x)\\
  & \Rightarrow x \in f(A_1 \backslash A_2)
 \end{align*}
\end{description}

\subsection*{6}

\begin{description}
 \item[``$\subset$'']
 \begin{align*}
  & \forall x \in f^{-1}(B_1 \backslash B_2) \exists y \in B_1 \backslash B_2 f(y) = x\\
  & \Rightarrow y \in B_1 \land (\forall z \in B_2 f^{-1}(z) \ne x)\\
  & \Rightarrow x \in f^{-1}(B_1) \backslash f^{-1}(B_2)
 \end{align*}
 
 \item[``$\supset$'']
 \begin{align*}
  & \forall x \in (f^{-1}(B_1) \backslash f^{-1}(B_2)) ((\exists y \in B_1 f^{-1}(y) = x) \land (\forall z \in B_2 f^{-1}(z) \ne x))\\
  & \Rightarrow (y \in B_1 \backslash B_2 \Rightarrow f^{-1}(y) = x)\\
  & \Rightarrow x \in f^{-1}(B_1 \backslash B_2)
 \end{align*}
\end{description}

\section*{Übung 2}

\subsection*{$R_1$}

\begin{itemize}
 \item nicht reflexiv: $x \not< x$
 \item nicht symmetrisch: $x \le y \Rightarrow y \nleq x$
 \item transitiv: $x < y \land y < z \Rightarrow x < z$
\end{itemize}

\subsection*{$R_2$}

\begin{itemize}
 \item reflexiv: $x \le x$
 \item anti-symmetrisch: $x \le y \land y \le x \Rightarrow x = y$
 \item transitiv: $x \le y \land y \le z \Rightarrow x \le z$
\end{itemize}

\subsection*{$R_3$}

\begin{itemize}
 \item nicht reflexiv: $x + x = 15 \Leftrightarrow x = \{\}$
 \item symmetrisch: $x + y = y + x = 15$
 \item $x + y = 15 \land y + z = 15 \Rightarrow x + y = y + z \Rightarrow x = z$. Da $R_3$ nicht reflexiv ist, ist es auch nicht transitiv.
\end{itemize}

\subsection*{$R_4$}

\begin{itemize}
 \item reflexiv: $x \text{ ist eine Primzahl} \Leftrightarrow x \text{ ist eine Primzahl}$
 \item symmetrisch, weil die Bedingung beide Elemente einzeln untersucht und daher unabhängig von der Reihenfolge ist
 \item transitiv, weil wenn $x$, $y$ und $z$ Primzahlen sind, sind auch $x$ und $z$ Primzahlen
\end{itemize}

\subsection*{$R_5$}

\begin{itemize}
 \item reflexiv: $x = x$
 \item nicht symmetrisch, weil $7 \ne 5 \land (7,5) \ne (5,7) \Rightarrow 5 \sim_{R_5} 7 \land 7 \not\sim_{R_5} 5$
 \item transitiv: $x = y = z$ oder $(x,y) = (5,7) \Rightarrow z = 7$ oder $(y,z) = (5,7) \Rightarrow x = 5$
\end{itemize}

\subsection*{$R_6$}

\begin{itemize}
 \item reflexiv: $x = x$
 \item symmetrisch: $x = y = x \lor xy = yx = 72$
 \item transitiv: $x = y = z \lor xy = yz = 72 \Rightarrow x = z$
\end{itemize}

\subsection*{$R_7$}

\begin{itemize}
 \item reflexiv: $x - x = 0 = 0 * 2$
 \item symmetrisch: $\exists n \in \mathbb{Z} (x - y = 2n) \Rightarrow y - x = 2 * (-n)$
 \item transitiv: $\exists n \in \mathbb{Z} (x - y = 2n) \land \exists k \in \mathbb{Z} (y - z = 2k) \Rightarrow x - z = 2 * (n + k)$
\end{itemize}

\subsection*{$R_8$}

\begin{itemize}
 \item reflexiv: $x = 1x$
 \item anti-symmetrisch: $y = n_0 * x \land x = \frac{1}{n_0} * y \Rightarrow y = x$
 \item transitiv: $y = n_0 * x \land z = n_1 * y \Rightarrow z = n_0 * n_1 * x$
\end{itemize}

\section*{Übung 3}

Es gilt immer $n \in \mathbb{Z}$.

\subsection*{1}

R ist \dots
\begin{itemize}
 \item reflexiv: $x - x = 0 = 0 * 5$
 \item symmetrisch: $x - y = n * 5 \Rightarrow y - x = (-n) * 5$
 \item transitiv: $x - y = 5n_0 \land y - z = 5n_1 \Rightarrow x - z = 5(n_0 + n_1)$
\end{itemize}

\subsection*{2}

\begin{equation*}
 [0]_{\sim_{R}} = \{x \in \mathbb{Z} \mid x = 5n\} = \{0, 5, -5, 10, -10, \dots\}
\end{equation*}

\subsection*{3}

\begin{claim}
 Es gibt $5$ Äquivalenzklassen in $M\diagup R$.
\end{claim}

Der Beweis setzt auf dem Lemma aus der Vorlesung auf, nach dem $[x] = [y] \Leftrightarrow x \sim_R y$ gilt.

\begin{proof}
 Wir schreiben jede Äquivalenzklasse als $[p]$.
 Wir schreiben $p$ als $5k + p - 5k$, wobei $k \in \mathbb{Z}$ so gewählt ist, dass $5k \le p < 5(k + 1)$, i.e. $p - 5k \in \{0, 1, 2, 3, 4\}$.
 Es gilt 
 \begin{align*}
  [5k + p - 5k] = [p - 5k] & \Leftrightarrow 5k + p - 5k \sim_R p - 5k\\
  & \Leftrightarrow 5k + p - 5k - (p - 5k) = 5n\\
  & \Leftrightarrow 5k = 5n \quad \text{wahr, weil nur gilt } k, n \in \mathbb{Z}
 \end{align*}
 Jede Äquivalenzklasse ist also gleich einer der Äquivalenzklassen $[0]$, $[1]$, $[2]$, $[3]$, $[4]$.
\end{proof}

\end{document}
