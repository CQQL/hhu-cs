\documentclass[a4paper,10pt]{article}
\usepackage[utf8]{inputenc}
\usepackage{amsmath}
\usepackage{amssymb}
\usepackage{amsthm}
\usepackage[german]{babel}

\title{LinA1, Übungsblatt 4}
\author{Marten Lienen (2126759), Gruppe 1}

\newtheorem*{claim}{Behauptung}
\newtheorem*{definition}{Definition}
\newtheorem*{notice}{Bemerkung}
\newtheorem*{lemma}{Lemma}
\newtheorem*{example}{Beispiel}

\begin{document}

\maketitle

\section*{Übung 1}

\subsection*{$(\mathbb{Z}_n, +)$ ist eine kommutative Gruppe}

Sei $x \in \mathbb{Z}_n$.
\begin{equation*}
 0 + x = r_n(0 + x) = r_n(x) = x
\end{equation*}
$0$ ist das neutrale Element.

\begin{align*}
 x + (n - x) = r_n(x + n - x) = r_n(n) = 0
\end{align*}
$n - x$ ist also das Inverse von $x \in \mathbb{Z}_n$.

\begin{align*}
 x + (y + z) = r_n(x + r_n(y + z)) = r_n(x + (y + z - qn)) = x + (y + z - qn) - pn = x + y + z
\end{align*}

Da $x + y = y + x$ ist auch $r_n(x + y) = r_n(y + x)$.

\subsection*{$(\mathbb{Z}_n, *)$ hat ein neutrales Element}

\begin{equation*}
 ex = r_n(ex) = r_n(x) \Rightarrow e = 1
\end{equation*}
$e = 1 \in \mathbb{Z}_n$, wenn $n \ge 2$.
Wenn $n = 1$, löst auch $e = 0 \in \mathbb{Z}_n$ die obige Gleichung für alle $x \in \mathbb{Z}_0$.

\subsection*{$*$ ist assoziativ}

Seien $x, y \in \mathbb{Z}_n$.
\begin{equation*}
 xy = r_n(xy) = r_n(yx) = yx
\end{equation*}

\subsection*{Es gilt das Distributivgesetz}

\begin{align*}
 x(y + z) = r_n(x(y + z)) = r_n(x * r_n(y + z)) = r_n(x * (y + z - qn)) = r_n(xy + xz - xqn) = r_n(xy + xz) = xy + xz
\end{align*}

\section*{Übung 2}

\subsection*{$(V, +)$ ist eine kommutative Gruppe}

Es gibt einen $0_V$-Vektor mit $n$ 0-Zeilen, sodass
\begin{equation*}
 0_V + v = \begin{pmatrix}x_1 + 0 \\ \vdots \\ x_n + 0\end{pmatrix} = \begin{pmatrix}x_1 \\ \vdots \\ x_n\end{pmatrix} = v
\end{equation*}

Jedes $v$ hat ein Inverses in $V$, das in jeder Spalte jeweils das Inverse in $K$ der entsprechenden Spalte von $v$ hat.
\begin{equation*}
 v + v^{-1} = \begin{pmatrix}x_1 \\ \vdots \\ x_n\end{pmatrix} + \begin{pmatrix}x_1^{-1} \\ \vdots \\ x_n^{-1}\end{pmatrix} = \begin{pmatrix}x_1 + x_1^{-1} \\ \vdots \\ x_n + x_n^{-1}\end{pmatrix} = \begin{pmatrix}0 \\ \vdots \\ 0\end{pmatrix}
\end{equation*}

Seien $x, y, z \in V$.
\begin{equation*}
 x + (y + z) = \begin{pmatrix}x_1 \\ \vdots \\ x_n\end{pmatrix} + \begin{pmatrix}y_1 + z_1 \\ \vdots \\ y_n + z_n\end{pmatrix} = \begin{pmatrix}x_1 + y_1 + z_1 \\ \vdots \\ x_n + y_n + z_n\end{pmatrix} = \begin{pmatrix}x_1 + y_1 \\ \vdots \\ x_n + y_n\end{pmatrix} + \begin{pmatrix}z_1 \\ \vdots \\ z_n\end{pmatrix} = (x + y) + z
\end{equation*}

Seien $v, w \in V$.
\begin{equation*}
 v + w = \begin{pmatrix}v_1 + w_1 \\ \vdots \\ v_n + w_n \end{pmatrix} = \begin{pmatrix}w_1 + v_1 \\ \vdots \\ w_n + v_n \end{pmatrix} = w + v
\end{equation*}

\subsection*{Weitere Bedingungen}

Seien $x, y \in K$ und $v \in V$.
\begin{equation*}
 x * (y * v) = x * \begin{pmatrix}y * v_1 \\ \vdots \\ y * v_n\end{pmatrix} = \begin{pmatrix} xy * v_1 \\ \vdots \\ xy * v_n\end{pmatrix} = (xy) * v
\end{equation*}

Das neutrale Element in $K$ ist auch das neutrale Element bezüglich der Skalarmultiplikation.
\begin{equation*}
 1_K * v = \begin{pmatrix}1_K * v_1 \\ \vdots \\ 1_K * v_n\end{pmatrix} = \begin{pmatrix}v_1 \\ \vdots \\ v_n\end{pmatrix} = v
\end{equation*}

Sei $x \in K$ und $v_1, v_2 \in V$.
\begin{align*}
 x * (v_1 + v_2) & = x * \begin{pmatrix}v_{1_1} + v_{2_1} \\ \vdots \\ v_{1_n} + v_{2_n} \end{pmatrix} = \begin{pmatrix}x * (v_{1_1} + v_{2_1}) \\ \vdots \\ x * (v_{1_n} + v_{2_n}) \end{pmatrix} = \begin{pmatrix}xv_{1_1} + xv_{2_1} \\ \vdots \\ xv_{1_n} + xv_{2_n} \end{pmatrix}\\
 & = \begin{pmatrix}xv_{1_1} \\ \vdots \\ xv_{1_n} \end{pmatrix} + \begin{pmatrix}xv_{2_1} \\ \vdots \\ xv_{2_n} \end{pmatrix} = x * \begin{pmatrix}v_{1_1} \\ \vdots \\ v_{1_n} \end{pmatrix} + x * \begin{pmatrix}v_{2_1} \\ \vdots \\ v_{2_n} \end{pmatrix} = xv_1 + xv_2
\end{align*}

Seien $x, y \in K$ und $v \in V$.
\begin{equation*}
 (x + y) * v = \begin{pmatrix}(x + y) * v_1 \\ \vdots \\ (x + y) * v_n \end{pmatrix} = \begin{pmatrix} xv_1 + yv_1 \\ \vdots \\ xv_n + yv_n \end{pmatrix} = x * \begin{pmatrix}v_1 \\ \vdots \\ v_n \end{pmatrix} + y * \begin{pmatrix}v_1 \\ \vdots \\ v_n \end{pmatrix} = xv + yv
\end{equation*}

\section*{Übung 3}

\subsection*{$(\mathbb{Q}(\sqrt{2}), +)$ ist eine Untergruppe von $(\mathbb{R}, +)$}

\subsubsection*{$(\mathbb{Q}(\sqrt{2}), +)$ besitzt ein neutrales Element}

Sei $x, e \in \mathbb{Q}(\sqrt{2})$.
\begin{equation*}
 x + e = x_1 + e_1 + (x_2 + e_2)\sqrt{2} = x_1 + x_2\sqrt{2} = x \Rightarrow e = 0 + 0\sqrt{2} \in \mathbb{Q}(\sqrt{2})
\end{equation*}
$e$ ist das neutrale Element.

\subsubsection*{$x + y \in \mathbb{Q}(\sqrt{2}) \forall x, y \in \mathbb{Q}(\sqrt{2})$}

Seien $x, y \in \mathbb{Q}(\sqrt{2})$.
\begin{equation*}
 x + y = (x_1 + y_1) + (x_2 + y_2)\sqrt{2} \in \mathbb{R} \Rightarrow x + y \in \mathbb{Q}(\sqrt{2})
\end{equation*}

\subsubsection*{$x^{-1} \in \mathbb{Q}(\sqrt{2}) \forall x \in \mathbb{Q}(\sqrt{2})$}

\begin{equation*}
 x + x^{-1} = (x_1 + x^{-1}_1) + (x_2 + x^{-1}_2)\sqrt{2} = 0 \Rightarrow x^{-1} = -x_1 - x_2\sqrt{2} \in \mathbb{Q}(\sqrt{2})
\end{equation*}

\subsection*{$xw \in \mathbb{Q}(\sqrt{2})$ für alle $x \in \mathbb{Q}$ und $w \in \mathbb{Q}(\sqrt{2})$}

\begin{equation*}
 xw = x * (w_1 + w_2\sqrt{2}) = xw_1 + xw_2\sqrt{2} \in \mathbb{R} \Rightarrow xw \in \mathbb{Q}(\sqrt{2})
\end{equation*}

\section*{Übung 4}

\begin{proof}
 $\Rightarrow$: Angenommen $f$ ist linear.
 \begin{align*}
  f(x_1v_1 + x_2v_2) = f(x_1v_1) + f(x_2v_2) = x_1f(v_1) + x_2f(v_2)
 \end{align*}

 $\Leftarrow$: Angenommen $f(x_1v_1 + x_2v_2) = x_1f(v_1) + x_2f(v_2)$.

 \begin{description}
  \item[P] $f(v_1 + v_2) = f(v_1) + f(v_2)$
  \item[Q] $f(xv) = xf(v)$
 \end{description}

 Ich schreibe $Q$, wenn $Q$ für alle $v \in V, x \in K$ gilt, und $\neg{Q}$, wenn es $v \in V, x \in K$ gibt, für die $Q$ nicht gilt.

 \subsubsection*{$\neg P \land \neg Q$}
 
 Wenn weder $P$ noch $Q$ gilt, haben wir keine Regeln, um $f(x_1v_1 + x_2v_2)$ oder $x_1f(v_1) + x_2f(v_2)$ umzuformen und es gilt $f(x_1v_1 + x_2v_2) \ne x_1f(v_1) + x_2f(v_2)$ im Widerspruch zur Annahme.

 \subsubsection*{$P \land \neg Q$}
 
 Es gibt $x_1, x_2 \in K, v_1, v_2 \in V$, sodass:
 \begin{equation*}
  f(x_1v_1 + x_2v_2) = f(x_1v_1) + f(x_2v_2) \ne x_1f(v_1) + x_2f(v_2)
 \end{equation*}
 Widerspruch.

 \subsubsection*{$\neg P \land Q$}
 
 Es gibt $x_1, x_2 \in K, v_1, v_2 \in V$, sodass:
 \begin{equation*}
  f(x_1v_1 + x_2v_2) \ne f(x_1v_1) + f(x_2v_2) = x_1f(v_1) + x_2f(v_2)
 \end{equation*}
 Widerspruch.

 \subsubsection*{$P \land Q$}
 
 Für alle $x_1, x_2 \in K, v_1, v_2 \in V$ gilt:
 \begin{equation*}
  f(x_1v_1 + x_2v_2) = f(x_1v_1) + f(x_2v_2) = x_1f(v_1) + x_2f(v_2)
 \end{equation*}
 
 Es müssen also $P$ und $Q$ gelten, wenn $f(x_1v_1 + x_2v_2) = x_1f(v_1) + x_2f(v_2)$.
\end{proof}

\section*{Übung 5}

\begin{proof}
 Da $f$ bijektiv ist, gilt nach Satz 1.4.11 $f^{-1} \circ f = Id_V$ und $f \circ f^{-1} = Id_W$.
 Somit ist $f^{-1}$ auch bijektiv.
 
 Es gibt $w_1, w_2 \in W$ mit $f(v_1) = w_1, f(v_2) = w_2$ und $f^{-1}(w_1) = v_1, f^{-1}(w_2) = v_2$.
 Da $f$ isomorph ist, gilt $f(v_1 + v_2) = f(v_1) + f(v_2) = w_1 + w_2$.
 Durch Anwenden von $f^{-1}$ erhalten wir $v_1 + v_2 = f^{-1}(w_1 + w_2)$.
 Durch Ersetzen von $v_1, v_2$ erhalten wir $f^{-1}(w_1 + w_2) = v_1 + v_2 = f^{-1}(w_1) + f^{-1}(w_2)$.
 
 Es gibt $w \in W$ mit $f(v) = w$ und $f^{-1}(w) = v$.
 Da $f$ isomorph ist, gilt $f(xv) = xw$.
 Durch Anwenden von $f^{-1}$ kommen wir zu $xv = f^{-1}(xw)$.
 Durch Ersetzen von $v$ kommen wir zu $f^{-1}(xw) = xv = xf^{-1}(w)$.
\end{proof}

\end{document}
