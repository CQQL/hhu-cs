\documentclass[a4paper,10pt]{article}
\usepackage[utf8]{inputenc}
\usepackage[german]{babel}
\usepackage{amsmath}
\usepackage{amssymb}
\usepackage{amsthm}
\usepackage{stmaryrd}

\title{LinA1, Übungsblatt 9}
\author{Marten Lienen (2126759), Gruppe 1; Fabian Schmittmann (2083559), Gruppe 5}

\newtheorem{lemma}{Lemma}

\begin{document}

\maketitle

\section*{Übung 1}

\subsection*{Teil 1}

\begin{equation}
 Mat_{\mathcal{B}, \mathcal{B}'}(f) =
   \begin{pmatrix}
    1 & 5 & -3 & 1\\
    0 & 1 & 0 & 2\\
    0 & 2 & 1 & 3
   \end{pmatrix}
\end{equation}

\subsection*{Teil 2}

\begin{equation}
 Basis(Ker(f)) = (\begin{pmatrix}12\\-2\\1\\1\end{pmatrix})
\end{equation}

Sei $w \in f(\mathbb{R}^4)$.
Weil $f$ linear ist, gilt
\begin{align}
 w & = x_1 * f(v_1) + x_2 * f(v_2) + x_3 * f(v_3) + x_4 * f(v_4)\\
 & =
   x_1 * \begin{pmatrix}1\\0\\0\end{pmatrix} +
   x_2 * \begin{pmatrix}5\\1\\2\end{pmatrix} +
   x_3 * \begin{pmatrix}-3\\0\\1\end{pmatrix} +
   x_4 * \begin{pmatrix}1\\2\\3\end{pmatrix}
\end{align}
$(f(v_1), f(v_2), f(v_3), f(v_4))$ ist also ein EZS von $f(\mathbb{R}^4)$.
Es gilt aber
\begin{equation}
 \begin{pmatrix}5\\1\\2\end{pmatrix} = 6 * \begin{pmatrix}1\\0\\0\end{pmatrix} + 0,5 * \begin{pmatrix}-3\\0\\1\end{pmatrix} + 0,5 * \begin{pmatrix}1\\2\\3\end{pmatrix}
\end{equation}
Da $f(v_2)$ sich als Linearkombination der restlichen Vektoren darstellen lässt, können wir das EZS auf $(f(v_1), f(v_3), f(v_4))$ reduzieren.
\begin{align*}
 & x_1 * \begin{pmatrix}1\\0\\0\end{pmatrix} + x_2 * \begin{pmatrix}-3\\0\\1\end{pmatrix} + x_3 * \begin{pmatrix}1\\2\\3\end{pmatrix} = 0\\
 \Rightarrow & x_1 - 3x_2 + x_3 = 0 \land 2x_3 = 0 \land x_2 + 3x_3 = 0\\
 \Rightarrow & x_3 = 0 \Rightarrow x_2 = 0 \Rightarrow x_1 = 0
\end{align*}
$(f(v_1), f(v_3), f(v_4))$ ist linear unabhängig und somit eine Basis von $f(\mathbb{R}^4)$.

\subsection*{Teil 3}

$v_5$ ist in der Basis vom Kern von $f$.
Weil $Ker(f) \subset \mathbb{R}^4$, ist $(v_5)$ auf linear unabhängig in $\mathbb{R}^4$.
Nach dem Basisergänzungssatz können wir $(v_5)$ mit Elementen von $(v_1, v_2, v_3, v_4)$ zu einer Basis von $\mathbb{R}^4$ ergänzen.
Angenommen $v_5$ sei eine Linearkombination von $(v_1, v_2, v_3)$.
\begin{equation}
 \begin{pmatrix}12\\-2\\1\\1\end{pmatrix} = x_1\begin{pmatrix}1\\0\\0\\0\end{pmatrix} + x_2\begin{pmatrix}0\\1\\0\\0\end{pmatrix} + x_3\begin{pmatrix}0\\0\\1\\0\end{pmatrix}
 \Rightarrow 1 = 0 \lightning
\end{equation}
Da $(v_5)$ keine Linearkombination von $(v_1, v_2, v_3)$ ist, können wir $(v_5)$ nacheinander mit $v_1, v_2, v_3$ ergänzen, wobei es aber linear unabhängig bleibt.
$(v_1, v_2, v_3, v_5)$ ist also eine Basis von $\mathbb{R}^4$.

\subsection*{Teil 4}

\begin{equation}
 f(v_1) = \begin{pmatrix}1\\0\\0\end{pmatrix},
 f(v_2) = \begin{pmatrix}5\\1\\2\end{pmatrix},
 f(v_3) = \begin{pmatrix}-3\\0\\1\end{pmatrix}
\end{equation}

\begin{align}
 & \begin{pmatrix}1\\0\\0\end{pmatrix} = 1 * f(v_1) + 0 * f(v_2) + 0 * f(v_3)\\
 & \begin{pmatrix}0\\1\\0\end{pmatrix} = -11 * f(v_1) + 1 * f(v_2) - 2 * f(v_3)\\
 & \begin{pmatrix}0\\0\\1\end{pmatrix} = 3 * f(v_1) + 0 * f(v_2) + 1 * f(v_3)
\end{align}

Weil die Basis $w_1, w_2, w_3 \in \langle f(v_1), f(v_2), f(v_3) \rangle$, ist $(f(v_1), f(v_2), f(v_3))$ ein EZS von $\mathbb{R}^3$.

\begin{align}
 & x_1\begin{pmatrix}1\\0\\0\end{pmatrix}
 + x_2\begin{pmatrix}5\\1\\2\end{pmatrix}
 + x_3 \begin{pmatrix}-3\\0\\1\end{pmatrix} = 0\\
 \Rightarrow & x_2 = 0 \Rightarrow x_3 = 0 \Rightarrow x_1 = 0
\end{align}

$(f(v_1), f(v_2), f(v_3))$ ist also auch linear unabhängig und damit eine Basis von $\mathbb{R}^3$.

\subsection*{Teil 5}

\begin{equation}
 Mat_{\mathcal{B}'', \mathcal{B}'''}(f) =
   \begin{pmatrix}
    1 & 0 & 0 & 0\\
    0 & 1 & 0 & 0\\
    0 & 0 & 1 & 0
   \end{pmatrix}
\end{equation}

\section*{Übung 2}

\subsection*{Teil 1}

\begin{equation}
 A^2 = 
   \begin{pmatrix}
    3 & -1 & -1\\
    -1 & 3 & -1\\
    -1 & -1 & 3
   \end{pmatrix} = 
   2\begin{pmatrix}
    1 & 0 & 0\\
    0 & 1 & 0\\
    0 & 0 & 1
   \end{pmatrix} +
   \begin{pmatrix}
    -1 & 1 & 1\\
    1 & -1 & 1\\
    1 & 1 & -1
   \end{pmatrix} = 2I_3 - A
\end{equation}

\subsection*{Teil 2}

\begin{equation}
 A^{-1} =
   \begin{pmatrix}
    0 & 0,5 & 0,5\\
    0,5 & 0 & 0,5\\
    0,5 & 0,5 & 0
   \end{pmatrix}
\end{equation}
\begin{equation}
 AA^{-1} =
   \begin{pmatrix}
    1 & 0 & 0\\
    0 & 1 & 0\\
    0 & 0 & 1
   \end{pmatrix} = I_3 = A^{-1}A
\end{equation}

\section*{Übung 3}

\subsection*{Teil 1}

\begin{equation}
 E_{k,l}A = C
\end{equation}
mit $C = (c_{ij})$, wobei
\begin{equation}
 c_{ij} = \sum_{q = 1}^n e_{iq}a_{qj} =
   \begin{cases}
    0 & \text{wenn $(i,j) \ne (k,l)$}\\
    a_{lj} & \text{andernfalls}
   \end{cases}
\end{equation}
Es wird also die $l$-te Zeile in die $k$-te Zeile kopiert, während alles andere auf $0$ gesetzt wird.

\begin{equation}
 AE_{k,l} = D
\end{equation}
mit $D = (d_{ij})$, wobei
\begin{equation}
 d_{ij} = \sum_{q = 1}^n e_{qj}a_{iq} =
   \begin{cases}
    0 & \text{wenn $(i,j) \ne (k,l)$}\\
    a_{ik} & \text{andernfalls}
   \end{cases}
\end{equation}
Es wird also die $k$-te Spalte in die $l$-te Spalte kopiert, während alles andere auf $0$ gesetzt wird.

\subsection*{Teil 2}

Sei $AB = (c_{ij})$ und $BA = (d_{ij})$.
Dann muss gelten
\begin{align}
 & c_{ij} = \sum_{q = 1}^n a_{iq}b_{qj} = \sum_{q = 1}^n b_{iq}a_{qj} = d_{ij}\\
 \Leftrightarrow & \sum_{q = 1}^n a_{iq}b_{qj} - b_{iq}a_{qj} = 0
\end{align}
Dies ist mindestens wahr für alle $A$ deren $(a_{ij})$ folgende Werte annehmen:
\begin{equation}
 a_{ij} =
   \begin{cases}
    x \in K & \text{wenn $i = j$}\\
    0 & \text{andernfalls}
   \end{cases}
\end{equation}
In diesem Fall sind die Summanden nämlich entweder
\begin{equation}
 0b_{qj} - b_{iq}0 = 0
\end{equation}
oder
\begin{equation}
 a_{ii}b_{ii} - b_{ii}a_{ii} = 0
\end{equation}
Es sind also alle $xI_n \in Z(M_n(K))$ mit $x \in K$.

\end{document}
