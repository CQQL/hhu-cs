\documentclass[10pt,a4paper]{article}
\usepackage[utf8]{inputenc}
\usepackage[german]{babel}
\usepackage{amsmath}
\usepackage{amsfonts}
\usepackage{amssymb}
\usepackage{amsthm}
\usepackage[left=2cm,right=2cm,top=2cm,bottom=2cm]{geometry}

\begin{document}

\section*{Aufgabe 37}

\begin{proof}
Das Maximum existiert, weil $\{ x \mid x \in K^{n}\ \land\ ||x|| = 1 \}$ abgeschlossen ist.
Da alle Elemente in $\{ ||Ax|| \mid x \in K^{n}\ \land\ ||x|| = 1 \}$ $\ge 0$ sind, ist auch deren Maximum  größergleich $0$.

\begin{equation}
||A|| = 0 \Rightarrow ||Ax|| = 0\ \forall x \Rightarrow A = 0\ \textit{weil $x \ne 0$, weil $||x|| = 1$}
\end{equation}

\begin{align*}
||\lambda A|| & = \max \{ ||\lambda Ax|| \mid x \in K^{n}\ \land\ ||x|| = 1 \}\\
& = \max \{ |\lambda| \cdot ||Ax|| \mid x \in K^{n}\ \land\ ||x|| = 1 \}\\
& = |\lambda| \cdot \max \{ ||Ax|| \mid x \in K^{n}\ \land\ ||x|| = 1 \}\\
& = |\lambda| \cdot ||A||
\end{align*}

\begin{align*}
||A + B|| & = \max \{ ||(A + B)x|| \mid x \in K^{n}\ \land\ ||x|| = 1 \}\\
& = \max \{ ||Ax + Bx|| \mid x \in K^{n}\ \land\ ||x|| = 1 \}\\
& \le \max \{ ||Ax|| + ||Bx|| \mid x \in K^{n}\ \land\ ||x|| = 1 \}\\
& \le \max \{ ||Ax|| \mid x \in K^{n}\ \land\ ||x|| = 1 \} + \max \{ ||Bx|| \mid x \in K^{n}\ \land\ ||x|| = 1 \}\\
& = ||A|| + ||B||\\
\end{align*}
\end{proof}

\section*{Aufgabe 38}

\subsection*{Teil a}

Eine Lösung der zugehörigen homogenen DGl ist
\begin{equation}
\varphi(x) = e^{x\begin{pmatrix}
0 & 1\\
1 & 0
\end{pmatrix}
} y_{0} = \begin{pmatrix}
\frac{1}{2} & \frac{1}{2}\\
\frac{1}{2} & -\frac{1}{2}
\end{pmatrix}
e^{x \begin{pmatrix}
1 & 0\\
0 & -1
\end{pmatrix}
}
\begin{pmatrix}
1 & 1\\
1 & -1
\end{pmatrix}
= \begin{pmatrix}
\frac{1}{2} & \frac{1}{2}\\
\frac{1}{2} & -\frac{1}{2}
\end{pmatrix}
\begin{pmatrix}
e^{x} & 0\\
0 & -e^{x}
\end{pmatrix}
\begin{pmatrix}
1 & 1\\
1 & -1
\end{pmatrix}
= \begin{pmatrix}
0 & e^{x}\\
e^{x} & 0
\end{pmatrix}
\end{equation}

\begin{equation}
u(x) = \int_{x_{0}}^{x} 
\begin{pmatrix}
0 & e^{-x}\\
e^{-x} & 0
\end{pmatrix}
\begin{pmatrix}
e^{x}\\0
\end{pmatrix}\ dx
= \begin{pmatrix}
c_{1}\\x + c_{2}
\end{pmatrix}
\end{equation}

\begin{equation}
\begin{pmatrix}
\varphi_{1}\\
\varphi_{2}
\end{pmatrix}
= \begin{pmatrix}
e^{x}(x + c_{2})\\
c_{1}e^{x}
\end{pmatrix}
\end{equation}

\subsection*{Teil b}

\section*{Aufgabe 39}

\section*{Aufgabe 40}

\end{document}