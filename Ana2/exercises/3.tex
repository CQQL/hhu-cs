\documentclass[10pt,a4paper]{article}
\usepackage[utf8]{inputenc}
\usepackage[german]{babel}
\usepackage{amsmath}
\usepackage{amsfonts}
\usepackage{amssymb}
\usepackage{amsthm}
\usepackage[left=2cm,right=2cm,top=2cm,bottom=2cm]{geometry}

\begin{document}

\section*{Übung 9}

\subsection*{a}

\begin{proof}
Sei $A = \{ x \in X \mid f(x) \le g(x) \}$ und $x$ ein Berührpunkt von $A$.
Nehmen wir an $f(x) > g(x)$.
Weil $f$ und $g$ stetig sind, existiert $B_{\delta}(x)$ mit $f(B_{\delta}(x)) \subseteq B_{\frac{f(x) - g(x)}{2}}(f(x))$ und $B_{\gamma}(x)$ mit $g(B_{\gamma}(x)) \subseteq B_{\frac{f(x) - g(x)}{2}}(g(x))$ und $\delta > 0$ und $\gamma > 0$.
Sei $y \in B_{\min(\delta, \gamma)}(x)$.
Dann gilt
\begin{equation}
f(y) > f(y) - \frac{f(x) - g(x)}{2} = g(y) + \frac{f(y) - g(y)}{2} < g(y)
\end{equation}
Also sind $B_{\min(\delta, \gamma)}(x)$ und $A$ disjunkt und es gibt eine $min(\delta, \gamma)$-Umgebung von $x$, die keinen Punkt von $A$ enthält.
Dies steht im Widerspruch dazu, dass $x$ ein Berührpunkt ist und es folgt, dass $f(x) \le g(x)$ und somit $x \in A$ und $A$ ist abschlossen in $X$.
\end{proof}

\subsection*{b}

\begin{proof}
Sei $A = \{ x \in X \mid f(x) < g(x) \}$ und $x \in A$.
Weil $f$ und $g$ stetig sind, existiert $B_{\delta}(x)$ mit $f(B_{\delta}(x)) \subseteq B_{\frac{g(x) - f(x)}{2}}(f(x))$ und $B_{\gamma}(x)$ mit $g(B_{\gamma}(x)) \subseteq B_{\frac{g(x) - f(x)}{2}}(g(x))$ und $\delta > 0$ und $\gamma > 0$.
Sei $y \in B_{\min(\delta, \gamma)}(x)$.
Dann gilt
\begin{equation}
f(y) < f(y) + \frac{g(x) - f(x)}{2} = g(y) - \frac{g(y) - f(y)}{2} < g(y)
\end{equation}
Also $B_{\min(\delta, \gamma)}(x) \subseteq A$ und $A$ ist offen in $X$.
\end{proof}

\section*{Übung 10}

\subsection*{a}

\begin{proof}
Sei $\varepsilon > 0$, $x_{0} \in X$ und $x \in X$ mit $d(x, x_{0}) < \delta$, wobei $\delta = \frac{\varepsilon}{2}$.
Dann gilt
\begin{equation}
|f(x) - f(x_{0})| = |d(x, A) - d(x_{0}, A)| \le |d(x, x_{0})| = \frac{\varepsilon}{2} < \varepsilon
\end{equation}
wie wir in Aufgabe 5 gezeigt haben.
\end{proof}

\subsection*{b}

\begin{proof}
Seien
\begin{align*}
U = \bigcup_{x \in A} B_{\frac{d(x, B)}{2}}(x)\\
V = \bigcup_{x \in B} B_{\frac{d(x, A)}{2}}(x)
\end{align*}
Dann sind $U$ und $V$ offen in $X$ als Vereinigung beliebig vieler offener Mengen.

Sei $a \in A$.
Da $B_{\frac{d(a, B)}{2}}(a) \subseteq U$, ist $a \in U$ und $A \subseteq U$.

Sei $b \in B$.
Da $B_{\frac{d(b, B)}{2}}(b) \subseteq V$, ist $b \in U$ und $B \subseteq U$.

Sei $\gamma \in U \cap V$.
Es gibt also $a \in A, b \in B$, sodass $\gamma \in B_{\frac{d(a, B)}{2}}(a)$, $\gamma \in B_{\frac{d(b, A)}{2}}(b)$ und $d(a, b) = d(a, B) = d(b, A)$.

\end{proof}

\section*{Übung 11}

\begin{proof}
Sei $x \in X \setminus \overline{B}_{r}(a)$, und $y \in B_{d(x, a) - r}(x)$.
Dann ist $d(y, a) > d(x, a) - (d(x, y) - r) = r$, also $y \notin \overline{B}_{r}(a)$.
Also ist $X \setminus \overline{B}_{r}(a)$ offen und sein Komplement, $\overline{B}_{r}(a)$, offen.
\end{proof}

\section*{Übung 12}

\subsection*{$f_{1}$}

\subsection*{$f_{2}$}

$g_{2}$ ist nicht stetig in $(0, 0)$, weil $g_{2}(0, 0) = 0$, aber $g_{2}(x, x) = \frac{1}{2}$ für $x \ne 0$.

\subsection*{$f_{3}$}

\subsection*{$f_{4}$}

\section*{Übung 13}

\begin{proof}
Sei $A = \{0\} \subset V$ und $d$ die zu $||.||$ gehörige Metrik.
Dann ist $f(x) = ||x|| = d(x, A)$ stetig nach Aufgabe 10a).
\end{proof}

\end{document}