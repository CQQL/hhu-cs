\documentclass[10pt,a4paper]{article}
\usepackage[utf8]{inputenc}
\usepackage[german]{babel}
\usepackage{amsmath}
\usepackage{amsfonts}
\usepackage{amssymb}
\usepackage{amsthm}
\usepackage[left=2cm,right=2cm,top=2cm,bottom=2cm]{geometry}

\begin{document}

\section*{Aufgabe 49}

\begin{equation}
a = 1
\end{equation}

\begin{proof}
Sei $x \in \mathbb{R}$ mit $|x| \le a$.
\begin{equation}
|f(x)| = x^{3} \le 1 \Leftrightarrow x \le 1^{\frac{1}{3}} = 1 = a
\end{equation}

\begin{equation}
\lim_{x \rightarrow 1} |f(x) - f(0)| = \lim_{x \rightarrow 1} |f(x)| = \lim_{x \rightarrow 1} |x^{3}| = 1 = |x - y|
\end{equation}
$f$ ist also nicht kontrahierend auf $]-1, 1[$, weil es zu jedem $C \in [0, 1[$ ein $x$ gibt, sodass $|f(x) - f(0)| > C|x - 0|$.

Seien $x, y \in [-b, b]$ mit $0 < b < 1$ und $x < y$.
\begin{align*}
|f(y) - f(x)| & = |y^{3} - x^{3}|\\
& = |(y - x)(x^{2} + xy + y^{2})|\\
& \le |(y - x)((-b)^{2} - bb + b^{2})|\\
& = |b^{2} - b^{2} + b^{2}| \cdot |(y - x)|\\
& = b^{2} \cdot |(y - x)|\\
\end{align*}
$f$ ist auf $[-b, b]$ kontrahierend mit Kontraktionskonstante $b^{2} < 1$.
\end{proof}

\section*{Aufgabe 50}

\begin{proof}
Sei $v \in \mathbb{R}^{n}$ mit $||v|| = 1$, sodass $||Df(x_{0})v|| = ||Df(x_{0})||$.
Dann gibt es $\alpha \in \mathbb{R}_{> 0}$, sodass für $v' = \alpha v + x_{0}$ gilt
\begin{equation}
||f(v') - f(x_{0})|| = 
\end{equation}
Dann ist $U = B_{||v' - x_{0}||}(x_{0})$ die gesuchte Umgebung.
\end{proof}

\section*{Aufgabe 51}

\section*{Aufgabe 52}

\subsection*{Teil a}

Diese Menge ist gleichgradig stetig.

\begin{proof}
Sei $\varepsilon > 0$.
Sei $\delta = $.
Seien $x, y \in \mathbb{R}$ mit $|x - y| < \delta$.
Sei $f_{\alpha} \in M$.
\begin{align*}
|f_{\alpha}(x) - f_{\alpha}(y)| & = |\frac{1}{\alpha}\sin(\alpha x) - \frac{1}{\alpha}\sin(\alpha y)|\\
& = \frac{1}{\alpha}|\sin(\alpha x) - \sin(\alpha y)|\\
\end{align*}
\end{proof}

\subsection*{Teil b}

Diese Menge ist nicht gleichgradig stetig.

\begin{proof}
Sei $\varepsilon > 0$.
Sei $\delta > 0$.
Seien $x = 1$ und $y = 1 - \frac{\delta}{2}$.
Sei $g_{\alpha} \in M$.
\begin{align*}
|g_{\alpha}(x) - g_{\alpha}(y)| & = |\frac{\alpha}{1 + x^{2}} - \frac{\alpha}{1 + y^{2}}|\\
& = \alpha \cdot |\frac{1}{1 + 1^{2}} - \frac{1}{1 + (1 - \frac{\delta}{2})^{2}}|\\
& = \alpha \cdot |\frac{1}{2} - \frac{1}{2 - \delta + \delta^{2}}|
\end{align*}
Es gibt also für jedes $\delta$ Funktionen $g_{\alpha}$, sodass die Werte sich in einer $\frac{\delta}{2}$-Umgebung von $1$ beliebig unterscheiden, also auch um mehr als jedes $\varepsilon$.
\end{proof}

\subsection*{Teil c}

Diese Menge ist gleichgradig stetig.

\begin{proof}
Sei $\varepsilon > 0$.
Sei $\delta = $.
Seien $x, y \in \mathbb{R}$ mit $|x - y| < \delta$.
Sei $g_{\alpha} \in M$.
\begin{align*}
|g_{\alpha}(x) - g_{\alpha}(y)| & = |\frac{\alpha}{1 + x^{2}} - \frac{\alpha}{1 + y^{2}}|\\
& = \alpha \cdot |\frac{1}{1 + x^{2}} - \frac{1}{1 + y^{2}}|\\
& = \alpha \cdot |\frac{1 + y^{2} - 1 - x^{2}}{(1 + x^{2})(1 + y^{2})}|\\
& = \alpha \cdot |\frac{y^{2} - x^{2}}{(1 + x^{2})(1 + y^{2})}|\\
& = \alpha \cdot |\frac{(y - x)(y + x)}{(1 + x^{2})(1 + y^{2})}|\\
& \le \alpha \cdot |(y - x)(y + x)|\\
& = \alpha \cdot |-(x - y)(y + x)|\\
& = \alpha \cdot |x - y| \cdot |y + x|\\
\end{align*}
\end{proof}

\section*{Aufgabe 53}

\begin{proof}
Sei $\varepsilon > 0$.
Sei $N = \max \{ 1, \lceil \frac{1}{\varepsilon} \rceil \}$.
Seien $n \ge m \ge N$ und $x \in \mathbb{R}$.
\begin{equation}
|f_{n}(x) - f_{m}(x)| = |x + \frac{1}{n} - x - \frac{1}{m}| = |\frac{1}{n} - \frac{1}{m}| \le |\frac{1}{n}| \le \varepsilon
\end{equation}
$f$ konvergiert also gleichmäßig.

\begin{equation}
h_{n}(x) := (g \circ f_{n})(x) = x^{2} + \frac{2}{n}x + \frac{1}{n^{2}}
\end{equation}
Seien jetzt $\varepsilon > 0$, $N \in \mathbb{N}$, $n \ge m \ge N$ und $x > \frac{\varepsilon \cdot m}{2}$.
\begin{align*}
|h_{n}(x) - h_{m}(x)| & = |x^{2} + \frac{2}{n}x + \frac{1}{n^{2}} - x^{2} - \frac{2}{m}x - \frac{1}{m^{2}}|\\
& = |\frac{2}{n}x + \frac{1}{n^{2}} - \frac{2}{m}x - \frac{1}{m^{2}}|\\
& = |(\frac{2}{n} - \frac{2}{m})x + \frac{1}{n^{2}} - \frac{1}{m^{2}}|\\
& \ge |-\frac{2}{m}x - \frac{1}{m^{2}}| \qquad \textit{weil $n \ge m$}\\
& > |-\frac{2}{m}x| \qquad \textit{weil $m \ge 1$}\\
& = \frac{2}{m}x > \varepsilon\\
\end{align*}
$(g \circ f_{n})_{n}$ konvergiert nicht gleichmäßig, weil es zu jedem $n$ und $m$ ein $x$ gibt, sodass $(g \circ f_{n})(x)$ und $(g \circ f_{m})(x)$ beliebig weit auseinander liegen.
\end{proof}

\end{document}