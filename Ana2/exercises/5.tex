\documentclass[10pt,a4paper]{article}
\usepackage[utf8]{inputenc}
\usepackage[german]{babel}
\usepackage{amsmath}
\usepackage{amsfonts}
\usepackage{amssymb}
\usepackage{amsthm}
\usepackage[left=2cm,right=2cm,top=2cm,bottom=2cm]{geometry}

\begin{document}

\section*{Aufgabe 19}

\subsection*{Teil a}

\subsection*{Teil b}

\section*{Aufgabe 20}

\section*{Aufgabe 21}

\begin{proof}
$f$ ist in $(0, 0)$ partiell differenzierbar, weil alle partiellen Ableitungen existieren
\begin{equation}
D_{1}f(x) = F_{1}'(x) = f'(x, 0) = 0
\end{equation}
\begin{equation}
D_{2}f(y) = F_{2}'(y) = f'(0, y) = 0
\end{equation}

Es ist hingegen nicht differenzierbar, weil, wie wir aus Satz 5 des Paragraphen 5 wissen, für differenzierbare Funktionen gilt
\begin{equation}
D_{v}f(x) = <v, \nabla f(x)>
\end{equation}
In unserem Fall haben wir jedoch beispielsweise
\begin{align*}
D_{(1, 1)}f(0, 0) & = \lim_{t \rightarrow 0} \frac{1}{t} (f((0, 0) + t \cdot (1, 1)) + f(0, 0))\\
& = \lim_{t \rightarrow 0} \frac{1}{t} f(t, t)\\
& = \lim_{t \rightarrow 0} \frac{1}{t} \frac{t^{3}}{2t^{2}} = \frac{1}{2}
\end{align*}
und
\begin{equation}
<(1, 1), \nabla f(0, 0)> = <(1, 1), (0, 0)> = 0
\end{equation}
Also ist $D_{(1, 1)}f(0, 0) \ne <(1, 1), \nabla f(0, 0)>$ und $f$ kann in $(0, 0)$ nicht differenzierbar sein.
\end{proof}

\section*{Aufgabe 22}

\subsection*{Teil a}

Nach dem Satz von Schwartz gilt
\begin{align*}
rot \nabla f & = (D_{2}D_{3}f - D_{3}D_{2}f, D_{3}D_{1}f - D_{1}D_{3}f, D_{1}D_{2}f - D_{2}D_{1}f)\\
& = (D_{2}D_{3}f - D_{2}D_{3}f, D_{3}D_{1}f - D_{3}D_{1}f, D_{1}D_{2}f - D_{1}D_{2}f)\\
& = (0, 0, 0) = 0\\
\end{align*}
weil $f \in C^{2}$.

\subsection*{Teil b}

\begin{align*}
div\ rot\ g & = div (D_{2}g_{3} - D_{3}g_{2}, D_{3}g_{1} - D_{1}g_{3}, D_{1}g_{2} - D_{2}g_{1})\\
& = D_{1}(D_{2}g_{3} - D_{3}g_{2}) + D_{2}(D_{3}g_{1} - D_{1}g_{3}) + D_{3}(D_{1}g_{2} - D_{2}g_{1})\\
& = D_{1}D_{2}g_{3} - D_{1}D_{3}g_{2} + D_{2}D_{3}g_{1} - D_{2}D_{1}g_{3} + D_{3}D_{1}g_{2} - D_{3}D_{2}g_{1}\\
& = D_{1}D_{2}g_{3} - D_{1}D_{3}g_{2} + D_{2}D_{3}g_{1} - D_{1}D_{2}g_{3} + D_{1}D_{3}g_{2} - D_{2}D_{3}g_{1} = 0\\
\end{align*}

\section*{Aufgabe 23}

\subsection*{Teil a}

\begin{align*}
rot\ g & = (D_{2}g_{3} - D_{3}g_{2}, D_{3}g_{1} - D_{1}g_{3}, D_{1}g_{2} - D_{2}g_{1})\\
& = (D_{2}e^{xy} - D_{3}(xze^{xy}), D_{3}(yze^{xy}) - D_{1}e^{xy}, D_{1}(xze^{xy}) - D_{2}(yze^{xy}))\\
& = (xe^{xy} - xe^{xy}, ye^{xy} - ye^{xy}, ze^{xy} - ze^{xy})\\
& = (0, 0, 0) = 0\\
\end{align*}

\subsection*{Teil b}

\begin{equation}
f(x, y, z) = z \cdot e^{xy}
\end{equation}
\begin{equation}
D_{1}f(x, y, z) = yze^{xy}
\end{equation}
\begin{equation}
D_{2}f(x, y, z) = xze^{xy}
\end{equation}
\begin{equation}
D_{3}f(x, y, z) = e^{xy}
\end{equation}

\end{document}