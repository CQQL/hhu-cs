\documentclass[10pt,a4paper]{article}
\usepackage[utf8]{inputenc}
\usepackage[german]{babel}
\usepackage{amsmath}
\usepackage{amsfonts}
\usepackage{amssymb}
\usepackage{amsthm}
\usepackage[left=2cm,right=2cm,top=2cm,bottom=2cm]{geometry}

\begin{document}

\section*{Aufgabe 24}

\subsection*{Teil a}

Die kritischen Stellen von $f$ sind genau die Stellen, für die $\nabla f = 0$.

\begin{equation}
\nabla f = \begin{pmatrix}
2x + y + 1\\
2y + x + 1
\end{pmatrix}
\end{equation}
\begin{equation}
2x + y + 1 = 0 \land 2y + x + 1 = 0 \Leftrightarrow x = -\frac{1}{3} \land y = -\frac{1}{3}
\end{equation}

$(-\frac{1}{3}, -\frac{1}{3})$ ist also eine kritische Stelle von $f$.

\begin{equation}
Hf(x, y) = \begin{pmatrix}
2 & 1\\
1 & 2
\end{pmatrix} = Hf(-\frac{1}{3}, -\frac{1}{3})
\end{equation}
Nach dem Kriterium von Hurwitz ist $Hf(-\frac{1}{3}, -\frac{1}{3})$ positiv definit und $(-\frac{1}{3}, -\frac{1}{3})$ ist ein striktes lokales Minimum von $f$.

\begin{equation}
f(-\frac{1}{3}, -\frac{1}{3}) = \frac{1}{9} + \frac{1}{9} + \frac{1}{9} - \frac{1}{3} - \frac{1}{3} + 1 = \frac{2}{3}
\end{equation}

\subsection*{Teil b}

Offensichtlich ist das lokale Minimum aus Teil a in $Q$ enthalten.
Dann muss man jetzt die Randwerte untersuchen.
\begin{equation}
Q_{1} = \{ (-1, t) \mid t \in [-1, 1[ \}
\end{equation}
\begin{equation}
Q_{2} = \{ (1, t) \mid t \in ]-1, 1] \}
\end{equation}
\begin{equation}
Q_{3} = \{ (t, -1) \mid t \in ]-1, 1] \}
\end{equation}
\begin{equation}
Q_{4} = \{ (t, 1) \mid t \in [-1, 1[ \}
\end{equation}
$Q_{1}, \dots, Q_{4}$ sind eine Partition von $\partial Q$.

\begin{equation}
g_{1}(t) = f(-1, t) = 1 - t + t^{2} - 1 + t + 1 = t^{2} + 1
\end{equation}
\begin{equation}
g_{1}'(t) = 2t
\end{equation}
\begin{equation}
g_{1}''(t) = 2
\end{equation}
\begin{equation}
g_{2}(t) = f(1, t) = 1 + t + t^{2} + 1 + t + 1 = t^{2} + 2t + 3
\end{equation}
\begin{equation}
g_{2}'(t) = 2t + 2
\end{equation}
\begin{equation}
g_{2}''(t) = 2
\end{equation}
\begin{equation}
g_{3}(t) = f(t, -1) = t^{2} - t + 1 + t - 1 + 1 = t^{2} + 1
\end{equation}
\begin{equation}
g_{4}(t) = f(t, 1) = t^{2} + t + 1 + t + 1 + 1 = t^{2} + 2t + 3
\end{equation}
\begin{equation}
g_{1}'(t) = 0 \Leftrightarrow t = 0
\end{equation}
\begin{equation}
g_{2}'(t) = 0 \Leftrightarrow t = -1
\end{equation}
Für $g_{3}$ und $g_{4}$ gilt es genauso aufgrund von Gleichheit und an den zweiten Ableitungen sieht man, dass beides Minima sind.
\begin{equation}
f(0, -1) = 1 > f(-\frac{1}{3}, -\frac{1}{3})
\end{equation}
\begin{equation}
f(1, -1) = 2 > f(-\frac{1}{3}, -\frac{1}{3})
\end{equation}
$f(-\frac{1}{3}, -\frac{1}{3}) = \frac{2}{3}$ ist also das globale Minimum auf $Q$.
Die Maxima von $g_{k}$ sind die Randwerte.
\begin{equation}
g_{1}(-1) = 2
\end{equation}
\begin{equation}
g_{1}(1) = 2
\end{equation}
\begin{equation}
g_{2}(-1) = 2
\end{equation}
\begin{equation}
g_{2}(1) = 6
\end{equation}
Also ist das globale Maximum von $f$ auf $Q$ $f(1, 1) = 6$.

\section*{Aufgabe 25}

\begin{equation}
x + y + z = 60 \Leftrightarrow z = 60 - x - y
\end{equation}

Ich definiere $f: \{ (x, y) \mid x, y \ge 0 \land x + y \le 60 \} \rightarrow \mathbb{R}$ durch
\begin{equation}
f(x, y, z) = xyz = xy(60 - x - y) = 60xy - x^{2}y - xy^{2} := f(x, y)
\end{equation}
\begin{equation}
\nabla f = \begin{pmatrix}
60y - 2yx - y^{2}\\
60x - x^{2} - 2xy
\end{pmatrix}
\end{equation}
\begin{align*}
& 60y - 2yx - y^{2} = 0 \land 60x - x^{2} - 2xy = 0\\
\Leftrightarrow & y(60 - 2x - y) = 0 \land x(60 - x - 2y) = 0\\
\Leftrightarrow & y = 0 \lor 60 - 2x - y = 0 \land x = 0 \lor 60 - x - 2y = 0\\
\Leftrightarrow & y = 0 \lor y = 60 - 2x \land x = 0 \lor x = 60 - 2y\\
\Leftrightarrow & y = 0 \lor y = 60 - 2x \land x = 0 \lor x = 60 \lor x = 60 - 2(60 - 2x)\\
\Leftrightarrow & y = 0 \lor y = 60 - 2x \land x = 0 \lor x = 60 \lor x = 20\\
\Leftrightarrow & y = 0 \lor y = 60 \lor y = 20 \land x = 0 \lor x = 60 \lor x = 20\\
\end{align*}
Weil $f(x, y, z) = 0$ wenn $x = 0$, $y = 0$ oder $z = 0$ und beide andere Variablen $0$ sein müssen, wenn eine $60$ ist, wegen $x + y + z = 60$, bleibt nur $(20, 20)$ als kritische Stelle, die Sinn ergibt.
Aus diesem Grund, kann das Maximum auch nicht auf dem Rand liegen.

\begin{equation}
Hf(x, y) = \begin{pmatrix}
2y & 60 - 2x - 2y\\
60 - 2x - 2y & 2x
\end{pmatrix}
\end{equation}
\begin{equation}
Hf(20, 20) = \begin{pmatrix}
40 & -20\\
-20 & 40
\end{pmatrix}
\end{equation}

Nach dem Kriterium von Hurwitz ist diese Matrix positiv definit und $(20, 20)$ somit ein Minimum, was aber nicht stimmt.

Das maximale Produkt erhält man mit der Belegung $(20, 20, 20)$.

\section*{Aufgabe 26}

\subsection*{Teil a}

\subsubsection*{Teil i}

\begin{proof}
\begin{equation}
\nabla f = \begin{pmatrix}
-8x^{3} + 6xy\\
3x^{2} - 2y
\end{pmatrix}
\end{equation}
\begin{align*}
& -8x^{3} + 6xy = 0 \land 3x^{2} - 2y = 0\\
\Leftrightarrow & -8x^{3} + 6xy = 0 \land 3x^{2} = 2y\\
\Leftrightarrow & -8x^{3} + 3x3x^{2} = 0 \land 3x^{2} = 2y\\
\Leftrightarrow & x^{3} = 0 \land 3x^{2} = 2y\\
\Leftrightarrow & x = 0 \land 3x^{2} = 2y\\
\Leftrightarrow & x = 0 \land y = 0
\end{align*}
\end{proof}

\subsubsection*{Teil ii}

\begin{proof}
Sei $v = (x, y) \in \mathbb{R}^{2} \setminus \{0\}$.
\begin{align*}
g(t) = f(tv) & = -(ty - t^{2}x^{2})(ty - 2t^{2}x^{2})\\
& = -2t^{4}x^{4} + 3t^{2}x^{2}ty - t^{2}y^{2}\\
& = -2x^{4}t^{4} + 3x^{2}yt^{3} - y^{2}t^{2}\\
\end{align*}
\begin{equation}
g'(t) = -8x^{4}t^{3} + 9x^{2}yt^{2} - 2y^{2}t
\end{equation}
\begin{align*}
& g(t) = 0\\
\Leftrightarrow & t(-8x^{4}t^{2} + 9x^{2}yt - 2y^{2}) = 0\\
\Leftrightarrow & t = 0 \lor -8x^{4}t^{2} + 9x^{2}yt - 2y^{2} = 0\\
\end{align*}
$0$ ist also eine kritische Stelle von $g$.

\begin{equation}
g''(t) = -24x^{4}t^{2} + 18x^{2}yt - 2y^{2}
\end{equation}
\begin{equation}
g''(0) = -2y^{2} < 0 \qquad \textit{weil $y \ne 0$}
\end{equation}
$0$ ist also ein striktes lokales Maximum.
\end{proof}

\subsubsection*{Teil iii}

\begin{proof}
\begin{equation}
Hf(x, y) = \begin{pmatrix}
-24x^{2} + 6y & 6x\\
6x & -2
\end{pmatrix}
\end{equation}
\begin{equation}
Hf(0, 0) = \begin{pmatrix}
0 & 0\\
0 & -2
\end{pmatrix}
\end{equation}
\end{proof}

\subsection*{Teil b}

\subsubsection*{Teil i}

\subsubsection*{Teil ii}

\subsubsection*{Teil iii}

\section*{Aufgabe 27}

\begin{equation}
\nabla f = \begin{pmatrix}
2x\\
2y + 1
\end{pmatrix}
\end{equation}
Die kritischen Stellen von $f$ sind $(x, y)$ mit
\begin{equation}
2x = 0 \land 2y + 1 = 0 \Leftrightarrow x = 0 \land y = -\frac{1}{2}
\end{equation}
\begin{equation}
Hf(x, y) = \begin{pmatrix}
2 & 0\\
0 & 2
\end{pmatrix} = Hf(0, -\frac{1}{2})
\end{equation}
$(0, -\frac{1}{2})$ ist also ein striktes lokales Minimum, weil die Eigenwerte von $Hf(0, \frac{1}{2})$ alle ($2$ und $2$) positiv sind.

Um das globale Minimum und Maximum auf $D$ zu bestimmen, muss man jetzt noch die Randwerte untersuchen.
Wir wissen, dass $\partial D = \{ \xi \in \mathbb{R}^{2} \mid ||\xi||_{2} = 1 \}$, weil $D = \overline{B}_{1}(0, 0)$.
Es ist $\partial D = \{ (\sin t, \cos t) \mid t \in [0, 2\pi[ \} =: B$.

\begin{proof}
Sei $(x, y) \in \partial D$.
\begin{align*}
& ||(x, y)||_{2} = 1\\
\Leftrightarrow & \sqrt{x^{2} + y^{2}} = 1\\
\Leftrightarrow & x^{2} + y^{2} = 1\\
\Rightarrow & \exists t \in [0, 2\pi[ \quad \sin(t)^{2} + \cos(t)^{2} = 1 = x^{2} + y^{2}\\
\Rightarrow & (x, y) \in B \quad \textit{weil es genau eine solche Aufteilung von $1$ gibt}
\end{align*}

Sei $(\sin t, \cos t) \in B$.
\begin{equation}
||(\sin t, \cos t)||_{2} = \sqrt{\sin(t)^{2}, \cos(t)^{2}} = \sqrt{1} = 1 \Rightarrow (\sin t, \cos t) \in \partial D
\end{equation}
\end{proof}

Ich definiere $g: \mathbb{R} \rightarrow \mathbb{R}$.
\begin{equation}
g(t) = f(\sin t, \cos t) = \sin(t)^{2} + \cos(t)^{2} + \cos(t) = \cos(t) + 1
\end{equation}
\begin{equation}
g'(t) = -\sin(t)
\end{equation}
\begin{equation}
-\sin(t) = 0 \Leftrightarrow \sin(t) = 0 \Leftrightarrow t = 0 \lor t = \frac{1}{2}\pi
\end{equation}
$0$ und $\frac{1}{2}\pi$ sind also die kritischen Stellen von $g$.
\begin{equation}
g''(t) = -\cos(t)
\end{equation}
\begin{equation}
g''(0) = -1
\end{equation}
\begin{equation}
g''(\frac{1}{2}\pi) = 1
\end{equation}
Also ist $0$ ein lokales Maximum und $\frac{1}{2}\pi$ ein lokales Minimum von $g$.
Dann ist $f(0, -\frac{1}{2})$ ein lokales Minimum von $f$, $g(0)$ der maximale Randwert von $f$ und $g(\frac{1}{2}\pi)$ der minimale Randwert von $f$.
Ein Vergleich der Werte ergibt
\begin{equation}
f(0, -\frac{1}{2}) = -\frac{1}{4}
\end{equation}
\begin{equation}
g(0) = f(0, 1) = 2
\end{equation}
\begin{equation}
g(\frac{1}{2}\pi) = f(1, 0) = 1
\end{equation}
Also ist $f(0, -\frac{1}{2})$ das globale Minimum und $f(0, 1)$ das globale Maximum auf $D$.

Das Minimum ist natürlich in $D$.
\begin{equation}
||(0, -\frac{1}{2}||_{2} = \frac{1}{2} \le 1
\end{equation}

\end{document}