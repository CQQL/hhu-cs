\documentclass[10pt,a4paper]{article}
\usepackage[utf8]{inputenc}
\usepackage[german]{babel}
\usepackage{amsmath}
\usepackage{amsfonts}
\usepackage{amssymb}
\usepackage{amsthm}
\usepackage[left=2cm,right=2cm,top=2cm,bottom=2cm]{geometry}

\begin{document}

\section*{Übung 28}

\subsection*{Teil a}

Zuerst löse ich die zugehörige homogene DGl
\begin{equation}
y' = y
\end{equation}
\begin{align*}
& y' = \frac{dy}{dx} = y\\
\Leftrightarrow & \frac{1}{y} dy = dx\\
\Leftrightarrow & \int \frac{1}{y} dy = \int dx\\
\Leftrightarrow & \log(y) + c_{2} = x + c_{1}\\
\Leftrightarrow & \log(y) = x + c_{1} - c_{2}\\
\Leftrightarrow & y = \exp(x + c_{1} - c_{2}) = \exp(c_{1} - c_{2}) \exp(x)\\
\Leftrightarrow & y = c \cdot \exp(x)\\
\end{align*}
Da $c_{1}$ und $c_{2}$ frei wählbare Konstanten sind, ist $\exp(c_{1} - c_{2})$ auch konstant und ich schreibe dafür einfach $c$.
Die Lösung der homogenen DGl nenne ich $\varphi$.

Sei $\psi$ eine Lösung der DGl und $u$ eine Funktion, sodass $\psi = u \cdot \varphi$.
\begin{align*}
\psi' = u' \varphi + u \varphi' = u' \varphi + u \varphi = \psi + u' \varphi
\end{align*}
Ich setze $u' \varphi = 1$.
\begin{equation}
u' = \frac{1}{\varphi} \Leftrightarrow u = \int \frac{1}{\varphi} dx = \int \frac{1}{c \cdot \exp(x)}dx = \frac{1}{c} \int \exp(-x) dx = \frac{1}{c} \left( -\exp(-x) + d \right)
\end{equation}
Durch Einsetzen kann ich nun $\psi$ bestimmen.
\begin{equation}
\psi = \frac{1}{c} \left( -\exp(-x) + d \right) \cdot c \cdot \exp(x) = \left( -\exp(-x) + d \right) \cdot \exp(x) = d \exp(x) - \exp(x - x) = d \exp(x) - 1
\end{equation}
$\psi$ sind alle Lösungen für alle $d \in \mathbb{R}$.

\subsection*{Teil b}

Ich gehe genauso vor wie in Teil a bis zu dem Punkt, wo ich $u'\varphi = 1$ setze.
Stattdessen setze ich $u' \varphi = x$.
\begin{equation}
u' = \frac{x}{\varphi} \Leftrightarrow u = \int \frac{x}{\varphi} dx = \int \frac{x}{c \cdot \exp(x)} = \frac{1}{c} \int x \cdot \exp(-x) dx = \frac{1}{c} \left[ (-x - 1) \exp(-x) + k \right]\ \ \textit{siehe Nebenrechnung}
\end{equation}
Nebenrechnung: Ich setze $f = x$ und $g' = \exp(-x)$ und wende partielle Integration an.
Es sind $f' = 1$ und $g = d - \exp(-x)$ wie in Teil a.
\begin{align*}
\int f \cdot g' dx = fg - \int f' \cdot g dx & = fg - \int d - \exp(-x) dx\\
& = fg - (dx + e_{1}) + \int \exp(-x) dx\\
& = fg - (dx + e_{1}) + (e_{2} - \exp(-x))\\
& = x \cdot (d - \exp(-x)) - (dx + e_{1}) + (e_{2} - \exp(-x))\\
& = xd - x\exp(-x) - dx - e_{1} + e_{2} - \exp(-x)\\
& = - x\exp(-x) - \exp(-x) - e_{1} + e_{2} \qquad \textit{Ich definiere $k := - e_{1} + e_{2}$}\\
& = (-x - 1)\exp(-x) + k\\
\end{align*}

Wieder bestimme ich $\psi$ durch Einsetzen:
\begin{align*}
\psi & = c \cdot \exp(x) \cdot \frac{1}{c} \left[ (-x - 1) \exp(-x) + k \right]\\
& = \exp(x) \cdot \left[ (-x - 1) \cdot \exp(-x) + k \right]\\
& = \exp(x) \cdot \left[ \frac{(-x - 1)}{\exp(x)} + k \right]\\
& = \frac{(-x - 1)}{\exp(x)} \exp(x) + k\exp(x)\\
& = k\exp(x) + (-x - 1)\\
& = k\exp(x) - x - 1\\
\end{align*}
Dann ist $\psi$ alle Lösungen für alle $k \in \mathbb{R}$.

\section*{Übung 29}

\begin{proof}
Um die Form von $\varphi$ zu bestimmen, löse ich zunächst die zugehörige homogene DGl $y' = y$.
Diese wird, wie in Übung 28 gesehen, durch $\psi_{c} = c \cdot \exp(x)$ gelöst.
Nun wende ich die Methode der Variation der Konstanten an und wähle die Funktion $u$ so, dass $\varphi = u \psi_{c}$.
\begin{equation}
\varphi' = u' \psi_{c} + u \psi_{c}' = u'\psi_{c} + u \psi_{c} = \varphi + u' \psi_{c}
\end{equation}
Ich setze $u' \psi_{c} = p$.
\begin{align*}
u = \int \frac{p}{\psi_{c}} = \int \frac{p}{c \cdot \exp(x)} = \frac{1}{c} \int p \cdot \exp(-x)
\end{align*}
\end{proof}

\section*{Übung 30}

\section*{Übung 31}

\subsection*{Teil a}

\subsection*{Teil b}

\section*{Übung 32}

\end{document}