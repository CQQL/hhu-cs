\documentclass[10pt,a4paper]{article}
\usepackage[utf8]{inputenc}
\usepackage[german]{babel}
\usepackage{amsmath}
\usepackage{amsfonts}
\usepackage{amssymb}
\usepackage{amsthm}
\usepackage[left=2cm,right=2cm,top=2cm,bottom=2cm]{geometry}

\begin{document}

\section*{Übung 14}

\subsection*{Teil a}

\begin{equation}
D_{1} f(x) = 1
\end{equation}
\begin{equation}
D_{1} D_{1} f(x) = 0
\end{equation}
\begin{equation}
D_{2} D_{1} f(x) = 0
\end{equation}

\begin{equation}
D_{2} f(x) = 3y^{2}
\end{equation}
\begin{equation}
D_{1} D_{2} f(x) = 0
\end{equation}
\begin{equation}
D_{2} D_{2} f(x) = 6y
\end{equation}

\subsection*{Teil b}

\begin{equation}
D_{1} f(x) = y^{3}
\end{equation}
\begin{equation}
D_{1} D_{1} f(x) = 0 
\end{equation}
\begin{equation}
D_{2} D_{1} f(x) = 3y^{2}
\end{equation}

\begin{equation}
D_{2} f(x) = 3xy^{2}
\end{equation}
\begin{equation}
D_{1} D_{2} f(x) = 3y^{2}
\end{equation}
\begin{equation}
D_{2} D_{2} f(x) = 6xy
\end{equation}

\subsection*{Teil c}

\begin{equation}
D_{1} f(x) = \cos (x + y^{3})
\end{equation}
\begin{equation}
D_{1} D_{1} f(x) = -\sin (x + y^{3})
\end{equation}
\begin{equation}
D_{2} D_{1} f(x) = -3y^{2} \sin(x + y^{3})
\end{equation}

\begin{equation}
D_{2} f(x) = 3y^{2} \cos(x + y^{3})
\end{equation}
\begin{equation}
D_{1} D_{2} f(x) = -3y^{2} \sin(x + y^{3})
\end{equation}
\begin{equation}
D_{2} D_{2} f(x) = 6y \cos(x + y^{3}) - 9y^{4} \sin (x + y^{3})
\end{equation}

\subsection*{Teil d}

\begin{equation}
D_{1} f(x) = y^{3} \cos (xy^{3})
\end{equation}
\begin{equation}
D_{1} D_{1} f(x) = -y^{6} \sin (xy^{3})
\end{equation}
\begin{equation}
D_{2} D_{1} f(x) = 3y^{2} \cos(xy^{3}) - 3xy^{5} \sin(xy^{3})
\end{equation}

\begin{equation}
D_{2} f(x) = 3xy^{2} \cos(xy^{3})
\end{equation}
\begin{equation}
D_{1} D_{2} f(x) = 3y^{2} \cos (xy^{3}) - 3xy^{5} \sin(xy^{3})
\end{equation}
\begin{equation}
D_{2} D_{2} f(x) = 6xy \cos(xy^{3}) - 9x^{2}y^{4} \sin(xy^{3})
\end{equation}

\section*{Übung 15}

\subsection*{Teil a}

Nach dem Satz von H.A. Schwarz gilt für solche $f \in C^{2}$
\begin{equation}
\frac{\partial^{2} f}{\partial y \partial x} = \frac{\partial^{2} f}{\partial x \partial y}
\end{equation}
aber
\begin{equation}
\frac{\partial^{2} f}{\partial x \partial y} = x^{3}
\end{equation}
\begin{equation}
\frac{\partial^{2} f}{\partial y \partial x} = 4x^{3}
\end{equation}
Das ist ein Widerspruch und folglich gibt es kein solches $f$.

\subsection*{Teil b}

Sei $c \in \mathbb{R}$.
\begin{equation}
f_{c}(x, y) = \frac{1}{5}x^{5} + \frac{1}{4}x^{4}y + \frac{1}{3}y^{3} + y + c
\end{equation}

\section*{Übung 16}

\begin{proof}
Für $(x, y) \ne (0, 0)$ ist $f$ partiell differenzierbar nach Analysis 1.
Seien $(x_{n})$ und $(y_{n})$ Nullfolgen.
Dann ist 
\begin{equation}
\lim_{n \rightarrow \infty} \frac{f(x_{n}, 0) - f(0, 0)}{x_{n}} = 0
\end{equation}
und \begin{equation}
\lim_{n \rightarrow \infty} \frac{f(0, y_{n}) - f(0, 0)}{y_{n}} = 0
\end{equation}
$f$ ist also auch in $(0, 0)$ partiell differenzierbar.

Konkret ergeben sich
\begin{align}
D_{1}f(x, y) & =
\begin{cases}
\frac{yx^{4} + 4y^{3}x^{2} - y^{5}}{(x^{2} + y^{2})^{2}} & \textit{$(x, y) \ne (0, 0)$}\\
0 & \textit{sonst}
\end{cases}\\
D_{2}f(x, y) & = 
\begin{cases}
-\frac{xy^{4} + 4x^{3}y^{2} - x^{5}}{(x^{2} + y^{2})^{2}} & \textit{$(x, y) \ne (0, 0)$}\\
0 & \textit{sonst}
\end{cases}
\end{align}

Für $(x, y) \ne (0, 0)$ sind $D_{1}f$ und $D_{2}f$ partiell differenzierbar nach Analysis 1.
Seien $(x_{n})$ und $(y_{n})$ Nullfolgen.
Dann ist
\begin{equation}
\lim_{n \rightarrow \infty} \frac{D_{1}f(x_{n}, 0) - D_{1}f(0, 0)}{x_{n}} = 0
\end{equation}
\begin{equation}
\lim_{n \rightarrow \infty} \frac{D_{2}f(x_{n}, 0) - D_{2}f(0, 0)}{x_{n}} = -1 = D_{1}D_{2}f(0, 0)
\end{equation}
\begin{equation}
\lim_{n \rightarrow \infty} \frac{D_{1}f(0, y_{n}) - D_{1}f(0, 0)}{y_{n}} = 1 = D_{2}D_{1}f(0, 0)
\end{equation}
\begin{equation}
\lim_{n \rightarrow \infty} \frac{D_{2}f(0, y_{n}) - D_{2}f(0, 0)}{y_{n}} = 0
\end{equation}
Da alle Grenzwerte existieren, sind $D_{1}f$ und $D_{2}f$ auch in $(0, 0)$ differenzierbar.

Außerdem ist $D_{1}D_{2}f(0, 0) \ne D_{2}D_{1}f(0, 0)$.
\end{proof}

\section*{Übung 17}

\subsection*{Teil a}

\begin{equation}
D_{1} f(1, 0) = 1
\end{equation}
\begin{equation}
D_{2} f(1, 0) = 0
\end{equation}

\subsection*{Teil b}

\begin{proof}
Ich betrachte $D_{1}f$ im Punkt $(1, 1)$.
Sei $h \in \mathbb{R}$ mit $0 < h < 1$.
Dann ist
\begin{equation}
\lim_{h \rightarrow 0} \frac{F_{1}(1 - h) - F_{1}(1)}{h} = \lim_{h \rightarrow 0} \frac{1 - 1}{h} = 0
\end{equation}
und
\begin{equation}
\lim_{h \rightarrow 0} \frac{F_{1}(1 + h) - F_{1}(1)}{h} = \lim_{h \rightarrow 0} \frac{1 + h - 1}{h} = 1
\end{equation}
$F_{1}$ ist also in $(1, 1)$ nicht differenzierbar und somit auch $f$ nicht.
\end{proof}

\end{document}