\documentclass[10pt,a4paper]{article}
\usepackage[utf8]{inputenc}
\usepackage[german]{babel}
\usepackage{amsmath}
\usepackage{amsfonts}
\usepackage{amssymb}
\usepackage{amsthm}
\usepackage[left=2cm,right=2cm,top=2cm,bottom=2cm]{geometry}

\begin{document}

\subsection*{Aufgabe 54}

\begin{proof}
Da $F$ Summe/Produkt von $C^{\infty}$-Funktionen ist, ist es selbst in $C^{\infty}$.
Seien $x_{0} = 0 = y_{0}$.
Dann ist $F(x_{0}, y_{0}) = 0$ und $\frac{\partial F}{\partial y}(x, y) = ye^{y} + e^{y} + x$ und $\frac{\partial F}{\partial y}(x_{0}, y_{0}) = 1 \ne 0$.
Also ist $\frac{\partial F}{\partial y}(x_{0}, y_{0})$ invertierbar.
Damit sind alle Bedingungen für den Satz über implizite Funktionen erfüllt.
Es gibt also eine offene Kugel in $\mathbb{R}$ ($]-\varepsilon, \varepsilon[$) um $x_{0} = 0$ und eine eindeutig bestimmte Funktion $f : ]-\varepsilon, \varepsilon[ \rightarrow \mathbb{R}$ mit $f(x_{0}) = f(0) = y_{0} = 0$ und $F(x, f(x)) = 0$ für alle $x \in ]-\varepsilon, \varepsilon[$.
Weil $F$ in $C^{\infty}$ ist, ist $f$ es ebenfalls.

\begin{equation}
\frac{\partial F}{\partial x}(x, y) = xe^{x} + e^{x} + y
\end{equation}
\begin{equation}
Df(x_{0}) = f'(x_{0}) = f'(0) = \frac{\partial F}{\partial y}(x_{0}, y_{0})^{-1} \cdot \frac{\partial F}{\partial x}(x_{0}, y_{0}) = 1 \cdot 1 = 1
\end{equation}
\end{proof}

\subsection*{Aufgabe 55}

\subsection*{Teil a}

\begin{proof}
\begin{align*}
& (x, y) \in L\\
\Leftrightarrow & \sqrt{(x - 1)^{2} + y^{2}}\sqrt{(x + 1)^{2} + y^{2}} = \sqrt{((x - 1)^{2} + y^{2})((x + 1)^{2} + y^{2})} = 1\\
\Leftrightarrow & ((x - 1)^{2} + y^{2})((x + 1)^{2} + y^{2}) = 1\\
\Leftrightarrow & (x - 1)^{2}(x + 1)^{2} + (x - 1)^{2}y^{2} + (x + 1)^{2}y^{2} + y^{4} = 1\\
\Leftrightarrow & (x^{2} - 2x + 1)(x^{2} + 2x + 1) + (x^{2} - 2x + 1)y^{2} + (x^{2} + 2x + 1)y^{2} + y^{4} = 1\\
\Leftrightarrow & x^{4} + 2x^{3} + x^{2} - 2x^{3} - 4x^{2} - 2x + x^{2} + 2x + 1 + x^{2}y^{2} - 2xy^{2} + y^{2} + x^{2}y^{2} + 2xy^{2} + y^{2} + y^{4} = 1\\
\Leftrightarrow & x^{4} + 2x^{2}y^{2} + y^{4} - 2x^{2} + 2y^{2} + 1 = 1\\
\Leftrightarrow & x^{4} + 2x^{2}y^{2} + y^{4} - 2x^{2} + 2y^{2} = 0\\
\Leftrightarrow & (x^{2} + y^{2})^{2} - 2(x^{2} - y^{2}) = 0\\
\Leftrightarrow & F(x, y) = 0\\
\end{align*}
\end{proof}

\subsection*{Teil b}

\begin{equation}
\frac{\partial F}{\partial y}(x, y) = 4x^{2}y + 4y^{3} + 4y
\end{equation}
\begin{equation}
\frac{\partial F}{\partial y}(x, y) = 0 \Leftrightarrow 4x^{2}y + 4y^{3} + 4y = 0 \Leftrightarrow \begin{cases}
x \in \mathbb{R} & \textit{wenn $y = 0$}\\
x^{2} = - y^{2} - 1& \textit{wenn $y \ne 0$}
\end{cases}
\end{equation}
Für $y \ne 0$, gibt es also keine Lösung.
Es bleibt zu bestimmen, welche $(x, 0) \in L$ sind.
\begin{equation}
F(x, 0) = 0 \Leftrightarrow x^{4} - 2x^{2} = x^{2}(x^{2} - 2) = 0 \Leftrightarrow x = 0\ \lor\ x = \sqrt{2}\ \lor\ x = -\sqrt{2}
\end{equation}

\subsection*{Teil c}

\subsection*{Teil d}

Weil $F(x, y) = 0$ für alle $(x, y) \in L$ und die sonstigen Bedingungen auch erfüllt sind für $y \ne 0$, gibt es dort immer eine Funktion $f$, sodass $F(x, f(x)) = 0$ und $f$ nimmt sein Maximum an wegen Teil c.
\begin{equation}
f'(x) = 
\end{equation}

\subsection*{Teil e}

\subsection*{Aufgabe 56}

\subsection*{Teil a}

\begin{equation}
f(x, y) = \begin{pmatrix}
x^{2} + y^{2}\\
2yx
\end{pmatrix}
\end{equation}
\begin{equation}
Df(x, y) = \begin{pmatrix}
2x & 2y\\
2y & 2x
\end{pmatrix}
\end{equation}
\begin{equation}
Df(x, y)\textit{ ist invertierbar} \Leftrightarrow \det(Df(x, y)) = 4(x^{2} - y^{2}) \ne 0 \Leftrightarrow x \ne y\ \land\ x \ne -y
\end{equation}
$Df(x, y)$ ist also für alle Punkte in $\mathbb{R}^{2}$ invertierbar, die nicht auf einer der zwei Hauptdiagonalen liegen.

\subsection*{Teil b}

\begin{proof}
Da die beiden Komponentenfunktionen stetig sind, ist $f$ ebenfalls stetig.

\end{proof}

\subsection*{Aufgabe 57}

\subsection*{Teil a}

\begin{equation}
f(a, b) = \begin{pmatrix}
\sum_{k = 0\ \textit{$k$ gerade}}^{n} \binom{n}{k} (-1)^{\frac{k}{2}} a^{n - k}b^{k}\\
\sum_{k = 1\ \textit{$k$ ungerade}}^{n} \binom{n}{k} (-1)^{\frac{k - 1}{2}} a^{n - k}b^{k}
\end{pmatrix}
\end{equation}
\begin{equation}
Df(a, b) = \begin{pmatrix}
\sum_{k = 0\ \textit{$k$ gerade}}^{n} \binom{n}{k} (-1)^{\frac{k}{2}} (n - k) a^{n - k - 1}b^{k} & \sum_{k = 0\ \textit{$k$ gerade}}^{n} \binom{n}{k} (-1)^{\frac{k}{2}} k a^{n - k}b^{k - 1}\\
\sum_{k = 1\ \textit{$k$ ungerade}}^{n} \binom{n}{k} (-1)^{\frac{k - 1}{2}} (n - k) a^{n - k - 1}b^{k} &
\sum_{k = 1\ \textit{$k$ ungerade}}^{n} \binom{n}{k} (-1)^{\frac{k - 1}{2}} k a^{n - k}b^{k - 1}
\end{pmatrix}
\end{equation}

\subsection*{Teil b}

\end{document}