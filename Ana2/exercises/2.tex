\documentclass[10pt,a4paper]{article}
\usepackage[utf8]{inputenc}
\usepackage[german]{babel}
\usepackage{amsmath}
\usepackage{amsfonts}
\usepackage{amssymb}
\usepackage{amsthm}
\usepackage[left=2cm,right=2cm,top=2cm,bottom=2cm]{geometry}

\begin{document}

\section*{Aufgabe 5}

\subsection*{a}

\begin{align}
|d(x, z) - d(y, z)| \le |d(x, y) + d(y, z) - d(y, z)| = |d(x, y)| = d(x, y)
\end{align}

\subsection*{b}

Sei $a \in A$.
Nach Teil a) gilt
\begin{equation}
|d(x, A) - d(y, A)| \le |d(x, a) - d(y, a)| \le d(x, y)
\end{equation}

\subsection*{c}

\begin{equation}
d((0, 0), A) = 0
\end{equation}
\begin{equation}
d((3, 0), A) = 1
\end{equation}
\begin{equation}
d((0, 1), A) = \sqrt{2} - 1
\end{equation}

\section*{Aufgabe 6}

\begin{equation}
\mathring{A} = A - \{(x, y) \in A \mid x^{2} + y^{2} = 1\}
\end{equation}
\begin{equation}
\overline{A} = A + \{0\}
\end{equation}
\begin{equation}
\partial A = \{(x, y) \in A \mid x^{2} + y^{2} = 1\}
\end{equation}

\section*{Aufgabe 7}

\subsection*{$A_{1}$}

Sei $(x, y) \in A_{1}$, $r = \min \{|x|, y\}$ und $(a, b) \in B_{r}((x, y))$.
Dann gilt
\begin{align*}
d((a, b), (x, y)) = |(a - x, b - y)| = & \sqrt{|a - x|^{2} + |b - y|^{2}} < r\\
\Rightarrow & |a - x|^{2} + |b - y|^{2} < r^{2}\\
\Rightarrow & |b - y|^{2} < r^{2} - |a - x|^{2} < r^{2}\\
\Rightarrow & |b - y| < r\\
\Rightarrow & b \in ]y - r, y + r[\\
\Rightarrow & b > 0
\end{align*}
Also $B_{r}((x, y)) \in A_{1}$ für alle $(x, y) \in A_{1}$ und $A_{1}$ ist offen in $X$.

Sei $q = (1, 0)$, $0 < r < 1$ und $y = \frac{r}{2} > 0$.
Dann ist $(1, y) \in A_{1}$ und $(1, y) \in B_{r}(q)$.
\begin{equation}
d((1, y), (1, 0)) = y < r
\end{equation}
Das heißt, dass $q$ ein Berührpunkt von $A_{1}$ ist, aber $q \notin A_{1}$. 
$A_{1}$ ist also nicht abgeschlossen in $X$.

\subsection*{$A_{2}$}

Sei $y = (1, 0)$ und $0 < r < 1$.
Dann ist $y \in A_{2}$.
\begin{equation}
d(y, (1, -\frac{r}{2})) = \frac{r}{2} < r \Rightarrow (1, -\frac{r}{2}) \in B_{r}(y)
\end{equation}
Aber es gilt auch
\begin{equation}
-\frac{r}{2} < 0 \Rightarrow (1, -\frac{r}{2}) \notin A_{2} \Rightarrow B_{r}(y) \nsubseteq A_{2}
\end{equation}
$A_{2}$ ist also nicht offen in $X$.

Sei $(x, y)$ ein Berührpunkt von $A_{2}$.

$A_{2}$ ist geschlossen in $X$.

\subsection*{$A_{3}$}

Sei $(x, y) \in A_{3}$ und $(a, b) \in B_{x}((x, y))$.
\begin{align}
d((x, y), (a, b)) = & \sqrt{|x - a|^{2} + |y - b|^{2}} < x\\
\Rightarrow & |x - a|^{2} + |y - b|^{2} < x^{2}\\
\Rightarrow & |x - a|^{2} < x^{2} - |y - b|^{2} \le x^{2}\\
\Rightarrow & |x - a| < x\\
\Rightarrow & a \in ]0, 2x[ \Rightarrow a > 0
\end{align}
$A_{3}$ ist also Umgebung aller seiner Punkte und offen in $X$.

$A_{3}$ ist auch geschlossen in $X$.

\subsection*{$A_{4}$}

\section*{Aufgabe 8}

\subsection*{a}

\begin{proof}
Seien $x, y \in B_{r}(0)$.
\begin{align}
|\lambda y + (1 - \lambda)x| & \le |\lambda y| + |(1 - \lambda) x|\\
& = \lambda |y| + (1 - \lambda) |x|\\
& < \lambda r + (1 - \lambda) r = r\\
& \Rightarrow |\lambda y + (1 - \lambda)x| \in B_{r}(0)\\
\end{align}

Seien $x, y \in \overline{B}_{r}(0)$.
\begin{align}
|\lambda y + (1 - \lambda)x| & \le |\lambda y| + |(1 - \lambda) x|\\
& = \lambda |y| + (1 - \lambda) |x|\\
& \le \lambda r + (1 - \lambda) r = r\\
& \Rightarrow |\lambda y + (1 - \lambda)x| \in \overline{B}_{r}(0)\\
\end{align}
\end{proof}

\subsection*{b}

\begin{proof}
Seien $x, y \in B_{r}(a)$ und $0 \le \lambda \le 1$.
\begin{align}
d(\lambda y + (1 - \lambda)x, a) & = |\lambda y + (1 - \lambda)x - a|\\
& = |\lambda y + (1 - \lambda)x - (\lambda + (1 - \lambda))a|\\
& = |\lambda y - \lambda a + (1 - \lambda)x - (1 - \lambda)a|\\
& \le |\lambda y - \lambda a| + |(1 - \lambda)x - (1 - \lambda)a|\\
& = \lambda |y - a| + (1 - \lambda) |x - a|\\
& = \lambda d(y, a) + (1 - \lambda) d(x, a)\\
& < \lambda r + (1 - \lambda) r = r\\
& \Rightarrow \lambda y + (1 - \lambda)x \in B_{r}(a)
\end{align}

Seien $x, y \in \overline{B}_{r}(a)$ und $0 \le \lambda \le 1$.
\begin{align}
d(\lambda y + (1 - \lambda)x, a) & = |\lambda y + (1 - \lambda)x - a|\\
& = |\lambda y + (1 - \lambda)x - (\lambda + (1 - \lambda))a|\\
& = |\lambda y - \lambda a + (1 - \lambda)x - (1 - \lambda)a|\\
& \le |\lambda y - \lambda a| + |(1 - \lambda)x - (1 - \lambda)a|\\
& = \lambda |y - a| + (1 - \lambda) |x - a|\\
& = \lambda d(y, a) + (1 - \lambda) d(x, a)\\
& \le \lambda r + (1 - \lambda) r = r\\
& \Rightarrow \lambda y + (1 - \lambda)x \in \overline{B}_{r}(a)
\end{align}
\end{proof}

\subsection*{c}

Seien $n \ge 2$, $0 < p < 1$ und $v, w \in \mathbb{R}^{n}$ mit $v = (1, 0, 0, \dots, 0)$ und $w = (0, 1, 0, \dots, 0)$.
\begin{align}
||v + w||_{p} & = (1^{p} + 1^{p})^{\frac{1}{p}}\\
& = 2^{\frac{1}{p}}\\
& > 2\\
& = 1 + 1\\
& = (1^{p})^{\frac{1}{p}} + (1^{p})^{\frac{1}{p}}\\
& = ||v||_{p} + ||w||_{p}
\end{align}
Es gibt also mindestens ein Gegenbeispiel zur Dreiecksungleichung, wenn $0 < p < 1$, sodass $||.||_{p}$ keine Norm ist.

\end{document}