\documentclass[10pt,a4paper]{article}
\usepackage[utf8]{inputenc}
\usepackage[german]{babel}
\usepackage{mathrsfs}
\usepackage{amsmath}
\usepackage{amsfonts}
\usepackage{amssymb}
\usepackage{amsthm}
\usepackage[left=2cm,right=2cm,top=2cm,bottom=2cm]{geometry}

\begin{document}

\section{Aufgabe 21}

\subsection{Teil a}
\begin{equation}
  P(A) = \frac{1}{2}, P(A^{C}) = \frac{1}{2}, P(B \mid A^{C}) = \frac{1}{288}, P(B \mid A) = \frac{1}{12}
\end{equation}

\subsection{Teil b}
\begin{equation}
  P(B^{C} \mid A^{C}) = 1 - P(B \mid A^{C}) = \frac{287}{288}
\end{equation}
\begin{equation}
  P(A \cap B) = P(A) \cdot P(B \mid A) = \frac{1}{2} \cdot \frac{1}{12} = \frac{1}{24}
\end{equation}

\subsection{Teil c}
\begin{equation}
  P(B) = P(B \cap A) + P(B \cap A^{C}) = P(B \cap A) + P(B \mid A^{C}) \cdot P(A^{C}) = \frac{1}{24} + \frac{1}{576} = \frac{25}{576}
\end{equation}

\subsection{Teil d}
\begin{equation}
  P(A \mid B) = \frac{P(B \mid A) \cdot P(A)}{P(B)} = \frac{24}{25}
\end{equation}

\section{Aufgabe 22}
Wir definieren die folgenden Ereignisse
\begin{description}
\item[A] Die Alarmanlage geht los
\item[B] Ein Mensch macht sich zu schaffen
\item[C] Heiner ist am Werk
\end{description}
Aber B und C sind das selbe, weil Heiner der einzige Mensch ist, der von den Kois weiß.
Dann kennen wir die folgenden Wahrscheinlichkeiten
\begin{equation}
  P(A \mid C) = 0.99
\end{equation}
\begin{equation}
  P(A \mid C^{C}) = 0.02
\end{equation}
\begin{equation}
  P(C) = 0.001
\end{equation}
Gesucht ist $P(C \mid A)$.
\begin{equation}
  P(A) = P(A \cap C) + P(A \cap C^{C}) = P(A \mid C) \cdot P(C) + P(A \cap C^{C}) \cdot P(C^{C}) = 0.99 \cdot 0.001 + 0.02 \cdot (1 - 0.001) = 0.02097
\end{equation}
\begin{equation}
  P(C \mid A) = \frac{P(A \mid C) \cdot P(C)}{P(A)} = \frac{0.99 \cdot 0.001}{0.02097} = 0.0472103004292
\end{equation}

\section{Aufgabe 23}

\subsection{Teil a}

\subsection{Teil b}

\section{Aufgabe 24}

\subsection{Teil a}

\subsection{Teil b}

\end{document}