\documentclass[10pt,a4paper]{article}
\usepackage[utf8]{inputenc}
\usepackage[german]{babel}
\usepackage{mathrsfs}
\usepackage{amsmath}
\usepackage{amsfonts}
\usepackage{amssymb}
\usepackage{amsthm}
\usepackage[left=2cm,right=2cm,top=2cm,bottom=2cm]{geometry}

\begin{document}

Stochastik 1, Blatt 9\\
Gruppe 1\\
Marten Lienen (2126759)\\
Fabian Schmittmann (2083559)

\section{Aufgabe 33}

\subsection{Teil a}

\begin{align*}
  f_{X_{i}}(t) & = \sum_{k = 0}^{\infty} P(X_{i} = k) \cdot t^{k}\\
  & = \sum_{k = 0}^{\infty} \frac{\lambda_{i}^{k}}{k!} \cdot e^{-\lambda_{i}} \cdot t^{k}\\
  & = e^{-\lambda_{i}} \cdot \sum_{k = 0}^{\infty} \frac{\lambda_{i}^{k}}{k!} \cdot t^{k}\\
  & = e^{-\lambda_{i}} \cdot e^{t\lambda_{i}}\\
  & = e^{(t - 1) \cdot \lambda_{i}} = e^{(t - 1) \cdot i^{-1}}
\end{align*}

\begin{align*}
  P\left( S_{n} = k \right) & =
\end{align*}

\begin{align*}
  f_{S_{n}} & = \sum_{k = 0}^{\infty} P(S_{n} = k) \cdot t^{k}\\
  & = \sum_{k = 0}^{\infty}  \cdot t^{k}\\
\end{align*}

\subsection{Teil b}

\begin{proof}
  Sei $\varepsilon > 0$.
\end{proof}

\section{Aufgabe 34}

\subsection{Teil a}

\subsection{Teil b}

\section{Aufgabe 35}

\subsection{Teil a}

\subsection{Teil b}

\subsection{Teil c}

\subsection{Teil d}

\subsection{Teil e}

\section{Aufgabe 36}

\subsection{Teil a}

$Y$ ist $B_{100}, \frac{2}{5}$-verteilt.

\begin{equation}
  E(Y) = 100 \cdot \frac{2}{5} = 40
\end{equation}

\begin{equation}
  V(Y) = 100 \cdot \frac{2}{5} \cdot \frac{3}{5} = 24
\end{equation}

\subsection{Teil b}

\begin{equation}
  P(Y \ge 40) \simeq 0.53792
\end{equation}

\begin{equation}
  P(Y = 40) = P(Y \ge 40) - P(Y \ge 41) \simeq 0.53792 - 0.45671 = 0.08121
\end{equation}

\begin{equation}
  P(30 < Y \le 40) = P(Y \ge 31) - P(Y \ge 41) \simeq 0.97522 - 0.45671 = 0.5185
\end{equation}

\subsection{Teil c}

\begin{align*}
  P(30 < Y \le 40) & = P(31 \le Y \le 40)\\
  & = \Phi\left( \frac{40 + 0.5 - 40}{\sqrt{24}} \right) - \Phi\left( \frac{31 - 0.5 - 40}{\sqrt{24}} \right)\\
  & = \Phi\left( \frac{0.5}{\sqrt{24}} \right) - \Phi\left( -\frac{9.5}{\sqrt{24}} \right)\\
  & \simeq \Phi(0.102062072616) - \Phi(-1.9391793797)\\
  & = \Phi(0.102062072616) - \Phi(-1.9391793797)\\
  & = \Phi(0.102062072616) - (1 - \Phi(1.9391793797))\\
  & \simeq 0.539828 - (1 - 0.973197) = 0.513025
\end{align*}

\end{document}