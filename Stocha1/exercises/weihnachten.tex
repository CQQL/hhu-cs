\documentclass[10pt,a4paper]{article}
\usepackage[utf8]{inputenc}
\usepackage[german]{babel}
\usepackage{mathrsfs}
\usepackage{amsmath}
\usepackage{amsfonts}
\usepackage{amssymb}
\usepackage{amsthm}
\usepackage[left=2cm,right=2cm,top=2cm,bottom=2cm]{geometry}

\begin{document}

\section{Aufgabe W1}

\subsection{Teil a}
Da Vor- und Nachname buchstabendisjunkt sind, ist die Wahrscheinlichkeit, dass ein Buchstabe im richtigen Korb landet, $\frac{1}{2}$.
Die Gesamtwahrscheinlichkeit ist dann $\frac{1}{2}^{13}$.

\subsection{Teil b}
Die Wahrscheinlichkeit den Vornamen in der richtigen Reihenfolge zu ziehen, ist
\begin{equation}
  \frac{3}{7} \cdot \frac{1}{6} \cdot \frac{2}{5} \cdot \frac{1}{4} \cdot \frac{1}{3} \cdot \frac{2}{2} \cdot \frac{1}{1} = \frac{1}{420}
\end{equation}
Beim Nachnamen ist es
\begin{equation}
  \frac{1}{6} \cdot \frac{2}{5} \cdot \frac{2}{4} \cdot \frac{1}{3} \cdot \frac{1}{2} \cdot \frac{1}{1} = \frac{1}{180}
\end{equation}
Also muss Philipp Berger bei dem Papierkorb mit seinem Vornamen die Nachhilfestunden wahrscheinlicher nicht geben.

\subsection{Teil c}

\subsubsection{Teil i}

\begin{equation}
  \frac{419}{420}^{50} \cdot \frac{179}{180}^{100} \simeq 0.51
\end{equation}

\subsubsection{Teil ii}

\begin{equation}
  50 \cdot 20 \cdot \frac{1}{420} + 100 \cdot 20 \cdot \frac{1}{180} = \frac{850}{63} \simeq 13.5
\end{equation}

\subsubsection{Teil iii}

\begin{equation}
  V =
\end{equation}

\subsection{Teil d}

Ich nehme an, dass nur die Studierenden betrachtet werden, die auch auf Blättern stehen/standen.

\subsubsection{Teil i}

\begin{equation}
  P(A) = 0.8
\end{equation}
\begin{equation}
  P(A^{C}) = 0.2
\end{equation}
\begin{equation}
  P(B | A) = 0.75
\end{equation}
\begin{equation}
  P(B | A^{C}) = 0.1
\end{equation}

\subsubsection{Teil ii}

\begin{equation}
  P(B^{C} | A) = 1 - P(B | A) = 0.25
\end{equation}
\begin{equation}
  P(B^{C} | A^{C}) = 1 - P(B | A^{C}) = 0.9
\end{equation}
\begin{equation}
  P(B) = P(B | A) \cdot P(A) + P(B | A^{C}) \cdot P(A^{C}) = 0.75 \cdot 0.8 + 0.1 \cdot 0.2 = 0.62
\end{equation}
\begin{equation}
  P(B^{C}) = 1 - P(B) = 0.38
\end{equation}
\begin{equation}
  P(A | B) = \frac{P(A \cap B)}{P(B)} = \frac{P(B | A) \cdot P(A)}{P(B)} = \frac{0.75 \cdot 0.8}{0.62} = \frac{30}{31} \simeq 0.97
\end{equation}
\begin{equation}
  P(A^{C} | B) = 1 - P(A | B) = \frac{1}{31} \simeq 0.03
\end{equation}
\begin{equation}
  P(A | B^{C}) = \frac{P(A \cap B^{C})}{P(B^{C})} = \frac{P(B^{C} | A) \cdot P(A)}{P(B^{C})} = \frac{0.25 \cdot 0.8}{0.38} = \frac{10}{19} \simeq 0.53
\end{equation}
\begin{equation}
  P(A^{C} | B^{C}) = 1 - P(A | B^{C}) = \frac{9}{19} \simeq 0.47
\end{equation}

\subsubsection{Teil iii}

$P(B^{C} | A)$ ist die Wahrscheinlichkeit, dass ein Student die erste Klausur nicht besteht, wenn er die Aufgaben selbst bearbeitet hat.
$P(B | A^{C})$ ist die Wahrscheinlichkeit, dass ein Student die erste Klausur besteht, wenn er die Aufgaben nicht selbst bearbeitet hat.

\section{Aufgabe W2}

\subsubsection{Teil a}
\begin{equation}
  \frac{\frac{6!}{2!}}{\frac{42!}{38!}} = \frac{6 \cdot 5 \cdot 4 \cdot 3}{42 \cdot 41 \cdot 40 \cdot 39} = \frac{1}{7462}
\end{equation}

\subsubsection{Teil b}
\begin{equation}
  1 - \frac{\frac{36!}{32!}}{\frac{42!}{38!}} = 1 - \frac{561}{1066} = \frac{505}{1066} \simeq 0.47
\end{equation}

\subsubsection{Teil c}
\begin{equation}
  \frac{6}{42} \cdot \frac{36}{41} \cdot \frac{5}{40} \cdot \frac{35}{39} = \frac{15}{1066} \simeq 0.01
\end{equation}

\subsubsection{Teil d}
Nein, die Wahrscheinlichkeit aus c) ist kleiner, denn das Ereignis d ist eine echte Obermenge des Ereignisses c.
c ist in d enthalten, aber auch das Ereignis, dass der 1. und 2. Student ein Ei finden, und der 3. und 4. nicht.

\subsubsection{Teil e}
\begin{equation}
  \frac{4!}{42!} = \frac{1}{111930} \simeq 0
\end{equation}

\section{Aufgabe W3}

\subsubsection{Teil a}

\subsubsection{Teil b}

\subsubsection{Teil c}

\subsubsection{Teil d}

\subsubsection{Teil e}

\end{document}