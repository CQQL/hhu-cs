\documentclass[10pt,a4paper]{article}
\usepackage[utf8]{inputenc}
\usepackage[german]{babel}
\usepackage{mathrsfs}
\usepackage{amsmath}
\usepackage{amsfonts}
\usepackage{amssymb}
\usepackage{amsthm}
\usepackage[left=2cm,right=2cm,top=2cm,bottom=2cm]{geometry}

\begin{document}

\section{Aufgabe W1}

\subsection{Teil a}
Da Vor- und Nachname buchstabendisjunkt sind, ist die Wahrscheinlichkeit, dass ein Buchstabe im richtigen Korb landet, $\frac{1}{2}$.
Die Gesamtwahrscheinlichkeit ist dann $\frac{1}{2}^{13}$.

\subsection{Teil b}
Die Wahrscheinlichkeit den Vornamen in der richtigen Reihenfolge zu ziehen, ist
\begin{equation}
  \frac{3}{7} \cdot \frac{1}{6} \cdot \frac{2}{5} \cdot \frac{1}{4} \cdot \frac{1}{3} \cdot \frac{2}{2} \cdot \frac{1}{1} = \frac{1}{420}
\end{equation}
Beim Nachnamen ist es
\begin{equation}
  \frac{1}{6} \cdot \frac{2}{5} \cdot \frac{2}{4} \cdot \frac{1}{3} \cdot \frac{1}{2} \cdot \frac{1}{1} = \frac{1}{180}
\end{equation}
Also muss Philipp Berger bei dem Papierkorb mit seinem Vornamen die Nachhilfestunden wahrscheinlicher nicht geben.

\subsection{Teil c}

\subsubsection{Teil i}

\begin{equation}
  \frac{419}{420}^{50} \cdot \frac{179}{180}^{100} \simeq 0.51
\end{equation}

\subsubsection{Teil ii}

\begin{equation}
  50 \cdot 20 \cdot \frac{1}{420} + 100 \cdot 20 \cdot \frac{1}{180} = \frac{850}{63} \simeq 13.5
\end{equation}

\subsubsection{Teil iii}

\begin{equation}
  V =
\end{equation}

\subsection{Teil d}

\subsubsection{Teil i}

\subsubsection{Teil ii}

\subsubsection{Teil iii}

\section{Aufgabe W2}

\subsubsection{Teil a}

\subsubsection{Teil b}

\subsubsection{Teil c}

\subsubsection{Teil d}

\subsubsection{Teil e}

\section{Aufgabe W3}

\subsubsection{Teil a}

\subsubsection{Teil b}

\subsubsection{Teil c}

\subsubsection{Teil d}

\subsubsection{Teil e}

\end{document}