\documentclass[10pt,a4paper]{article}
\usepackage[utf8]{inputenc}
\usepackage[german]{babel}
\usepackage{mathrsfs}
\usepackage{amsmath}
\usepackage{amsfonts}
\usepackage{amssymb}
\usepackage{amsthm}
\usepackage[left=2cm,right=2cm,top=2cm,bottom=2cm]{geometry}

\begin{document}

\section{Aufgabe 25}

\subsection{Teil a}

\subsection{Teil b}

\subsection{Teil c}

\section{Aufgabe 26}

\subsection{Teil a}
\begin{equation}
  P(A_{k}) = \sum_{n = 1}^{\infty} P(\{kn\}) = \sum_{n = 1}^{\infty} \frac{1}{a k^{2} n^{2}} = \frac{1}{ak^{2}} \cdot \sum_{n = 1}^{\infty} \frac{1}{n^{2}} = \frac{1}{ak^{2}} \cdot \frac{\pi^{2}}{6} = \frac{1}{k^{2}}
\end{equation}

\subsection{Teil b}
Sei $J$ eine endliche Teilmenge von $\mathbb{N}$.
\begin{align*}
  P(\bigcap_{j \in J} A_{p_{j}}) & = \sum_{n = 1}^{\infty} P(\{ n \cdot \prod_{j \in J} p_{j} \})\\
  & = \sum_{n = 1}^{\infty} \frac{1}{an^{2} \cdot \prod_{j \in J} p_{j}^{2}}\\
  & = \frac{1}{a \cdot \prod_{j \in J} p_{j}^{2}} \cdot \sum_{n = 1}^{\infty} \frac{1}{n^{2}}\\
  & = \frac{1}{a \cdot \prod_{j \in J} p_{j}^{2}} \cdot \frac{\pi^{2}}{6}\\
  & = \prod_{j \in J} \frac{1}{p_{j}^{2}}\\
  & = \prod_{j \in J} P(A_{p_{j}})
\end{align*}

\subsection{Teil c}
Ja, weil alle endlichen Teilmengen stochastisch unabhängig sind, siehe Definition 7.2.

\subsection{Teil d}


\section{Aufgabe 27}

\subsection{Teil a}

\subsection{Teil b}

\section{Aufgabe 28}

\subsection{Teil a}

\subsubsection{Method i}

\subsubsection{Method ii}

\subsection{Teil b}

\subsection{Teil c}

\end{document}