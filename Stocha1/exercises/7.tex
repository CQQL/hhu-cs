\documentclass[10pt,a4paper]{article}
\usepackage[utf8]{inputenc}
\usepackage[german]{babel}
\usepackage{mathrsfs}
\usepackage{amsmath}
\usepackage{amsfonts}
\usepackage{amssymb}
\usepackage{amsthm}
\usepackage[left=2cm,right=2cm,top=2cm,bottom=2cm]{geometry}

\begin{document}

\section{Aufgabe 25}
Seien $G$ und $B$ zwei Zufallsvariablen.
Dann bildet $G$ männliche Studenten auf $M$ und weibliche auf $W$ ab, und $B$ Hauptfachstudierende auf $H$ und Nebenfächler auf $N$.

\subsection{Teil a}
Die Verteilungen sind eindeutig durch ihre Werte bestimmt.
\begin{equation}
  P_{G}(M) = P(G^{-1}(M)) = \frac{0.6 \cdot 240}{240} = \frac{3}{5}
\end{equation}
\begin{equation}
  P_{G}(W) = P(G^{-1}(W)) = 1 - P_{G}(M) = \frac{2}{5}
\end{equation}
\begin{equation}
  P_{B}(H) = P(B^{-1}(H)) = \frac{180}{240} = \frac{3}{4}
\end{equation}
\begin{equation}
  P_{B}(N) = P(B^{-1}(N)) = 1 - P_{B}(H) = \frac{1}{4}
\end{equation}
\begin{equation}
  P_{G,B}(W, N) = P(G^{-1}(W) \cap B^{-1}(N)) = \frac{30}{240} = \frac{1}{8}
\end{equation}
\begin{equation}
  P_{G,B}(M, N) = P(G^{-1}(M) \cap B^{-1}(N)) = \frac{30}{240} = \frac{1}{8}
\end{equation}
\begin{equation}
  P_{G,B}(M, H) = P(G^{-1}(M) \cap B^{-1}(H)) = \frac{114}{240} = \frac{19}{40}
\end{equation}
\begin{equation}
  P_{G,B}(W, H) = P(G^{-1}(W) \cap B^{-1}(H)) = \frac{66}{240} = \frac{11}{40}
\end{equation}

\subsection{Teil b}
Nein.
\begin{equation}
  P_{G,B}(W, H) = \frac{11}{40} \neq \frac{3}{10} = P_{G}(W) \cdot P_{B}(H)
\end{equation}

\subsection{Teil c}
Sei $P$ die gesuchte Wahrscheinlichkeit.
\begin{equation}
  P = \frac{P_{G,B}(M, N)}{P_{B}(N)} = \frac{\frac{1}{8}}{\frac{1}{4}} = \frac{1}{2}
\end{equation}

\section{Aufgabe 26}

\subsection{Teil a}
\begin{equation}
  P(A_{k}) = \sum_{n = 1}^{\infty} P(\{kn\}) = \sum_{n = 1}^{\infty} \frac{1}{a k^{2} n^{2}} = \frac{1}{ak^{2}} \cdot \sum_{n = 1}^{\infty} \frac{1}{n^{2}} = \frac{1}{ak^{2}} \cdot \frac{\pi^{2}}{6} = \frac{1}{k^{2}}
\end{equation}

\subsection{Teil b}
Sei $J$ eine endliche Teilmenge von $\mathbb{N}$.
\begin{align*}
  P(\bigcap_{j \in J} A_{p_{j}}) & = \sum_{n = 1}^{\infty} P(\{ n \cdot \prod_{j \in J} p_{j} \})\\
  & = \sum_{n = 1}^{\infty} \frac{1}{an^{2} \cdot \prod_{j \in J} p_{j}^{2}}\\
  & = \frac{1}{a \cdot \prod_{j \in J} p_{j}^{2}} \cdot \sum_{n = 1}^{\infty} \frac{1}{n^{2}}\\
  & = \frac{1}{a \cdot \prod_{j \in J} p_{j}^{2}} \cdot \frac{\pi^{2}}{6}\\
  & = \prod_{j \in J} \frac{1}{p_{j}^{2}}\\
  & = \prod_{j \in J} P(A_{p_{j}})
\end{align*}

\subsection{Teil c}
Nein.
Man betrachte $A_{4}$ und $A_{8}$.
\begin{equation}
  P(A_{4} \cap A_{8}) = P(A_{8}) = \frac{1}{64} \ge \frac{1}{1024} = \frac{1}{4^{2}} \cdot \frac{1}{8^{2}} = P(A_{4}) \cdot P(A_{8})
\end{equation}

\subsection{Teil d}
Es gilt
\begin{equation}
  P_{X_{i}}(1) = P(X_{i}^{-1}(1)) = P(A_{p_{i}}) = \frac{1}{p_{i}^{2}}
\end{equation}
und entsprechend
\begin{equation}
  P_{X_{i}}(0) = 1 - P_{X_{i}}(1) = \frac{p_{i}^{2} - 1}{p_{i}^{2}}
\end{equation}

Sei $\omega \in \{ 0, 1 \}^{n}$.
\begin{align*}
  P(\omega) & = P \left( \cap_{i = 1}^{n} X_{i}^{-1}(\omega_{i}) \right)\\
  & = \prod_{i = 1}^{n} P \left( X_{i}^{-1}(\omega_{i}) \right)\\
  & = \prod_{i = 1}^{n} P_{X_{i}}(\omega_{i})\\
  & = \frac{\prod_{i = 1, \omega_{i} = 0}^{n} (p_{i}^{2} - 1)}{\left( \prod_{i = 1}^{n} p_{i} \right)^{2}}
\end{align*}

\section{Aufgabe 27}

\subsection{Teil a}
Man kann die Wahrscheinlichkeit hier geometrisch bestimmen.
Die Länge von $[0, 1]$ ist $1$ und da die einzelnen Teilmengen eines $A_{n}$ disjunkt sind, kann man einfach die Länge dieser aufaddieren, um die Länge von $A_{n}$ zu bestimmen.
\begin{equation}
  P(A_{n}) = \frac{\textit{Länge}(A_{n})}{1} = 2^{n - 1} \cdot \frac{2i - 2i + 1}{2^{n}} = \frac{1}{2}
\end{equation}

\subsection{Teil b}
Sei o.B.d.A. $n_{1} < n_{2}$.
Dann ist die Länge von $A_{n_{1}} \cap A_{n_{2}}$ die Hälfte der Länge von $A_{n_{1}}$, weil $A_{n_{1}} \cap A_{n_{2}} \subset A_{n_{1}}$ und jedes der disjunkten Teilintervalle von $A_{n_{1}}$ beim Schneiden mit $A_{n_{2}}$ in $2^{n_{2} - n_{1}}$ Teilintervalle aufgeteilt wird, von denen jedes zweite verworfen wird.
Damit ergibt sich
\begin{align*}
  P(A_{n_{1}} \cap A_{n_{2}}) & = \frac{1}{2} \cdot P(A_{n_{1}})\\
  & = \frac{1}{2} \cdot \frac{1}{2}\\
  & = P(A_{n_{1}}) \cdot P(A_{n_{2}})
\end{align*}

\section{Aufgabe 28}

\subsection{Teil a}

\subsubsection{Method i}
Da jede Probe untersucht wird, ist die erwartete Anzahl gleich der Gesamtanzahl $n$.

\subsubsection{Method ii}
Die Wahrscheinlichkeit, dass von $k$ Fischen mindestens
Sei $X$ die Anzahl an Untersuchungen.
Sei $Y_{i}$ $1$, wenn die $i$-te Gruppe eine Infektion aufweist, sonst $0$, für alle Gruppen $1 \le i \le m$.
Dann ist
\begin{equation}
  X = m + \sum_{i = 1}^{m} k \cdot Y_{i}
\end{equation}
und
\begin{align*}
  E(X) & = E(m + \sum_{i = 1}^{m} k \cdot Y_{i})\\
  & = m + E(\sum_{i = 1}^{m} k \cdot Y_{i})\\
  & = m + k \cdot E(\sum_{i = 1}^{m} Y_{i})\\
  & = m + k \cdot \sum_{i = 1}^{m} E(Y_{i})\\
  & = m + k \cdot \sum_{i = 1}^{m} \left( 1 - (1 - p)^{k} \right)\\
  & = m + k \cdot m \cdot \left( 1 - (1 - p)^{k} \right)\\
  & = m \cdot \left( 1 + k \cdot \left( 1 - (1 - p)^{k} \right) \right)\\
  & = \frac{n}{k} \cdot \left( 1 + k \cdot \left( 1 - (1 - p)^{k} \right) \right)\\
\end{align*}

\subsection{Teil b}
Die zweite Methode ist vorzuziehen, wenn der Erwartungswert geringer ist.
\begin{align*}
  E(X) = & m \cdot \left( 1 + k \cdot \left( 1 - (1 - p)^{k} \right) \right) < n\\
  \Leftrightarrow & 1 + k \cdot \left( 1 - (1 - p)^{k} \right) < \frac{n}{m}\\
  \Leftrightarrow & 1 + k \cdot \left( 1 - (1 - p)^{k} \right) < \frac{n}{\frac{n}{k}}\\
  \Leftrightarrow & 1 + k \cdot \left( 1 - (1 - p)^{k} \right) < k\\
  \Leftrightarrow & \frac{1}{k} + \left( 1 - (1 - p)^{k} \right) < 1\\
  \Leftrightarrow & \frac{1}{k} - (1 - p)^{k} < 0\\
  \Leftrightarrow & \frac{1}{k} < (1 - p)^{k}\\
  \Leftrightarrow & 1 < k \cdot (1 - p)^{k}\\
\end{align*}

\subsection{Teil c}
\begin{equation}
  E(X) = \frac{1000}{k} \cdot \left( 1 + k \cdot \left( 1 - 0.99^{k} \right) \right)
\end{equation}
\begin{align*}
  \frac{\partial E(X)}{\partial k} & = -\frac{1000}{k^{2}} \cdot \left( 1 + k \cdot \left( 1 - 0.99^{k} \right) \right) + \frac{1000}{k} \cdot ((1 - 0.99^{k}) - k \cdot \log(0.99) \cdot 0.99^{k})\\
  & = \frac{1000}{k} \cdot \left( 1 - 0.99^{k} - k \cdot \log(0.99) \cdot 0.99^{k} - \frac{1}{k} - 1 + 0.99^{k} \right)\\
  & = \frac{1000}{k} \cdot \left( k \cdot \log(0.99) \cdot 0.99^{k} - \frac{1}{k} \right)\\
  & = 1000 \cdot \left( \log(0.99) \cdot 0.99^{k} - \frac{1}{k^{2}} \right)
\end{align*}
Diese Ableitung sollte eine Nullstelle irgendwo zwischen 10 und 11 haben, aber irgendwie kommt da nur Unsinn raus.
Mein Ergebnis sieht man, wenn man sich $E(X)$ plotten lässt.

$k = 11$ ergibt die niedrigste erwartete Anzahl an Untersuchungen.
\begin{equation}
  E(X) \simeq 195.617 \textit{ wenn $k = 10$}
\end{equation}
\begin{equation}
  E(X) \simeq 195.571 \textit{ wenn $k = 11$}
\end{equation}
\begin{equation}
  E(X) \simeq 196.949 \textit{ wenn $k = 12$}
\end{equation}

\end{document}