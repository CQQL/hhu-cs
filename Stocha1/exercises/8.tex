\documentclass[10pt,a4paper]{article}
\usepackage[utf8]{inputenc}
\usepackage[german]{babel}
\usepackage{mathrsfs}
\usepackage{amsmath}
\usepackage{amsfonts}
\usepackage{amssymb}
\usepackage{amsthm}
\usepackage[left=2cm,right=2cm,top=2cm,bottom=2cm]{geometry}

\begin{document}

\section{Aufgabe 29}

\subsection{Teil a}
\begin{align*}
  E(X) & = \int_{-\infty}^{\infty} x \lambda e^{-\lambda x} 1_{[0, \infty)}(x)\ dx\\
  & = \lambda \int_{0}^{\infty} x e^{-\lambda x}\ dx\\
  & = \lambda \left( \left[ -\frac{1}{\lambda} x e^{-\lambda x} \right]_{0}^{\infty} - \int_{0}^{\infty} -\frac{1}{\lambda} e^{-\lambda x}\ dx \right)\\
  & = -\left( -\int_{0}^{\infty} e^{-\lambda x}\ dx \right)\\
  & = \left[ -\frac{1}{\lambda} e^{-\lambda x} \right]_{0}^{\infty}\\
  & = -\frac{1}{\lambda} \left[ e^{-\lambda x} \right]_{0}^{\infty} = \frac{1}{\lambda}\\
\end{align*}
\begin{align*}
  Var(X) & = E(X^{2}) - E(X)^{2}\\
  & = \lambda \int_{0}^{\infty} x^{2} e^{-\lambda x}\ dx - \frac{1}{\lambda^{2}}\\
  & = -\left( \left[ x^{2}e^{-\lambda x} \right]_{0}^{\infty} - 2 \int_{0}^{\infty} x e^{-\lambda x} dx \right) - \frac{1}{\lambda^{2}}\\
  & = 2 \int_{0}^{\infty} x e^{-\lambda x} dx - \frac{1}{\lambda^{2}}\\
  & = 2 \left( \left[ -\frac{1}{\lambda} x e^{-\lambda x} \right]_{0}^{\infty} - \int_{0}^{\infty} -\frac{1}{\lambda} e^{-\lambda x}\ dx \right) - \frac{1}{\lambda^{2}}\\
  & = -\frac{2}{\lambda} \left( \left[ x e^{-\lambda x} \right]_{0}^{\infty} - \int_{0}^{\infty} e^{-\lambda x}\ dx \right) - \frac{1}{\lambda^{2}}\\
  & = \frac{2}{\lambda} \int_{0}^{\infty} e^{-\lambda x}\ dx - \frac{1}{\lambda^{2}}\\
  & = \frac{2}{\lambda^{2}} - \frac{1}{\lambda^{2}} = \frac{1}{\lambda^{2}}\\
\end{align*}

\subsection{Teil b}
\begin{align*}
  E(X) & = \sum_{k = 0}^{\infty} kp(1 - p)^{k}\\
  & = p \sum_{k = 0}^{\infty} k(1 - p)^{k}\\
  & = p \frac{1 - p}{p^{2}}\\
  & = \frac{1 - p}{p}\\
\end{align*}
\begin{align*}
  Var(X) & =
\end{align*}

\subsection{Teil c}
\begin{align*}
  E(X) & = \int_{0}^{\infty} x \frac{1}{\sqrt{2\pi\sigma^{2}}}\frac{1}{x}\exp \left( -\frac{(\ln(x) - \mu)^{2}}{2\sigma^{2}} \right) dx\\
  & = \frac{1}{\sqrt{2\pi\sigma^{2}}} \int_{0}^{\infty} \exp \left( -\frac{(\ln(x) - \mu)^{2}}{2\sigma^{2}} \right) dx\\
  & = \frac{1}{\sqrt{2\pi\sigma^{2}}} \int_{0}^{\infty} \exp \left( -\frac{1}{2\sigma^{2}}\ln(x)^{2} + \frac{2\mu}{2\sigma^{2}}\ln(x) - \frac{\mu^{2}}{2\sigma^{2}} \right) dx\\
  & = \frac{1}{\sqrt{2\pi\sigma^{2}}} \int_{0}^{\infty} \exp \left( -\frac{1}{2\sigma^{2}}\ln(x)^{2} \right) \exp \left( \frac{\mu}{\sigma^{2}}\ln(x) \right) \exp \left( -\frac{\mu^{2}}{2\sigma^{2}} \right) dx\\
  & = \frac{1}{\sqrt{2\pi\sigma^{2}}} \int_{0}^{\infty} x^{-\frac{1}{2\sigma^{2}}\ln(x)} x^{\frac{\mu}{\sigma^{2}}} \exp \left( -\frac{\mu^{2}}{2\sigma^{2}} \right) dx\\
  & = \frac{1}{\sqrt{2\pi\sigma^{2}}} \int_{0}^{\infty} x^{-\frac{1}{2\sigma^{2}}\ln(x) + \frac{\mu}{\sigma^{2}}} \exp \left( -\frac{\mu^{2}}{2\sigma^{2}} \right) dx\\
  & = \frac{1}{\sqrt{2\pi\sigma^{2}}} \int_{0}^{\infty} x^{\frac{1}{\sigma^{2}} \left( \mu - \frac{1}{2\sigma^{2}}\ln(x) \right)} \exp \left( -\frac{\mu^{2}}{2\sigma^{2}} \right) dx\\
\end{align*}

\section{Aufgabe 30}
Da für beide Variablen der Erwartungswert der Quadrate existiert, existiert auch das 2te absolute Moment und somit auch das erste und der einfache Erwartungswert und alles davon ist endlich.
Sei $f$ die Dichte von $X$ und $g$ die Dichte von $Y$.

\subsection{Teil a}
\begin{align*}
  Var(X) & = E((X - E(X))^{2})\\
  & = \int_{-\infty}^{\infty} (x - E(X))^{2} \cdot f(x)\ dx\\
  & = \int_{-\infty}^{\infty} (x^{2} - 2xE(X) + E(X)^{2}) \cdot f(x)\ dx\\
  & = \int_{-\infty}^{\infty} x^{2}f(x) - 2xE(X)f(x) + E(X)^{2}f(x)\ dx\\
  & = \int_{-\infty}^{\infty} x^{2}f(x)\ dx - \int_{-\infty}^{\infty} 2xE(X)f(x)\ dx + \int_{-\infty}^{\infty} E(X)^{2}f(x)\ dx\\
  & = \int_{-\infty}^{\infty} x^{2}f(x)\ dx - 2E(X) \int_{-\infty}^{\infty} xf(x)\ dx + E(X)^{2} \int_{-\infty}^{\infty} f(x)\ dx\\
  & = E(X^{2}) - 2E(X) E(X) + E(X)^{2}\\
  & = E(X^{2}) - E(X)^{2}
\end{align*}

\subsection{Teil b}
\begin{align*}
  Var(aX + b) & = E(a^{2}X^{2} + 2abX + b^{2}) - E(aX + b)^{2}\\
  & = a^{2} E(X^{2}) + 2abE(X) + b^{2} - \left( a^{2}E(X)^{2} + 2abE(X) + b^{2} \right)\\
  & = a^{2} (E(X^{2}) - E(X)^{2})\\
  & = a^{2} Var(X)
\end{align*}

\subsection{Teil c}
\begin{align*}
  Var(X + Y) & = E((X + Y)^{2}) - E(X + Y)^{2}\\
  & = E(X^{2} + 2XY + Y^{2}) - \left( E(X)^{2} + 2E(X)E(Y) + E(Y)^{2} \right)\\
  & = E(X^{2}) + 2E(XY) + E(Y^{2}) - E(X)^{2} - 2E(X)E(Y) - E(Y)^{2}\\
  & = Var(X) + Var(Y) + 2 \left( E(XY) - E(X)E(Y) \right)
\end{align*}

\section{Aufgabe 31}

\subsection{Teil a}

\subsection{Teil b}

\subsection{Teil c}
Wenn $n$ gerade ist, sei $m = \frac{n}{2}$.
Dann ist
\begin{equation}
  P(X \le m) = P(\{ 1, \dots, \frac{n}{2} \}) = \sum_{k = 1}^{\frac{n}{2}} P(\{k\}) = \frac{\frac{n}{2}}{n} = \frac{1}{2}
\end{equation}
und
\begin{equation}
  P(X \ge m) = P(\{ \frac{n}{2}, \dots, n \}) = \sum_{k = \frac{n}{2}}^{n} P(\{k\}) = \frac{1}{n} \cdot (n - \frac{n}{2} + 1) = \frac{1}{2} + \frac{1}{n} \ge \frac{1}{2}
\end{equation}

Wenn $n$ ungerade ist, sei $m = \frac{n + 1}{2}$.
Dann ist
\begin{equation}
  P(X \le m) = P(\{ 1, \dots, \frac{n + 1}{2} \}) = \sum_{k = 1}^{\frac{n + 1}{2}} P(\{k\}) = \frac{1}{n} \cdot \frac{n + 1}{2} = \frac{1}{2} + \frac{1}{2n} \ge \frac{1}{2}
\end{equation}
und
\begin{equation}
  P(x \ge m) = P(\{ \frac{n + 1}{2}, \dots, n \}) = \sum_{k = \frac{n + 1}{2}}^{n} P(\{k\}) = \frac{1}{n} \cdot \left( n - \frac{n + 1}{2} + 1 \right) = \frac{1}{n} \cdot \frac{n + 1}{2} = \frac{1}{2} + \frac{1}{2n} \ge \frac{1}{2}
\end{equation}

\section{Aufgabe 32}

\subsection{Teil a}

\subsection{Teil b}

\subsection{Teil c}

\subsection{Teil d}

\end{document}