\documentclass[10pt,a4paper]{article}
\usepackage[utf8]{inputenc}
\usepackage[german]{babel}
\usepackage{mathrsfs}
\usepackage{amsmath}
\usepackage{amsfonts}
\usepackage{amssymb}
\usepackage{amsthm}
\usepackage[left=2cm,right=2cm,top=2cm,bottom=2cm]{geometry}

\begin{document}

\section{Aufgabe 29}

\subsection{Teil a}
\begin{align*}
  E(X) & = \int_{-\infty}^{\infty} x \lambda e^{-\lambda x} 1_{[0, \infty)}(x)\ dx\\
  & = \lambda \int_{0}^{\infty} x e^{-\lambda x}\ dx\\
  & = \lambda \left( \left[ -\frac{1}{\lambda} x e^{-\lambda x} \right]_{0}^{\infty} - \int_{0}^{\infty} -\frac{1}{\lambda} e^{-\lambda x}\ dx \right)\\
  & = -\left( -\int_{0}^{\infty} e^{-\lambda x}\ dx \right)\\
  & = \left[ -\frac{1}{\lambda} e^{-\lambda x} \right]_{0}^{\infty}\\
  & = -\frac{1}{\lambda} \left[ e^{-\lambda x} \right]_{0}^{\infty} = \frac{1}{\lambda}\\
\end{align*}

\subsection{Teil b}

\subsection{Teil c}

\section{Aufgabe 30}
Da für beide Variablen der Erwartungswert der Quadrate existiert, existiert auch das 2te absolute Moment und somit auch das erste und der einfache Erwartungswert und alles davon ist endlich.
Sei $f$ die Dichte von $X$ und $g$ die Dichte von $Y$.

\subsection{Teil a}
\begin{align*}
  Var(X) & = E((X - E(X))^{2})\\
  & = \int_{-\infty}^{\infty} (x - E(X))^{2} \cdot f(x)\ dx\\
  & = \int_{-\infty}^{\infty} x^{2} \cdot f(x)\ dx - \left( \int_{-\infty}^{\infty} x \cdot f(x)\ dx \right)^{2}\\
  & = E(X^{2}) - E(X)^{2}
\end{align*}

\subsection{Teil b}

\subsection{Teil c}

\section{Aufgabe 31}

\subsection{Teil a}

\subsection{Teil b}

\subsection{Teil c}
Wenn $n$ gerade ist, sei $m = \frac{n}{2}$.
Dann ist
\begin{equation}
  P(X \le m) = P(\{ 1, \dots, \frac{n}{2} \}) = \sum_{k = 1}^{\frac{n}{2}} P(\{k\}) = \frac{\frac{n}{2}}{n} = \frac{1}{2}
\end{equation}
und
\begin{equation}
  P(X \ge m) = P(\{ \frac{n}{2}, \dots, n \}) = \sum_{k = \frac{n}{2}}^{n} P(\{k\}) = \frac{1}{n} \cdot (n - \frac{n}{2} + 1) = \frac{1}{2} + \frac{1}{n} \ge \frac{1}{2}
\end{equation}

Wenn $n$ ungerade ist, sei $m = \frac{n + 1}{2}$.
Dann ist
\begin{equation}
  P(X \le m) = P(\{ 1, \dots, \frac{n + 1}{2} \}) = \sum_{k = 1}^{\frac{n + 1}{2}} P(\{k\}) = \frac{1}{n} \cdot \frac{n + 1}{2} = \frac{1}{2} + \frac{1}{2n} \ge \frac{1}{2}
\end{equation}
und
\begin{equation}
  P(x \ge m) = P(\{ \frac{n + 1}{2}, \dots, n \}) = \sum_{k = \frac{n + 1}{2}}^{n} P(\{k\}) = \frac{1}{n} \cdot \left( n - \frac{n + 1}{2} + 1 \right) = \frac{1}{n} \cdot \frac{n + 1}{2} = \frac{1}{2} + \frac{1}{2n} \ge \frac{1}{2}
\end{equation}

\section{Aufgabe 32}

\subsection{Teil a}

\subsection{Teil b}

\subsection{Teil c}

\subsection{Teil d}

\end{document}