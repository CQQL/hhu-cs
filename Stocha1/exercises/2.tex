\documentclass[10pt,a4paper]{article}
\usepackage[utf8]{inputenc}
\usepackage[german]{babel}
\usepackage{mathrsfs}
\usepackage{amsmath}
\usepackage{amsfonts}
\usepackage{amssymb}
\usepackage{amsthm}
\usepackage[left=2cm,right=2cm,top=2cm,bottom=2cm]{geometry}

\begin{document}

\section{Aufgabe 5}

\subsection{Teil a}

\subsubsection{Teil i}
\begin{equation}
  \left( \bigcup_{k = 1}^{n} A_{k} \right)^{C}
\end{equation}

\subsubsection{Teil ii}
\begin{equation}
  \bigcup_{k = 1}^{n} \left( A_{k} \setminus \bigcup_{\substack{j = 1 \\ j \neq k}}^{n} A_{j} \right)
\end{equation}

\subsubsection{Teil iii}
\begin{equation}
  \bigcup_{k = 1}^{n} \left( \bigcap_{\substack{j = 1\\j \neq k}}^{n} A_{j} \right)
\end{equation}

\subsection{Teil b}

\subsubsection{Teil i}
\begin{equation}
  P(A) = \frac{1}{3}
\end{equation}
\begin{equation}
  P(B) = \frac{1}{3}
\end{equation}
\begin{equation}
  P(C) = \frac{1}{3}
\end{equation}

\subsubsection{Teil ii}
\begin{equation}
  P(C \setminus B) = \frac{1}{6}
\end{equation}

\subsubsection{Teil iii}
\begin{equation}
  P(A \cup B \cup C) = \frac{3}{4}
\end{equation}

\section{Aufgabe 6}

\subsection{Teil a}
Dieses Experiment lässt sich durch ein Ziehen von $n$ Kugeln ohne Zurücklegen und ohne Berücksichtigung der Reihenfolge modellieren, also als Kombinationen ohne Wiederholung.

\subsection{Teil b}
\begin{proof}
  Wenn man im $k$-ten Versuch ist, hat man bereits $k - 1$ Schlüssel ausprobiert und es verbleiben noch $n - (k - 1)$ Schlüssel.
  Dann ist $\frac{n!}{(n - (k - 1))!}$ die Anzahl der Kombinationen, wie man die ersten $k - 1$ Schlüssel ziehen konnte, und $\frac{(n - 1)!}{(n - k)!}$ die Anzahl der Kombinationen, wie man $k - 1$ mal den falschen Schlüssel ziehen konnte.
  \begin{equation}
    P(A_{k}) = \frac{\frac{n!}{(n - (n - 1))!}}{\frac{(n - 1)!}{(n - k)!}} \cdot \frac{1}{n - (k - 1)} = \frac{(n - 1)!(n - (k - 1))!}{n!(n - k)!} \cdot \frac{1}{n - (k - 1)} = \frac{1}{n}
  \end{equation}
\end{proof}

\section{Aufgabe 7}

\subsection{Teil a}
Er sollte eine schwarze Kugel in die eine Schachtel und die anderen Kugeln in die andere Schachtel tun.
Dann wird er im einen Fall gewinnen und im anderen Fall zu 50\%.

\subsection{Teil b}
\begin{equation}
  P(\textit{schwarz}) = \frac{1}{2} \cdot 1 + \frac{1}{2} \cdot \frac{99}{199} = \frac{297}{398} \simeq \frac{3}{4}
\end{equation}

\section{Aufgabe 8}

\subsection{Teil a}
Da zwei Belegungen mit denselben Stühlen, aber unterschiedlichen Studenten darauf, unterschielich sind, suchen wir alle Permutationen ohne Wiederholung von $30$ dieser $36$ Stühle.
\begin{equation}
  \frac{36!}{6!} 
\end{equation}

\subsection{Teil b}
Hier sucht man die Kombinationen ohne Zurücklegen von $4$ von $30$ Studenten.
\begin{equation}
  \binom{30}{4}
\end{equation}

\subsection{Teil c}
Hier sucht man die Kombinationen mit Zurücklegen von $4$ von $30$ Studenten.
\begin{equation}
  \binom{30 + 4 - 1}{4} = \binom{33}{4}
\end{equation}

\subsection{Teil d}
Hier sucht man die Permutationen mit Zurücklegen.
\begin{equation}
  30^{4}
\end{equation}

\end{document}