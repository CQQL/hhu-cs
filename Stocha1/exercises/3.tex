\documentclass[10pt,a4paper]{article}
\usepackage[utf8]{inputenc}
\usepackage[german]{babel}
\usepackage{mathrsfs}
\usepackage{amsmath}
\usepackage{amsfonts}
\usepackage{amssymb}
\usepackage{amsthm}
\usepackage[left=2cm,right=2cm,top=2cm,bottom=2cm]{geometry}

\begin{document}

Stochastik 1, Blatt 3\\
Gruppe 1\\
Marten Lienen (2126759)\\
Fabian Schmittmann (2083559)

\section{Aufgabe 9}
Es gibt die vier verschiedenen Kombinationen $(0, 0), (0, 1), (1, 0), (1, 1)$ von zwei aufeinanderfolgenden Bits.
Die Wahrscheinlichkeit, dass zwei aufeinanderfolgende Bits gleich sind, ist also $0,5$, ebenso wie die Gegenwahrscheinlichkeit.
Da unterschiedliche, aufeinanderfolgende Bits einen neuen Run starten, besitzt dies dieselbe Wahrscheinlichkeit.
Und weil das Verhältnis aufeinanderfolgender Bits nicht von den restlichen Bits abhängt, ist die Wahrscheinlichkeit binomialverteilt.
Genau $l$ Runs treten also mit der Wahrscheinlichkeit
\begin{equation}
  p(l) = B_{n - 1, 0,5}(l - 1)
\end{equation}
auf.
Das ist die Wahrscheinlichkeit in den $n - 1$ aufeinanderfolgenden Paaren genau $l - 1$ neue Runs zu starten.

\section{Aufgabe 10}

\section{Aufgabe 11}

\subsection{Teil a}

\subsection{Teil b}

\section{Aufgabe 12}

\subsection{Teil a}

\subsubsection{Teil i}

\begin{equation}
  P = \frac{4}{10} \times \frac{4}{10} \times \frac{6}{10} \times \frac{6}{10} = \frac{576}{10000}
\end{equation}

\subsubsection{Teil ii}

\begin{equation}
  P = 1 - 
\end{equation}

\subsubsection{Teil iii}

\subsection{Teil b}

\end{document}