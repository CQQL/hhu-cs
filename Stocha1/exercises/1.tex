\documentclass[10pt,a4paper]{article}
\usepackage[utf8]{inputenc}
\usepackage[german]{babel}
\usepackage{mathrsfs}
\usepackage{amsmath}
\usepackage{amsfonts}
\usepackage{amssymb}
\usepackage{amsthm}
\usepackage[left=2cm,right=2cm,top=2cm,bottom=2cm]{geometry}

\begin{document}

\section{Aufgabe 1}
In Fall 1 ist das Komplementärereignis, dass man gar keine 6 würfelt, also:
\begin{equation}
  P(A) = 1 - P(A^{C}) = 1 - \frac{5}{6}^{4} \simeq 0.52 
\end{equation}
In Fall 2 ist das Komplementärereignis, dass man gar keinen Pasch würfelt, also:
\begin{equation}
  P(A) = 1 - P(A^{C}) = 1 - \frac{35}{36}^{24} \simeq 0.49
\end{equation}
Nein, es ist nicht so.
Die Wahrscheinlichkeit ist in Fall 2 niedriger.

\section{Aufgabe 2}

\subsection{Teil a}
\begin{proof}
  \begin{align*}
    \sum_{k = 0}^{n} p(k) & = \sum_{k = 0}^{n} \binom{n}{k} p^{k} (1 - p)^{n - k}\\
    & = (p + 1 - p)^{n} = 1^{n} = 1
  \end{align*}
\end{proof}

\subsection{Teil b}
\begin{proof}
  \begin{align*}
    \sum_{k = 0}^{\infty} p(k) & = \sum_{k = 0}^{\infty} \frac{\lambda^{k}}{k!} e^{-\lambda}\\
    & = e^{-\lambda} \sum_{k = 0}^{\infty} \frac{\lambda^{k}}{k!}\\
    & = e^{-\lambda} e^{\lambda} = e^{\lambda - \lambda} = e^{0} = 1
  \end{align*}
\end{proof}

\subsection{Teil c}
\begin{proof}
  \begin{align*}
    \sum_{k = 1}^{\infty} p(k) & = \sum_{k = 1}^{\infty} -\frac{p^{k}}{k \cdot \log(1 - p)} \\
    & = 1
  \end{align*}
\end{proof}

\section{Aufgabe 3}
\begin{align*}
  P(A_{1} \cup A_{2} \cup A_{3}) & = P(A_{1}) + P(A_{2} \setminus A_{1}) + P(A_{3} \setminus (A_{1} \cup A_{2}))\\
  & = P(A_{1}) + P(A_{2}) - P(A_{1} \cap A_{2}) + P(A_{3}) - P(A_{3} \cap (A_{1} \cup A_{2}))\\\
  & = P(A_{1}) + P(A_{2}) - P(A_{1} \cap A_{2}) + P(A_{3}) - P((A_{3} \cap A_{1}) \cup (A_{3} \cap A_{2}))\\\
  & = P(A_{1}) + P(A_{2}) - P(A_{1} \cap A_{2}) + P(A_{3}) - (P(A_{1} \cap A_{3}) + P(A_{2} \cap A_{3}) - P(A_{1} \cap A_{3} \cap A_{2} \cap A_{3}))\\\
  & = P(A_{1}) + P(A_{2}) - P(A_{1} \cap A_{2}) + P(A_{3}) - P(A_{1} \cap A_{3}) - P(A_{2} \cap A_{3}) + P(A_{1} \cap A_{2} \cap A_{3})\\\
  & = P(A_{1}) + P(A_{2}) + P(A_{3}) - P(A_{1} \cap A_{2}) - P(A_{1} \cap A_{3}) - P(A_{2} \cap A_{3}) + P(A_{1} \cap A_{2} \cap A_{3})
\end{align*}

\section{Aufgabe 4}
\begin{proof}
  
\end{proof}

\end{document}