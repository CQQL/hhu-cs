\documentclass[10pt,a4paper]{article}
\usepackage[utf8]{inputenc}
\usepackage[german]{babel}
\usepackage{mathrsfs}
\usepackage{amsmath}
\usepackage{amsfonts}
\usepackage{amssymb}
\usepackage{amsthm}
\usepackage[left=2cm,right=2cm,top=2cm,bottom=2cm]{geometry}

\begin{document}

\section{Aufgabe 1}
In Fall 1 ist das Komplementärereignis, dass man gar keine 6 würfelt, also:
\begin{equation}
  P(A) = 1 - P(A^{C}) = 1 - \frac{5}{6}^{4} \simeq 0.52 
\end{equation}
In Fall 2 ist das Komplementärereignis, dass man gar keinen Pasch würfelt, also:
\begin{equation}
  P(A) = 1 - P(A^{C}) = 1 - \frac{35}{36}^{24} \simeq 0.49
\end{equation}
Nein, es ist nicht so.
Die Wahrscheinlichkeit ist in Fall 2 niedriger.

\section{Aufgabe 2}

\subsection{Teil a}

\subsection{Teil b}

\subsection{Teil c}

\section{Aufgabe 3}

\section{Aufgabe 4}

\end{document}