\documentclass[10pt,a4paper]{article}
\usepackage[utf8]{inputenc}
\usepackage[german]{babel}
\usepackage{mathrsfs}
\usepackage{amsmath}
\usepackage{amsfonts}
\usepackage{amssymb}
\usepackage{amsthm}
\usepackage[left=2cm,right=2cm,top=2cm,bottom=2cm]{geometry}

\begin{document}

Stochastik 1, Blatt 10\\
Gruppe 1\\
Marten Lienen (2126759)\\
Fabian Schmittmann (2083559)

\section{Aufgabe 37}

Sei $X$ eine Zufallsvariable für die Anzahl der Kranken $k$.
Dann ist $X$ $B_{n,p}$ verteilt.
Die Likelihood-Funktion ist
\begin{equation}
  L(n; X) = B_{n, p}(k) = \binom{n}{k} \cdot p^{k} \cdot (1 - p)^{n - k}
\end{equation}
Der Übergang zur log-Likelihood liefert
\begin{equation}
  \log L(n; X) = \log(n!) - \log(k!) - \log((n - k)!) + k \cdot \log(p) + (n - k) \cdot \log(1 - p)
\end{equation}
Und die erste Ableitung nach $n$ ist
\begin{equation}
  \frac{\partial}{\partial n} \log L(n; X) = \frac{(n!)'}{n!} - \frac{((n - k)!)'}{(n - k)!} + \log(1 - p)
\end{equation}

\section{Aufgabe 38}

\subsection{Teil a}

\subsection{Teil b}

\subsection{Teil c}

\subsection{Teil d}

\section{Aufgabe 39}

\subsection{Teil a}

\begin{align*}
  L(\sigma; x_{1}, \dots, x_{n}) & = \prod_{i = 1}^{n} \frac{1}{\sqrt{2 \pi} \cdot \sigma} \cdot e^{-\frac{x_{i}^{2}}{2\sigma^{2}}}\\
\end{align*}
Log-Likelihood
\begin{align*}
  \log L(\sigma; x_{1}, \dots, x_{n}) & = n \cdot \log(\frac{1}{\sqrt{2 \pi} \cdot \sigma}) + \sum_{i = 1}^{n} \log\left( e^{-\frac{x_{i}^{2}}{2\sigma^{2}}} \right)\\
  & = n \cdot \log(\frac{1}{\sqrt{2 \pi} \cdot \sigma}) - \sum_{i = i}^{n} \frac{x_{i}^{2}}{2\sigma^{2}}\\
  & = n \cdot \log(\frac{1}{\sqrt{2 \pi} \cdot \sigma}) - \frac{1}{2\sigma^{2}} \cdot \sum_{i = i}^{n} x_{i}^{2}
\end{align*}
\begin{align*}
  \frac{\partial}{\partial \sigma} \log(L(\sigma; x_{1}, \dots, x_{n})) & = n \cdot \frac{\frac{1}{\sqrt{2 \pi}}}{\frac{1}{\sqrt{2 \pi} \cdot \sigma}} + \frac{1}{\sigma^{3}} \cdot \sum_{i = i}^{n} x_{i}^{2}\\
  & = n \cdot \sigma + \frac{1}{\sigma^{3}} \cdot \sum_{i = i}^{n} x_{i}^{2}
\end{align*}
\begin{align*}
  n \cdot \sigma + \frac{1}{\sigma^{3}} \cdot \sum_{i = i}^{n} x_{i}^{2} & = 0\\
  \Leftrightarrow n \cdot \sigma & = -\frac{1}{\sigma^{3}} \cdot \sum_{i = i}^{n} x_{i}^{2}\\
  \Leftrightarrow \sigma^{4} & = -\frac{1}{n} \cdot \sum_{i = i}^{n} x_{i}^{2}\\
  \Leftrightarrow \sigma^{4} & = -\frac{1}{n} \cdot \sum_{i = i}^{n} x_{i}^{2}\\
  \Leftrightarrow \sigma & = 0
\end{align*}

\subsection{Teil b}

\subsection{Teil c}

\section{Aufgabe 40}

\subsection{Teil a}

\subsection{Teil b}

\subsection{Teil c}

\end{document}