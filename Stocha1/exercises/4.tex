\documentclass[10pt,a4paper]{article}
\usepackage[utf8]{inputenc}
\usepackage[german]{babel}
\usepackage{mathrsfs}
\usepackage{amsmath}
\usepackage{amsfonts}
\usepackage{amssymb}
\usepackage{amsthm}
\usepackage[left=2cm,right=2cm,top=2cm,bottom=2cm]{geometry}

\begin{document}

\section{Aufgabe 13}

\subsection{Teil a}
Es gibt also $\binom{n - 2k + 2}{k}$ $k$-elementige Teilmengen dieser Art.

\subsection{Teil b}

\section{Aufgabe 14}

\subsection{Teil a}
$X$ beschreibt die Anzahl der Versuche, die derjenige brauchte, der zuerst eine Goldforelle gefangen hat.

\subsection{Teil b}
\begin{align*}
  P(X = d) & = P((X_{1} = d \land X_{2} > d) \lor (X_{1} > d \land X_{2} = d) \lor (X_{1} = 1 \land X_{2} = d))\\
  & = P(X_{1} = d \land X_{2} > d) + P(X_{1} > d \land X_{2} = d) + P(X_{1} = 1 \land X_{2} = d)\\
  & = G_{\frac{k}{n}}(d - 1) \cdot \left( \sum_{i = d}^{m - 1} G_{\frac{l}{m}}(i) \right) + \left( \sum_{i = d}^{n - 1} G_{\frac{k}{n}}(i) \right) \cdot G_{\frac{l}{m}}(d - 1) + G_{\frac{k}{n}}(d - 1) \cdot G_{\frac{l}{m}}(d - 1)\\
  & = \frac{n - k}{n}^{d - 1} \cdot \frac{k}{n} \cdot \left( \sum_{i = d}^{m - 1} \frac{m - l}{m}^{i} \cdot \frac{l}{m} \right) + \left( \sum_{i = d}^{n - 1} \frac{n - k}{n}^{i} \cdot \frac{k}{n} \right) \cdot \frac{m - l}{m}^{d - 1} \cdot \frac{l}{m} + \frac{n - k}{n}^{d - 1} \cdot \frac{k}{n} \cdot \frac{m - l}{m}^{d - 1} \cdot \frac{l}{m}\\
  & = \frac{k}{n} \cdot \frac{l}{m} \cdot \left( \frac{n - k}{n}^{d - 1} \cdot \left( \sum_{i = d}^{m - 1} \frac{m - l}{m}^{i} \right) + \left( \sum_{i = d}^{n - 1} \frac{n - k}{n}^{i} \right) \cdot \frac{m - l}{m}^{d - 1} + \frac{n - k}{n}^{d - 1} \cdot \frac{m - l}{m}^{d - 1} \right)\\
  & = \frac{k}{n} \cdot \frac{l}{m} \cdot \left( \frac{n - k}{n}^{d - 1} \cdot \frac{m}{l} \cdot \frac{m - l}{m}^{d} \cdot \left(1 - \frac{m - l}{m}^{m - d} \right) + \frac{n}{k} \cdot \frac{n - k}{n}^{d} \cdot \left( 1 - \frac{n - k}{n}^{n - d} \right) \cdot \frac{m - l}{m}^{d - 1} + \frac{n - k}{n}^{d - 1} \cdot \frac{m - l}{m}^{d - 1} \right)\\
  & = \frac{k}{n} \cdot \frac{l}{m} \cdot \frac{n - k}{n}^{d - 1} \cdot \left( \frac{m}{l} \cdot \frac{m - l}{m}^{d} \cdot \left(1 - \frac{m - l}{m}^{m - d} \right) + \frac{n}{k} \cdot \frac{n - k}{n} \cdot \left( 1 - \frac{n - k}{n}^{n - d} \right) \cdot \frac{m - l}{m}^{d - 1} + \frac{m - l}{m}^{d - 1} \right)\\
  & = \frac{k}{n} \cdot \frac{l}{m} \cdot \frac{n - k}{n}^{d - 1} \cdot \frac{m - l}{m}^{d - 1} \cdot \left( \frac{m}{l} \cdot \frac{m - l}{m} \cdot \left(1 - \frac{m - l}{m}^{m - d} \right) + \frac{n}{k} \cdot \frac{n - k}{n} \cdot \left( 1 - \frac{n - k}{n}^{n - d} \right) + 1 \right)\\
  & = \frac{k}{n} \cdot \frac{l}{m} \cdot \frac{n - k}{n}^{d - 1} \cdot \frac{m - l}{m}^{d - 1} \cdot \left( \frac{m - l}{l} \cdot \left(1 - \frac{m - l}{m}^{m - d} \right) + \frac{n - k}{k} \cdot \left( 1 - \frac{n - k}{n}^{n - d} \right) + 1 \right)\\
\end{align*}

\section{Aufgabe 15}

\subsection{Teil a}
Da $0$ nicht-negativ ist und $3$ die Basis der Potenzen ist, sind alle Werte von $f$ nicht-negativ.
\begin{align*}
  \sum_{j = 1}^{\infty} \left( \sum_{k = 1}^{j - 1} 3^{-j -k} + 3^{-2j + 1} \right) & = \sum_{j = 1}^{\infty} \left( 3^{-j - 1} \left( \sum_{k = 0}^{j - 2} 3^{-k} \right) + 3^{-2j + 1} \right)\\
  & = \sum_{j = 1}^{\infty} \left( 3^{-j - 1} \frac{3}{2} \left( 1 - \frac{1}{3}^{j - 1} \right) + 3^{-2j + 1} \right)\\
  & = \sum_{j = 1}^{\infty} \left( 3^{-j} \frac{1}{2} \left( 1 - \frac{1}{3}^{j - 1} \right) + 3^{-2j + 1} \right)\\
  & = \sum_{j = 1}^{\infty} \left( \frac{1}{2} \left( 3^{-j} - \frac{3^{-j}}{3^{j - 1}} \right) + 3^{-2j + 1} \right)\\
  & = \sum_{j = 1}^{\infty} \left( \frac{1}{2} \left( 3^{-j} - \frac{1}{3^{2j - 1}} \right) + 3^{-2j + 1} \right)\\
  & = \sum_{j = 1}^{\infty} \left( \frac{1}{2} \left( 3^{-j} - \frac{3}{3^{2^{j}}} \right) + 3^{-2j + 1} \right)\\
  & = \sum_{j = 1}^{\infty} \left( \frac{1}{2} \left( 3^{-j} - 3 \cdot 9^{-j} \right) + 3^{-2j + 1} \right)\\
  & = \sum_{j = 1}^{\infty} \left( \frac{1}{2} 9^{-j} \left( 3^{j} - 3 \right) + 3^{-2j + 1} \right)\\
  & = \sum_{j = 1}^{\infty} \left( \frac{1}{2} 9^{-j} \left( 3^{j} - 3 \right) + 9^{-j} \cdot 3 \right)\\
  & = \sum_{j = 1}^{\infty} 9^{-j} \left( \frac{1}{2} \left( 3^{j} - 3 \right) + 3 \right)\\
  & = \frac{1}{2} \sum_{j = 1}^{\infty} 9^{-j} \left( 3^{j} + 3 \right)\\
  & = \frac{1}{2} \sum_{j = 1}^{\infty} \left( 3^{-j} + 3 \cdot 9^{-j} \right)\\
  & = \frac{1}{2} \sum_{j = 1}^{\infty} 3^{-j} + \frac{3}{2} \sum_{j = 1}^{\infty} 9^{-j}\\
  & = \frac{1}{2} \sum_{j = 1}^{\infty} \frac{1}{3}^{j} + \frac{3}{2} \sum_{j = 1}^{\infty} \frac{1}{9}^{j}\\
  & = \frac{1}{2} \left( \sum_{j = 0}^{\infty} \frac{1}{3}^{j} - 1 \right) + \frac{3}{2} \left( \sum_{j = 0}^{\infty} \frac{1}{9}^{j} - 1\right)\\
  & = \frac{1}{2} \left( \frac{3}{2} - 1 \right) + \frac{3}{2} \left( \frac{9}{8} - 1\right) = \frac{1}{2} \cdot \frac{1}{2} + \frac{3}{2} \cdot \frac{1}{8} = \frac{1}{4} + \frac{3}{16} = \frac{7}{16}
\end{align*}
Mit $c = \frac{16}{7}$ ist diese Summe $1$ und $q(j, k)$ eine Zähldichte.

\subsection{Teil b}

\section{Aufgabe 16}

\subsection{Teil a}
Hier haben wir ein Urnenmodell mit 4 Sorten und 19 Kugeln, die wir ohne Zurücklegen ziehen.
Dabei gibt es $K_{1} = 6$, $K_{2} = 5$, $K_{3} = 4$ und $K_{4} = 4$ Kugeln, die jeweils einem Ticket für das $i$-te Abteil entsprechen.
Wenn die an $k$-ter Stelle gezogene Kugel der $k$-ten Person ein Abteil zuweist, suchen wir also die Anzahl der Permutationen der Länge 19 dieser Kugeln.


\subsection{Teil b}

\end{document}