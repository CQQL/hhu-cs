\documentclass[a4paper,10pt]{article}
\usepackage[utf8]{inputenc}
\usepackage[german]{babel}
\usepackage{amsmath}
\usepackage{amssymb}
\usepackage{amsthm}
\usepackage{stmaryrd}

\title{Ana1, Übungsblatt 11}
\author{Marten Lienen (2126759), Gruppe 8; Fabian Schmittmann (2083559), Gruppe 0}

\begin{document}

\maketitle

\section*{Aufgabe 47}

\subsection*{Teil a}

Sei $x \ne 0$.
\begin{equation}
 f'(x) = \frac{2}{x^3} e^{-\frac{1}{x^2}}
\end{equation}
\begin{equation}
 f''(x) = \frac{2}{x^3} \frac{2}{x^3} e^{-\frac{1}{x^2}} + \frac{-6}{x^4} e^{-\frac{1}{x^2}} = (\frac{4}{x^6} - \frac{6}{x^4}) e^{-\frac{1}{x^2}}
\end{equation}

\subsection*{Teil b}

\begin{proof}
 Wir zeigen es durch Induktion über $n$.
 Für $n = 0$ ist $p_0 = 1$.
 Für $n > 0$ nehmen wir an, dass wir bereits ein Polynom $p_{n - 1}$ vom Grad $3n - 3$ haben und dass $f^{(n - 1)}$ die geforderte Form hat.
 \begin{align*}
  f^{(n)} & = p_{n - 1}(\frac{1}{x}) f'(x) + p_{n - 1}'(\frac{1}{x}) f(x)\\
  & = \frac{2}{x^3} p_{n - 1}(\frac{1}{x}) f(x) + p_{n - 1}'(\frac{1}{x}) f(x)\\
  & = \left( \frac{2}{x^3} p_{n - 1}(\frac{1}{x}) + p_{n - 1}'(\frac{1}{x}) \right) f(x)
 \end{align*}
 Wir haben also ein Polynom $p_n = \frac{2}{x^3} p_{n - 1}(\frac{1}{x}) + p_{n - 1}'(\frac{1}{x})$ gefunden.
 Wegen $\frac{2}{x^3} p_{n - 1}$ und dass $p_{n - 1}$ Grad $3n - 3$ hat, hat $p_n$ Grad $3n$.
\end{proof}

\subsection*{Teil c}

\begin{proof}
 Wir zeigen $f^{(n)}(0) = 0$ durch Induktion über $n$.
 Für $n = 0$ ist es per Definition wahr.
 Für $n > 0$ nehmen wir an, dass $f^{(n - 1)}(0) = 0$.
 Sei $x_n$ eine Nullfolge mit $x_k \ne 0$ und $y_n = -x_n$.
 \begin{align*}
  \lim_{k \rightarrow \infty} \frac{f^{(n - 1)}(x_k) - f^{(n - 1)}(0)}{x_k - 0} & = \lim_{k \rightarrow \infty} \frac{f^{(n - 1)}(x_k)}{x_k}
 \end{align*}

\end{proof}

\subsection*{Teil d}

$f'$ hat genau eine Nullstelle, $0$, wie in $c$ gezeigt, weil weder $\frac{2}{x^3}$ noch $e^{-\frac{1}{x^2}}$ eine Nullstelle haben.
Da $e^k > 0 \quad k \in \mathbb{R}$ und $\frac{2}{x^3} < 0 \quad \forall x \in \mathbb{R}_{< 0}$ und $\frac{2}{x^3} > 0 \quad \forall x \in \mathbb{R}_{> 0}$, ist $f'(\mathbb{R}_{< 0}) < 0$ und $f'(\mathbb{R}_{> 0}) > 0$.
Also hat $f$ an der Stelle $0$ ein Minimum.

\subsection*{Teil e}

Sei $x \ne 0$.
\begin{align*}
 & f''(x) = (\frac{4}{x^6} - \frac{6}{x^4}) e^{-\frac{1}{x^2}} \ge 0\\
 \Leftrightarrow & (\frac{4}{x^6} - \frac{6}{x^4}) \ge 0\\
 \Leftrightarrow & -\frac{6}{x^6}(x^2 - \frac{2}{3}) \ge 0\\
 \Leftrightarrow & x^2 - \frac{2}{3} \le 0\\
 \Leftrightarrow & x^2 \le \frac{2}{3}\\
 \Leftrightarrow & |x| \le \sqrt{\frac{2}{3}}\\
\end{align*}
Wegen $f''(0) = 0$, liegt auch $0$ in der Lösungsmenge und $f$ ist nach Satz 8.7 konvex in $[-\sqrt{\frac{2}{3}}, \sqrt{\frac{2}{3}}]$.

\section*{Aufgabe 48}

\subsection*{Teil a}

Weil $f$ eine Treppenfunktion ist, deren Stufen alle Höhe $0$ haben, ist sie integrierbar und
\begin{equation}
 \int_0^1 f(x) dx = 0
\end{equation}

\subsection*{Teil b}

Weil $f$ eine Treppenfunktion ist, deren Stufen alle Höhe $0$ haben, ist sie integrierbar und
\begin{equation}
 \int_0^1 f(x) dx = 0
\end{equation}

\subsection*{Teil c}

Weil $f$ auch eine Treppenfunktion ist, ist sie auch integrierbar.
\begin{align*}
 \int_0^1 f(x) dx & = \sum_{k = 0}^\infty \frac{1}{2^k} \cdot \left( \frac{1}{2^k} - \frac{1}{2^{k + 1}} \right)\\
 & = \sum_{k = 0}^\infty \frac{1}{2^k} \cdot \frac{2 - 1}{2^{k + 1}}\\
 & = \sum_{k = 0}^\infty \frac{1}{2^{2k + 1}}\\
 & = \frac{1}{2} \cdot \sum_{k = 0}^\infty \frac{1}{2^{2k}}\\
 & = \frac{1}{2} \cdot \sum_{k = 0}^\infty \frac{1}{4^k}\\
 & = \frac{2}{3}
\end{align*}

\subsection*{Teil d}

$f$ ist eine Treppenfunktion mit schrägen Stufen.
\begin{align*}
 \int_0^1 f(x) dx & = \sum_{k = 0}^\infty \frac{1}{2} \cdot \left( \frac{1}{2^k} - \frac{1}{2^{k + 1}} \right)\\
 & = \sum_{k = 0}^\infty \frac{1}{2^{k + 2}}\\
 & = \frac{1}{4} \sum_{k = 0}^\infty \frac{1}{2^k}\\
 & = \frac{1}{2}
\end{align*}

\section*{Aufgabe 49}

\begin{equation}
 h(x) = f(x) - g(x)
\end{equation}
\begin{equation}
 h'(x) = e^x - 2x + 1
\end{equation}
\begin{equation}
 h''(x) = e^x - 2
\end{equation}
\begin{equation}
 h'''(x) = e^x
\end{equation}
Wir suchen nun das globale Minimum von $h'$.
\begin{equation}
 h''(x) = 0 \Leftrightarrow x = \log 2
\end{equation}
\begin{equation}
 h'''(\mathbb{R}) > 0
\end{equation}
$h'$ hat also ein lokales Minimum bei $\log 2$, das auch ein globales ist, weil $h''$ immer positiv ist.
\begin{equation}
 h'(\log 2) = 2 - 2\log 2 + 1 = 3 - 2 \log 2 > 0
\end{equation}
Da $h'(\mathbb{R}) > 0$, ist $h$ streng monoton steigend.
\begin{equation}
 h(0) = -1
\end{equation}
\begin{equation}
 h(3) = e^3 - 2 + 3^2 - 3 \approx 2
\end{equation}
Da $h$ streng monoton wachsend ist und $h(0) < 0$ und $h(3) > 0$ gibt es genau eine Stelle $\xi$ mit $h(\xi) = 0$, also $f(\xi) = g(\xi)$.

\section*{Aufgabe 50}

\subsection*{Teil a}

Sei $f'$ beschränkt durch $M$.
Wähle $\delta = \frac{\varepsilon}{M}$.
Seien $x, x' \in D$ mit $|x - x'| < \delta$.
\begin{equation}
 |f(x) - f(x')| \le M \cdot |x - x'| < M \cdot \frac{\varepsilon}{M} = \varepsilon
\end{equation}

\subsection*{Teil b}

Wähle $\delta = $.
Seien $x, x' \in D$ mit $|x - x'| < \delta$.
\begin{equation}
 |f(x) - f(x')|  = \varepsilon
\end{equation}

\subsection*{Teil c}

\end{document}
