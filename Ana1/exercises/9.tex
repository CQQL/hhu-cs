\documentclass[a4paper,10pt]{article}
\usepackage[utf8]{inputenc}
\usepackage[german]{babel}
\usepackage{amsmath}
\usepackage{amssymb}
\usepackage{amsthm}
\usepackage{stmaryrd}

\title{Ana1, Übungsblatt 9}
\author{Marten Lienen (2126759), Gruppe 8; Fabian Schmittmann (2083559), Gruppe 0}

\begin{document}

\maketitle

\section*{Übung 37}

\subsection*{a}

Mit den Regeln 2.a, der Kettenregel und verschiedenen Beispielen aus der Vorlesung ergibt sich
\begin{align*}
 f'(x) & = \frac{2x}{(x^2 + 1)^2} + \frac{2x}{(x^2 + 1)} + \frac{2x}{(x^2 + 1)^2}\frac{1}{\frac{1}{(x^2 + 1)^2} + 1}\\
 & = \frac{2x}{(x^2 + 1)^2} + \frac{2x}{(x^2 + 1)} + \frac{2x}{(x^2 + 1)^2 + 1}
\end{align*}

\subsection*{b}

Mit den Regeln 2.a, 2.b und dem Beispiel aus der Vorlesung zu $\log'$ ergibt sich
\begin{equation}
 f'(x) = \log x
\end{equation}

\subsection*{c}

\begin{equation}
 f(x) = a^x = \exp(x \cdot \log a)
\end{equation}

Mit der Regel 2.b, der Kettenregel und dem Beispiel aus der Vorlesung zu $\log'$ ergibt sich
\begin{equation}
 f'(x) = \log a \cdot \exp(x \cdot \log a) = \log a \cdot a^x
\end{equation}

\section*{Übung 38}

\begin{equation}
 f'(x) = 2\cos (2x) - 2\sin x
\end{equation}

\begin{align*}
 f'(x) = 0 \Leftrightarrow & 2\cos (2x) = 2\sin x\\
 \Leftrightarrow & \cos (2x) = \sin x\\
 \Leftrightarrow & \cos(2x) - \sin x = 0\\
 \Leftrightarrow & \cos^2 x - \sin^2 x - \sin x = 0\\
 \Leftrightarrow & 1 - \sin^2 x - \sin^2 x - \sin x = 0\\
 \Leftrightarrow & \sin^2 x + \frac{1}{2}\sin x - \frac{1}{2} = 0\\
 \Leftrightarrow & \sin x = -\frac{1}{4} \pm \sqrt{\frac{1}{16} + \frac{1}{2}} = -\frac{1}{4} \pm \frac{3}{4}\\
 \Leftrightarrow & \sin x = \frac{1}{2} \lor \sin x = -1\\
 \Leftrightarrow & x = \arcsin \frac{1}{2} = \lor x = \arcsin -1
\end{align*}

$f'$ hat in $[0, 2\pi]$ 3 Nullstellen: $\frac{3}{2}\pi$, $\frac{1}{6}\pi$ und $\frac{5}{6}\pi$.

\begin{equation}
 f''(x) = -4 \sin (2x) - 2 \cos x
\end{equation}
\begin{align*}
 f''(\frac{1}{6}\pi) < 0\\
 f''(\frac{5}{6}\pi) > 0\\
 f''(\frac{3}{2}\pi) = 0
\end{align*}

Nach dem Satz aus der Vorlesung hat $f$ an $\frac{1}{6}\pi$ ein lokales Maximum, an $\frac{5}{6}\pi$ ein lokales Minimum und bei $\frac{3}{2}\pi$ kann der Satz nicht angewendet werden.
Die nächste kritische Stelle der Ableitung ist $\frac{13}{6}\pi$, die wegen der Periodenlänge von $2\pi$ ebenfalls ein Maximum von $f$ ist.
Nach der Definition ist $f'$ rechts von $\frac{1}{6}\pi$ positiv und links von $\frac{13}{6}\pi$ negativ.
Wenn $\frac{3}{2}\pi$ ein Maximum wäre, so wäre $f'$ rechts von $\frac{3}{2}\pi$ negativ und es müsste zwischen $\frac{3}{2}\pi$ und $\frac{13}{6}\pi$ eine weiter kritische Stelle in $f'$ geben, gibt es aber nicht.
Wenn $\frac{3}{2}\pi$ ein Minimum wäre, so wäre $f'$ links von $\frac{3}{2}\pi$ negativ und es müsste zwischen $\frac{3}{2}\pi$ und $\frac{1}{6}\pi$ eine weiter kritische Stelle in $f'$ geben, gibt es aber nicht.

\section*{Übung 39}

\begin{align*}
 \lim_{x \searrow 0} x^x & = \lim_{x \searrow 0} \exp(x \cdot \log x)\\
 & = \exp\left(\lim_{x \searrow 0} (x \cdot \log x)\right) \text{substituiere $x = \frac{1}{t}$}\\
 & = \exp\left(\lim_{t \nearrow \infty} (\frac{1}{t} \cdot \log \frac{1}{t})\right)\\
 & = \exp\left(\lim_{t \nearrow \infty} (-\frac{\log t}{t})\right) = exp(0) = 1\\
\end{align*}

\begin{align*}
 f'(x) = \left(exp(x \log x)\right)' = (1 + \log x) \cdot exp(x \log x) = (1 + \log x) \cdot x^x
\end{align*}

\begin{align*}
 \lim_{x \searrow 0} f'(x) & = \lim_{x \searrow 0} \left( (1 + \log x) \cdot x^x \right)\\
 & = \lim_{x \searrow 0} (1 + \log x) \cdot \lim_{x \searrow 0} x^x\\
 & = \lim_{x \searrow 0} (1 + \log x) \cdot 1\\
 & = \lim_{x \searrow 0} (1 + \log x) = -\infty
\end{align*}

\section*{Übung 40}

\subsection*{a}

\begin{equation}
 \lim_{x \searrow 0} \frac{x}{x} = 1 \ne -1 = \lim_{x \nearrow 0} \frac{-x}{x}
\end{equation}

\subsection*{b}

$\sqrt[3]{x}$ ist definiert auf $\mathbb{R}_{\ge 0}$.
Da $0$ aber der untere Rand von $\mathbb{R}_{\ge 0}$ ist, ist $\mathbb{R}_{\ge 0}$ nicht offen und deshalb ist $\sqrt[3]{x}$ in $0$ nicht differenzierbar.

\section*{Übung 41}

Nach den Bedingungen kann man $g$ ganz allgemein definieren als
\begin{equation}
 g(x) = 
  \begin{cases}
   0 & \text{wenn $x = 0$}\\
   \frac{x}{f(x)} & \text{andernfalls}
  \end{cases}
\end{equation}
Wenn $f$ nicht differenzierbar wäre, könnte die Bedingung gar nicht mehr eingehalten werden, dass $f$ und $g$ differenzierbar sind.
Sei also $f$ differenzierbar.

\end{document}
