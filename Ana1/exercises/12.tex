\documentclass[a4paper,10pt]{article}
\usepackage[utf8]{inputenc}
\usepackage[german]{babel}
\usepackage{amsmath}
\usepackage{amssymb}
\usepackage{amsthm}
\usepackage{stmaryrd}

\title{Ana1, Übungsblatt 12}
\author{Marten Lienen (2126759), Gruppe 8; Fabian Schmittmann (2083559), Gruppe 0}

\begin{document}

\maketitle

\section*{Übung 51}

\subsection*{Teil a}

\subsubsection*{Teil 1}

\begin{align*}
 & \frac{1}{(x^2 + 9)(x + 3)(x - 3)} = \frac{a}{x^2 + 9} + \frac{b}{x + 3} + \frac{c}{x - 3}\\
 \Leftrightarrow & \frac{1}{(x^2 + 9)(x + 3)(x - 3)} = \frac{a(x + 3)(x - 3) + b(x - 3)(x^2 + 9) + c(x^2 + 9)(x + 3)}{(x^2 + 9)(x + 3)(x - 3)}\\
 \Leftrightarrow & 1 = a(x + 3)(x - 3) + b(x - 3)(x^2 + 9) + c(x^2 + 9)(x + 3)\\
 \Leftrightarrow & 1 = x^2a - 9a + x^3b + 9xb - 3x^2b - 27b + x^3c + 3x^2c + 9xc + 27c\\
 \Leftrightarrow & 1 = x^3b + x^3c + x^2a - 3x^2b + 3x^2c + 9xb + 9xc + 27c - 27b - 9a\\
 \Leftrightarrow & 1 = x^3(b + c) + x^2(a - 3b + 3c) + x(9b + 9c) + (27c - 27b - 9a)\\
 \Leftrightarrow &
  \begin{cases}
   b + c & = 0\\
   a - 3b + 3c & = 0\\
   27c - 27b - 9a & = 1
  \end{cases}\\
 \Leftrightarrow &
  \begin{cases}
   c & = -b\\
   a & = 3b - 3c\\
   b & = c - \frac{1}{3}a - \frac{1}{27}
  \end{cases}\\
 \Leftrightarrow &
  \begin{cases}
   c & = -b\\
   a & = 6b\\
   b & = -\frac{1}{6}a - \frac{1}{54}
  \end{cases}\\
 \Leftrightarrow &
  \begin{cases}
   c & = -b\\
   a & = -\frac{1}{18}\\
   b & = -\frac{1}{6}a - \frac{1}{54}
  \end{cases}\\
 \Leftrightarrow &
  \begin{cases}
   c & = \frac{1}{108}\\
   a & = -\frac{1}{18}\\
   b & = -\frac{1}{108}
  \end{cases}\\
\end{align*}

\begin{align*}
 \int \frac{1}{x^4 - 81}\ dx & = \int \frac{1}{(x^2 + 9)(x + 3)(x - 3)}\ dx\\
 & = \int \frac{1}{18(x^2 + 9)} - \frac{1}{108(x + 3)} + \frac{1}{108(x - 3)}\ dx\\
 & = \frac{1}{18} \int \frac{1}{x^2 + 9}\ dx - \frac{1}{108} \int \frac{1}{x + 3}\ dx + \frac{1}{108} \int \frac{1}{x - 3}\ dx\\
 & = \frac{1}{162} \int \frac{1}{(\frac{x}{3})^2 + 1}\ dx - \frac{\log(x + 3)}{108} + \frac{\log(x - 3)}{108} \qquad \text{substituiere $u = \frac{x}{3}$}\\
 & = \frac{1}{162} \int \frac{1}{u^2 + 1}\ dx - \frac{\log(x + 3)}{108} + \frac{\log(x - 3)}{108}\\
 & = \frac{1}{162} \arctan(u) - \frac{\log(x + 3)}{108} + \frac{\log(x - 3)}{108} \qquad \text{resubstituiere}\\
 & = \frac{1}{162} \arctan(\frac{x}{3}) - \frac{\log(x + 3)}{108} + \frac{\log(x - 3)}{108}\\
\end{align*}

\subsubsection*{Teil 2}

\begin{align*}
 \int \frac{1}{x^2 + 6x + 8}\ dx & = \int \frac{1}{(x + 3)^2 - 1}\ dx\\
 & = \int \frac{1}{u^2 - 1}\ dx \qquad \text{substituiere $u = x + 3$}\\
 & = \int \frac{1}{(u - 1)(u + 1)}\ dx\\
 & = \int \frac{1}{2}\frac{1}{u - 1} + \frac{1}{2}\frac{1}{u + 1}\ dx \qquad \text{nach Beispiel aus der Vorlesung}\\
 & = \frac{1}{2} \left( \log(u - 1) + \log(u + 1) \right)\\
 & = \frac{1}{2} \left( \log(x + 2) + \log(x + 4) \right)\\
\end{align*}

\subsection*{Teil b}

\section*{Übung 52}

\subsection*{Teil a}

\subsection*{Teil b}

\subsection*{Teil c}

\section*{Übung 53}

\subsection*{Teil a}

\subsection*{Teil b}

\subsection*{Teil c}

\section*{Übung 54}

\subsection*{Teil a}

\begin{equation}
 f(x) = x,\ f'(x) = 1,\ g(x) = (\log x)^2,\ g'(x) = \frac{2 \log x}{x}
\end{equation}
\begin{align*}
 \int (\log x)^2\ dx & = x \cdot (\log x)^2 - \int x \cdot \frac{2 \log x}{x}\ dx\\
 & = x \cdot (\log x)^2 - 2 \int \log x\ dx\\
 & = x \cdot (\log x)^2 - 2 (x \log x + x)\\
 & = x \cdot ((\log x)^2 - 2 \log x + 2)\\
\end{align*}

\subsection*{Teil b}

\begin{equation}
 f(x) = x,\ f'(x) = 1,\ g(x) = (\log x)^3,\ g'(x) = \frac{3 (\log x)^2}{x}
\end{equation}
\begin{align*}
 \int (\log x)^3\ dx & = x \cdot (\log x)^3 - \int x \cdot \frac{3 (\log x)^2}{x}\ dx\\
 & = x \cdot (\log x)^3 - 3 \int (\log x)^2\ dx\\
 & = x \cdot (\log x)^3 - 3 \left( x \cdot ((\log x)^2 - 2 \log x + 2) \right)\\
 & = x \cdot (\log x)^3 - 3x \cdot ((\log x)^2 - 2 \log x + 2)\\
 & = x \cdot (\log x)^3 - 3x(\log x)^2 + 6x \log x - 6x\\
 & = x \cdot \left( (\log x)^3 - 3(\log x)^2 + 6 \log x - 6 \right)\\
\end{align*}

\subsection*{Teil c}

\begin{equation}
 f(x) = -\frac{\cos(bx)}{b},\ f'(x) = \sin(bx),\ g(x) = e^{ax},\ g'(x) = a \cdot e^{ax}
\end{equation}
\begin{align*}
 \int e^{ax} \sin(bx)\ dx & = -\frac{\cos(bx)}{b} \cdot e^{ax} + \int \frac{\cos(bx)}{b} \cdot a \cdot e^{ax}\ dx\\
 & = -\frac{\cos(bx)}{b} \cdot e^{ax} + \frac{a}{b} \int e^{ax}\cos(bx)\ dx\\
\end{align*}

Nebenrechnung:
\begin{equation}
 h(x) = \frac{\sin(bx)}{b},\ h'(x) = \cos(bx),\ i(x) = e^{ax},\ i'(x) = a \cdot e^{ax}
\end{equation}
\begin{align*}
 \int e^{ax}\cos(bx)\ dx & = e^{ax} \cdot \frac{\sin(bx)}{b} - \int \frac{\sin(bx)}{b} \cdot a \cdot e^{ax} \ dx\\
 & = \frac{1}{b} \left( e^{ax} \sin(bx) - a \int e^{ax} \sin(bx) \ dx \right)\\
\end{align*}

Nun zurück zur eigentlichen Rechnung:
\begin{align*}
 \int e^{ax} \sin(bx)\ dx & = -\frac{\cos(bx)}{b} \cdot e^{ax} + \frac{a}{b} \int e^{ax}\cos(bx)\ dx\\
 & = -\frac{\cos(bx)}{b} \cdot e^{ax} + \frac{a}{b^2} \left( e^{ax} \sin(bx) - a \int e^{ax} \sin(bx) \ dx \right)\\
 & = -\frac{\cos(bx)}{b} \cdot e^{ax} + \frac{a}{b^2} e^{ax} \sin(bx) - \frac{a^2}{b^2} \int e^{ax} \sin(bx) \ dx\\
\end{align*}

Daraus folgt:
\begin{align*}
 & \frac{b^2 + a^2}{b^2} \int e^{ax} \sin(bx)\ dx = -\frac{\cos(bx)}{b} \cdot e^{ax} + \frac{a}{b^2} e^{ax} \sin(bx)\\
 \Leftrightarrow & \int e^{ax} \sin(bx)\ dx = \frac{b^2}{b^2 + a^2} \cdot \left( -\frac{\cos(bx)}{b} \cdot e^{ax} + \frac{a}{b^2} e^{ax} \sin(bx) \right)\\
\end{align*}

\begin{align*}
 \int e^{ax} \sin(bx)\ dx & = \frac{b^2}{b^2 + a^2} \cdot \left( -\frac{\cos(bx)}{b} \cdot e^{ax} + \frac{a}{b^2} e^{ax} \sin(bx) \right)\\
 & = \frac{b^2}{b^2 + a^2} \cdot e^{ax} \cdot \left( \frac{a}{b^2} \sin(bx) - \frac{\cos(bx)}{b} \right)\\
 & = \frac{e^{ax}}{b^2 + a^2} \cdot \left( a \sin(bx) - b \cdot \cos(bx) \right)\\
\end{align*}

\section*{Übung 55}

\section*{Übung 56}

\subsection*{Teil a}

\begin{align*}
 \int_0^1 f(x) \sin(\pi x)\ dx & = \int_0^1 \frac{1}{n!} x^n (1 - x)^n \sin(\pi x)\ dx\\
 & > \int_0^1 \frac{1}{n!} x^n (1 - x)^n \sin(\pi)\ dx\\
 & = \int_0^1 \frac{1}{n!} x^n (1 - x)^n \cdot 0\ dx\\
 & = \int_0^1 0\ dx = 0\\
\end{align*}

\begin{align*}
 \int_0^1 f(x) \sin(\pi x)\ dx & = \int_0^1 \frac{1}{n!} x^n (1 - x)^n \sin(\pi x)\ dx\\
 & = \frac{1}{n!} \int_0^1 x^n (1 - x)^n \sin(\pi x)\ dx\\
 & \le \frac{1}{n!} \int_0^1 x^n (1 - x)^n\ dx\\
 & < \frac{1}{n!} \int_0^1 1 \ dx\\
 & \le \frac{1}{n!}
\end{align*}

\subsection*{Teil b}

\subsection*{Teil c}

\subsection*{Teil d}

\end{document}
