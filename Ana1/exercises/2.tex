\documentclass[a4paper,10pt]{article}
\usepackage[utf8]{inputenc}
\usepackage{amsmath}
\usepackage{amssymb}
\usepackage{amsthm}
\usepackage[german]{babel}

\title{Übungsblatt 2, Ana1}
\author{Marten Lienen (2126759), Gruppe 8}

\newtheorem*{claim}{Behauptung}
\newtheorem*{lemma}{Lemma}

\begin{document}

\maketitle

\section*{1.}

\subsection*{a}



\subsection*{b}

\section*{2.}

\subsection*{a}

\subsection*{b}

\subsection*{c}

Für $a = 1, b = 1$ ist $3 = 3 * 1 * 1 > 1^2 + 1^2 = 2$, was im Widerspruch zur Behauptung steht.

\section*{3.}

\begin{lemma}
 \begin{equation*}
  x \le y \land -x \le y \Leftrightarrow |x| \le y
 \end{equation*}
\end{lemma}

\begin{proof}
 Angenommen $x \le y$ und $-x \le y$.
 \begin{align*}
  & x < 0 \Rightarrow |x| = -x \Rightarrow |x| \le y\\
  & x \ge 0 \Rightarrow |x| = x \Rightarrow |x| \le y
 \end{align*}

 Andersherum angenommen $|x| \le y$. Es gilt $x \le |x|$ und $-x \le |x|$. So gilt
 \begin{equation*}
  |x| \le y \Rightarrow x \le y \land -x \le y
 \end{equation*}
\end{proof}

\begin{claim}
 \begin{equation*}
  [a - \varepsilon, a + \varepsilon] = \{x \in \mathbb{R} \mid |x - a| \le \varepsilon\}
 \end{equation*}
\end{claim}

\begin{proof}
 \begin{align*}
  & [a - \varepsilon, a + \varepsilon] = \{x \in \mathbb{R} \mid a - \varepsilon \le x \le a + \varepsilon\}\\
  = & \{x \in \mathbb{R} \mid (a - \varepsilon \le x) \land (x \le a + \varepsilon)\}\\
  = & \{x \in \mathbb{R} \mid (-\varepsilon \le x - a) \land (x - a \le \varepsilon)\}\\
  = & \{x \in \mathbb{R} \mid (-(x - a) \le \varepsilon) \land (x - a \le \varepsilon)\}\\
  = & \{x \in \mathbb{R} \mid |x - a| \le \varepsilon\}
 \end{align*}
\end{proof}

\section*{4.}

\subsection*{a}

\begin{claim}
 \begin{equation*}
  P(n) := x, y > 0 \land x < y \Rightarrow x^n < y^n
 \end{equation*}
\end{claim}

\begin{proof}
 $P$ gilt für $1$: $x^1 = x < y = y^1$.
 
 Wenn $0 < x < y$ und gilt, dass $x^n < y^n$, dann
 \begin{equation*}
  x^{n + 1} = xx^n < yx^n < yy^n = y^{n + 1}
 \end{equation*}
 Somit gilt $P$ auch für $n + 1$.
\end{proof}

\subsection*{b}

\begin{claim}
 Sei $n = 2k - 1$ mit $k \in \mathbb{N}$ und $x, y \in \mathbb{R}$.
 Dann ist $x < y \Leftrightarrow x^n < y^n$.
\end{claim}

\begin{proof}
 Ich beweise es mit vollständiger Induktion über $k$.
 
 Die Aussage ist wahr für $k = 1$: $x < y \Leftrightarrow x^1 < y^1$.
 
 ``$\Rightarrow$'': Wenn $x < y \Rightarrow x^{2k - 1} < y^{2k - 1}$ gilt, dann gilt auch:
 \begin{equation*}
  x < y \Rightarrow 
 \end{equation*}
\end{proof}

\section*{5.}

\subsection*{a}

\subsection*{b}

\end{document}
