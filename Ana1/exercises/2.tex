\documentclass[a4paper,10pt]{article}
\usepackage[utf8]{inputenc}
\usepackage{amsmath}
\usepackage{amssymb}
\usepackage{amsthm}
\usepackage[german]{babel}

\title{Übungsblatt 2, Ana1}
\author{Marten Lienen (2126759), Gruppe 8}

\newtheorem*{claim}{Behauptung}
\newtheorem*{lemma}{Lemma}

\begin{document}

\maketitle

\section*{1.}

\subsection*{a}

\begin{lemma}
 $c$ ist genau dann obere Schranke von $M \cup N$, wenn es obere Schranke von $M$ und $N$ ist.
\end{lemma}

\begin{proof}
 \begin{align*}
  \forall x \in M \cup N (x \le c) & \Leftrightarrow \forall x \in \{d \mid d \in M \lor d \in N\} (x \le c)\\
  & \Leftrightarrow (\forall m \in M (m \le c)) \land (\forall n \in N (n \le c))
 \end{align*}
\end{proof}

\begin{proof}
 Sei $c$ das Supremum von $M \cup N$.
 $c$ ist also auch obere Schranke von $M$ und $N$ und $\forall c' < c \exists x \in M \cup N (x > c')$.
 Wäre $c < max \{sup(M), sup(N)\}$, dann wäre $c < sup(M) \lor c < sup(N)$, könnte also nicht obere Schranke von $M$ oder $N$ sein, was im Widerspruch zur Vorraussetzung steht.
 Wäre $c > max \{sup(M), sup(N)\}$, so gäbe es die obere Schranke $max \{sup(M), sup(N)\} < d < c$ von $M$ und $N$, die auch obere Schranke von $M \cup N$ ist.
 Sie kann jedoch nicht existieren, da $c$ die kleinste obere Schranke von $M \cup N$ ist.
\end{proof}

\subsection*{b}

\begin{proof}
 Seien $c = sup(M \cap N)$, $d = sup(M)$ und $e = sup(N)$.
 Da $M \cap N \subseteq M$ und $M \cap N \subseteq N$ sind $d$ und $e$ auch obere Schranken von $M \cap N$, sodass gilt $c \le d \le c \le e \Rightarrow c \le min \{d, e\}$.
\end{proof}

Das nicht immer Gleichheit gilt, sieht man an folgendem Beispiel
\begin{align*}
 & M = \{1, 2\}\\
 & N = \{1, 3\}\\
 & sup(M \cap N) = 1 < 2 = min \{2, 3\} = min \{sup(M), sup(N)\}
\end{align*}

\section*{2.}

\subsection*{a}

\begin{proof}
 Nach Satz 4 gilt
 \begin{equation*}
  \forall x \in \mathbb{R} x^2 \ge 0
 \end{equation*}

 \begin{align*}
  2ab \le a^2 + b^2 \Leftrightarrow (a^2 + b^2) - 2ab = a^2 + b^2 - 2ab = (a - b)^2 \ge 0
 \end{align*}
\end{proof}

\subsection*{b}

\begin{proof}
 Wenn $ab \ge 0$, gilt $ab < 2ab \le a^2 + b^2$.
 Wenn $ab < 0$, gilt $ab < 0 < a^2 + b^2$, weil für $x \ne 0 \in \mathbb{R} x^2 > 0$ gilt.
\end{proof}

\subsection*{c}

Für $a = 1, b = 1$ ist $3 = 3 * 1 * 1 > 1^2 + 1^2 = 2$, was im Widerspruch zur Behauptung steht.

\section*{3.}

\begin{lemma}
 \begin{equation*}
  x \le y \land -x \le y \Leftrightarrow |x| \le y
 \end{equation*}
\end{lemma}

\begin{proof}
 Angenommen $x \le y$ und $-x \le y$.
 \begin{align*}
  & x < 0 \Rightarrow |x| = -x \Rightarrow |x| \le y\\
  & x \ge 0 \Rightarrow |x| = x \Rightarrow |x| \le y
 \end{align*}

 Andersherum angenommen $|x| \le y$. Es gilt $x \le |x|$ und $-x \le |x|$. So gilt
 \begin{equation*}
  |x| \le y \Rightarrow x \le y \land -x \le y
 \end{equation*}
\end{proof}

\begin{claim}
 \begin{equation*}
  [a - \varepsilon, a + \varepsilon] = \{x \in \mathbb{R} \mid |x - a| \le \varepsilon\}
 \end{equation*}
\end{claim}

\begin{proof}
 \begin{align*}
  & [a - \varepsilon, a + \varepsilon] = \{x \in \mathbb{R} \mid a - \varepsilon \le x \le a + \varepsilon\}\\
  = & \{x \in \mathbb{R} \mid (a - \varepsilon \le x) \land (x \le a + \varepsilon)\}\\
  = & \{x \in \mathbb{R} \mid (-\varepsilon \le x - a) \land (x - a \le \varepsilon)\}\\
  = & \{x \in \mathbb{R} \mid (-(x - a) \le \varepsilon) \land (x - a \le \varepsilon)\}\\
  = & \{x \in \mathbb{R} \mid |x - a| \le \varepsilon\}
 \end{align*}
\end{proof}

\section*{4.}

\subsection*{a}

\begin{claim}
 \begin{align*}
  & x, y > 0\\
  & P(n) := x < y \Leftrightarrow x^n < y^n
 \end{align*}
\end{claim}

\begin{proof}
 $P$ gilt für $1$: $x^1 = x < y = y^1$.
 
 Wenn $x < y$ und $x^n < y^n$ gilt, dann
 \begin{equation*}
  x^{n + 1} = xx^n < yx^n < yy^n = y^{n + 1}
 \end{equation*}
 Somit gilt $P$ auch für $n + 1$.
\end{proof}

\subsection*{b}

\begin{claim}
 Sei $n = 2k - 1$ mit $k \in \mathbb{N}$ und $x, y \in \mathbb{R}$.
 Dann ist $x < y \Leftrightarrow x^n < y^n$.
\end{claim}

\begin{equation*}
 x^n = x^{2k}x^{-1} = (x^2)^kx^{-1}
\end{equation*}
\begin{equation*}
 y^n = y^{2k}y^{-1} = (y^2)^ky^{-1}
\end{equation*}

\begin{proof}
 Ich beweise es mit Induktion über $k$.
 Für $k = 1$ gilt die Behauptung: $x = x^1 < y^1 = y$.
 
 Wenn $x < y$ und $x^{2k - 1} < y^{2k - 1}$ gelten, gilt auch
 \begin{align*}
  x^{2(k + 1) - 1} = x^{2k - 1 + 2} = x^{2k - 1}x^2 < y^{2k - 1}y^2 = y^{2k - 1 + 2} = y^{2(k + 1) - 1}
 \end{align*}
 Somit gilt die Behauptung auch für $k + 1$.
\end{proof}

\section*{5.}

\subsection*{a}

\begin{proof}
 Wenn man die Folge auflöst entsteht
 \begin{equation*}
  1 \cdot \frac{1}{n} \cdot 2 \cdot \frac{2}{n} \cdots (n - 1) \cdot \frac{1}{n} \cdot 1
 \end{equation*}
 In der Vorlesung wurde gezeigt, dass $(\frac{1}{n})$ eine Nullfolge ist.
 Es wird also immer paarweise eine beschränkte (konstante) Folge mit einer Nullfolge multipliziert, was wiederum eine Nullfolge ergibt.
 Das Produkt dieser $n - 1$ Nullfolgen ergibt nach den Rechenregeln für Grenzwerte wieder eine Nullfolge.
\end{proof}

\subsection*{b}

\end{document}
