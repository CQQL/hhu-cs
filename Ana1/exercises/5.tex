\documentclass[a4paper,10pt]{article}
\usepackage[utf8]{inputenc}
\usepackage{amsmath}
\usepackage{amssymb}
\usepackage{amsthm}
\usepackage[german]{babel}

\title{Ana1, Übungsblatt 5}
\author{Marten Lienen (2126759), Gruppe 8, Fabian Schmittmann}

\newtheorem{claim}{Behauptung}

\begin{document}

\maketitle

\section*{Übung 18}

\subsection*{a}

Nach dem binomischen Lehrsatz mit $x = 1$ und $y = -1$ gilt
\begin{equation}
 \sum_{k = 0}^n \begin{pmatrix}n\\k\end{pmatrix} * 1^k * (-1)^{n - k} = (1 + (-1))^n = 0^n = 0
\end{equation}

\subsection*{b}

Nach dem binomischen Lehrsatz mit $x = 3$ und $y = 1$ gilt
\begin{equation}
 \sum_{k = 0}^n \begin{pmatrix}n\\k\end{pmatrix} * 3^k * (1)^{n - k} = (3 + 1)^n = 4^n
\end{equation}

\subsection*{c}

\begin{proof}
 Nach dem binomischen Lehrsatz gilt
 \begin{equation}
  \sum_{k = 0}^{2n + 1} \begin{pmatrix}2n + 1\\k\end{pmatrix} = 2^{2n + 1} = 2 * 2^{2n}
 \end{equation}
 Durch Aufteilen der Summe in 2 Teilsummen und eine Indexverschiebung erhalten wir
 \begin{align}
  \sum_{k = 0}^{2n + 1} \begin{pmatrix}2n + 1\\k\end{pmatrix} & = \sum_{k = 0}^{n} \begin{pmatrix}2n + 1\\k\end{pmatrix} + \sum_{k = n + 1}^{2n + 1} \begin{pmatrix}2n + 1\\k\end{pmatrix}\\
  & = \sum_{k = 0}^{n} \begin{pmatrix}2n + 1\\k\end{pmatrix} + \sum_{k = 0}^{n} \begin{pmatrix}2n + 1\\n + 1 + k\end{pmatrix}\\
  & = \sum_{k = 0}^{n} \begin{pmatrix}2n + 1\\k\end{pmatrix} + \sum_{k = 0}^{n} \begin{pmatrix}2n + 1\\2n + 1 - k\end{pmatrix}\\
  & = \sum_{k = 0}^{n} \begin{pmatrix}2n + 1\\k\end{pmatrix} + \begin{pmatrix}2n + 1\\2n + 1 - k\end{pmatrix}
 \end{align}
 Die Umformung von (5) nach (6) ist gültig, weil gilt
 \begin{align}
  \sum_{k = 0}^{n} \begin{pmatrix}2n + 1\\n + 1 + k\end{pmatrix} = \sum_{k = 0}^{n} \begin{pmatrix}2n + 1\\2n + 1 - k\end{pmatrix}
 \end{align}
 Dies gilt, weil der untere Teil des Binomialkoeffizienten auf beiden Seiten die gleichen Werte annimmt:
 \begin{equation*}
  \{ x \mid n + 1\le x \le n + 1 + n = 2n + 1 \} = \{ x \mid 2n + 1 - n = n + 1 \le x \le 2n + 1 - 0 = 2n + 1 \}
 \end{equation*}
 
 Nach Bemerkung a) aus dem Skript gilt
 \begin{equation}
  \begin{pmatrix}2n + 1\\k\end{pmatrix} = \begin{pmatrix}2n + 1\\2n + 1 - k\end{pmatrix}
 \end{equation}
 Damit ergibt sich
 \begin{align}
  \sum_{k = 0}^{n} \begin{pmatrix}2n + 1\\k\end{pmatrix} + \begin{pmatrix}2n + 1\\2n + 1 - k\end{pmatrix} & = \sum_{k = 0}^{n} 2 * \begin{pmatrix}2n + 1\\k\end{pmatrix}\\
  & = 2 * \sum_{k = 0}^{n} \begin{pmatrix}2n + 1\\k\end{pmatrix} = 2 * 2^{2n}\\
  \Rightarrow & \sum_{k = 0}^{n} \begin{pmatrix}2n + 1\\k\end{pmatrix} = 2^{2n}
 \end{align}
\end{proof}

\subsection*{d}



\section*{Übung 19}

\subsection*{a}

\begin{align}
 \frac{|a_{n + 1}|}{|a_n|} = \frac{\frac{(n + 1)^k}{2^{ + 1}}}{\frac{n^k}{2^n}} = \frac{(n + 1)^k * 2^n}{2^{n + 1} * n^k} = \frac{1}{2} * (1 + \frac{1}{n})^k
\end{align}
Da $(1 + \frac{1}{n})^k \rightarrow 1$, ist $\frac{|a_{n + 1}|}{|a_n|} < \frac{3}{4}$ für fast alle $n \in \mathbb{N}$.

\subsection*{b}

Da $\sum \frac{n^k}{2^n}$ nach a) konvergiert, folgt aus Satz 1 §3, dass $\frac{n^k}{2^n}$ eine Nullfolge ist, d.h.
\begin{equation}
 \lim_{n \rightarrow \infty} \frac{n^k}{2^n} = 0
\end{equation}

\section*{Übung 20}

\begin{claim}
 $f$ ist an den Stellen $\frac{1}{2}, \frac{1}{3}, \frac{1}{4}, \dots$ unstetig.
\end{claim}

\begin{proof}
 Sei $n \in \mathbb{N}$.
 Wir suchen ein $\varepsilon > 0$, sodass $|f(x) - f(\frac{1}{n + 1})| = |f(x) - \frac{1}{n}| \ge \varepsilon$ für einige $x \in \mathbb{R}$ mit $|x - \frac{1}{n + 1}| < \delta$ für jedes $\delta > 0$.
 Wähle $\varepsilon = \frac{1}{n^2 + n + 1}$.
 Wenn $\frac{1}{n + 1} - \delta < x < \frac{1}{n + 1}$, gilt
 \begin{equation}
  |f(x) - \frac{1}{n}| = -(f(x) - \frac{1}{n}) = \frac{1}{n} - f(x) \ge \frac{1}{n} - \frac{1}{n + 1} = \frac{1}{n^2 + n} > \frac{1}{n^2 + n + 1} = \varepsilon
 \end{equation}
 Folglich gibt es mindestens ein solches $x$ und $f$ ist unstetig an allen Stellen $\frac{1}{n + 1}$ für alle $n \in \mathbb{N}$.
\end{proof}

\begin{claim}
 $f$ ist an allen anderen Stellen stetig.
\end{claim}

\begin{proof}
 Nach Satz 2 und Beispiel 1 und 2 ist $f$ stetig an allen Stellen außer $< 0$ und $> 1$.
 
 Sei $\varepsilon > 0$.
 Sei $0 < x_0 \le 1$ und $x \ne \frac{1}{n + 1} \forall n \in \mathbb{N}$.
 Wir suchen ein $\delta$, sodass $|f(x) - f(x_0)| < \varepsilon$ für alle $x$ mit $|x - x_0| < \delta$.
 Wir wählen $n \in \mathbb{N}$, sodass $\frac{1}{n + 1} < x < \frac{1}{n}$.
 Wir wählen $\delta = min \{ |x_0 - \frac{1}{n + 1}|, |x_0 - \frac{1}{n}| \}$.
 \begin{equation}
  |f(x) - f(x_0)| = |\frac{1}{n} - \frac{1}{n}| = 0 < \varepsilon
 \end{equation}
 
 Abschließend betrachten wir die Stelle $0$.
 Sei $\varepsilon > 0$.
 Wir suchen ein $\delta$, sodass $|f(x) - f(0)| = |f(x)| < \varepsilon$ für alle $x$ mit $|x| < \delta$.
 Wir wählen $\delta = min \{5, max \{ \{ \frac{1}{n} \mid \frac{1}{n} < \varepsilon \} \backslash \{ max \{ \frac{1}{n} \mid \frac{1}{n} < \varepsilon \} \} \}$.
 Wenn $x < 0$, gilt
 \begin{equation}
  |f(x)| = 0 < \varepsilon
 \end{equation}
 Wenn $x > 0$, gilt
 \begin{equation}
  |f(x)| < \frac{1}{n + 1} < \varepsilon
 \end{equation}
\end{proof}

\section*{Übung 21}

\subsection*{Stetigkeit}

Nach Satz 2 und Beispiel 1 und 2 ist $f$ stetig an allen Stellen außer $2$.
Wir zeigen, dass $f$ auch an der Stelle $2$ stetig ist.
Sei $\varepsilon > 0$.
Wir suchen ein $\delta$, sodass $|f(x) - f(2)| = |f(x) - 3| < \varepsilon$ für alle $x \in \mathbb{R}$ mit $|x - 2| < \delta$.
Wähle $\delta = min \{ -\frac{\varepsilon - 4}{2}, \varepsilon + 2 \}$.
Wenn $x < 2$, gilt
\begin{equation}
 |f(x) - 3| = |2x - 1 - 3| = |2x - 4| = -(2x - 4) < \varepsilon
\end{equation}
Wenn $x > 2$, gilt
\begin{equation}
 |f(x) - 3| = |x + 1 - 3| = |x - 2| = x - 2 < \varepsilon
\end{equation}

\subsection*{Umkehrfunktion und Bijektivität}

\begin{equation}
 f^{-1}(x) = \begin{cases}
              \frac{1}{2} x + \frac{1}{2} & \text{ wenn $x \le 3$}\\
              x - 1 & \text{ wenn $x > 3$}
             \end{cases}
\end{equation}

Wenn $x \le 2$
\begin{align}
 (f \circ f^{-1})(x) = f(\frac{1}{2} x + \frac{1}{2}) = 2 * (\frac{1}{2} x + \frac{1}{2}) - 1 = x + 1 - 1 = x = id_\mathbb{R}(x)\\
 (f^{-1} \circ f)(x) = f^{-1}(2x - 1) = \frac{2x - 1}{2} + \frac{1}{2} = x - \frac{1}{2} + \frac{1}{2} = x = id_\mathbb{R}(x)
\end{align}

Wenn $2 < x \le 3$
\begin{align}
 (f \circ f^{-1})(x) = f(\frac{1}{2} x + \frac{1}{2}) = 2 * (\frac{1}{2} x + \frac{1}{2}) - 1 = x = id_\mathbb{R}(x)\\
 (f^{-1} \circ f)(x) = f^{-1}(x + 1) = (x + 1) - 1 = x = id_\mathbb{R}(x)
\end{align}

Wenn $x > 3$
\begin{align}
 (f \circ f^{-1})(x) = f(x - 1) = (x - 1) + 1 = x = id_\mathbb{R}(x)\\
 (f^{-1} \circ f)(x) = f^{-1}(x + 1) = (x + 1) - 1 = x = id_\mathbb{R}(x)
\end{align}

$f^{-1}$ ist also die Umkehrfunktion von $f$ und $f$ ist bijektiv.

\subsection*{Graph}

\end{document}
