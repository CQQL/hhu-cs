\documentclass[a4paper,10pt]{article}
\usepackage[utf8]{inputenc}
\usepackage[german]{babel}
\usepackage{amsmath}
\usepackage{amssymb}
\usepackage{amsthm}
\usepackage{stmaryrd}

\title{Ana1, Übungsblatt 10}
\author{Marten Lienen (2126759), Gruppe 8; Fabian Schmittmann (2083559), Gruppe 0}

\begin{document}

\maketitle

\section*{Übung 42}

\subsection*{a}

\begin{equation}
 \lim_{x \rightarrow 0} \frac{3 \cos 3x}{\frac{1}{1 + x}} = \lim_{x \rightarrow 0}(3 \cos(3x)(1 + x)) = 3
\end{equation}

Nach dem Satz von L'Hospital gilt

\begin{equation}
 \lim_{x \rightarrow 0} \frac{\sin 3x}{\log (1 + x)} = 3
\end{equation}

\subsection*{b}

\begin{equation}
 \lim_{x \rightarrow 0} \left( \frac{1}{\log(1 + x) + \frac{1 + x}{1 + x} + 1} \right) = \frac{1}{2}
\end{equation}

Nach dem Satz von L'Hospital gilt

\begin{equation}
 \lim_{x \rightarrow 0} \left( \frac{1 - \frac{1}{1 + x}}{\log(1 + x) + \frac{x}{1 + x}} \right) = \lim_{x \rightarrow 0} \left( \frac{x}{\log(1 + x)(1 + x) + x} \right) = \frac{1}{2}
\end{equation}

Nach dem Satz von L'Hospital gilt

\begin{equation}
 \lim_{x \rightarrow 0} \left( \frac{1}{\log(1 + x)} - \frac{1}{x} \right) = \lim_{x \rightarrow 0} \left( \frac{x - \log(1 + x)}{\log(1 + x)x} \right) = \frac{1}{2}
\end{equation}

\subsection*{c}

\begin{equation}
 \lim_{x \rightarrow 0} \frac{(\cos x)^2 - 1}{(\cos x)^2 * (1 - \cos x)} = \lim_{x \rightarrow 0} -\frac{(\cos x + 1)}{(\cos x)^2} = 2
\end{equation}

Nach dem Satz von L'Hospital gilt

\begin{equation}
 \lim_{x \rightarrow 0} \frac{x - \tan x}{x - \sin x} = \lim_{x \rightarrow 0} \frac{x - \frac{\sin x}{\cos x}}{x - \sin x} = 2
\end{equation}

\section*{Übung 43}

Sei $r$ der Radius des Papierstücks, $0 < \alpha \le 2 \pi$ der Sektorwinkel, den man nicht ausschneidet, dann ist $q$ der Restumfang des Papierstücks, $r_k$ der Radius der Kegelgrundfläche, $h$ die Höhe des Kegels, $G$ die Grundfläche des Kegels und $v$ sein Volumen und es gilt
\begin{align}
 q & = \alpha r\\
 r_k & = \frac{q}{2 \pi} = \frac{\alpha r}{2 \pi}\\
 h & = \sqrt{r^2 - r_k^2} = \sqrt{\left( 1 - \left( \frac{\alpha}{2 \pi} \right)^2 \right) r^2}\\
 G & = \pi r_k^2 = \pi \left( \frac{\alpha r}{2 \pi} \right)^2\\
 v(\alpha) & = \frac{1}{3} G h = \frac{1}{3} \pi r^3 \left( \frac{\alpha}{2 \pi} \right)^2 \sqrt{1 - \left( \frac{\alpha}{2 \pi} \right)^2}\\
 v'(\alpha) & = \frac{1}{3} r^3 \frac{1}{\sqrt{\left( 1 - \left( \frac{\alpha}{2 \pi} \right)^2 \right)}} \left( \frac{1}{2} \left( \frac{\alpha}{2 \pi} \right)^3 - \frac{\alpha}{2 \pi} \left( 1 - \left( \frac{\alpha}{2 \pi} \right)^2 \right) \right)\\
 & = \frac{1}{3} r^3 \frac{1}{\sqrt{\left( 1 - \left( \frac{\alpha}{2 \pi} \right)^2 \right)}} \left( \frac{1}{2} \left( \frac{\alpha^3}{8 \pi^3} \right) - \frac{\alpha}{2 \pi} \left( 1 - \left( \frac{\alpha^2}{4 \pi^2} \right) \right) \right)\\
 & = \frac{1}{3} r^3 \frac{1}{\sqrt{\left( 1 - \left( \frac{\alpha}{2 \pi} \right)^2 \right)}} \left( \frac{1}{2} \frac{\alpha^3}{8 \pi^3} - \frac{\alpha}{2 \pi} + \frac{\alpha^3}{8 \pi^3} \right)\\
 & = \frac{1}{3} r^3 \frac{1}{\sqrt{\left( 1 - \left( \frac{\alpha}{2 \pi} \right)^2 \right)}} \left( \frac{3}{2} \frac{\alpha^3}{8 \pi^3} - \frac{\alpha}{2 \pi} \right)\\
 & = \frac{1}{6 \pi} r^3 \frac{\alpha}{\sqrt{\left( 1 - \left( \frac{\alpha}{2 \pi} \right)^2 \right)}} \left( \frac{3 \alpha^2}{8 \pi^2} - 1 \right)\\
 & = \frac{1}{16 \pi^3} r^3 \frac{\alpha}{\sqrt{\left( 1 - \left( \frac{\alpha}{2 \pi} \right)^2 \right)}} \left( \alpha^2 - \frac{8 \pi^2}{3} \right)\\
 & = \frac{1}{16 \pi^3} r^3 \frac{\alpha}{\sqrt{\left( 1 - \left( \frac{\alpha}{2 \pi} \right)^2 \right)}} \left( \alpha - \sqrt{\frac{8}{3}}\pi \right) \left( \alpha + \sqrt{\frac{8}{3}}\pi \right)
\end{align}
Aufgrund der Einschränkungen für $\alpha$ hat $v'$ nur eine Nullstelle im Definitionsbereich:
\begin{equation}
 \alpha = \sqrt{\frac{8}{3}}\pi
\end{equation}

Weil $\lim_{\alpha \rightarrow 0} v(\alpha) = 0$, $v(2 \pi) = 0$, $v'$ nur eine Nullstelle hat und $v$ im Definitionsbereich immer positiv ist, muss $\sqrt{\frac{8}{3}}\pi$ ein Maximum sein.
Man erreicht also das größte Kegelvolumen, wenn man einen Sektor mit dem Winkel $2 \pi - \sqrt{\frac{8}{3}}\pi$ herausschneidet.

\section*{Übung 44}

\subsection*{a}

\begin{equation}
 f'(x) = \log(1 + \frac{1}{x}) - \frac{1}{x + 1} = \log (1 + \frac{1}{x}) - \frac{\frac{1}{x}}{1 + \frac{1}{x}} = \varphi \left(\frac{1}{x}\right)
\end{equation}

\subsection*{b}

\begin{equation}
 \varphi'(x) = \frac{1}{1 + x} - \frac{1}{(1 + x)^2} = \frac{1 + x - 1}{(1 + x)^2} = \frac{x}{(1 + x)^2}
\end{equation}
Da für $x > 0$ sowohl Zähler als auch Nenner größer als 0 sind, ist auch $\varphi'(x) > 0$ und somit ist $\varphi$ streng monoton wachsend auf $\mathbb{R}_{> 0}$.
Da $\varphi(0) = 0$, was kleiner ist als jedes Element von $\varphi(\mathbb{R}_{> 0})$, ist $\varphi$ auch streng monoton wachsend auf $\mathbb{R}_{\ge 0}$.

\subsection*{c}

Da $\varphi(0) = 0$ und $\varphi$ streng monoton wachsend ist, folgt $\varphi(\mathbb{R}_{> 0}) > 0$.

\subsection*{d}

Die Ableitung der Funktion ist
\begin{equation}
 \left( \log \left( 1 + \frac{1}{x} \right) + \frac{x}{1 + \frac{1}{x}} \right) \left( 1 + \frac{1}{x} \right)^x
\end{equation}
Diese ist für alle $x > 0$ ebenfalls echt größer $0$ und somit ist die Funktion aus der Aufgabenstellung streng monoton wachsend auf $\mathbb{R}_{> 0}$.

\section*{Übung 45}

\subsection*{a}

Sei $g(x)$ definiert als
\begin{equation}
 g(x) := \frac{f(x)}{e^{a(x)}}
\end{equation}
Dann ist
\begin{equation}
 g'(x) = \frac{f'(x)e^{a(x)} - f(x)a(x)e^{a(x)}}{e^{2a(x)}} = \frac{a(x)f(x)e^{a(x)} - f(x)a(x)e^{a(x)}}{e^{2a(x)}} = 0
\end{equation}
$g$ ist folglicht konstant und $g(x) = c$. Daraus folgt
\begin{equation}
 f(x) = c e^{a(x)}
\end{equation}


\subsection*{b}

\section*{Übung 46}

\begin{align*}
 \lim_{n \rightarrow \infty} \left( 1 + \frac{x}{n} \right)^n & = \lim_{n \rightarrow \infty} \exp \left( n \cdot \log \left( 1 + \frac{x}{n} \right) \right)\\
 & = \exp \lim_{n \rightarrow \infty} \left( n \cdot \log \left( 1 + \frac{x}{n} \right) \right)\\
 & = \exp \lim_{t \rightarrow 0} \left( \frac{\log \left( 1 + xt \right)}{t} \right)\\
 & = \exp \lim_{t \rightarrow 0} \left( \frac{x \cdot \frac{1}{1 + xt}}{1} \right) = \exp x = e^x\\
\end{align*}

\end{document}
