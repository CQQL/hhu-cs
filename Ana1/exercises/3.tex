\documentclass[a4paper,10pt]{article}
\usepackage[utf8]{inputenc}
\usepackage{amsmath}
\usepackage{amssymb}
\usepackage{amsthm}
\usepackage[german]{babel}

\title{Ana1, Übungsblatt 3}
\author{Marten Lienen (2126759), Gruppe 8}

\newtheorem*{claim}{Behauptung}
\newtheorem*{definition}{Definition}
\newtheorem*{notice}{Bemerkung}
\newtheorem*{lemma}{Lemma}
\newtheorem*{example}{Beispiel}

\begin{document}

\maketitle

\section*{9}

\subsection*{a}

\begin{align*}
 \lim_{n \rightarrow \infty} \frac{n^2 - n + 1}{2n^3 + n + 1} = \lim_{n \rightarrow \infty} \frac{\frac{1}{n} - \frac{1}{n^2} + \frac{1}{n^3}}{2 + \frac{1}{n^2} + \frac{1}{n^3}} = \frac{0}{2} = 0
\end{align*}

\subsection*{b}

\begin{align*}
 \lim_{n \rightarrow \infty} \frac{n^2 - n + 1}{8n^2 - 3} = \lim_{n \rightarrow \infty} \frac{1 - \frac{1}{n} + \frac{1}{n^2}}{8 - \frac{3}{n^2}} = \frac{1}{8}
\end{align*}

\subsection*{c}

\begin{align*}
 \lim_{n \rightarrow \infty} \frac{n^3 - 17n + 1}{412n^2 + 13} = \lim_{n \rightarrow \infty} \frac{1 - \frac{17}{n^2} + \frac{1}{n^3}}{\frac{412}{n} + \frac{13}{x^3}}
\end{align*}

Es konvergiert nicht, weil $\frac{412}{n} + \frac{13}{x^3}$ den Grenzwert $0$ hat.

\subsection*{Regeln}

Man scheint folgende Regel erkennen zu können

\begin{description}
 \item[Grad(Zähler) $<$ Grad(Nenner)] Grenzwert ist $0$
 \item[Grad(Zähler) $=$ Grad(Nenner)] Grenzwert ist eine Zahl aus $\mathbb{Q}$
 \item[Grad(Zähler) $>$ Grad(Nenner)] Konvergiert nicht
\end{description}

\section*{10}

\subsection*{a}

\begin{proof}
 \begin{align*}
  a_n \rightarrow 0\\
  \sqrt{a_n}\sqrt{a_n} \rightarrow 0
 \end{align*}

 Nach Satz 3 muss mindestens einer der Faktoren eine Nullfolge sein.
 Da hier beide gleich sind, folgt, dass $(\sqrt{a_n})$ eine Nullfolge sein muss.
\end{proof}

\subsection*{b}

\begin{proof}
 Angenommen $a < 0$.
 $(a_n)$ müsste kleiner sein als $0$ für fast alle $n \in \mathbb{N}$, was ein Widerspruch dazu ist, dass $(a_n)$ eine Folge nicht-negativer, reeller Zahlen ist.
 
 \begin{equation*}
  a_n = \sqrt{a_n}^2 \rightarrow a = \sqrt{a}^2
 \end{equation*}
 
 Nach der Rechenregel für Grenzwerte $a_nb_n \rightarrow ab$ folgt:
 \begin{equation*}
  \sqrt{a_n}\sqrt{a_n} \rightarrow \sqrt{a}\sqrt{a} \Rightarrow \sqrt{a_n} \rightarrow \sqrt{a}
 \end{equation*}
\end{proof}

\section*{11}

\subsection*{a}



\subsection*{b}

Da $n > \sqrt{n}$ für alle $n \in \mathbb{N}$ und $|n - \sqrt{2}|$ streng monoton wächst, muss $b_n < 1$ sein und streng monoton fallen.

\subsection*{c}

\begin{align*}
 a_n := n^{\frac{1}{2n}} = \sqrt[2n]{n}\\
 1 \rightarrow 1\\
 (b_n) \rightarrow 0
\end{align*}
 
Nach den Rechenregeln für Grenzwerte folgt aus $a_n = 1 + b_n$, dass $\sqrt[2n]{n}~\rightarrow~1$.

\section*{12}

\subsection*{a}

\begin{proof}
 \begin{equation*}
  a_1 = \sqrt{2} < 2
 \end{equation*}

 Es bleibt zu zeigen, dass $a_{n + 1} < 2$, wenn $a_n < 2$.
 
 \begin{align*}
  a_n < 2 \Leftrightarrow 2 + a_n < 4 \Leftrightarrow \sqrt{2 + a_n}^2 < 2^2 \Rightarrow \sqrt{2 + a_n} < 2 \Leftrightarrow a_{n + 1} < 2
 \end{align*}
\end{proof}

\subsection*{b}

\begin{proof}
 \begin{equation*}
  a_1 < a_2 \Leftrightarrow \sqrt{2} < \sqrt{2 + \sqrt{2}}
 \end{equation*}

 Es bleibt zu zeigen, dass $a_{n + 1} < a_{n + 2}$, wenn $a_n < a_{n + 1}$.
 
 \begin{align*}
  a_n < a_{n + 1} & \Leftrightarrow 2 + a_n < 2 + a_{n + 1}\\
  & \Leftrightarrow \sqrt{2 + a_n} < \sqrt{2 + a_{n + 1}}\\
  & \Leftrightarrow a_{n + 1} < a_{n + 2}
 \end{align*}
\end{proof}

\subsection*{c}

Nach Satz 8 konvergiert die Folge $(a_n)$, weil sie streng monoton wachsend und nach oben beschränkt ist.
Sei $a = \lim_{n \rightarrow \infty} a_n$.
Es ist $a_{n + 1} = \sqrt{2 + a_n}$.
Durch Übergang zum Limes wie im Beispiel zu Satz 8 erhält man:

\begin{align*}
 a = \sqrt{2 + a} \Leftrightarrow a^2 = 2 + a \Leftrightarrow a^2 - a = 2
\end{align*}

Durch Einsetzen von $2$ erhält man $2^2 - 2 = 2$.
$a = 2$ löst also obige Gleichung.
Da wir wissen, dass $(a_n)$ genau einen Grenzwert hat, ist somit gezeigt, dass $\lim_{n \rightarrow \infty} a_n = 2$.

\end{document}
