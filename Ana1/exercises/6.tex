\documentclass[a4paper,10pt]{article}
\usepackage[utf8]{inputenc}
\usepackage[german]{babel}
\usepackage{amsmath}
\usepackage{amssymb}
\usepackage{amsthm}

\title{Ana1, Übungsblatt 6}
\author{Marten Lienen (2126759), Gruppe 1}

\begin{document}

\maketitle

\section*{Übung 23}

\begin{proof}
 Sei $\varepsilon > 0$, $x_0 \in \mathbb{R}$.
 Wir suchen $\delta$, sodass $|x^3 - x_0^3| < \varepsilon$ für alle $x$ mit $|x - x_0| < \delta$.
 Wir können $x$ schreiben als $x_0 + h$ mit $|h| < \delta$.
 Wähle $\delta = min \{1, \frac{\varepsilon}{|3x_0^2 + 3|x_0| + 1|} \}$.
 \begin{align*}
  |(x_0 + h)^3 - x_0^3| & = |x_0^3 + 3x_0^2h + 3x_0h^2 + h^3 - x_0^3|\\
  & = |3x_0^2h + 3x_0h^2 + h^3|\\
  & = |h * (3x_0^2 + 3x_0h + h^2)|\\
  & \le |h| * |3x_0^2 + 3|x_0||h| + h^2|\\
  & \le |h| * |3x_0^2 + 3|x_0| + 1|< \varepsilon
 \end{align*}
\end{proof}

\section*{Übung 24}

\begin{proof}
 Sei $\varepsilon > 0$, $x_0 \in D$.
 Wir suchen $\delta$, sodass $|h(x) - h(x_0)| < \varepsilon$ für alle $x \in D$ mit $|x - x_0| < \delta$.
 Weil $f$ und $g$ stetig sind, gibt es $\delta_f$ und $\delta_g$, sodass $|f(x) - f(x_0)| < \varepsilon$ für alle $x \in D$ mit $|x - x_0| < \delta_f$ und $|g(x) - g(x_0)| < \varepsilon$ für alle $x \in D$ mit $|x - x_0| < \delta_g$.
 Wir wählen $\delta = min \{\delta_f, \delta_g\}$.
 Zu zeigen ist
 \begin{equation}
  |h(x) - h(x_0)| < \varepsilon
 \end{equation}
 Wenn gilt
 \begin{align*}
  & (h(x) = f(x) \land h(x_0) = f(x_0))\\
  \lor & (h(x) = g(x) \land h(x_0) = g(x_0))
 \end{align*}
 ist es bereits gezeigt, weil $f$ und $g$ stetig sind.
 Wir nehmen nun o.B.d.A. an, dass $h(x) = f(x)$ und $h(x_0) = g(x_0)$.
 Für den umgekehrten Fall geht der Beweis analog.
 Zu zeigen ist
 \begin{equation}
  |f(x) - g(x_0)| < \varepsilon
 \end{equation}
 Es gilt $f(x) > g(x)$ und $g(x_0) > f(x_0)$.
 Wenn $f(x) > g(x_0)$, gilt
 \begin{equation}
  |f(x) - g(x_0)| < |f(x) - f(x_0)| < \varepsilon
 \end{equation}
 Wenn $f(x) \le g(x_0)$, gilt
 \begin{equation}
  |f(x) - g(x_0)| < |g(x) - g(x_0)| < \varepsilon
 \end{equation}
\end{proof}

\section*{Übung 25}

In der Vorlesung wurde folgende Formel gezeigt
\begin{equation}
 |e - \sum_{n = 0}^N \frac{1}{n!}| \le \frac{2}{(N + 1)!}
\end{equation}
Diese sagt aus, dass man nach $N + 1$ Iterationen von $\sum_{n = 0}^N \frac{1}{n!}$ nur noch höchstens $\frac{2}{(N + 1)!}$ von $e = exp(1)$ entfernt ist.
Um $e$ auf drei Nachkommastellen genau zu bestimmen, müssen wir uns $e$ bis auf weniger als $\frac{1}{1000}$ annähern.
\begin{equation}
 \frac{2}{(6 + 1)!} = \frac{1}{2520}
\end{equation}
Es reichen also $7$ Iterationen:
\begin{equation}
 \sum_{n = 0}^6 \frac{1}{n!} = 1 + 1 + \frac{1}{2} + \frac{1}{6} + \frac{1}{24} + \frac{1}{120} + \frac{1}{720} = \frac{1957}{720}
\end{equation}
$e_6 = \frac{1957}{720} = 2,71805556$ ist also eine Näherung an $e$ bis auf drei Nachkommastellen.

\end{document}
