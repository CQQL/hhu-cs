\documentclass[a4paper,10pt]{article}
\usepackage[utf8]{inputenc}
\usepackage[german]{babel}
\usepackage{amsmath}
\usepackage{amssymb}
\usepackage{amsthm}

\title{Ana1, Übungsblatt 6}
\author{Marten Lienen (2126759), Gruppe 1}

\begin{document}

\maketitle

\section*{Übung 23}

\begin{proof}
 Sei $\varepsilon > 0$, $x_0 \in \mathbb{R}$.
 Wir suchen $\delta$, sodass $|x^3 - x_0^3| < \varepsilon$ für alle $x$ mit $|x - x_0| < \delta$.
 Wähle $\delta = \xi$.
 \begin{equation}
  |x^3 - x_0^3| < \varepsilon
 \end{equation}

\end{proof}

\section*{Übung 24}

\section*{Übung 25}

In der Vorlesung wurde folgende Formel gezeigt
\begin{equation}
 |e - \sum_{n = 0}^N \frac{1}{n!}| \le \frac{2}{(N + 1)!}
\end{equation}
Diese sagt aus, dass man nach $N + 1$ Iterationen von $\sum_{n = 0}^N \frac{1}{n!}$ nur noch höchstens $\frac{2}{(N + 1)!}$ von $e = exp(1)$ entfernt ist.
Um $e$ auf drei Nachkommastellen genau zu bestimmen, müssen wir uns $e$ bis auf weniger als $\frac{1}{1000}$ annähern.
\begin{equation}
 \frac{2}{(6 + 1)!} = \frac{1}{2520}
\end{equation}
Es reichen also $7$ Iterationen:
\begin{equation}
 \sum_{n = 0}^6 \frac{1}{n!} = 1 + 1 + \frac{1}{2} + \frac{1}{6} + \frac{1}{24} + \frac{1}{120} + \frac{1}{720} = \frac{1957}{720}
\end{equation}
$e_6 = \frac{1957}{720}$ ist also eine Näherung an $e$ bis auf drei Nachkommastellen.

\end{document}
