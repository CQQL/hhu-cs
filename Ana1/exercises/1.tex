\documentclass[a4paper,10pt]{article}
\usepackage[utf8]{inputenc}
\usepackage{amsmath}
\usepackage{amssymb}
\usepackage{amsthm}
\usepackage[german]{babel}

\title{Übungsblatt 1}
\author{Marten Lienen (2126759)}

\begin{document}

\newcommand{\eqstep}[1]{\overset{\text{(#1)}}{\Longleftrightarrow}}

\newtheorem*{claim}{Behauptung}

\maketitle

\section*{Aufgabe 1}

\subsection*{a}

\begin{equation}
 4 - x < 3 - 2x \eqstep{9} 4 + x < 3 \eqstep{9} x < -1
\end{equation}

$-1$ ist das Supremum.

\subsection*{b}

\begin{equation}
 5 - x^2 < 8 \eqstep{9} -x^2 < 3 \eqstep{6} x^2 > -3 \eqstep{4} x
\end{equation}

Die Menge ist unbeschränkt.

\subsection*{c}

\begin{equation}
 5 - x^2 < -4 \eqstep{9} -x^2 < -9 \eqstep{6} x^2 > 9 \eqstep{12} x > 3
\end{equation}

$3$ ist das Infimum.

\subsection*{d}

\begin{equation}
 (x - 1)(x - 3) > 0 \eqstep{7} x > 1 \land x > 3 \Leftrightarrow x > 3
\end{equation}

$3$ ist das Infimum.

\subsection*{e}

\begin{equation}
 (x - 4)(x + 5)(x - 3) > 0 \eqstep{7} x > 4 \land (x + 5)(x - 3) > 0 \eqstep{7} x > 4 \land x > -5 \land x > 3 \Leftrightarrow x > 4
\end{equation}

$4$ ist das Infimum.

\subsection*{f}

\begin{equation}
 x^2 - 2x - 3 \le 0 \Leftrightarrow (x - 1)^2 - 4 \le 0 \eqstep{9} (x - 1)^2 \le 4 \Leftrightarrow |x - 1| \le 2
\end{equation}

\subsubsection*{Fall $x \ge 1$}

\begin{equation}
 |x - 1| \le 2 \Leftrightarrow x - 1 \le 2 \eqstep{9} x \le 3
\end{equation}

\subsubsection*{Fall $x < 1$}

\begin{equation}
 |x - 1| \le 2 \Leftrightarrow -(x - 1) \le 2 \Leftrightarrow -x + 1 \le 2 \eqstep{9} -x \le 1 \eqstep{6} x \ge -1
\end{equation}

Daraus folgt
\begin{equation}
 -1 \le x \le 3
\end{equation}

\begin{itemize}
 \item $-1$ ist Infimum.
 \item $3$ ist Supremum.
\end{itemize}

\section*{Aufgabe 2}

\begin{description}
 \item[a] $\{11, -5\}$
 \item[b] $]-5, 11[$
 \item[c] $[-6, -2]$
 \item[d] $\{ x \in \mathbb{R} | x \le 0\}$
 \item[e] $\{ x \in \mathbb{R} | x \ge 3 \lor x \le \frac{1}{3}\}$
 \item[f] $\{1, -1\}$
\end{description}

\section*{Aufgabe 3}

\subsection*{a}

\begin{proof}
 Es muss gelten $c|x - 1| \le c'|x - 1|$, damit alle $x \in M_c$ auch Element von $M_{c'}$ sind.
 Wenn $x = 1$, dann ist $0 \le 0$.
 Wenn $x \ne 1$, dann ist $|x - 1| > 0$ und so vereinfacht sich die Bedingung zu $c \le c'$ durch dividieren durch $|x - 1|$, was unsere Bedingung war und somit wahr ist.
\end{proof}

\subsection*{b}

Für $c < 0$ und $x \ne 1$ ist $M_c$ die leere Menge, weil dann $|x - 1| > 0$ und deshalb $c|x - 1| < 0$.
Und $|x| < 0$ ist nach Definition immer falsch.
Wenn $x = 1$, ist die Bedingung $|2| < c|0|$, was auch immer falsch ist.

Für $c = 0$ enthält $M_c$ das Element $0$, das dann nach (a) auch in allen $M_{c'}$ mit $c' > 0$ enthalten ist.

\begin{equation*}
 c < 0 \Rightarrow M_c = \emptyset
\end{equation*}

\subsection*{c}

\begin{equation}
 c \le 1
\end{equation}

\subsection*{d}

\begin{equation}
 c < 1
\end{equation}

\subsection*{e}

\begin{equation}
 c < 1
\end{equation}

\end{document}
