\documentclass[a4paper,10pt]{article}
\usepackage[utf8]{inputenc}
\usepackage{amsmath}
\usepackage{amssymb}
\usepackage{amsthm}
\usepackage[german]{babel}

\title{Übungsblatt 1}
\author{Marten Lienen (2126759)}

\begin{document}

\newcommand{\eqstep}[1]{\overset{\text{(#1)}}{\Longleftrightarrow}}

\newtheorem*{claim}{Behauptung}

\maketitle

\section*{Aufgabe 1}

\subsection*{a}

\begin{equation}
 4 - x < 3 - 2x \eqstep{9} 4 + x < 3 \eqstep{9} x < -1
\end{equation}

$-1$ ist das Supremum.

\subsection*{b}

\begin{equation}
 5 - x^2 < 8 \eqstep{9} -x^2 < 3 \eqstep{6} x^2 > -3 \eqstep{4} x
\end{equation}

Die Menge ist unbeschränkt.

\subsection*{c}

\begin{equation}
 5 - x^2 < -4 \eqstep{9} -x^2 < -9 \eqstep{6} x^2 > 9 \eqstep{12} x > 3
\end{equation}

$3$ ist das Infimum.

\subsection*{d}

\begin{equation}
 (x - 1)(x - 3) > 0 \eqstep{7} x > 1 \land x > 3 \Leftrightarrow x > 3
\end{equation}

$3$ ist das Infimum.

\subsection*{e}

\begin{equation}
 (x - 4)(x + 5)(x - 3) > 0 \eqstep{7} x > 4 \land (x + 5)(x - 3) > 0 \eqstep{7} x > 4 \land x > -5 \land x > 3 \Leftrightarrow x > 4
\end{equation}

$4$ ist das Infimum.

\subsection*{f}

\begin{equation}
 x^2 - 2x - 3 \le 0 \Leftrightarrow (x - 1)^2 - 4 \le 0 \eqstep{9} (x - 1)^2 \le 4 \Leftrightarrow |x - 1| \le 2
\end{equation}

\subsubsection*{Fall $x \ge 1$}

\begin{equation}
 |x - 1| \le 2 \Leftrightarrow x - 1 \le 2 \eqstep{9} x \le 3
\end{equation}

\subsubsection*{Fall $x < 1$}

\begin{equation}
 |x - 1| \le 2 \Leftrightarrow -(x - 1) \le 2 \Leftrightarrow -x + 1 \le 2 \eqstep{9} -x \le 1 \eqstep{6} x \ge -1
\end{equation}

Daraus folgt
\begin{equation}
 -1 \le x \le 3
\end{equation}

\begin{itemize}
 \item $-1$ ist Infimum.
 \item $3$ ist Supremum.
\end{itemize}

\section*{Aufgabe 2}

\begin{description}
 \item[a] $\{11, -5\}$
 \item[b] $]-5, 11[$
 \item[c] $[-6, -2]$
 \item[d] $\{ x \in \mathbb{R} | x \le 0\}$
 \item[e] $\{ x \in \mathbb{R} | x \ge 3 \lor x \le \frac{1}{3}\}$
 \item[f] $\{1, -1\}$
\end{description}

\section*{Aufgabe 3}

\subsection*{a}

\begin{claim}
 \begin{align*}
  & M_c := \{ x \in \mathbb{R} | |x + 1| \le c|x - 1| \}\\
  & c, c' \in \mathbb{R} \quad (c \le c' \quad (\forall x \in M_c \quad (x \in M_{c'})))
 \end{align*}
\end{claim}

\begin{proof}
 Man forme die Bedingung der Menge $M_c$ um
 \begin{align*}
  \begin{cases}
   x \ge \frac{c + 1}{c - 1} &\text{wenn } x \ge 1\\
   x \le \frac{c - 1}{c + 1} &\text{wenn } -1 \le x < 1\\
   x \le \frac{c + 1}{c - 1} &\text{wenn } x < -1
  \end{cases}
 \end{align*}
\end{proof}

\end{document}
