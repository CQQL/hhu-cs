\documentclass[a4paper,10pt]{article}
\usepackage[utf8]{inputenc}
\usepackage{amsmath}

\title{Ana1, Übungsblatt 4}
\author{Marten Lienen (2126759), Gruppe 8}

\begin{document}

\maketitle

\section*{13}

\section*{14}

\subsection*{a}

Diese Summer konvergiert, weil sie streng monoton fällt und nach unten durch $0$ beschränkt ist.

\subsection*{b}



\subsection*{c}

\section*{15}

\section*{16}

\subsection*{a}

\begin{equation}
 1 - \frac{1}{3} + \frac{1}{5} - \frac{1}{7} + \dots = \sum_{n = 1}^\infty (-1)^{n + 1}\frac{1}{2n - 1}
\end{equation}

Ich betrachte die Folge $a_n := \frac{1}{2n - 1}$.
Sei $\varepsilon > 0$.
Wähle $N \in \mathbb{N}$ mit .
Für alle $n \ge N$ gilt
\begin{align}
 |a_n| = a_n = < \varepsilon
\end{align}

Da $(a_n)$ eine Nullfolge ist, ist die Reihe konvergent nach dem Konvergenzkriterium von Leibniz.

\subsection*{b}

\section*{17}

\subsection*{a}

Sei $0 < q < 1$.
\begin{align}
 \frac{(n + 1)x^{n + 1}}{nx^n} = \frac{(n + 1)x}{n} = \frac{nx + x}{n} = x + \frac{x}{n}
\end{align}

\subsection*{b}

\end{document}
