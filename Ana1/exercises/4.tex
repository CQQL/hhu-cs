\documentclass[a4paper,10pt]{article}
\usepackage[utf8]{inputenc}
\usepackage{amsmath}
\usepackage{amssymb}

\title{Ana1, Übungsblatt 4}
\author{Marten Lienen (2126759), Gruppe 8}

\begin{document}

\maketitle

\section*{13}

Ich bezeichne die Grenzwerte mit $a_n \rightarrow a$ und $c_n \rightarrow c$.
Da $(a_n)$ und $(c_n)$ beide denselben Grenzwert haben, gilt $a = c$.
Da $(c_n)$ konvergiert, gibt es für jedes $\varepsilon > 0$ ein $N \in \mathbb{N}$, so dass für alle $n \ge N$
\begin{equation}
 |a_n - a| < \varepsilon \land |c_n - a| < \varepsilon
\end{equation}
\begin{align*}
 & a_n \le b_n \le c_n\\
 \Leftrightarrow & a_n - a \le b_n - a \le c_n - a\\
 \Leftrightarrow & -\varepsilon < a_n - a  \le b_n - a \le c_n - a < \varepsilon\\
 \Rightarrow & |b_n - a| < \varepsilon
\end{align*}
$(b_n)$ konvergiert folglich mit dem Grenzwert $a$.

\section*{14}

\subsection*{a}

Sei $s_k = \sum_{n = 1}^k \frac{1}{\sqrt{n}}$.
\begin{equation}
 s_k = \sum_{n = 1}^k \frac{1}{\sqrt{n}} \ge \sum_{n = 1}^k \frac{1}{n}
\end{equation}
Da es immer größer als die harmonische Reihe ist, divergiert es.

\subsection*{b}

Sei $s_k = \sum_{n = 2}^k \frac{1}{n^2 - 1} = \sum_{n = 1}^{k - 1} \frac{1}{(n + 1)^2 - 1} = \sum_{n = 1}^{k - 1} \frac{1}{n^2 + 2n}$.
\begin{equation}
 |s_k| = |\sum_{n = 1}^{k - 1} \frac{1}{n^2 + 2n}| < \sum_{n = 1}^{k - 1} \frac{1}{n^2}
\end{equation}
Da es immer kleiner als $\sum_{n = 1}^{k - 1} \frac{1}{n^2}$ ist, konvergiert es nach dem Majorantenkriterium.

\subsection*{c}

Sei $s_k = \sum_{n = 1}^k \frac{n}{n^2 + 1} = \sum_{n = 1}^k (\frac{1}{n} - \frac{1}{n^3 + n}) = \sum_{n = 1}^k \frac{1}{n} - \sum_{n = 1}^k \frac{1}{n^3 + n}$.
Der Minuent ist die harmonische Reihe, die divergiert, und der Subtrahent konvergiert mit der Majorante $\sum_{n = 1}^k \frac{1}{n^2}$.
Die Reihe als Differenz divergiert folglich.

\section*{15}

Die Folge $(a_n)$ lässt sich in zwei Folgen zerlegen:
\begin{align}
 & a_{2n} = \frac{1}{4^{2n}} = (\frac{1}{16})^n\\
 & a_{2n - 1} = \frac{1}{2^{2n - 1}} = (\frac{1}{2})^{2n - 1} = \frac{2}{4^n}
\end{align}
Als Reihe konvergieren beide Folgen, da sie Abwandlungen der geometrischen Reihe sind.
Da beide Reihen konvergieren, konvergiert auch ihre Summe, die Reihe $\sum a_n$.



\section*{16}

\subsection*{a}

\begin{equation}
 1 - \frac{1}{3} + \frac{1}{5} - \frac{1}{7} + \dots = \sum_{n = 1}^\infty (-1)^{n + 1}\frac{1}{2n - 1}
\end{equation}

Ich betrachte die Folge $a_n := \frac{1}{2n - 1}$.
Da $2n - 1 \ge n$ für alle $n \in \mathbb{N}$, ist $\frac{1}{n}$ eine konvergente Abschätzung nach oben von $\frac{1}{2n - 1}$.

Da $(a_n)$ eine Nullfolge ist, ist die Reihe konvergent nach dem Konvergenzkriterium von Leibniz.

\begin{equation}
 \sum_{n = 1}^\infty |(-1)^{n + 1}\frac{1}{2n - 1}| = \sum_{n = 1}^\infty \frac{1}{2n - 1} = \sum_{n = 1}^\infty \frac{1}{2} * \frac{1}{n} + \frac{1}{4n^2 - 2n}
\end{equation}
Die Reihe ist nicht absolut konvergent, weil sie die harmonische Reihe als Summand enthält, die nicht konvergiert.

\subsection*{b}

Die Reihe konvergiert, weil $\sum a_{2n}$ konvergiert nach dem Leibnizkriterium und $\sum a_{2n - 1}$ konvergiert, weil es die geometrische Reihe für $x = \frac{1}{2}$ ist.

Sie ist jedoch nicht absolut konvergent, weil $\sum a_{2n}$ nicht absolut konvergent ist, da $\sum |a_{2n}|$ die harmonische Reihe ist. 

\section*{17}

\subsection*{a}

Sei $0 < q < 1$.
\begin{align}
 |\frac{(n + 1)x^{n + 1}}{nx^n}| = |\frac{(n + 1)x}{n}| = |\frac{nx + x}{n}| = |x + \frac{x}{n}| = |x(1 + \frac{1}{n})|
\end{align}
Da $x$ gegen $x$ konvergiert und $(1 + \frac{1}{n})$ gegen $1$, wissen wir nach den Rechenregeln für Folgen, dass $x(1 + \frac{1}{n}) \rightarrow 1x = x$.
Wenn $-1 < x < 1$, können wir $q$ wählen mit $q = 1 - \frac{1 - |x|}{2}$, sodass $|(1 + \frac{1}{n})x| < q < 1$ für fast alle $n \in \mathbb{N}$.

\subsection*{b}

\begin{align}
 |\frac{\frac{1}{n + 1}x^{n + 1}}{\frac{1}{n}x^n}| = |\frac{n}{n + 1}x| = |\frac{1}{1 + \frac{1}{n}}x|
\end{align}
Da $x$ gegen $x$ konvergiert und $\frac{1}{1 + \frac{1}{n}}$ gegen $1$, wissen wir nach den Rechenregeln für Folgen, dass $x(\frac{1}{1 + \frac{1}{n}}) \rightarrow 1x = x$.
Wenn $-1 < x < 1$, können wir $q$ wählen mit $q = 1 - \frac{1 - |x|}{2}$, sodass $|(\frac{1}{1 + \frac{1}{n}})x| < q < 1$ für fast alle $n \in \mathbb{N}$.

Unter Benutzung des Leibnizkriteriums ergibt sich aber zusätzlich, dass auch $x = -1$ eine gültige Lösung ist, da $\frac{1}{n}$ eine monoton fallende Nullfolge ist und $-1^n$ es in die Form des Leibnizkriteriums bringt.

\end{document}
