\documentclass[a4paper,10pt]{article}
\usepackage[utf8]{inputenc}
\usepackage{amsmath}
\usepackage{amssymb}
\usepackage{amsthm}
\usepackage[german]{babel}

\title{Folgen}
\author{Marten Lienen}

\newtheorem*{definition}{Definition}
\newtheorem*{notice}{Bemerkung}
\newtheorem*{lemma}{Lemma}
\newtheorem*{example}{Beispiel}

\begin{document}

\maketitle

\section*{Zusammenfassung}

Sei $(a_n)$ eine Folge und sei $x_0 \in \mathbb{R}$.
Wir sagen, dass $(a_n)$ gegen $x_0$ konvergiert, wenn gilt: Für jedes $\varepsilon \in \mathbb{R}_{> 0}$ gibt es ein $N \in \mathbb{N}$, sodass $|x_0 - a_n| < \varepsilon \forall n \ge N$.

Die Folge $(a_n)$ ist eigentlich eine Abbildung $a: \mathbb{N} \mapsto \mathbb{R}$; wir schreiben $a_n := a(n)$.

Etwas allgemeiner betrachtet man für ein $n_0 \in \mathbb{Z}$ Abbildungen $a: \{n_0, n_0 + 1, \dots\} \mapsto \mathbb{R}$ und spricht dann von der Folge $(a_n)_{n > n_0}$.

\subsection*{Satz 2}

Jede konvergente Folge ist beschränkt.

\begin{definition}
 Jede Folge mit $\lim_{n \rightarrow \infty} a_n = 0$ heißt Nullfolge.
\end{definition}

\begin{notice}
 Sei $(a_n)$ eine Folge.
 Genau dann konvergiert die Folge $(a_n)$ gegen $a$, wenn die Folge $(a_n - a)$ eine Nullfolge ist.
\end{notice}

\begin{lemma}
 Sind $(a_n)$ und $(b_n)$ Nullfolgen, so auch $(a_n + b_n)$ und $(a_n - b_n)$.
\end{lemma}

\begin{proof}
 Sei $\varepsilon > 0$.
 Weil $(a_n)$ Nullfolge ist, gibt es $N_1 \in \mathbb{N}$ mit $|a_n| < \frac{\varepsilon}{2} \forall n \ge N_1$.
 Weil $(b_n)$ Nullfolge ist, gibt es $N_2 \in \mathbb{N}$ mit $|b_n| < \frac{\varepsilon}{2} \forall n \ge N_2$.
 
 Sei $N := max \{N_1, N_2\}$. Ist $n \ge N$, so ist $n \ge N_1 \land n \ge N_2$, also $|a_n + b_n| \le |a_n| + |b_n| < \frac{\varepsilon}{2} + \frac{\varepsilon}{2} = \varepsilon$.
\end{proof}

\section*{Satz 3}

Ist $(a_n)$ eine Nullfolge und $(b_n)$ eine beschränkte Folge, so ist $(a_n \cdot b_n)$ eine Nullfolge.

\begin{proof}
 Weil $(b_n)$ beschränkt ist, gibt es ein $B \in \mathbb{R}_{> 0}$ mit der Eigenschaft mit $|b_n| \le B \forall n \in \mathbb{N}$.
 Sei $\varepsilon > 0$.
 Weil $(a_n)$ eine Nullfolge ist, gibt es ein $N \in \mathbb{N}$ mit $|a_n| < \frac{\varepsilon}{B} \forall n \in \ge N$.
 Ist $n \ge N$, so ist $|a_n \cdot b_n| = |a_n| \cdot |b_n| < \frac{\varepsilon}{B} \cdot B = \varepsilon$.
\end{proof}

\section*{Satz 4: Rechenregeln für Grenzwerte}

Seien $(a_n)$ und $(b_n)$ Folgen mit $a_n \mapsto a$ und $b_n \mapsto b$.

\begin{itemize}
 \item $a_n + b_n \mapsto a + b$ und $a_n - b_n \mapsto a - b$
 \item $a_nb_n \mapsto ab$
 \item Ist $b \ne 0$, so ist $b_n \ne 0$ für fast alle $n$ und $\frac{a_n}{b_n} \mapsto \frac{a}{b}$.
\end{itemize}

\begin{proof}
 1: $(a_n - a)$ und $(b_n - b)$ sind Nullfolgen.
 Nach dem Lemma ist $((a_n - a) \pm (b_n - b)) = (a_n \pm b_n) - (a \pm b)$ auch eine Nullfolge.
 Deswegen konvergiert $(a_n \pm b_n)$ gegen $a \pm b$.
 
 2: $a_nb_n - ab = a_n(b_n - b) + b(a_n - a) \rightarrow 0$ nach Satz 4 und Lemma.
 
 3: $|b| > 0$. Aus $b_n \rightarrow b$ folgt: Für fast alle $n$ ist $|b - b_n| < \frac{|b|}{2}$.
 Dann ist $|b| = |b - b_n + b_n| \le |b - b_n| + |b_n| < \frac{|b|}{2} + |b_n|$ für fast alle $n$.
 $\Rightarrow |b_n| > |b| - \frac{|b|}{2} = \frac{|b|}{2}$ für fast alle $n$.
 
 \begin{align*}
  |\frac{1}{b_n} - \frac{1}{b}| = |\frac{b - b_n}{b_nb} = |\frac{1}{b_n}| \cdot |\frac{1}{b}| \cdot |b - b_n|
 \end{align*}
 
 Für fast alle $n$ ist $|\frac{1}{b_n}| = \frac{1}{|b|} < \frac{2}{|b|}$.
 Deswegen ist die Folge $(|\frac{1}{b_n}|)$ beschränkt.
 Aus Satz 3 folgt, dass $(\frac{1}{b_n} - \frac{1}{b})$ eine Nullfolge ist, also $\frac{1}{b_n} \rightarrow \frac{1}{b}$.
 Aus Satz 2 folgt: $\frac{a_n}{b_n} = a_n \cdot \frac{1}{b_n} \rightarrow a \cdot \frac{1}{b} = \frac{a}{b}$.
\end{proof}

\begin{example}
 \begin{equation*}
  \lim_{n \rightarrow \infty} \frac{2n^2 - 3n + 6}{3n^2 - 8n + 1} = \lim_{n \rightarrow \infty} \frac{2 - \frac{3}{n} + \frac{6}{n^2}}{3 - \frac{8}{n} + \frac{1}{n^2}} = \frac{2}{3}
 \end{equation*}
\end{example}

\begin{example}
 \begin{equation*}
  \lim_{n \rightarrow \infty} \frac{1000n + 312}{n^2 + 1} = \lim_{n \rightarrow \infty} \frac{\frac{1000}{n} + \frac{312}{n^2}}{1 \frac{1}{n^2}} = 0
 \end{equation*}
\end{example}

\section*{Satz 5}

Seien $(a_n)$ und $(b_n)$ Folgen mit $a_n \rightarrow a$ und $b_n \rightarrow b$.
Ist $a_n \ge b_n \forall n \in \mathbb{N}$, so ist $a \ge b$.

\begin{proof}
 Es ist $a_n - b_n \ge 0 \forall n \in \mathbb{N}$ und $a_n - b_n \rightarrow a - b$.
 
 Deswegen ist $a - b \ge 0$.
\end{proof}

\begin{example}
 Für welche $a \in \mathbb{R}$ konvergiert die Folge $(a^n)$?
 Wir benutzen Satz 6.
\end{example}

\section*{Satz 6: Bernoullische Ungleichung}

Sei $x \in \mathbb{R}$ und $x > -1$.
Für alle $n \in \mathbb{N}$ gilt $(x + 1)^n \ge 1 + nx$.

\begin{proof}
 Wir benutzen vollständige Induktion nach $n$:
 
 Für $n = 1$ steht auf beiden Seiten $1 + x$.
 
 Sei $n \in \mathbb{N}$ und sei bereits gezeigt, dass $(1 + x)^n \ge 1 + nx$.
 
 \begin{align*}
  (1 + x)^{n + 1} = (1 + x)^n(1 + x) & \ge (1 + nx)(1 + x)
  & = 1 + x + nx + nx^2
  & = 1 + (n + 1)x + nx^2 \ge 1 + (n + 1)x
 \end{align*}
\end{proof}

\section*{Satz 7}

Für $|a| > 1$ divergiert die Folge $(a^n)$. Für $|a| < 1$ ist $(a^n)$ eine Nullfolge.

\begin{proof}
 Wir betrachten nur den Fall $a > 0$. (Ist $a < 0$, so betrachte den Fall $-a = (-1) * a$ und benutze z.B. Satz 4.2.). 
 
 1. Fall: Sei $a > 1$. Schreibe $a = 1 + x$ mit $x > 0$. Nach Satz 6 ist $a^n = (1 + x)^n \ge 1 + nx$.
 Für jedes $M \in \mathbb{R}_{> 0}$ gibt es ein $n \in \mathbb{N}$ mit $1 + nx \ge M$, also erst recht $a^n \ge M$.
 Deswegen ist die Folge $(a^n)$ unbeschränkt, also nach dem Satz 2 nicht konvergent.
 
 2. Fall: Sei $0 < a < 1$.
 Sei $b := \frac{1}{a}$. Dann ist $b > 1$. 
 Sei $\varepsilon > 0$.
 Nach dem 1. Fall gibt es ein $N \in \mathbb{N}$ mit $b^N > \frac{1}{\varepsilon}$.
 Dann ist auch $b^n \ge b^N > \frac{1}{\varepsilon}$ für $n \ge N$, also $a^n = \frac{1}{b^n} < \varepsilon$.
\end{proof}

\begin{definition}
 Eine Folge $(a_n)$ heißt monoton wachsend, falls $a_n \le a_{n + 1} \forall n \in \mathbb{N}$.
\end{definition}

\begin{definition}
 Eine Folge $(a_n)$ heißt streng monoton wachsend, falls $a_n < a_{n + 1} \forall n \in \mathbb{N}$.
\end{definition}

\begin{definition}
 Eine Folge $(a_n)$ heißt monoton fallend, falls $a_n \ge a_{n + 1} \forall n \in \mathbb{N}$.
\end{definition}

\begin{definition}
 Eine Folge $(a_n)$ heißt streng monoton fallend, falls $a_n > a_{n + 1} \forall n \in \mathbb{N}$.
\end{definition}

\section*{Satz 8}

Sei $(a_n)$ eine beschränkte und monoton wachsende Folge.
Dann ist ist $(a_n)$ konvergent und der $\lim_{n \rightarrow \infty} a_n = sup\{a_n \mid n \in \mathbb{N}\}$.

\end{document}
