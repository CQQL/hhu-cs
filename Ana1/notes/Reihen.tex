\documentclass[a4paper,10pt]{article}
\usepackage[utf8]{inputenc}
\usepackage{amsmath}
\usepackage{amssymb}
\usepackage{amsthm}
\usepackage[german]{babel}

\title{§3 Reihen}
\author{Marten Lienen}

\newtheorem{definition}{Definition}
\newtheorem{notice}{Bemerkung}
\newtheorem{lemma}{Lemma}
\newtheorem{example}{Beispiel}
\newtheorem{vorbemerkung}{Vorbemerkung zur Notation}
\newtheorem{satz}{Satz}

\begin{document}

\maketitle

\section*{Das Summenzeichen}

Sind $h, k \in \mathbb{Z}$ und $h \le k$ und sind $a_h, a_{h + 1}, \dots, a_k \in \mathbb{R}$, so schreiben wir
\begin{equation*}
 \sum_{n=h}^k a_n := 
\end{equation*}

Ist M eine endliche Menge und ist für jedes $n \in M$ ein $a_n \in \mathbb{R}$ gebeben, so schreiben wir $\sum_{n \in M} a_n$ für die Summe aller dieser $a_n$.
Dabei ist $M = \emptyset$ zugelassen.
Dann sei $\sum_{n \in M} a_n := 0$.

\begin{definition}
 Sei $(a_n)$ eine Folge reeller Zahlen.
 Wir setzen $s_k = \sum_{n = 1}^k a_n$ für $k \in \mathbb{N}$.
 Wenn die Folge der $(s_k)_k$ konvergiert, so sagt man: Die Reihe $\sum_{n = 1}^\infty a_n$ konvergiert.
 Wenn dies der Fall ist, so schreibt man $\sum_{n = 1}^\infty a_n = \lim_{k \rightarrow \infty} s_k$.
 Wenn die Folge $(s_k)$ divergiert, so sagt man: Die Reihe $\sum_{n = 1}^\infty a_n$ divergiert.
 Man nennt die Zahlen $s_k$ die Partialsummen der Reihe $\sum_{n = 1}^\infty a_n$.
 
 Allgemeiner: Ist $n_0 \in \mathbb{Z}$ und geht man von der Folge $(a_n)_{n \ge n_0}$ aus, so erhält man die Reihe $\sum_{n = n_0}^\infty a_n$.
\end{definition}

\begin{satz}
 Wenn die Reihe $\sum_{n = 1}^\infty a_n$ konvergiert, so ist $(a_n)$ eine Nullfolge.
\end{satz}

\begin{proof}
 $s: \sum_{n = 1}^\infty a_n$, $s_k = \sum_{n = 1}^k a_n$.
 
 Sei $\varepsilon > 0$.
 Es gibt ein $N \in \mathbb{N}$ mit $|s - s_k| < \frac{\varepsilon}{2}$ für alle $k \ge N$.
 Ist $k \ge N$, so ist $|s_{k + 1} - s_k| = |(s_{k + 1} - s) + (s - s_k)| \le |s_{k + 1} - s| + |s - s_k| < \frac{\varepsilon}{2} + \frac{\varepsilon}{2} = \varepsilon$.
 $s_{k + 1} - s_k = (a_1 + a_2 + \dots + a_k + a_{k + 1} - (a_n + a_2 + \dots + a_k) = a_{k + 1}$.
 Ist $n \ge N + 1$, so ist $|a_n| < \varepsilon$.
\end{proof}

\begin{example}
 Sei $x \in \mathbb{R}$.
 Die geometrische Reihe $\sum_{n = 0}^\infty x^n$ konvergiert genau dann, wenn $|x| < 1$.
\end{example}

\begin{proof}
 Ist $|x| \ge 1$, so ist $|x^n| \ge 1$ für alle $n \in \mathbb{N}$.
 Also ist $(x^n)_n$, keine Nullfolge und $\sum_{n = 0}^\infty x^n$ divergiert nach Satz 1.
 
 Ist $x \in \mathbb{R}$, $x \ne 1$, so ist $\sum_{n = 0}^\infty x^n = \frac{1 - x^{k + 1}}{1 - x}$.
 Denn $(1 + x + x^2 + \dots + x^k) * (1 - x) = (1 + x + x^2 + \dots + x^k) - (x + x^2 + \dots + x^k + x^{k + 1}) = 1 - x^{k + 1}$.
 
 Ist $|x| < 1$.
 Nach §2 ist $\lim_{k \rightarrow \infty} x^{k + 1} = 0$.
 Also ist $\lim_{k \rightarrow \infty} \frac{1 - x^{k + 1}}{1 - x} = \frac{1}{1 - x}$.
 Für $|x| < 1$, ist $\sum_{n = 0}^\infty x^n = \frac{1}{1 - x}$.
\end{proof}

\begin{example}
 Die harmonische Reihe $\sum_{n = 1}^\infty \frac{1}{n}$ divergiert.
\end{example}

\begin{proof}
 $1 + \frac{1}{2} + \frac{1}{3} + \dots + \frac{1}{k} \ge 1 + \frac{1}{2} + \frac{1}{4} + \frac{1}{4} + \frac{1}{8} + \frac{1}{8} + \frac{1}{8} + \frac{1}{8}$.
 Also ist die Folge der Partialsummen von $\sum \frac{1}{n}$ unbeschränkt, also divergent nach §2.2.
\end{proof}

\begin{example}
 $\sum_{n = 1}^\infty \frac{1}{n(n + 1)}$.
\end{example}

\begin{proof}
 \begin{align*}
  \frac{1}{n(n + 1)} = \frac{1}{n} - \frac{1}{n + 1} \Rightarrow \sum_{n = 1}^k \frac{1}{n(n + 1)} = (1 - \frac{1}{2} + (\frac{1}{2} - \frac{1}{3}) + \dots + (\frac{1}{k} - \frac{1}{k + 1}) = 1 - \frac{1}{k + 1} \rightarrow 1 \text{für k }\rightarrow \infty
 \end{align*}
\end{proof}

\begin{satz}
 (Konvergenzkriterium von Leibniz)
 Sei $(b_n)_{n \ge 0}$ eine monoton fallende Nullfolge.
 Dann konvergiert $\sum_{n = 0}^\infty (-1)^n b_n$.
\end{satz}

\begin{proof}
 $s_k = \sum_{n = 0}^k (-1)^n b_n$.
 
 \begin{align*}
  s_{2k + 2} - s_{2k} & = (-1)^{2k + 2} b_{2k + 2} + (-1)^{2k + 1} b_{2k + 1}\\
  & = b_{2k + 2} - b_{2k + 1} \le 0\\
  & \Rightarrow s_0 \ge s_2 \ge s_4 \ge s_6 \ge \dots
 \end{align*}
 
 \begin{align*}
  s_{2k + 3} - s_{2k + 1} & = (-1)^{2k + 3} b_{2k + 3} + (-1)^{2k + 2} b_{2k + 2}\\
  & = -b_{2k + 3} + b_{2k + 3} \ge 0\\
  & \Rightarrow s_1 \le s_3 \le s_5 \le s_7 \le \dots
 \end{align*}
 
 \begin{align*}
  s_{2k + 1} - s_{2k} & = (-1)^{2k + 1} b_{2k + 1}\\
  & = -b_{2k + 1} \le 0\\
  & \Rightarrow s_1 \le s_{2k + 1} \le s_{2k} \le s_0
 \end{align*}
 
 Deswegen bilden die $s_{2k}$ eine monton fallende, durch $s_1$ nach unten beschränkte Folge, konvergieren also gegen ein $S \in \mathbb{R}$ nach §2.8.
 Und die $s_{2k + 1}$ bilden eine monoton wachsende, nach oben beschränkte Folge, konvergieren also gegen ein $S' \in \mathbb{R}$.
 Noch zu zeigen $S = S'$.
 
 \begin{align*}
  S - S' = \lim_{k \rightarrow \infty} (s_{2k} - s_{2k + 1}) = \lim_{k \rightarrow \infty} b_{2k + 1} = 0
 \end{align*}
\end{proof}

\begin{example}
 Nach Satz 2 konvergiert die Reihe $\sum_{n = 1}^\infty (-1)^n \frac{1}{n} = -1 + \frac{1}{2} - \frac{1}{3} + \frac{1}{4} + \dots = -ln 2$.
 Das werden wir gegen Ende des Semesters beweisen.
\end{example}

\begin{definition}
 Eine Reihe $\sum_{n = 1}^\infty a_n$ heißt absolut konvergent, wenn $\sum_{n = 1}^\infty |a_n|$ konvergiert.
\end{definition}

\begin{satz}
 Jede absolut konvergente Reihe ist konvergent.
\end{satz}

\begin{proof}
 Sei $\sum_{n = 1}^\infty |a_n|$ konvergent.
 Sei $s_k = \sum_{n = 1}^k a_n$ und $S_k = \sum_{n = 1}^k |a_n|$.
 Nach Vorraussetzung ist die Folge $(S_k)$ konvergent.
 Wir wollen die Konvergenz von $(s_k)$ mit dem Kriterium von Cauchy zeigen.
 Wir müssen also zeigen: Zu jedem $\varepsilon > 0$ gibt es ein $N \in \mathbb{N}$ mit $|s_n - s_m| < \varepsilon \forall n, m \ge N$.
 Dabei können wir annehmen, dass $n > m$.
 Dann ist $s_n - s_m = \sum_{k = m + 1}^n a_k$.
 $|s_n - s_m| \le |a_{m + 1}| + |a_{m + 2}| + \dots + |a_n| = S_n - S_m$.
 Nach der leichten Richtung von Cauchy gibt es ein $N \in \mathbb{N}$ mit $S_n - S_m < \varepsilon$ für alle $n, m \ge 0$.
 Deswegen ist $|s_n - s_m| < \varepsilon$ für alle $n, m \ge N$.
\end{proof}

\begin{notice}
 Die Folge der Partialsummen von $\sum_{n = 1}^\infty |a_n|$ ist monoton wachsend.
 Das heißt, dass $\sum_{n = 1}^\infty a_n$ konvergiert genau dann absolut, wenn die Folge der Partialsummen von $\sum_{n = 1}^\infty |a_n|$ beschränkt ist.
\end{notice}

\begin{notice}
 Ist $s_k = \sum_{n = 1}^k a_n$ und $S_k = \sum_{n = 1}^k |a_n|$, so ist $|s_k| \le S_k$, und wenn wir voraussetzen, dass $\sum a_n$ absolut konvergiert, so folgt: $|\sum_{n = 1}^\infty a_n| \le \sum_{n = 1}^\infty |a_n|$ (Nach §2.5).
 Das ist eine Verallgemeinerung der Dreiecksungleichung.
\end{notice}

\begin{satz}
 (Majorantenkriterium)
 
 Sei $(a_n)$ und $(b_n)$ Folgen mit $|a_n| \le b_n \forall n \in \mathbb{N}$.
 Wenn die Reihe $\sum_{n = 1}^\infty b_n$ konvergiert, so konvergiert die Reihe $\sum_{n = 1}^\infty a_n$ absolut.
 (Man nennt dann $\sum b_n$ eine konvergente Majorante von $\sum a_n$.
\end{satz}

\begin{proof}
 Sei $s_k = \sum_n^k |a_n|$ und $S_k = \sum_n^k b_n$.
 Dann ist $0 \le s_k \le S_k \forall k \in \mathbb{N}$.
 Wende nun Bemerkung 1 an.
\end{proof}

\begin{example}
 Die Reihe $sum_{n = 1}^\infty \frac{1}{n^2}$ konvergiert.
\end{example}

\begin{proof}
 Für $n \ge 2$ ist $\frac{1}{n^2} \le \frac{1}{n * (n - 1)}$ und die Reihe $\sum_{n = 2}^\infty \frac{1}{n * (n - 1)} = \frac{1}{1 * 2} + \frac{1}{2 * 3} + \dots$ konvergiert (Beispiel 3).
 Also konvergiert $\sum \frac{1}{n^2}$ nach dem Majorantenkriterium.
 (Wir werden sehen: $\sum_{n = 1}^\infty \frac{1}{n^2} = \frac{\pi^2}{6}$ (Satz von Euler).
\end{proof}

\begin{example}
 Sei $k \in \mathbb{N}$, $k \ge 2$.
 Dann konvergiert $\sum_{n = 1}^\infty \frac{1}{n^k}$.
\end{example}

\begin{proof}
 Es ist $\frac{1}{n^k} \le \frac{1}{n^2} \forall n \in \mathbb{N}$.
 Die Behauptung folgt also aus Beispiel 5 und dem Majorantenkriterium.
\end{proof}

\begin{satz}
 (Quotientenkriterium)
 
 Sei $(a_n)$ eine Folge.
 Es gebe ein $q \in \mathbb{R}$ mit $0 < q < 1$, sodass $\frac{|a_{n_ + 1}}{|a_n|} \le q$ für fast alle $n \in \mathbb{N}$.
 Dann konvergiert die Reihe $\sum_{n = 0}^\infty a_n$ absolut.
\end{satz}

\begin{proof}
 Wir können o.B.d.A. annehmen, dass $\frac{|a_{n + 1}|}{|a_n|} \le q \forall n \in \mathbb{N}$.
 Dann ist $|a_n| \le q * |a_{n - 1}| \le q * q * |a_{n - 2}| = q^2 * |a_{n - 2}| \le q^3 * |a_{n - 3}| \le \dots \le q^n * |a_0|$.
 Also ist $\sum_{n = 0}^\infty q^n * |a_0|$ eine konvergente Majorante nach Beispiel 1.
\end{proof}

\begin{notice}
 Für die Konvergenz von $\sum a_n$ ist nicht hinreichend, dass $\frac{|a_{n + 1}|}{|a_n|} < 1 \forall n \in \mathbb{N}$.
 Siehe die harmonische Reihe $\frac{1}{n}$.
\end{notice}

\begin{example}
 Ist $x \in \mathbb{R}$, so konvergiert $\sum_{n = 0}^\infty \frac{1}{n!} * x^n$ absolut.
\end{example}

\begin{proof}
 Sei $x \ne 0$.
 Setze $a_n := \frac{1}{n!} * x^n$.
 
 \begin{align*}
  \frac{|a_{n + 1}|}{|a_n|} = \frac{|x|^{n + 1} * n!}{(n + 1)! * |x|^n} = \frac{|x|}{n + 1} \le \frac{1}{2} \text{ für fast alle } n \in \mathbb{N}
 \end{align*}

 Also folgt die Behauptung aus dem Quotientenkriterium.
\end{proof}

\begin{definition}
 \begin{equation*}
  exp(x) := \sum_{n = 0}^\infty \frac{1}{n!} x^n
 \end{equation*}
\end{definition}

\begin{notice}
 Sind $\sum a_n$ und $\sum b_n$ konvergent, so ist $\sum (a_n + b_n)$ konvergent und $\sum (a_n + b_n) = \sum (a_n) + \sum (b_n)$.
\end{notice}

\begin{notice}
 Ist $\lambda \in \mathbb{R}$ und $\sum a_n$ konvergent, so ist $\sum (\lambda a_n)$ konvergent und $\sum (\lambda a_n) = \lambda (\sum)$
\end{notice}

\begin{example}
 Die Reihe $\frac{1}{2} - \frac{1}{2} + \frac{1}{3} - \frac{1}{3} + \dots$ ist konvergent und hat den Wert $0$.
 Die Umordnung $\frac{1}{2} + \frac{1}{3} + \frac{1}{4} - \frac{1}{2} + \frac{1}{5} + \frac{1}{6} - \frac{1}{3} + \frac{1}{7} + \frac{1}{8} -++ \dots$ konvergiert nach dem Leibnizkriterium und ihr Wert ist $> \frac{1}{2}$.
 Die Umordnung 
 \begin{equation*}
  \frac{1}{2} + \frac{1}{3} + \frac{1}{4} - \frac{1}{2} + \frac{1}{5} + \frac{1}{6} + \frac{1}{7} + \frac{1}{8} + \frac{1}{9} + \dots + \frac{1}{16} - \frac{1}{3} + \frac{1}{17} + \dots - \frac{1}{4} \dots
 \end{equation*}
 ist divergent!
\end{example}

\begin{definition}
 Seien $X$ und $Y$ Mengen und sei $f: X \rightarrow Y$ eine Abbildung.
 $f$ heißt surjektiv, wenn gilt $\forall y \in Y \exists x \in X (f(x) = y)$.
 $f$ heißt injektiv, wenn gilt $\forall x, x' \in X, x \ne x' \Rightarrow f(x) \ne f(x')$.
 $f$ heißt bijektiv, wenn $f$ surjektiv und injektiv ist, d.h. wenn gilt: Für jedes $y \in Y$ gibt es genau ein $x \in X$ mit $f(x) = y$.
\end{definition}

\begin{example}
 Definiere $\sigma: \mathbb{N} \rightarrow \mathbb{N}$ durch $\sigma(1) := 2, \sigma(2) := 1, \sigma(3) := 4, \sigma(4) := 3, \dots$, also $\sigma(2n) := 2n - 1$, $\sigma(2n - 1) := 2n \forall n \in \mathbb{N}$. Dann ist $\sigma$ bijektiv.
\end{example}

\begin{satz}
 (``Kommutativität'' absolut konvergenter Reihen)

 Sei $\sum_{n = 1}^\infty a_n$ eine absolut konvergente Reihe und sei $\sigma: \mathbb{N} \rightarrow \mathbb{N}$ eine bijektive Abbildung.
 Wir setzen $b_n := a_{\sigma(n)}$.
 Dann ist $\sum_{n = 1}^\infty b_n$ absolut konvergent und $\sum_{n = 1}^\infty b_n = \sum_{n = 1}^\infty a_n$.
\end{satz}

\begin{proof}
 Sei $s_n := \sum_{k = 1}^n a_k$ und $t_n = \sum_{k = 1}^n b_k$.
 Für $n \in \mathbb{N}$ sei $m(n) := max \{\sigma(1), \sigma(2), \dots, \sigma(n)\}$.
 Dann gilt
 \begin{align*}
  \sum_{k = 1}^n |b_k| = \sum_{k = 1}^n |a_{\sigma(k)}| \le \sum_{k = 1}^{m(n)} |a_k| \le \sum_{k = 1}^\infty |a_k|
 \end{align*}
 Also sind alle Partialsummen von $\sum |b_k|$ beschränkt, d.h. $\sum b_k$ konvergiert absolut.
 Wir müssen zeigen: $\lim_{n \rightarrow \infty} (t_n - s_n) = 0$.
 
 Sei $\varepsilon > 0$.
 Wir müssen zeigen: Es gibt ein $M$ mit $|t_m - s_m| < \varepsilon$ für $m \ge M$.
 Es gibt $N$ mit $|a_{N + 1}| + |a_{N + 2}| + \dots < \varepsilon$.
 Es gibt ein $M$ mit $\{1, 2, \dots, N\} \subset \{\sigma(1), \dots, \sigma(M)\}$.
 Sei $m \ge M$.
 Dann ist $\{1, \dots, M\} \subseteq \{\sigma(1), \dots, \sigma(m)\}$.
 Aus $t_m - s_m = \sum_{k = 1}^m a_{\sigma(k)} - \sum_{k = 1}^m a_k$ heben sich mindestens die Terme $a_1, \dots, a_N$ auf, also ist $t_m - s_m = \varepsilon_{N + 1}a_{N + 1} + \varepsilon_{N + 2}a_{N + 2} + \dots$ mit $\varepsilon_i \in \{1, 0, -1\}$.
 \begin{equation*}
  |t_m - s_m| \le |a_{N + 1}| + |a_{N + 2}| + |a_{N + 3}| + \dots < \varepsilon
 \end{equation*}
\end{proof}

\end{document}
