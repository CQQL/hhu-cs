\documentclass[a4paper,10pt]{article}
\usepackage[utf8]{inputenc}
\usepackage{amsmath}
\usepackage{amssymb}
\usepackage{amsthm}
\usepackage[german]{babel}

\title{§4 Stetige Funktionen}
\author{Marten Lienen}

\newtheorem{definition}{Definition}
\newtheorem{notice}{Bemerkung}
\newtheorem{lemma}{Lemma}
\newtheorem{example}{Beispiel}
\newtheorem{vorbemerkung}{Vorbemerkung zur Notation}
\newtheorem{satz}{Satz}
\newtheorem{bezeichnung}{Bezeichnung}

\begin{document}

\maketitle

\section{Allgemeine Vorbemerkungen zu Abbildungen}

\begin{itemize}
 \item 
 \begin{itemize}
  \item Sind $X. Y, Z$ drei Mengen und sind $f: X \rightarrow Y$ und $g: Y \rightarrow Z$ Abbildungen, so erhält man eine Abbildung
  \begin{equation}
   g \circ f: X \rightarrow Z
  \end{equation}
  durch $(g \circ f)(x) := g(f(x)) \forall x \in X$.
 
  \item Ist $X$ eine Menge, so erhält man eine Abbildung $id_X: X \rightarrow X$ durch
  \begin{equation}
   id_X(x) = x \forall x \in X
  \end{equation}
  Ist klar, um welches $X$ es sich handelt, so schreibt man $id := id_X$.
 
  \item Ist $f: X \rightarrow Y$ eine bijektive Abbildung, so gibt es für jedes $y \in Y$ genau ein Element $x \in X$ mit $f(x) = y$.
  Man schreibt $x := f^{-1}(y)$ und erhält so eine Abbildung $f^{-1}: Y \rightarrow X$, die Umkehrabbildung von $f$.
  $f1{-1}$ ist ebenfalls bijektiv und es gilt:
  \begin{itemize}
   \item $(f^{-1})^{-1} = f$
   \item $f^{-1} \circ f = id_X$
   \item $f \circ f^{-1} = id_Y$
  \end{itemize}
 \end{itemize}

 \item Seien $X, Y$ Mengen, $f: X \rightarrow Y$ eine Abbildung.
 \begin{itemize}
  \item Ist $A \subseteq X$, so sei $f(A) := \{f(a) \mid a \in A\} = \{y \in Y \mid \exists a \in A (f(a) = y)\}$.
  
  \item Ist $B \subseteq Y$, so sei $f^{-1}(B) := \{x \in X \mid f(x) \in B \}$. Hierbei muss $f$ nicht bijektiv sein.
  
  \item Sind $B, C \subset Y$, so ist
  \begin{align}
   f^{-1}(B \cup C) = f^{-1}(B) \cup f^{-1}(C)\\
   f^{-1}(B \cap C) = f^{-1}(B) \cap f^{-1}(C)
  \end{align}
  
  \item Sind $A, D \subseteq X$, so ist 
  \begin{align}
   f(A \cup D) = f(A) \cup f(D)\\
   f(A \cap D) \subseteq f(A) \cap f(D)
  \end{align}
  
  \item Ist $B \subseteq Y$, so ist $f(f^{-1}(B)) \subseteq B$.
  Wenn $f$ surjektiv ist, gilt die Gleichheit $=$.
  
  \item Ist $A \subseteq X$, so ist $f^{-1}(f(A)) \supseteq A$.
  Ist $f$ injektiv, so gilt die Gleichheit $=$.
  
  \item Ist $y \in Y$, so schreibt man
  \begin{align}
   f^{-1}(y) = f^{-1}(\{y\}) = \{x \in X \mid f(x) = y\}
  \end{align}
  Dazu muss $f$ nicht bijektiv sein.
 \end{itemize}
 
 \item Sind $X, Y$ Mengen, so sei $X \times Y := \{(x, y) \mid x \in X, y \in Y\}$.
 Man schreibt $X \times X = X^2$, allgemeiner $X^n = X \times X \times \dots \times X$.
 Ist $D \subseteq X$ und $f: D \rightarrow Y$ eine Abbildung, so heißt $Graph(f) := \{(x, f(x)) \mid x \in D\} \subseteq X \times Y$ der Graph von $f$.
\end{itemize}

\begin{example}
 Sei $X = Y = \mathbb{R}$ und $f(x) = x^2$.
 \begin{align}
  f([1, 2]) = [1, 4]\\
  f(]-1, 1[) = [0, 1[\\
  f^{-1}([1, 4]) = [1, 2] \cup [-2, -1]\\
  f^{-1}([-4, -1]) = \emptyset
 \end{align}
\end{example}

\begin{example}
 Ist $A = [1, 2]$ und $D = [-2, -1]$, so ist $f(A) \cap f(D) = [1, 4]$ und $f(A \cap D) = \emptyset$.
\end{example}

\begin{definition}
 Sei $f: D \rightarrow \mathbb{R}$ eine Funktion und $x_0 \in D$.
 Dann heißt $f$ stetig in $x_0$, wenn gilt:
 Zu jedem $\varepsilon > 0$ gibt es ein $\delta > 0$ mit folgender Eigenschaft:
 Ist $x \in D$, mit $|x - x_0| < \delta$, so ist $|f(x) - f(x_0)| < \varepsilon$.
 $f$ heißt stetig, wenn $f$ in jedem Punkt von $D$ stetig ist.
\end{definition}

\begin{notice}
 Schreibt man $x = x_0 + h$, so gilt: Genau dann ist $f$ stetig in $x_0$, wenn es zu jedem $\varepsilon > 0$ ein $\delta > 0$ gibt, sodass
 \begin{align}
  |f(x_0 + h) - f(x_0)| < \varepsilon
 \end{align}
 für alle $h \in \mathbb{R}$ mit $|h| < \varepsilon$ und $x_0 + h \in D$.
\end{notice}

\begin{example}
 Sei $c \in \mathbb{R}$ fest und sei $f: \mathbb{R} \rightarrow \mathbb{R}$ definiert durch $f(x) := c \forall x \in \mathbb{R}$.
 Dann ist $f$ stetig.
\end{example}

\begin{proof}
 Sei $x_0 \in \mathbb{R}$ und $\varepsilon > 0$.
 Wähle ein beliebiges $\delta > 0$.
 Ist $|x - x_0| < \delta$, so ist $|f(x) - f(x_0)| = |c - c| = |0| = 0 < \varepsilon$.
\end{proof}

\begin{example}
 Definiere $f: \mathbb{R} \rightarrow \mathbb{R}$ durch $f(x) := x \forall x \in \mathbb{R}$, also $f = id_\mathbb{R}$.
 Dann ist $f$ stetig.
\end{example}

\begin{proof}
 Sei $x_0 \in \mathbb{R}$ und $\varepsilon > 0$.
 Sei $\delta = \varepsilon$.
 Ist $|x - x_0| < \delta$, so ist $|f(x) - f(x_0)| = |x - x_0| < \delta = \varepsilon$.
\end{proof}

\begin{bezeichnung}
 Sei $f: D \rightarrow \mathbb{R}$ eine Funktion, $x_0 \in D$ und $a \in \mathbb{R}$.
 Wir schreiben
 \begin{equation}
  \lim_{x \rightarrow x_0} f(x) = a
 \end{equation}
 wenn gilt: Für jede Folge $(x_n)$ in $D$ mit $\lim_{n \rightarrow \infty} x_n = x_0$ ist $\lim_{n \rightarrow \infty} f(x_n) = a$.
\end{bezeichnung}

\begin{satz}
 Sei $f: D \rightarrow \mathbb{R}$ eine Funktion und $x_0 \in D$.
 Dann sind äquivalent:
 \begin{itemize}
  \item $f$ ist stetig in $x_0$
  \item $\lim_{x \rightarrow x_0} f(x) = f(x_0)$
 \end{itemize}
\end{satz}

\begin{proof}
 $1 \Rightarrow 2$: Sei $f$ stetig in $x_0$.
 Sei $(x_n)$ eine Folge in $D$ mit $x_n \rightarrow x_0$.
 Wir müssen zeigen: $f(x_n) \rightarrow f(x_0)$.
 Sei $\varepsilon > 0$.
 Es gibt ein $\delta > 0$, sodass $|f(x) - f(x_0)| < \varepsilon$ für alle $x \in D$ mit $|x - x_0| < \delta$.
 Es gibt ein $N \in \mathbb{N}$ mit $|x_n - x_0| < \delta$ für $n \ge N$.
 Ist $n \ge N$, so ist $|f(x_n) - f(x_0)| < \varepsilon$.
 Also gilt $f(x_n) \rightarrow f(x_0)$.
 
 $2 \Rightarrow 1$: Sei $\lim_{x \rightarrow x_0} f(x) = f(x_0)$.
 Angenommen $f$ sei nicht stetig in $x_0$.
 Dann gibt es ein $\varepsilon_0 > 0$, sodass gilt: Für jedes $n \in \mathbb{N}$ gibt es ein $x_n \in D$ mit $|x_n - x_0| < \frac{1}{n}$ und $|f(x_n) - f(x_0)| \ge \varepsilon_0$.
 Also ist $(x_n)$ eine Folge in $D$ mit $x_n \rightarrow x_0$ und $f(x_n) \nrightarrow f(x_0)$, Widerspruch.
\end{proof}

\begin{example}
 Definiere $f: \mathbb{R} \rightarrow \mathbb{R}$ durch
 \begin{equation}
  f := \begin{cases}
        0 \text{ für } x \le 0\\
        1 \text{ für } x > 0
       \end{cases}
 \end{equation}
 Dann ist $f$ nicht stetig in $0$.
\end{example}

\begin{proof}
 $(\frac{1}{n})$ ist eine Folge in $\mathbb{R}$ mit $\frac{1}{n} \rightarrow 0$.
 $f(\frac{1}{n}) = 1$ und $f(0) = 0$, also $(f(\frac{1}{n}))_n$ konvergiert nicht gegen $f_0$.
\end{proof}

\begin{bezeichnung}
 Sind $f, g: D \rightarrow \mathbb{R}$ Funktionen, so definiert man Funktionen $f + g$, $f - g$, $f * g$, $\frac{f}{g}: D \rightarrow \mathbb{R}$ durch
 \begin{align}
  (f + g)(x) := f(x) + g(x)\\
  (f - g)(x) := f(x) - g(x)\\
  (f * g)(x) := f(x) * g(x)\\
  (\frac{f}{g})(x) := \frac{f(x)}{g(x)} \text{ wenn $g(x) \ne 0 \forall x \in D$}
 \end{align}
\end{bezeichnung}

\begin{satz}
 Sind $f, g: D \rightarrow \mathbb{R}$ Funktionen, die beide stetig in $x_0$ sind, so sind auch $f + g$, $f - g$, $f * g$, $\frac{f}{g}$ stetig in $x_0$ (letzteres nur, wenn $g(x_0) \ne 0$).
\end{satz}

\begin{proof}
 Satz 1 und ``Rechenregeln für Folgen''.
\end{proof}

\begin{example}
 Ist $n \in \mathbb{N}$, sind $a_0, a_1, \dots, a_n \in \mathbb{R}$ fest und definiert man $f: \mathbb{R} \rightarrow \mathbb{R}$ durch $f(x) = a_0 + a_1x + \dots + a_nx^n$, so ist $f$ stetig nach Beispiel 1 und 2 und Satz 2.
 Ein solches $f$ heißt Polynom.
\end{example}


\end{document}
