\documentclass[a4paper,10pt]{article}
\usepackage[utf8]{inputenc}
\usepackage[german]{babel}
\usepackage{amsmath}
\usepackage[left=2cm,right=2cm,top=2cm,bottom=2cm]{geometry}

\title{LinA2, Übungsblatt 0}
\author{Marten Lienen (2126759)}

\begin{document}

\maketitle

\section*{Übung 1}

\subsection*{1}

\begin{equation}
Rg(A) = 3
\end{equation}

\subsection*{2}

\begin{align}
\chi_A & = det(XI_3 - A) = det
\begin{pmatrix}
 X & -2 & 2\\
 -1 & X + 1 & -2\\
 -1 & 3 & X - 4
\end{pmatrix}\\
& = X \cdot det(A_{1,1}) + 2 \cdot det(A_{1,2}) + 2 \cdot det(A_{1,3})\\
& = X \cdot ((X + 1) \cdot (X - 4) + 6) + 2 \cdot ((-X + 4) - 2) + 2 \cdot (-3 + (X + 1))\\
& = X \cdot (X + 1) \cdot (X - 4) + 6X + (-2X) + 8 - 4 - 6 + 2X + 2\\
& = X \cdot (X + 1) \cdot (X - 4) + 6X\\
\end{align}

\subsection*{3}

\subsection*{4}

\subsection*{5}

\section*{Übung 2}

\subsection*{1}

\begin{equation}
Rg(A) = n
\end{equation}
weil jeder Basisvektor auf genau einen Basisvektor abgebildet wird.

\subsection*{2}

\subsection*{3}

\subsection*{4}

\subsection*{5}

\section*{Übung 3}

\subsection*{1}

$f$ ist trigonalisierbar, weil $\chi_{f}$ vollständig in Linearfaktoren zerfällt.

\subsection*{2}

$f$ ist trigonalisierbar, weil $\chi_{f}$ vollständig in Linearfaktoren zerfällt.

\subsection*{3}

$f$ ist trigonalisierbar, weil $\chi_{f}$ vollständig in Linearfaktoren zerfällt.

\subsection*{4}

\section*{Übung 4}

\subsection*{1}

\subsection*{2}

\end{document}