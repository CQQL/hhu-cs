\documentclass[10pt,a4paper]{article}
\usepackage[utf8]{inputenc}
\usepackage[german]{babel}
\usepackage{amsmath}
\usepackage{amsfonts}
\usepackage{amssymb}
\usepackage{amsthm}
\usepackage[left=2cm,right=2cm,top=2cm,bottom=2cm]{geometry}

\DeclareMathOperator{\id}{id}

\begin{document}

\section*{Übung 1}

\subsection*{Teil 1}

\subsection*{Teil 2}

\begin{align*}
\sum_{i = 0}^{d} \prod_{j = 1}^{n} (i + j) & = \sum_{i = 0}^{d} \frac{(n + i)!}{i!}\\
& = \sum_{i = 0}^{d} \frac{(n + i)!}{i!n!}n!\\
& = \sum_{i = 0}^{d} n! \cdot \binom{n + i}{i}\\
& = n! \cdot \sum_{i = 0}^{d} \binom{n + i}{i}\\
& = n! \cdot \binom{n + d + 1}{d}\\
\end{align*}

\section*{Übung 2}

\subsection*{Teil 1}

\begin{proof}
$B$ ist linear im ersten Argument, weil alle $\varphi \in V^{\vee}$ linear sind.
Seien $\varphi, \psi \in V^{\vee}$, $v \in V$ und $\lambda, \mu \in K$.
\begin{equation}
B(v, \lambda \varphi + \mu \psi) = (\lambda \varphi + \mu \psi)(v) = \lambda \varphi(v) + \mu \psi(v) = \lambda B(v, \varphi) + \mu B(v, \psi)
\end{equation}
\end{proof}

\subsection*{Teil 2}

\begin{equation}
B(e_{k}, e_{j}^{\vee}) = \begin{cases}
1 & \textit{wenn $k = j$}\\
0 & \textit{sonst}
\end{cases}
\end{equation}
Also ist die Matrix von $B$
\begin{equation}
(b_{ij}) = \begin{cases}
1 & \textit{wenn $i = j$}\\
0 & \textit{sonst}
\end{cases}
= I_{d}
\end{equation}

\subsection*{Teil 3}

\begin{proof}
Angenommen $B$ sei ausgeartet.
Dann gibt es $v \ne 0 \in V$, sodass $B(v, \varphi) = 0$ für alle $\varphi \in V^{\vee}$.
Sei $\varphi = e_{1}^{\vee} + \dots + e_{d}^{\vee}$.
\begin{equation}
B(v, \varphi) = \varphi(v) = (e_{1}^{\vee} + \dots + e_{d}^{\vee})(v) = v_{1} + \dots + v_{d} \ne 0 \quad \textit{weil $v \ne 0$}
\end{equation}
Dies ist ein Widerspruch zu der Existenz von $v$ und $B$ ist nicht ausgeartet.
\end{proof}

\section*{Übung 3}

\subsection*{Teil 1}

\begin{proof}
\begin{equation}
M = (m_{i,j})_{i,j} \quad N = (n_{i,j})_{i,j}
\end{equation}
\begin{align*}
\sigma(MN) & = \sigma\left( \sum_{k = 1}^{n} m_{i,k}n_{k,j} \right)_{i,j}\\
& = \left( \sum_{k = 1}^{n} \sigma(m_{i,k}n_{k,j}) \right)_{i,j}\\
& = \left( \sum_{k = 1}^{n} \sigma(m_{i,k})\sigma(n_{k,j}) \right)_{i,j}\\
& = (\sigma(m_{i,j}))_{i,j} \cdot (\sigma(n_{i,j}))_{i,j}\\
& = \sigma((m_{i,j})_{i,j})\sigma((n_{i,j})_{i,j})\\
& = \sigma(M)\sigma(N)
\end{align*}
\end{proof}

\subsection*{Teil 2}

Die Identitätsfunktion und die folgende Funktion $\overline{x}$
\begin{equation}
\overline{a + b\sqrt{2}} \Rightarrow a - b\sqrt{2}
\end{equation}

\begin{proof}
Seien $a, b \in \mathbb{Q}\sqrt{2}$.
\begin{equation}
\id(1) = 1
\end{equation}
\begin{equation}
\id(a + b) = a + b = \id(a) + \id(b)
\end{equation}
\begin{equation}
\id(ab) = ab = \id(a)\id(b)
\end{equation}

Seien $a + b\sqrt{2}, c + d\sqrt{2} \in \mathbb{Q}\sqrt{2}$.
\begin{equation}
\overline{1} = 1
\end{equation}
\begin{equation}
\overline{a + b\sqrt{2} + c + d\sqrt{2}} = a - b\sqrt{2} + c - d\sqrt{2} = \overline{a + b\sqrt{2}} + \overline{c + d\sqrt{2}}
\end{equation}
\begin{equation*}
\overline{(a + b\sqrt{2}) \cdot (c + d\sqrt{2})} = \overline{ac + 2bd + (ad + bc)\sqrt{2}} = ac + 2bd - (ad + bc)\sqrt{2} = (a - b\sqrt{2}) \cdot (c - d\sqrt{2}) = \overline{a + b\sqrt{2}} \cdot \overline{c + d\sqrt{2}}
\end{equation*}
\end{proof}

\section*{Übung 4}

\subsection*{Teil 1}

\begin{proof}
Sei $(e_{1}, \dots, e_{n})$ eine Basis von $V$.
Sei $M = (m_{ij})$ die Matrix von $B$ in dieser Basis.
\begin{align*}
\textit{B ist symmetrisch} & \Leftrightarrow B(x, y) = B(y, x)\ \forall x, y\\
& \Leftrightarrow B(e_{i}, e_{j}) = B(e_{j}, e_{i})\ \forall i, j \in [1, n]\\
& \Leftrightarrow m_{ij} = m_{ji}\ \forall i, j \in [1, n]\\
& \Leftrightarrow \textit{M ist symmetrisch}
\end{align*}
\begin{align*}
\textit{B ist antisymmetrisch} & \Leftrightarrow B(x, y) = -B(y, x)\ \forall x, y\\
& \Leftrightarrow B(e_{i}, e_{j}) = -B(e_{j}, e_{i})\ \forall i, j \in [1, n]\\
& \Leftrightarrow m_{ij} = -m_{ji}\ \forall i, j \in [1, n]\\
& \Leftrightarrow \textit{M ist antisymmetrisch}
\end{align*}
\begin{align*}
\textit{B ist alternierend} & \Leftrightarrow B(x, x) = 0\ \forall x\\
& \Leftrightarrow B(e_{i}, e_{i}) = 0\ \forall i \in [1, n]\ \land\ B(e_{i}, e_{j}) = -B(e_{i}, e_{j})\ \forall i, j \in [1, n], i \ne j\\
& \Leftrightarrow m_{ii} = 0 \forall i \in [1, n]\ \land\ m_{ij} = -m_{ji}\ \forall i, j \in [1, n], i \ne j\\
& \Leftrightarrow \textit{M ist alternierend}
\end{align*}
\end{proof}

\subsection*{Teil 2}

\begin{proof}
Sei $(e_{1}, \dots, e_{n})$ eine Basis von $V$.
Sei $M = (m_{ij})$ die Matrix von $B$ in dieser Basis.
\begin{align*}
\textit{B ist hermitsch} & \Leftrightarrow B(x, y) = \sigma(B(y, x))\\
& \Leftrightarrow B(e_{i}, e_{j}) = \sigma(B(e_{j}, e_{i}))\\
& \Leftrightarrow m_{ij} = \sigma(m_{ji})\\
& \Leftrightarrow \textit{M ist hermitsch}
\end{align*}
\begin{align*}
\textit{B ist antihermitsch} & \Leftrightarrow B(x, y) = -\sigma(B(y, x))\\
& \Leftrightarrow B(e_{i}, e_{j}) = -\sigma(B(e_{j}, e_{i}))\\
& \Leftrightarrow m_{ij} = -\sigma(m_{ji})\\
& \Leftrightarrow \textit{M ist antihermitsch}
\end{align*}
\end{proof}

\section*{Übung 5}

\subsection*{Teil 1}

\begin{proof}
Seien $A = \begin{pmatrix}a&b\\c&d\end{pmatrix}$, $C = \begin{pmatrix}e&f\\g&h\end{pmatrix}$, $D = \begin{pmatrix}i&j\\k&l\end{pmatrix}$ und $\lambda, \mu \in \mathbb{R}$.
\begin{align*}
B(A, \lambda C + \mu D) & = 
\end{align*}
\end{proof}

\subsection*{Teil 2}

\begin{equation}
B(e_{k}, e_{j}) = \det(e_{k} + e_{j}) - \det(e_{k} - e_{j})
\end{equation}
Damit ergibt sich die Matrix
\begin{equation}
M_{\mathcal{B}}(B) = \begin{pmatrix}
0 & 0 & 0 & 2\\
0 & 0 & -2 & 0\\
0 & -2 & 0 & 0\\
2 & 0 & 0 & 0
\end{pmatrix}
\end{equation}

\subsection*{Teil 3}

$B$ ist nicht ausgeartet.
\begin{proof}
Wenn $B$ ausgeartet wäre, gäbe es ein $v^{T} = (a, b, c, d)$, sodass $B(v, w) = 0$ für alle $w = \begin{pmatrix}
e\\f\\g\\h
\end{pmatrix}$, wobei $v, w$ Matrizen in Koordinatenschreibweise sind.
\begin{equation}
B(v, w) = v^{T} \cdot M(B) \cdot w = (a, b, c, d) \cdot \begin{pmatrix}
2h\\-2g\\-2f\\2e
\end{pmatrix} = 2ha - 2gb - 2fc + 2ed = 2(ha - gb - fc + ed)
\end{equation}
Da $v \ne 0$ sind aber nicht $a = b = c = d = 0$.
Also ist $ha - gb - fc + ed$ mindestens für eine der Basismatrizen nicht $0$, sodass $B(v, e_{x}) \ne 0$, was der Ausgeartetheit von $B$ widerspricht.
\end{proof}

\end{document}