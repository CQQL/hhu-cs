\documentclass[10pt,a4paper]{article}
\usepackage[utf8]{inputenc}
\usepackage[german]{babel}
\usepackage{amsmath}
\usepackage{amsfonts}
\usepackage{amssymb}
\usepackage{amsthm}
\usepackage[left=2cm,right=2cm,top=2cm,bottom=2cm]{geometry}

\begin{document}

\section*{Übung 1}

\subsection*{1}

\subsection*{2}

\subsubsection*{a}

$P$ und $Q$ sind teilerfremd, weil $deg(P) > deg(Q)$ und $P / Q$ den Rest $1$ hat.
\begin{equation}
x^{3} = (x - 1)(x^{2} + x + 1) + 1
\end{equation}
\begin{equation}
R = 1
\end{equation}
\begin{equation}
S = -(x - 1)
\end{equation}

\subsubsection*{b}

$P$ und $Q$ sind teilerfremd, weil $deg(Q) > deg(P)$ und $Q / P$ den Rest $1$ hat.
\begin{equation}
(x - 1)^2 = (x - 2)x + 1
\end{equation}
\begin{equation}
S = 1
\end{equation}
\begin{equation}
R = -(x - 2)
\end{equation}

\subsubsection*{c}

$P$ und $Q$ sind teilerfremd, weil $deg(Q) > deg(P)$ und $Q / P$ den Rest $x + 1$ hat und $P / (x + 1)$ den Rest $1$.
\begin{equation}
R = x^4(x - 1) + x^2
\end{equation}
\begin{equation}
S = -(x - 1)
\end{equation}

\subsubsection*{d}

\section*{Übung 2}

\subsection*{1}

Ich schreibe Polynome hier als Vektoren ihrer Koeffizienten.
Als Basis von $V$ wähle ich die kanonische Basis.
Sei $v_{k}$ der Vektor der kanonischen Basis, der an der $k$-ten Stelle eine $1$ hat und sonst überall $0$ ist.
Dabei steht an der $k$-ten Stelle der Koeffizient von $x^{k}$.
Nach den normalen Ableitungsregeln bildet $f$ also wie folgt ab
\begin{equation}
f(v_{k}) = w
\end{equation}
wobei $w$ überall $0$ ist, und nur in der $k - 1$-ten Zeile den Wert $k$ hat.
Sei $A = Mat(f)$.
Dann steht in der $k$-ten Spalte der Wert $k$ in der $k - 1$-ten Zeile.
Da $k - 1 < k$, liegt dieser Wert immer oberhalb der Diagonalen und $A$ ist eine obere Dreiecksmatrix.
Dann ist auch $XI_{n + 1} - A$ eine obere Dreiecksmatrix und es ist
\begin{equation}
\chi_{f} = det(XI_{n + 1} - A) = X^{n + 1}
\end{equation}
So folgt, dass $f$ nur den Eigenwert $0$ hat.
Wenn $n = 0$, ist $f$ diagonalisierbar, weil es dann $dim V = 1$ Eigenwerte gibt.
Wenn $n > 0$, zerfällt $\mu_{f}$ zwar in Linearfaktoren, aber die Vielfachheit ist immer größer als $1$, weil $A \ne 0$ und $\mu_{f}$ deshalb nicht $X$ sein kann.
Also ist $f$ in diesem Fall nicht diagonalisierbar.

\subsection*{2}

Ich zeige
\begin{equation}
E_{k}(f, 0) = \langle v_{1}, \dots, v_{k} \rangle \quad \forall k \in [1, n + 1]
\end{equation}

\begin{proof}
Sei $k = 1$.
Wegen $f(v_1) = 0$ und $f(v_{j}) = j \cdot v_{j - 1} \ne 0$ für $j \in [2, n + 1]$, ist es wahr.

Sei $1 < k \le n + 1$ und $E_{k - 1}(f, 0) = \langle v_{1}, \dots, v_{k - 1} \rangle$.
\begin{align}
& f(v_{k}) = k \cdot v_{k - 1} \in E_{k - 1}(f, 0)\\
\Leftrightarrow & f^{k - 1}(f(v_{k})) = 0\\
\Leftrightarrow & f^{k}(v_{k}) = 0\\
\Leftrightarrow & v_{k} \in E_{k}(f, 0)\\
\Leftrightarrow & E_{k}(f, 0) = \langle v_{1}, \dots, v_{k} \rangle \quad \textit{weil $v_{k}$ linear unabhängig von $v_{1}, \dots, v_{k - 1}$ ist}
\end{align}
Für $j > k$ gilt außerdem
\begin{equation}
f(v_{j}) = j \cdot v_{j - 1} \notin E_{k - 1}(f, 0) \Rightarrow v_{j} \notin E_{k}(f, 0)
\end{equation}
\end{proof}

$E_{k}(f, 0)$ für $k \in [0, n + 1]$ sind also alle verallgemeinerten Unterräume, wobei $E_{0}(f, 0) = 0$ und $E_{n + 1}(f, 0) = V = H(f, 0)$ ist.

\section*{Übung 3}

\subsection*{1}

\begin{equation}
A = 
\begin{pmatrix}
0 & 1 & 0 & 0\\
0 & 0 & 0 & 0\\
0 & 0 & 0 & 1\\
0 & 0 & 0 & 0\\
\end{pmatrix}
\end{equation}
\begin{equation}
\chi_{f} = X^{4}
\end{equation}
\begin{equation}
A^{2} = 0 \Rightarrow \mu_{f} = X^{2}
\end{equation}
$A$ ist nicht diagonalisierbar, weil die Vielfachheit der Nullstelle $0$ im Minimalpolynom größer als $1$ ist.

\subsection*{2}

Sei $e_{1}, e_{2}, e_{3}, e_{4}$ die kanonische Basis von $V$.
\begin{equation}
E(A, 0) = Ker(A) = \langle e_{2}, e_{4} \rangle
\end{equation}
\begin{equation}
E_{2}(A, 0) = Ker(A^{2}) = Ker(0) = \langle e_{1}, e_{2}, e_{3}, e_{4} \rangle = H(A, 0)
\end{equation}
$E_{2}(A, 0)$ ist der Hauptraum von $A$, weil $E_{2}(A, 0) = V$.

\section*{Übung 4}

\subsection*{$A_{1}$}

\begin{equation}
\chi_{A_{1}} = x^{3}
\end{equation}
\begin{equation}
H(A_{1}, 0) = Ker \left( (0I_{3} - A_{1})^{3} \right) = \langle
\begin{pmatrix}
1\\0\\0
\end{pmatrix},
\begin{pmatrix}
0\\1\\0
\end{pmatrix},
\begin{pmatrix}
0\\0\\1
\end{pmatrix}
\rangle
\end{equation}

\subsection*{$A_{2}$}

\begin{equation}
\chi_{A_{2}} = x^{3}
\end{equation}
\begin{equation}
H(A_{2}, 0) = Ker \left( (0I_{3} - A_{2})^{3} \right) = \langle
\begin{pmatrix}
1\\0\\0
\end{pmatrix},
\begin{pmatrix}
0\\1\\0
\end{pmatrix},
\begin{pmatrix}
0\\0\\1
\end{pmatrix}
\rangle
\end{equation}

\subsection*{$A_{3}$}

\begin{equation}
\chi_{A_{3}} = x^{3}
\end{equation}
\begin{equation}
H(A_{3}, 0) = Ker \left( (0I_{3} - A_{3})^{3} \right) = \langle
\begin{pmatrix}
1\\0\\0
\end{pmatrix},
\begin{pmatrix}
0\\1\\0
\end{pmatrix},
\begin{pmatrix}
0\\0\\1
\end{pmatrix}
\rangle
\end{equation}

\subsection*{$A_{4}$}

\begin{equation}
\chi_{A_{4}} = (x - 1)(x - 2)^{2}
\end{equation}
\begin{equation}
H(A_{4}, 1) = Ker \left( (I_{3} - A_{4}) \right) = \langle
\begin{pmatrix}
-1\\-1\\1
\end{pmatrix}
\rangle
\end{equation}
\begin{equation}
H(A_{4}, 2) = Ker \left( (2I_{3} - A_{4})^{2} \right) = \langle
\begin{pmatrix}
1\\0\\1
\end{pmatrix},
\begin{pmatrix}
0\\2\\-\frac{3}{2}
\end{pmatrix}
\rangle
\end{equation}

\section*{Übung 5}

Alle Haupträume sind A-invariant.

\begin{equation}
\chi_{A} = X(X-2)(X-3)
\end{equation}
\begin{equation}
H(A,0) = \langle
\begin{pmatrix}
1\\1\\0
\end{pmatrix}
\rangle
\end{equation}
\begin{equation}
H(A,2) = \langle
\begin{pmatrix}
0\\0\\1
\end{pmatrix}
\rangle
\end{equation}
\begin{equation}
H(A,3) = \langle
\begin{pmatrix}
1\\-2\\0
\end{pmatrix}
\rangle
\end{equation}

Außerdem sind alle Vereinigungen der Haupträume A-invariant.
Natürlich sind auch $\emptyset$ und $\mathbb{R}^{3}$ A-invariant.

\end{document}