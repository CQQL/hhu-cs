\documentclass[10pt,a4paper]{article}
\usepackage[utf8]{inputenc}
\usepackage[german]{babel}
\usepackage{amsmath}
\usepackage{amsfonts}
\usepackage{amssymb}
\usepackage{amsthm}
\usepackage[left=2cm,right=2cm,top=2cm,bottom=2cm]{geometry}

\begin{document}

\section*{Übung 1}

\subsection*{Teil 1}

\subsection*{Teil 2}

\subsection*{Teil 3}

\subsection*{Teil 4}

\section*{Übung 2}

\begin{proof}
Seien $a, b \in A$, die sich mithilfe des EZS darstellen lassen als
\begin{equation}
a = \sum_{i \in I} \lambda_{a, i} a_{i}
\end{equation}
\begin{equation}
b = \sum_{i \in I} \lambda_{b, i} a_{i}
\end{equation}

\begin{align*}
f(ab) & = f \left( \left( \sum_{i \in I} \lambda_{a, i} a_{i} \right) \cdot \left( \sum_{i \in I} \lambda_{b, i} a_{i} \right) \right)\\
& = f \left( \sum_{i \in I} \sum_{j \in I} \lambda_{a, j} a_{j} \lambda_{b, i} a_{i} \right)\\
& = f \left( \sum_{i \in I} \sum_{j \in I} \lambda_{a, j} \lambda_{b, i} a_{j} a_{i} \right)\\
& = \sum_{i \in I} \sum_{j \in I} f \left( \lambda_{a, j} \lambda_{b, i} a_{j} a_{i} \right)\\
& = \sum_{i \in I} \sum_{j \in I} \lambda_{a, j} \lambda_{b, i} f \left( a_{j} a_{i} \right)\\
& = \sum_{i \in I} \sum_{j \in I} \lambda_{a, j} \lambda_{b, i} f(a_{j}) f(a_{i})\\
& = \sum_{i \in I} \sum_{j \in I} \lambda_{a, j} f(a_{j}) \lambda_{b, i} f(a_{i})\\
& = \left( \sum_{j \in I} \lambda_{a, j} f(a_{j}) \right) \cdot \left( \sum_{i \in I} \lambda_{b, i} f(a_{i}) \right)\\
& = f \left( \sum_{j \in I} \lambda_{a, j} a_{j} \right) \cdot f \left( \sum_{i \in I} \lambda_{b, i} a_{i} \right)\\
& = f(a) \cdot f(b)
\end{align*}
\end{proof}

\section*{Übung 3}

\subsection*{Teil 1}

\begin{proof}
Sei $i \in I$ und $a \in A$.
Weil alle $I_{j}$ Ideale sind, gilt
\begin{equation}
i \in I_{j}\ \forall j \in J \Rightarrow ai, ia \in I_{j}\ \forall j \in J \Rightarrow ai, ia \in I
\end{equation}
\end{proof}

\subsection*{Teil 2}

\begin{proof}
Seien $I_{j}$ alle Ideale, die $E$ enthalten.
Sei $I = \cap_{j} I_{j}$.
Dann ist $I$ nach Teil $1$ ein Ideal.
$I$ ist auch minimal, weil es in jedem Ideal $I_{j}$ enthalten ist.
\end{proof}

\subsection*{Teil 3}

\subsection*{Teil 4}

\subsection*{Teil 5}

\section*{Übung 4}

\subsection*{Teil 1}

Der einzige Basisvektor von $V^{\otimes n}$ ist $v_{1}$ $n$-mal mit sich selbst getensort,
weil eine Basis eines Tensorprodukts besteht aus den Tensoren, die alle möglichen Kombinationen der Basisvektoren der Faktorräume beinhalten.
Da $V$ nur einen Basisvektor hat, gibt es auch nur eine solche Kombination.
\begin{equation}
V^{\otimes n} = \langle v_{1} \otimes \dots \otimes v_{1} \rangle
\end{equation}
\begin{equation}
\dim_{K} V^{\otimes n} = 1
\end{equation}

\subsection*{Teil 2}

\begin{proof}
Sei $v \in T(V)$.
\begin{equation}
v = \sum_{n \ge 0} \lambda_{n} v_{1}^{\otimes n}
\end{equation}

\begin{align*}
H(v) & = H(\sum_{n \ge 0} \lambda_{n} v_{1}^{\otimes n})\\
& = \sum_{n \ge 0} \lambda_{n} H(v_{1}^{\otimes n})\\
& = \sum_{n \ge 0} \lambda_{n} H(v_{1}^{n})\\
& = \sum_{n \ge 0} \lambda_{n} H(v_{1})^{n}\\
& = \sum_{n \ge 0} \lambda_{n} X^{n}
\end{align*}

Also ist $H$ durch diese Vorgaben bereits eindeutig bestimmt für alle $v \in T(V)$.
Jeder gleichermaßen definierte Algebrahomomorphismus muss also gleich $H$ sein.

Sei $v \in V^{\otimes n}$.
\begin{equation}
H(v) = H(\lambda \cdot v_{1}^{\otimes n}) = \lambda \cdot H(v_{1}^{n}) = \lambda \cdot H(v_{1})^{n} = \lambda \cdot X^{n}
\end{equation}
Also ist $H$ graduiert.
\end{proof}

\subsection*{Teil 3}

\begin{proof}
Sei $P \in K[X]$.
\begin{equation}
P = \sum_{n \ge 0} a_{n} X^{n}
\end{equation}
\begin{equation}
H^{-1}(P) := \sum_{n \ge 0} a_{n} v_{1}^{\otimes n}
\end{equation}
\begin{align*}
H \circ H^{-1}(P) & = H(\sum_{n \ge 0} a_{n} v_{1}^{\otimes n})\\
& = \sum_{n \ge 0} a_{n} H(v_{1}^{\otimes n})\\
& = \sum_{n \ge 0} a_{n} H(v_{1}^{n})\\
& = \sum_{n \ge 0} a_{n} H(v_{1})^{n}\\
& = \sum_{n \ge 0} a_{n} X^{n} = P
\end{align*}

Sei $v$ wie in Teil 1.
\begin{align*}
H^{-1} \circ H(v) & = H^{-1}(H(\sum_{n \ge 0} \lambda_{n} v_{1}^{\otimes n}))\\
& = H^{-1}(\sum_{n \ge 0} \lambda_{n} H(v_{1}^{\otimes n}))\\
& = H^{-1}(\sum_{n \ge 0} \lambda_{n} X^{n})\\
& = \sum_{n \ge 0} \lambda_{n} H^{-1}(X^{n})\\
& = \sum_{n \ge 0} \lambda_{n} v_{1}^{n}\\
& = \sum_{n \ge 0} \lambda_{n} v_{1}^{\otimes n} = v\\
\end{align*}

Es gibt also eine Umkehrfunktion $H^{-1}$ zu $H$ und $H$ ist ein Isomorphismus.
\end{proof}

\section*{Übung 5}

\subsection*{Teil 1}

\begin{proof}
Sei $a_{n} \in A_{n}$ und $a_{m} \in A_{m}$.
\begin{equation}
a_{n} = \sum_{i = 1}^{n} \lambda_{i} X_{1}^{i - 1}X_{2}^{n - i + 1}
\end{equation}
\begin{equation}
a_{m} = \sum_{i = 1}^{m} \lambda_{i} X_{1}^{i - 1}X_{2}^{m - i + 1}
\end{equation}

\begin{align*}
a_{n} \cdot a_{m} & = \left( \sum_{i = 0}^{n} \lambda_{i} X_{1}^{i}X_{2}^{n - i} \right) \cdot \left( \sum_{i = 0}^{m} \lambda_{i} X_{1}^{i}X_{2}^{m - i} \right)\\
& = \sum_{i = 0}^{m} \left( \sum_{j = 0}^{n} \lambda_{j} X_{1}^{j}X_{2}^{n - j} \right) \cdot \lambda_{i} X_{1}^{i}X_{2}^{m - i}\\
& = \sum_{i = 0}^{m} \sum_{j = 0}^{n} \lambda_{j} X_{1}^{j}X_{2}^{n - j} \cdot \lambda_{i} X_{1}^{i}X_{2}^{m - i}\\
& = \sum_{i = 0}^{m} \sum_{j = 0}^{n} \lambda_{j}\lambda_{i} X_{1}^{j}X_{2}^{n - j}X_{1}^{i}X_{2}^{m - i}\\
& = \sum_{i = 0}^{m} \sum_{j = 0}^{n} \lambda_{j}\lambda_{i} X_{1}^{i + j}X_{2}^{m + n - j - i} \in A_{m + n}\\
\end{align*}
\end{proof}

\subsection*{Teil 2}

\subsection*{Teil 3}

\begin{proof}
Sei $P \in K[X_{1}, X_{2}]$.
\begin{equation}
P = \sum_{n \ge 0} \sum_{k \ge 0} a_{n, k} X_{1}^{n}X_{2}^{k}
\end{equation}
\begin{align*}
H(\sum_{n \ge 0} \sum_{k \ge 0} a_{n, k} v_{1}^{\otimes n} v_{2}^{\otimes k}) & = \sum_{n \ge 0} \sum_{k \ge 0} a_{n, k} H(v_{1}^{\otimes n} v_{2}^{\otimes k})\\
& = \sum_{n \ge 0} \sum_{k \ge 0} a_{n, k} H(v_{1}^{\otimes n})H(v_{2}^{\otimes k})\\
& = \sum_{n \ge 0} \sum_{k \ge 0} a_{n, k} H(v_{1}^{n})H(v_{2}^{k})\\
& = \sum_{n \ge 0} \sum_{k \ge 0} a_{n, k} H(v_{1})^{n}H(v_{2})^{k}\\
& = \sum_{n \ge 0} \sum_{k \ge 0} a_{n, k} X_{1}^{n}X_{2}^{k} = P
\end{align*}

Es gibt also zu jedem $P$ ein Urbild und $H$ ist surjektiv.
\end{proof}

\subsection*{Teil 4}

\begin{proof}
$H$ ist nicht injektiv, weil $H(v_{1} \otimes v_{2}) = X_{1}X_{2} = X_{2}X_{1} = H(v_{2} \otimes v_{1})$, aber $v_{1} \otimes v_{2} \ne v_{2} \otimes v_{1}$.
\end{proof}

\end{document}