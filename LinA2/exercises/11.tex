\documentclass[10pt,a4paper]{article}
\usepackage[utf8]{inputenc}
\usepackage[german]{babel}
\usepackage{amsmath}
\usepackage{amsfonts}
\usepackage{amssymb}
\usepackage{amsthm}
\usepackage[left=2cm,right=2cm,top=2cm,bottom=2cm]{geometry}

\DeclareMathOperator{\Mat}{Mat}

\begin{document}

\subsection*{Übung 1}

\subsection*{Teil 1}

\begin{equation}
(1, e_{1}, e_{2}, e_{1} \otimes e_{2})
\end{equation}

\subsection*{Teil 2}

\begin{tabular}{c|c|c|c|c}
$\otimes$ & $1$ & $e_{1}$ & $e_{2}$ & $e_{1} \otimes e_{2}$\\
\hline
$1$ & 1 & $e_{1}$ & $e_{2}$ & $e_{1} \otimes e_{2}$\\
\hline
$e_{1}$ & $e_{1}$ & 0 & $e_{1} \otimes e_{2}$ & 0\\
\hline
$e_{2}$ & $e_{2}$ & $-e_{1} \otimes e_{2}$ & 0 & 0\\
\hline
$e_{1} \otimes e_{2}$ & $e_{1} \otimes e_{2}$ & 0 & 0 & 0
\end{tabular}

\subsection*{Teil 3}

Seien $v, w \in \bigwedge(V)$ so, dass $v$ invertierbar und $w$ das Inverse ist.
\begin{align*}
v \otimes w & = \left( a + be_{1} + ce_{2} + d(e_{1} \otimes e_{2}) \right) \otimes \left( g + he_{1} + ie_{2} + j(e_{1} \otimes e_{2}) \right)\\
& = (a  \otimes \left( g + he_{1} + ie_{2} + j(e_{1} \otimes e_{2}) \right)) + (be_{1} \otimes \left( g + he_{1} + ie_{2} + j(e_{1} \otimes e_{2}) \right))\\
&\ \ \ + (ce_{2} \otimes \left( g + he_{1} + ie_{2} + j(e_{1} \otimes e_{2}) \right)) + (d(e_{1} \otimes e_{2}) \otimes \left( g + he_{1} + ie_{2} + j(e_{1} \otimes e_{2}) \right))\\
& = (a  \otimes g + a  \otimes he_{1} + a  \otimes ie_{2} + a  \otimes j(e_{1} \otimes e_{2})) + ( be_{1} \otimes g + be_{1} \otimes he_{1} + be_{1} \otimes ie_{2} + be_{1} \otimes j(e_{1} \otimes e_{2}) )\\
&\ \ \ + (ce_{2} \otimes g + ce_{2} \otimes he_{1} + ce_{2} \otimes ie_{2} + ce_{2} \otimes j(e_{1} \otimes e_{2})) + (d(e_{1} \otimes e_{2}) \otimes g + d(e_{1} \otimes e_{2}) \otimes he_{1} + d(e_{1} \otimes e_{2}) \otimes ie_{2} + d(e_{1} \otimes e_{2}) \otimes j(e_{1} \otimes e_{2}) )\\
& = (ag + ahe_{1} + aie_{2} + aj(e_{1} \otimes e_{2})) + (bge_{1} + bh(e_{1} \otimes e_{1}) + bi(e_{1} \otimes e_{2}) + bj(e_{1} \otimes e_{1} \otimes e_{2}) )\\
&\ \ \ + (cge_{2} + ch(e_{2} \otimes e_{1}) + ci(e_{2} \otimes e_{2}) + cj(e_{2} \otimes e_{1} \otimes e_{2})) + (dg(e_{1} \otimes e_{2}) + dh(e_{1} \otimes e_{2} \otimes e_{1}) + di(e_{1} \otimes e_{2} \otimes e_{2}) + dj(e_{1} \otimes e_{2} \otimes e_{1} \otimes e_{2}))\\
& = ag + ahe_{1} + aie_{2} + aj(e_{1} \otimes e_{2}) + bge_{1} + bi(e_{1} \otimes e_{2}) + cge_{2} - ch(e_{1} \otimes e_{2}) + dg(e_{1} \otimes e_{2})\\
& = ag + (ah + bg)e_{1} + (ai + cg)e_{2} + (aj + bi - ch + dg)(e_{1} \otimes e_{2}) = 1
\end{align*}
Weil beide Terme als Linearkombination einer Basis dargestellt sind, gilt die komponentenweise Gleichheit.
\begin{align*}
\begin{cases}
ag & = 1\\
ah + bg & = 0\\
ai + cg & = 0\\
aj + bi - ch + dg & = 0
\end{cases}
\end{align*}
Weil $V$ auf einem Körper definiert ist und $ag = 1$, sind $a, g \ne 0$.
\begin{align*}
& \begin{cases}
ag & = 1\\
ah + bg & = 0\\
ai + cg & = 0\\
aj + bi - ch + dg & = 0
\end{cases}\\
\Leftrightarrow &
\begin{cases}
g & = \frac{1}{a}\\
ah + bg & = 0\\
ai + cg & = 0\\
aj + bi - ch + dg & = 0
\end{cases}\\
\Leftrightarrow &
\begin{cases}
g & = \frac{1}{a}\\
ah + \frac{b}{a} & = 0\\
ai + \frac{c}{a} & = 0\\
aj + bi - ch + \frac{d}{a} & = 0
\end{cases}\\
\Leftrightarrow &
\begin{cases}
g & = \frac{1}{a}\\
h & = -\frac{b}{a^{2}}\\
i & = -\frac{c}{a^{2}}\\
aj + bi - ch + \frac{d}{a} & = 0
\end{cases}\\
\Leftrightarrow &
\begin{cases}
g & = \frac{1}{a}\\
h & = -\frac{b}{a^{2}}\\
i & = -\frac{c}{a^{2}}\\
aj - \frac{bc}{a^{2}} + \frac{cb}{a^{2}} + \frac{d}{a} & = 0
\end{cases}\\
\Leftrightarrow &
\begin{cases}
g & = \frac{1}{a}\\
h & = -\frac{b}{a^{2}}\\
i & = -\frac{c}{a^{2}}\\
j & = \frac{bc}{a^{3}} - \frac{cb}{a^{3}} - \frac{d}{a^{2}}
\end{cases}\\
\end{align*}
Es sind also alle $v \in \bigwedge(V) = a + be_{1} + ce_{2} + d(e_{1} \otimes e_{2})$ mit $a \ne 0$ invertierbar und das Inverse ist $\frac{1}{a} - \frac{b}{a^{2}}e_{1} - \frac{c}{a^{2}}e_{2} + \left( \frac{bc}{a^{3}} - \frac{cb}{a^{3}} - \frac{d}{a^{2}} \right)(e_{1} \otimes e_{2})$.

\subsection*{Übung 2}

\subsection*{Teil 1.1}

\begin{proof}
\begin{align*}
v_{1} \wedge v_{2} \wedge v_{3} & = v_{1} \wedge v_{2} \wedge \sum_{i = 1}^{3} a_{i, 3} e_{i}\\
& = \sum_{i = 1}^{3} a_{i, 3}\ v_{1} \wedge v_{2} \wedge e_{i}\\
& = \sum_{i = 1}^{3} a_{i, 3}\ v_{1} \wedge \left( \sum_{j = 1}^{3} a_{j, 3} e_{j} \right) \wedge e_{i}\\
& = \sum_{i = 1}^{3} a_{i, 3}\ \left( \sum_{j = 1}^{3} a_{j, 3} v_{1} \wedge e_{j} \right) \wedge e_{i}\\
& = \sum_{i = 1}^{3} a_{i, 3}\ \left( \sum_{j = 1}^{3} a_{j, 3} v_{1} \wedge e_{j}  \wedge e_{i}\right)\\
\end{align*}
\end{proof}

\subsection*{Teil 1.2}

\begin{proof}
Sei $v \in \bigwedge^{3}(V)$.
\end{proof}

\subsection*{Teil 1.3}

\begin{equation}
\chi_{f} = \det(XI_{3} - Mat(f)) = (x - 2)(x - 1)^{2}
\end{equation}
\begin{equation}
H(f, 2) = \langle \begin{pmatrix}
2\\1\\1
\end{pmatrix} \rangle
\end{equation}
\begin{equation}
E_{1}(f, 1) = \langle \begin{pmatrix}
1\\0\\0
\end{pmatrix} \rangle
\end{equation}
\begin{equation}
H(f, 1) = \langle
\begin{pmatrix}
1\\0\\0
\end{pmatrix},
\begin{pmatrix}
0\\1\\0
\end{pmatrix}
\rangle
\end{equation}
\begin{equation}
\mathcal{B}' = \left(
\begin{pmatrix}
2\\1\\1
\end{pmatrix},
\begin{pmatrix}
1\\0\\0
\end{pmatrix},
\begin{pmatrix}
0\\1\\0
\end{pmatrix}
\right)
\end{equation}

\subsection*{Teil 1.4}

\subsection*{Teil 2.1}

\begin{equation}
\chi_{f} = \det(XI_{4} - Mat(f)) = (x - 1)^{2}(x - 2)(x - 3)
\end{equation}
\begin{equation}
E_{1}(f, 1) = \langle
\begin{pmatrix}
-2\\0\\0\\1
\end{pmatrix}
\rangle
\end{equation}
\begin{equation}
H(f, 1) = \langle
\begin{pmatrix}
-3\\2\\0\\0
\end{pmatrix},
\begin{pmatrix}
-4\\0\\0\\2
\end{pmatrix}
\rangle
\end{equation}
\begin{equation}
H(f, 2) = \langle
\begin{pmatrix}
2\\1\\1\\-4
\end{pmatrix}
\rangle
\end{equation}
\begin{equation}
H(f, 3) = \langle
\begin{pmatrix}
0\\0\\0\\1
\end{pmatrix}
\rangle
\end{equation}
\begin{equation}
\mathcal{B}' = \left(
\begin{pmatrix}
-2\\0\\0\\1
\end{pmatrix},
\begin{pmatrix}
-3\\2\\0\\0
\end{pmatrix},
\begin{pmatrix}
2\\1\\1\\-4
\end{pmatrix},
\begin{pmatrix}
0\\0\\0\\1
\end{pmatrix}
\right)
\end{equation}

\subsection*{Teil 2.2}

Aus der Matrix erhält man
\begin{equation}
f(e_{1}) = e_{1} + e_{4} \quad f(e_{2}) = e_{1} + e_{2} + e_{4} \quad f(e_{3}) = e_{1} + e_{2} + 2e_{3} + e_{4} \quad f(e_{4}) = 3e_{4}
\end{equation}
Mit der kanonischen Basis erhält man die Basis $\mathcal{B}_{2} = (e_{1} \otimes e_{2}, e_{1} \otimes e_{3}, e_{1} \otimes e_{4}, e_{2} \otimes e_{3}, e_{2} \otimes e_{4}, e_{3} \otimes e_{4})$ von $\bigwedge(V)$.
\begin{equation}
\Mat_{\mathcal{B}_{2}}(\bigwedge(f)) = \begin{pmatrix}
1 & 1 & 0 & 0 & 0 & 0\\
0 & 2 & 0 & 2 & 0 & 0\\
0 & 0 & 1 & 0 & 3 & 3\\
0 & 0 & 0 & 2 & 0 & 0\\
-1 & -1 & 0 & 0 & 3 & 3\\
0 & -2 & 0 & -2 & 0 & 6
\end{pmatrix}
\end{equation}
\begin{equation}
\chi_{\bigwedge(f)} = \det(XI_{6} - \Mat_{\mathcal{B}_{2}}(f)) = (x - 6)(x - 3)(x - 2)^{2}(x - 1)^{2}
\end{equation}
\begin{equation}
H(\bigwedge(f), 6) = \langle
\begin{pmatrix}
0\\0\\6\\0\\5\\5
\end{pmatrix}
\rangle
\end{equation}
\begin{equation}
H(\bigwedge(f), 3) = \langle
\begin{pmatrix}
0\\0\\6\\0\\5\\5
\end{pmatrix}
\rangle
\end{equation}

\subsection*{Teil 3.1}

\begin{equation}
\chi_{f} = \det(XI_{2} - Mat(f)) = x^{2} - 1 = (x - 1)(x + 1) = \mu_{f}
\end{equation}

\subsection*{Teil 3.2}

Die Basis $\mathcal{B}_{2}$ von $T^{2}$ ist $e_{1} \otimes e_{1}, e_{1} \otimes e_{2}, e_{2} \otimes e_{1}, e_{2} \otimes e_{2}$.
Die Abbildung $T(f)$ bildet $v \otimes w$ auf $f(v) \otimes f(w)$ ab.
Aus der Matrix wissen wir, dass
\begin{equation}
f(e_{1}) = e_{2} \quad f(e_{2}) = e_{1}
\end{equation}
Damit bestimme ich jetzt
\begin{equation}
\Mat_{\mathcal{B}_{2}}(f) = \begin{pmatrix}
0 & 0 & 0 & 1\\
0 & 0 & 1 & 0\\
0 & 1 & 0 & 0\\
1 & 0 & 0 & 0
\end{pmatrix}
\end{equation}
Dann ist das charakteristische Polynom
\begin{equation}
\chi_{T_{2}(f)} = (x - 1)^{2}(x + 1)^{2}
\end{equation}
Das Minimalpolynom bleibt
\begin{equation}
\mu_{T_{2}(f)} = (x - 1)(x + 1)
\end{equation}

\subsection*{Teil 3.3}

\begin{equation}
\chi_{T_{n}(f)} = (x - 1)^{\frac{2^{n}}{2}}(x + 1)^{\frac{2^{n}}{2}}
\end{equation}
\begin{equation}
\mu_{T_{n}(f)} = (x - 1)(x + 1)
\end{equation}

\subsection*{Übung 3}

\begin{proof}
$\Rightarrow$: Ein System von einem Vektor ist immer linear unabhängig.
Also ist auch $(v')$ linear unabhängig.
Nach Lemma 4.2.1 ist dann $v = 0$.
Das gleiche gilt für $v$.

$\Leftarrow$: Sei o.B.d.A. $v = 0$.
Dann ist $v = 0 \cdot a$ für ein $a \in V$.
\begin{equation}
v \otimes v' = 0a \otimes v' = 0(a \otimes v') = 0
\end{equation}
\end{proof}

\end{document}