\documentclass[10pt,a4paper]{article}
\usepackage[utf8]{inputenc}
\usepackage[german]{babel}
\usepackage{amsmath}
\usepackage{amsfonts}
\usepackage{amssymb}
\usepackage{amsthm}
\usepackage[left=2cm,right=2cm,top=2cm,bottom=2cm]{geometry}

\title{LinA2, Übungsblatt 1}
\author{Marten Lienen (2126759)}

\begin{document}

\maketitle

\section*{Übung 1}

\subsection*{1}

\subsubsection*{1.a}

\begin{equation}
ZSF(A_{1}) = 
\begin{pmatrix}
1 & -1 & 0\\
0 & 0 & 1\\
0 & 0 & 0
\end{pmatrix}
\end{equation}
\begin{equation}
Rg(A_{1}) = 2
\end{equation}

\subsubsection*{1.b}

\begin{equation}
\chi_{A_{1}} = X(X-2)(X-3)
\end{equation}

\subsubsection*{1.c}

\begin{equation}
\mu_{A_{1}} = X(X-2)(X-3)
\end{equation}

\subsubsection*{1.d}

\subsubsection*{1.e}

\subsubsection*{2.a}

\begin{equation}
ZFS(A_{2}) = I_{3}
\end{equation}
\begin{equation}
Rg(A_{2}) = 3
\end{equation}

\subsubsection*{2.b}

\begin{equation}
\chi_{A_{2}} = X(X-2)(X-3)
\end{equation}

\subsubsection*{2.c}

\begin{equation}
\mu_{A_{2}} = X(X-2)(X-3)
\end{equation}

\subsubsection*{2.d}

\subsubsection*{2.e}

\subsubsection*{3.a}

\begin{equation}
ZSF(A_{3}) = 
\begin{pmatrix}
0 & 0 & 0\\
0 & 0 & 0\\
0 & 0 & 0
\end{pmatrix}
\end{equation}
\begin{equation}
Rg(A_{3}) = 
\end{equation}

\subsubsection*{3.b}

\begin{equation}
\chi_{A_{3}} = 
\end{equation}

\subsubsection*{3.c}

\begin{equation}
\mu_{A_{3}} = 
\end{equation}

\subsubsection*{3.d}

\subsubsection*{3.e}

\subsubsection*{4.a}

\begin{equation}
ZSF(A_{4}) = 
\begin{pmatrix}
0 & 0 & 0\\
0 & 0 & 0\\
0 & 0 & 0
\end{pmatrix}
\end{equation}
\begin{equation}
Rg(A_{4}) = 
\end{equation}

\subsubsection*{4.b}

\begin{equation}
\chi_{A_{4}} = 
\end{equation}

\subsubsection*{4.c}

\begin{equation}
\mu_{A_{4}} = 
\end{equation}

\subsubsection*{4.d}

\subsubsection*{4.e}

\subsubsection*{5.a}

\begin{equation}
ZSF(A_{5}) = 
\begin{pmatrix}
0 & 0 & 0\\
0 & 0 & 0\\
0 & 0 & 0
\end{pmatrix}
\end{equation}
\begin{equation}
Rg(A_{5}) = 
\end{equation}

\subsubsection*{5.b}

\begin{equation}
\chi_{A_{5}} = 
\end{equation}

\subsubsection*{5.c}

\begin{equation}
\mu_{A_{5}} = 
\end{equation}

\subsubsection*{5.d}

\subsubsection*{5.e}

\subsubsection*{6.a}

\begin{equation}
ZSF(A_{6}) = 
\begin{pmatrix}
0 & 0 & 0\\
0 & 0 & 0\\
0 & 0 & 0
\end{pmatrix}
\end{equation}
\begin{equation}
Rg(A_{6}) = 
\end{equation}

\subsubsection*{6.b}

\begin{equation}
\chi_{A_{6}} = 
\end{equation}

\subsubsection*{6.c}

\begin{equation}
\mu_{A_{6}} = 
\end{equation}

\subsubsection*{6.d}

\subsubsection*{6.e}

\subsubsection*{7.a}

\begin{equation}
ZSF(A_{7}) = 
\begin{pmatrix}
0 & 0 & 0\\
0 & 0 & 0\\
0 & 0 & 0
\end{pmatrix}
\end{equation}
\begin{equation}
Rg(A_{7}) = 
\end{equation}

\subsubsection*{7.b}

\begin{equation}
\chi_{A_{7}} = 
\end{equation}

\subsubsection*{7.c}

\begin{equation}
\mu_{A_{7}} = 
\end{equation}

\subsubsection*{7.d}

\subsubsection*{7.e}

\subsection*{2}

\section*{Übung 2}

\subsection*{1}

\subsection*{2}

\section*{Übung 3}

\subsection*{1}

Nein, sie sind nicht ähnlich, weil ihre Determinanten nicht gleich sind.
\begin{equation}
det(A) = -2
\end{equation}
\begin{equation}
det(B) = -1
\end{equation}

\subsection*{2}

Nein, sie sind nicht ähnlich, weil ihre Determinanten nicht gleich sind.
\begin{equation}
det(A) = 0
\end{equation}
\begin{equation}
det(B) = 1
\end{equation}

\end{document}