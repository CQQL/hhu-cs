\documentclass[10pt,a4paper]{article}
\usepackage[utf8]{inputenc}
\usepackage[german]{babel}
\usepackage{amsmath}
\usepackage{amsfonts}
\usepackage{amssymb}
\usepackage{amsthm}
\usepackage[left=2cm,right=2cm,top=2cm,bottom=2cm]{geometry}

\DeclareMathOperator{\ord}{ord}
\DeclareMathOperator{\id}{id}
\DeclareMathOperator{\Image}{Im}
\DeclareMathOperator{\Hom}{Hom}
\DeclareMathOperator{\Tr}{Tr}

\newtheorem*{lemma}{Lemma}

\begin{document}

\section*{Übung 1}

\subsection*{Teil 1}

\begin{equation}
\id^{1} = \id \Rightarrow \ord(\id) = 1
\end{equation}
\begin{equation}
s_{i}^{2} = \id \Rightarrow \ord(s_{i}) = 2
\end{equation}

\subsection*{Teil 2}

Mit Teil 1 und dem vorigen Übungsblatt ergibt sich
\begin{equation}
(1\ 2\ 3) = \id \Rightarrow \ord(1\ 2\ 3) = 1
\end{equation}
\begin{equation}
(1\ 3\ 2) = s_{2} \Rightarrow \ord(1\ 3\ 2) = 2
\end{equation}
\begin{equation}
(2\ 1\ 3) = s_{1} \Rightarrow \ord(2\ 1\ 3) = 2
\end{equation}
\begin{equation}
(3\ 2\ 1)^{2} = (s_{1}s_{2}s_{1})^{2} = \id \Rightarrow \ord(3\ 2\ 1) = 2
\end{equation}
\begin{equation}
(2\ 3\ 1)^{3} = (s_{2}s_{1})^{3} = \id \Rightarrow \ord(2\ 3\ 1) = 3
\end{equation}
\begin{equation}
(3\ 1\ 2)^{3} = (s_{1}s_{2})^{3} = \id \Rightarrow \ord(3\ 1\ 2) = 3
\end{equation}

\subsection*{Teil 3}

\begin{equation}
\ord(\sigma) = n
\end{equation}

\begin{lemma}
Sei $k \in [1, n]$.
Dann ist $\sigma^{n - k + 1}(k) = 1$ und $\sigma^{k - 1}(1) = k$, also $\sigma^{k - 1} \circ \sigma^{n - k + 1}(k) = \sigma^{n}(k) = k$.
\end{lemma}

\begin{proof}
Ich zeige es per Induktion.
\begin{equation}
\sigma^{1}(n) = \sigma^{n - n + 1}(n) = 1
\end{equation}
\begin{equation}
\sigma^{0}(1) = \sigma^{k - 1}(1) = 1
\end{equation}

Angenommen $\sigma^{n - k + 1}(k) = 1$ und $\sigma^{k - 1}(1) = k$.
\begin{equation}
\sigma^{n - (k - 1) + 1}(k - 1) = \sigma^{n - k + 1} \circ \sigma(k - 1) = \sigma^{n - k + 1}(k) = 1
\end{equation}
Dies gilt, weil $k \le n \Rightarrow k - 1 < n$.

Sei $k \le n - 1$.
\begin{equation}
\sigma^{k}(1) = \sigma \circ \sigma^{k - 1}(1) = \sigma(k) = k + 1
\end{equation}
\end{proof}

\section*{Übung 2}

\subsection*{Teil 1}

\begin{proof}
Seien $A, B \in M_{n}(\mathbb{R})$ und $\lambda, \mu \in \mathbb{R}$.
\begin{align*}
\varphi(\lambda A + \mu B) & = \sum_{i = 1}^{n} \sum_{j = 1}^{n} (\lambda a_{i, j} + \mu b_{i, j}) (e_{i} \otimes e_{j})\\
& = \lambda \cdot \sum_{i = 1}^{n} \sum_{j = 1}^{n} a_{i, j} (e_{i} \otimes e_{j}) + \mu \cdot \sum_{i = 1}^{n} \sum_{j = 1}^{n} b_{i, j} (e_{i} \otimes e_{j})\\
& = \lambda \varphi(A) + \mu \varphi(B)
\end{align*}
Also ist $\varphi$ linear.

Sei $v \in \mathbb{R}^{n} \otimes_{\mathbb{R}} \mathbb{R}^{n}$ mit $v = \sum_{i = 1}^{n} \sum_{j = 1}^{n} \lambda_{i, j} (e_{i} \otimes e_{j})$.
Dann definiere ich $\varphi^{-1}: \mathbb{R}^{n} \otimes_{\mathbb{R}} \mathbb{R}^{n} \rightarrow M_{n}(\mathbb{R})$ durch
\begin{equation}
\varphi^{-1}(v) = (a_{i, j})_{i \in [1, n], j \in [1, m]}
\end{equation}
mit
\begin{equation}
a_{i, j} = \lambda_{i, j}
\end{equation}

Sei $v$ wie oben.
\begin{align*}
\varphi \circ \varphi^{-1}(v) = \varphi((a_{i, j})) = \sum_{i = 1}^{n} \sum_{j = 1}^{n} \lambda_{i, j} (e_{i} \otimes e_{j}) = v
\end{align*}

Sei $A = (a_{i, j}) \in M_{n}(\mathbb{R})$.
\begin{align*}
\varphi^{-1} \circ \varphi(A) = \varphi^{-1}(\sum_{i = 1}^{n} \sum_{j = 1}^{n} a_{i, j} (e_{i} \otimes e_{j})) = A
\end{align*}

Dann ist $\varphi \circ \varphi^{-1} = \id = \varphi^{-1} \circ \varphi$ und $\varphi$ ist ein Isomorphismus.
\end{proof}

\subsection*{Teil 2}

\begin{proof}
\begin{align*}
\varphi(A) & = \sum_{i = 1}^{n} \sum_{j = 1}^{n} a_{i, j} (e_{i} \otimes e_{j})\\
& = \sum_{j = 1}^{n} \sum_{i = 1}^{n} a_{i, j} (e_{i} \otimes e_{j})\\
& = \sum_{j = 1}^{n} \left( \sum_{i = 1}^{n} a_{i, j} e_{i} \right) \otimes e_{j} \qquad \textit{weil $\otimes$ bilinear ist}\\
& = \sum_{j = 1}^{n} S_{j} \otimes e_{j}\\
\end{align*}
\end{proof}

\subsection*{Teil 3}

\begin{proof}
Wenn $Rg(A) = 0$, bedeutet das, dass alle Spalten von $A$ paarweise linear abhängig sind.
Seien $\mu_{k} \in \mathbb{R}$, sodass $S_{k} = \mu_{k} \cdot S_{j}$ für alle $k \in [1, n]$.
Sei $S \in \Image(A)$ und $\xi_{k} \in \mathbb{R}$ für alle $k \in [1, n]$.
Dann ist
\begin{align*}
S & = \xi_{1} S_{1} + \dots + \xi_{j} S_{j} + \dots + \xi_{n} S_{n}\\
& = \xi_{1} \mu_{1} S_{j} + \dots + \xi_{j} \mu_{j} S_{j} + \dots + \xi_{n} \mu_{n} S_{j}\\
& = \left( \xi_{1} \mu_{1} + \dots + \xi_{n} \mu_{n} \right) S_{j}\\
& \Rightarrow \lambda_{j} = \left( \xi_{1} \mu_{1} + \dots + \xi_{n} \mu_{n} \right)^{-1}
\end{align*}

\begin{align*}
\varphi(A) & = \sum_{j = 1}^{n} S_{j} \otimes e_{j}\\
& = \sum_{j = 1}^{n} \lambda_{j} S \otimes e_{j}\\
& = \sum_{j = 1}^{n} S \otimes \lambda_{j} e_{j} \qquad \textit{wegen der Bilinearität}\\
& = S \otimes \sum_{j = 1}^{n} \lambda_{j} e_{j} \qquad \textit{wegen der Bilinearität}\\
& = S \otimes v\\
\end{align*}
\end{proof}

\subsection*{Teil 4}

\subsection*{Teil 5}

\subsection*{Teil 6}

\section*{Übung 3}

\subsection*{Teil 1}

\begin{proof}
Sei $v \in V$, $\phi, \psi \in V^{\vee}$, $w, z \in W$ und $a, b \in K$.
\begin{align*}
\Phi(a \phi + b \psi, w)(v) & = (a \phi + b \psi)(v) w\\
& = (a \phi(v) + b \psi(v)) w\\
& = a \phi(v) w + b \psi(v)w = (a \Phi(\phi, w) + b \Phi(\psi, w))(v)\\
\end{align*}
\begin{align*}
\Phi(\phi, aw + bz)(v) & = \phi(v) (aw + bz)\\
& = a \phi(v)w + b \phi(v)z = a \Phi(\phi, w) + b \Phi(\phi, z)
\end{align*}

Weil $\Phi$ von $V^{\vee} \times W$ nach $\Hom_{K}(V, W)$ abbildet und wir wissen, dass das Tensorprodukt existiert, gibt es nach Definition des Tensorprodukts eine Abbildung $\pi: V^{\vee} \times W \rightarrow V^{\vee} \otimes W$ definiert durch $\pi(\phi, w) = \phi \otimes w$, sodass $L_{\Phi} \circ \pi (\phi, w)(v) = \Phi(\phi, w)(v) = \phi(v) w$ ist.
Durch Auflösen der Komposition erhält man $L_{\Phi}(\phi \otimes w)(v) = \phi(v)w$.
\end{proof}

\subsection*{Teil 2}

\begin{proof}
Seien $\varphi, \psi \in V^{\vee}$, $v \in V$ und $a, b \in W$, sodass
\begin{equation}
L_{\Phi}(\varphi \otimes a)(v) = \varphi(v) a = \psi(v) b = L_{\Phi}(\psi \otimes b)
\end{equation}
Dann muss $\psi(v)$ linear abhängig von $\varphi(v)$ und $b$ von $a$ sein.
Seien $\lambda_{b}$ und $\lambda_{\psi}$ so, dass
\begin{equation}
b = \lambda_{b} a \qquad \land \qquad \psi(v) = \lambda_{\psi} \varphi(v)
\end{equation}
Wegen $\varphi(v) a = \psi(v) b = \lambda_{\psi} \varphi(v) \lambda_{b} a$ gilt außerdem $\lambda_{b}\lambda_{\psi} = 1$.
Durch Einsetzen erhält man
\begin{align*}
\psi \otimes b & = (\lambda_{\psi} \varphi) \otimes (\lambda_{b} a)\\
& = \lambda_{\psi} \lambda_{b} (\varphi \otimes a) \qquad \textit{wegen der Bilinearität}\\
& = \varphi \otimes a \qquad \textit{wegen $\lambda_{b}\lambda_{\psi} = 1$}\\
\end{align*}
Somit ist $L_{\Phi}$ injektiv.
\end{proof}

\subsection*{Teil 3}



\section*{Übung 4}

\subsection*{Teil 1}

Nach der Definition des Tensorprodukts ist der Wert in der Spalte $v_{p} \otimes w_{q}$ und Zeile $v_{r} \otimes w_{s}$
\begin{equation}
a_{p, r} \cdot b_{q, s}
\end{equation}

\subsection*{Teil 2}

\begin{proof}
Nach Teil 1 ist
\begin{equation}
\Tr(A \otimes B) = \sum_{i = 1}^{n} \sum_{j = 1}^{m} a_{i, i} b_{j, j}
\end{equation}
Für die rechte Seite der Gleichung gilt
\begin{align*}
\Tr(A)\Tr(B) & = \left( \sum_{i = 1}^{n} a_{i, i} \right) \left( \sum_{j = 1}^{m} b_{j, j} \right)\\
& = \sum_{i = 1}^{n} \left( a_{i, i} \cdot \left( \sum_{j = 1}^{m} b_{j, j} \right) \right) \\
& = \sum_{i = 1}^{n} \sum_{j = 1}^{m} a_{i, i} b_{j, j}
\end{align*}

Also ist
\begin{equation}
\Tr(A \otimes B) = \Tr(A)\Tr(B)
\end{equation}
\end{proof}

\subsection*{Teil 3}

\begin{proof}
\begin{equation}
f \times g(v \otimes w) = f(v) \otimes g(w) = (\lambda v) \otimes (\mu w) = \lambda \mu (v \otimes w)
\end{equation}

$(v \otimes w)$ ist also ein Eigenvektor zum Eigenwert $\lambda \mu$.
\end{proof}

\subsection*{Teil 4}

Nach Teil 3 besteht die Basis $B \otimes B'$ aus Eigenvektoren von $f \otimes g$.
Sie ist deshalb diagonalisierbar und das charakteristische Polynom ist das Produkt der Diagonalwerte von $X I_{mn} - A \otimes B$.
Die Diagonalwerte von $A \otimes B$ sind wiederum nach Teil 3 $\lambda_{i} \mu_{j} \quad \forall i \in [1, n], j \in [1, m]$.
Insgesamt ist
\begin{equation}
\chi_{A \otimes B} = \prod_{i = 1}^{n} \prod_{j = 1}^{m} (X - \lambda_{i} \mu_{j})
\end{equation}

\end{document}