\documentclass[10pt,a4paper]{article}
\usepackage[utf8]{inputenc}
\usepackage[german]{babel}
\usepackage{amsmath}
\usepackage{amsfonts}
\usepackage{amssymb}
\usepackage{amsthm}
\usepackage[left=2cm,right=2cm,top=2cm,bottom=2cm]{geometry}

\begin{document}

\section*{Übung 1}

\subsection*{Teil 1}

Sei $(v_{1}, \dots, v_{n})$ die Basis in der $A$ ist.
Da $A$ der Jordan-Block $J(0, n + 1)$ ist, ist diese Basis eine Jordankette.

\subsection*{Teil 2}

Da $A$ eine obere Dreiecksmatrix ist, ist $XI_{n + 1} - A$ auch eine und $\chi_{A} = x^{n + 1}$.

\subsection*{Teil 3}

Da $A$ der Jordan-Block $J(0, n + 1)$ ist, gilt
\begin{equation}
\mu_{A} = x^{n + 1}
\end{equation}
Für einen Beweis siehe Übung 5.

\subsection*{Teil 4}

Sei $B = (b_{i,j})$ mit
\begin{equation}
b_{i,j} =
\begin{cases}
j & \textit{wenn $j = i + 1$}\\
0 & \textit{sonst}
\end{cases}
\end{equation}

Beispiel für $n = 3$
\begin{equation}
B = 
\begin{pmatrix}
0 & 1 & 0 & 0\\
0 & 0 & 2 & 0\\
0 & 0 & 0 & 3\\
0 & 0 & 0 & 0\\
\end{pmatrix}
\end{equation}

\subsection*{Teil 5}

Sei $v \in V$ überall $0$ und nur in der $n + 1$-ten Zeile $1$.
Dann ist $(B^{n}v, \dots, B^{2}, Bv, v)$ eine Jordankette.

\begin{proof}
Da $B$ eine obere Dreiecksmatrix ist, ist ihr einziger Eigenwert $\lambda = 0$ und $B - \lambda I_{n + 1} = B$.
Also ist $E_{k}(B, \lambda) = Ker(B^{k})$. 
Es ist zuzeigen, dass $v \in E_{n + 1}(B, 0) \setminus E_{n}(B, 0)$ ist, also $B^{n + 1}v = 0$, aber $B^{n}v \ne 0$.
Ich zeige es mit Induktion über $k$.

Sei $k = 0$.
Da $v \ne 0$ ist, ist $B^{0}v \ne 0$.
Außerdem ist $v$ überall $0$ außer in der $n + 1 - k$-ten Zeile.

Sei $0 < k < n + 1$, $w = B^{k - 1}v \ne 0$ und $w$ überall $0$ außer in der $n + 1 - (k - 1)$-ten Zeile.
Dann ist $x = Bw = (x_{m})$ mit $x_{m} = \sum_{q = 1}^{n + 1} b_{m, q} w_{q}$, aber $w_{q} \ne 0$ nur für $q = n + 1 - (k - 1)$, also $x_{m} = b_{m, n + 1 - (k - 1)} w_{n + 1 - (k - 1)}$.
Aber $b_{m, q} \ne 0$ nur für $q = m + 1 \Leftrightarrow m = q - 1$, also $x_{m} \ne 0$ nur für $m = q - 1 = n + 1 - k + 1 - 1 = n + 1 - k$.
Das heißt, $B^{k}v \ne 0$ und $B^{k}v$ ist überall $0$ außer in der $n + 1 - k$-ten Zeile.

Sei $k = n + 1$, $w = B^{n}v \ne 0$ und $w$ sei überall $0$ außer in der $1$-ten Zeile.
Dann ist $x = Bw = (x_{m})$ mit $x_{m} = \sum_{q = 1}^{n + 1} b_{m, q} w_{q}$, aber $w_{q} \ne 0$ nur für $q = 1$, also $x_{m} = b_{m, 1} w_{1}$.
Aber $b_{m, q} \ne 0$ nur für $q = m + 1 \Rightarrow m = 0$, aber $b_{0, 1}$ existiert nicht.
$b_{m, q}$ ist also immer $0$ somit auch $x$ und $BB_{n}v = 0$.

Damit sind alle Eigenschaften erfüllt, damit $(B^{n}v, \dots, B^{2}, Bv, v)$ eine Jordankette ist.
\end{proof}

\subsection*{Teil 6}

\section*{Übung 2}

\section*{Übung 3}

\subsection*{Teil 1}

\subsection*{Teil 2}

\subsection*{Teil 3}

\subsection*{Teil 4}

\subsection*{Teil 5}

\section*{Übung 4}

\subsection*{$A_{1}$}

\begin{equation}
\chi_{A_{1}} = x^{3} - 9x^{2} + 27x - 27 = (x - 3)^{3}
\end{equation}
\begin{equation}
\sigma(A_{1}) = \{3\}
\end{equation}

Jordankette zum Eigenwert $3$:
\begin{equation}
\left(
\begin{pmatrix}
1\\1\\1
\end{pmatrix},
\begin{pmatrix}
2\\0\\1
\end{pmatrix},
\begin{pmatrix}
2\\-1\\0
\end{pmatrix}
\right)
\end{equation}

\subsection*{$A_{2}$}

\begin{equation}
\chi_{A_{1}} = x^{3}
\end{equation}
\begin{equation}
\sigma(A_{1}) = \{0\}
\end{equation}

Jordankette zum Eigenwert $0$:
\begin{equation}
\left(
\begin{pmatrix}
1\\0\\0
\end{pmatrix},
\begin{pmatrix}
0\\1\\0
\end{pmatrix},
\begin{pmatrix}
0\\0\\1
\end{pmatrix}
\right)
\end{equation}

\subsection*{$A_{3}$}

\begin{equation}
\chi_{A_{1}} = (x - 3)^{3}
\end{equation}
\begin{equation}
\sigma(A_{1}) = \{3\}
\end{equation}

Jordankette zum Eigenwert $3$:
\begin{equation}
\left(
\begin{pmatrix}
-1\\1\\0
\end{pmatrix},
\begin{pmatrix}
-1\\0\\0
\end{pmatrix}
\right)
\end{equation}

\subsection*{$A_{4}$}

\begin{equation}
\chi_{A_{1}} = (x - 2)(x - 3)^{2}
\end{equation}
\begin{equation}
\sigma(A_{1}) = \{2, 3\}
\end{equation}

Jordankette zum Eigenwert $2$:
\begin{equation}
\left(
\begin{pmatrix}
1\\1\\1
\end{pmatrix}
\right)
\end{equation}

Jordankette zum Eigenwert $3$:
\begin{equation}
\left(
\begin{pmatrix}
-1\\1\\0
\end{pmatrix},
\begin{pmatrix}
-1\\0\\0
\end{pmatrix}
\right)
\end{equation}

\section*{Übung 5}

Weil es eine obere Dreiecksmatrix ist, ist auch $XI_{n} - J(\lambda, n)$ eine obere Dreiecksmatrix und es gilt
\begin{equation}
\chi_{J(\lambda, n)} = (X - \lambda)^{n}
\end{equation}

Für das Minimalpolynom gilt
\begin{equation}
\mu_{J(\lambda, 1)} = (X - \lambda)^{n}
\end{equation}

\begin{proof}
Sei $(v_{1}, \dots, v_{n})$ die Basis von $J(\lambda, n)$.
Dann ist die Basis eine Jordankette, weil $J(\lambda, n)$ ein Jordanblock ist.
Das heißt $v_{1}, \dots, v_{n} \in E_{n}(f, \lambda)$, aber $v_{k} \notin E_{n - 1}(f, \lambda)$ für $k \in [1, n]$.

Sei $k \in [1, n - 1]$ und $q \in [k, n - 1]$.
Dann ist
\begin{equation}
(X - \lambda)^{k}(J(\lambda, n))(v_{q}) \ne 0 \Rightarrow (X - \lambda)^{k}(J(\lambda, n)) \ne 0 \Rightarrow \mu_{J(\lambda, n)} \ne (X - \lambda)^{k}
\end{equation}
Sei $k = n$.
Dann gilt
\begin{equation}
(X - \lambda)^{k}(J(\lambda, n))(v_{q}) = 0 \quad \forall q \in [1, n] \Rightarrow (X - \lambda)^{k}(J(\lambda, n)) = 0
\end{equation}
$deg \mu_{J(\lambda, n)} = n$ ist also der kleinste Grad, sodass $(X - \lambda)^{k}(J(\lambda, n)) = 0$ und es gilt
\begin{equation}
\mu_{J(\lambda, n)} = (X - \lambda)^{n}
\end{equation}
\end{proof}

\end{document}