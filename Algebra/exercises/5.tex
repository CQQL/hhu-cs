\documentclass[10pt,a4paper]{article}
\usepackage[utf8]{inputenc}
\usepackage[german]{babel}
\usepackage{mathrsfs}
\usepackage{amsmath}
\usepackage{amsfonts}
\usepackage{amssymb}
\usepackage{amsthm}
\usepackage[left=2cm,right=2cm,top=2cm,bottom=2cm]{geometry}

\DeclareMathOperator{\id}{id}

\begin{document}

\section{Übung 1}

\begin{equation}
  |G| = 4! = 24 = 2^{3} \cdot 3
\end{equation}

\subsection{Teil 1}

Da $3^{1}$ die höchste Potenz von $3$ ist, die $|G|$ teilt, haben 3-Sylowuntergruppen von $G$ die Ordnung 3.
\begin{equation}
  H = \langle [143] \rangle = \{ \id, [143], [134] \}
\end{equation}
Dies ist eine Untergruppe, weil es die von $[143]$ erzeute Untergruppe ist, und eine 3-Sylowuntergruppe, weil die Ordnung von $[143]$ 3 ist.

\subsection{Teil 2}

Sei $k$ die Anzahl der 3-Sylowuntergruppen.
Dann teilt $k$ $2^{3}$ und ist $1$ modulo $3$.
Also ist $k \in \{ 1,4 \}$.
Da $\langle [142] \rangle$ eine andere Untergruppe der Ordnung $3$ ist, muss $k > 1$, also $k = 4$ sein.

\subsection{Teil 3}

$2$-Sylowuntergruppen von $G$ haben Ordnung $8$.

\subsection{Teil 4}

Sei $k$ die Anzahl der $2$-Sylowuntergruppen.
Dann ist $k \equiv 1 \pmod{2}$, also ungerade, und teilt 3.

\section{Übung 2}

\begin{proof}
  \begin{equation}
    |G| = 63 = 7 \cdot 3^{2}
  \end{equation}
  Sei $k$ die Anzahl der $7$-Sylowuntergruppen.
  Dann ist $k \equiv 1 \pmod{7}$ und $k\ \vert\ 3^{2}$.
  Also ist $k \in \{ 1, 8 \} \cap \{ 1, 3, 9 \} \Rightarrow k = 1$.
  Es gibt also genau eine $7$-Sylowuntergruppe und diese ist ein Normalteiler.
  Sie hat Ordnung $7$ und da $1 < 7 < 63$, ist $G$ nicht einfach.
\end{proof}

\section{Übung 3}

$\mathbb{H}$ ist auflösbar.
Ich werde eine Folge von Gruppen angeben, die die Definition von Auflösbarkeit erfüllen.

\begin{equation}
  G_{0} = \mathbb{H}, G_{1} = \langle I \rangle, G_{2} = \langle -1 \rangle, G_{3} = \{ 1 \}
\end{equation}

Dies sind alles Gruppen und jeweils Untergruppen von der vorherigen.
Die Ordnungen sind 8, 4, 2, 1.
Also sind die jeweiligen Quotienten isomorph zu $\mathbb{Z} / 2 \mathbb{Z}$ und kommutativ.
Somit ist $\mathbb{H}$ auflösbar.

\section{Übung 4}

\subsection{Teil 1}

\begin{proof}
  Sei $G = G_{1} \times \dots \times G_{r}$.
  Dann ist $G$ eine Gruppe mit der elementweisen Verknüpfung der Vektoren als Verknüpfung.

  $\Rightarrow$: Es gibt ein $m \in \mathbb{N}$, sodass $D^{m}(G) = \{ e_{G} \}$.
  Definiere $f_{i} : G \rightarrow G_{i}$ als Abbildung eines $g \in G$ auf seine $i$-te Komponente.
  Dies ist offensichtlich ein surjektiver Gruppenhomomorphismus.
  Nach einem Satz aus der Vorlesung ist dann
  \begin{equation}
    \{ e_{G_{i}} \} = f_{i}(\{ e_{G} \}) = f_{i}(D^{m}(G)) = D^{m}(f_{i}(G)) = D^{m}(G_{i})
  \end{equation}
  Also ist $G_{i}$ auflösbar.

  $\Leftarrow$: Seien $m_{i}$ so, dass $D^{m_{i}}(G_{i}) = \{ e_{G_{i}} \}$ für $i \in [1, r]$.
  Sei $m = \max \{ m_{i} \mid i \in [1, r] \}$.
  Nun kann man den gleichen Gruppenhomomorphismus wie im ersten Teil verwenden und erhält
  \begin{equation}
    D^{m}(G) = ???
  \end{equation}
\end{proof}

\subsection{Teil 2}

\begin{proof}
  Sei $\phi$ der Homomorphismus, mit dem das semidirekte Produkt gebildet wurde.

  $\Rightarrow$:

  $\Leftarrow$:
\end{proof}

\section{Übung 5}

\begin{proof}
  Da $D^{1}(G)$ von einer kommutativen Gruppe $G$ immer trivial ist, ist es wahr für alle kommutativen Gruppen.

  Sei also $G$ nicht kommutativ.
\end{proof}

\end{document}