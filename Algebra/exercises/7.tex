\documentclass[10pt,a4paper]{article}
\usepackage[utf8]{inputenc}
\usepackage[german]{babel}
\usepackage{mathrsfs}
\usepackage{amsmath}
\usepackage{amsfonts}
\usepackage{amssymb}
\usepackage{amsthm}
\usepackage[left=2cm,right=2cm,top=2cm,bottom=2cm]{geometry}

\begin{document}

\section{Übung 1}

$0$ und $R$ sind Ideale von $R$.
Wenn ein Ideal $I$ eine invertierbare Matrix $m$ enthält, ist $I = R$, weil $m(m^{-1}a) = a$ für alle $a \in R$.
Also können weitere Ideale nur aus nicht-invertierbaren Matrizen bestehen.
Sei $J \ne \{ 0 \}$ ein Ideal, dass eine nicht-invertierbare Matrix $j$ enthält.
Durch Links- und Rechtsmultiplikation mit Elementarmatrizen kann man eine Spalte und eine Zeile der Matrix mit $0$ multiplizieren, sodass diese nur noch einen Nichtnullwert $\alpha$ hat.
Dann multipliziert man diese Zeile mit $\alpha^{-1}$ und vertauscht Zeilen und Spalten, sodass die $1$ auf einer der beiden Diagonalpositionen steht.
Da Ideale unter Links- und Rechtsmultiplikation abgeschlossen sind, sind alle Zwischenergebnisse wieder in $J$ und man erhält, dass $E_{1,1}$ und $E_{2,2}$ beide in $J$ enthalten sind.
Dabei ist $E_{i,j}$ die Matrix, die nur an der Stelle $(i, j)$ eine $1$ und sonst überall $0$ hat.
Durch erneute Addition dieser beiden erhält man $I_{2}$.
Also enthält $J$ auch eine invertierbare Matrix und somit ist $J = R$.
Es gibt also keine weiteren Ideale.

\section{Übung 2}

Seien $a, b \in K[X]$ mit $ab = 1$.
Also gilt
\begin{equation}
  a_{0}b_{0} = 1
\end{equation}
und
\begin{equation}
  \sum_{i = 0}^{k} a_{i}b_{k - i} = 0\ \forall k \ge 1
\end{equation}

Es ist $a_{k} = 0$ für alle $k \ge 1$ und $(K[X])^{\times} = K^{\times}$.

\begin{proof}
\end{proof}

\section{Übung 3}

\subsection{Teil 1}

\subsection{Teil 2}

\subsection{Teil 3}

\section{Übung 4}

\subsection{Teil 1}

\subsection{Teil 2}

\begin{proof}
  Dass $R$ eine Gruppe mit $+$ ist, sieht man direkt.
  Ebenso sieht man, dass $1 + 0i$ das neutrale Element der Multiplikation ist und $R$ unter Multiplikation abgeschlossen ist, da $\mathbb{Z}$ unter Addition und Multiplikation abgeschlossen ist.
  $R$ ist kommutativ, da die Addition und Multiplikation der komplexen Zahlen kommutativ ist.
  $R$ ist auch nicht nur das Nullelement.

  Sei $a + bi \ne 0$.
  \begin{equation}
    (a + bi)(c + di) = 0 \Leftrightarrow ac = bd \land ad = -bc
  \end{equation}
\end{proof}

\subsection{Teil 3}

\subsection{Teil 4}

\subsection{Teil 5}

\end{document}