\documentclass[10pt,a4paper]{article}
\usepackage[utf8]{inputenc}
\usepackage[german]{babel}
\usepackage{mathrsfs}
\usepackage{amsmath}
\usepackage{amsfonts}
\usepackage{amssymb}
\usepackage{amsthm}
\usepackage[left=2cm,right=2cm,top=2cm,bottom=2cm]{geometry}

\begin{document}

\section{Aufgabe 1}

\section{Aufgabe 2}

\subsection{Teil 1}

\begin{equation}
  P(0) = 1
\end{equation}
\begin{equation}
  P(1) = 1
\end{equation}
Da $P$ keine Nullstelle hat, ist es irreduzibel.

\subsection{Teil 2}

\subsection{Teil 3}

\section{Aufgabe 3}

\subsection{Teil 1.a}

\begin{equation}
  P(x) = 0 \Leftrightarrow x = \frac{-a \pm \sqrt{a^{2} - 4b}}{2} = \frac{-a \pm \sqrt{\Delta}}{2}
\end{equation}

\subsection{Teil 1.b}

\begin{proof}
  Wenn $\sqrt{\Delta} \in \mathbb{Q}$, sind die Nullstellen auch in $\mathbb{Q}$ und der Grad ist $1$ und $K = \mathbb{Q}$.

  Sei $\sqrt{\Delta} \not\in \mathbb{Q}$.
  Dann ist $K = \mathbb{Q}(\frac{-a + \sqrt{\Delta}}{2}, \frac{-a - \sqrt{\Delta}}{2})$.
  Betrachte $A := \mathbb{Q}(\sqrt{\Delta})$.
  Die Enthaltung $K \subset A$ ist offensichtlich richtig.
  \begin{equation}
    \frac{-a + \sqrt{\Delta}}{2} - \frac{-a - \sqrt{\Delta}}{2} = \sqrt{\Delta} \in K
  \end{equation}
  Somit ist $K = \mathbb{Q}(\sqrt{\Delta})$ und der Grad $2$.
\end{proof}

\subsection{Teil 2.a}

\begin{proof}
  \begin{equation}
    Q(X) = P(X^{2})
  \end{equation}
  Also sind die Nullstellen von $Q$ die Wurzeln der Nullstellen von $P$.
  \begin{equation}
    \pm \sqrt{\alpha} = \pm \sqrt{\frac{-a + \sqrt{\Delta}}{2}}
  \end{equation}
  \begin{equation}
    \pm \sqrt{\beta} = \pm \sqrt{\frac{-a - \sqrt{\Delta}}{2}}
  \end{equation}
  Also ist $L = \mathbb{Q}(\sqrt{\alpha}, -\sqrt{\alpha}, \sqrt{\beta}, -\sqrt{\beta})$.
  Dann ist ja auch offensichtlich $L = \mathbb{Q}(\sqrt{\alpha}, \sqrt{\beta})$.
\end{proof}

\subsection{Teil 2.b}

\subsection{Teil 2.c}

\subsection{Teil 2.d}

\end{document}