\documentclass[10pt,a4paper]{article}
\usepackage[utf8]{inputenc}
\usepackage[german]{babel}
\usepackage{mathrsfs}
\usepackage{amsmath}
\usepackage{amsfonts}
\usepackage{amssymb}
\usepackage{amsthm}
\usepackage[left=2cm,right=2cm,top=2cm,bottom=2cm]{geometry}

\begin{document}

\section{Übung 1}

\subsection{Teil 1}

\subsection{Teil 2}

\section{Übung 2}

\subsection{Teil 1}

\begin{proof}
  Die erste Gleichung ist genau das fünfte Kreisteilungspolynom, dessen Nullstellen genau die fünften Einheitswurzeln sind, mit der Einsetzung von $\zeta$.

  \begin{equation}
    \alpha^{2} + \alpha = \zeta^{2} + 2 + \frac{1}{\zeta^{2}} + \zeta + \frac{1}{\zeta} = (\zeta^{2} + \zeta + 1 + \frac{1}{\zeta}  + \frac{1}{\zeta^{2}}) + 1 = 1
  \end{equation}

  Aber der Klammerausdruck ist schon $0$ und somit ist es wahr.
  \begin{equation}
    (\zeta^{2} + \zeta + 1 + \frac{1}{\zeta}  + \frac{1}{\zeta^{2}}) = \frac{1}{\zeta^{2}}(\zeta^{4} + \zeta^{3} + \zeta^{2} + \zeta + 1) = \frac{1}{\zeta^{2}} \cdot 0 = 0
  \end{equation}
\end{proof}

\subsection{Teil 2}

\subsection{Teil 3}

\section{Übung 3}

\subsection{Teil 1}

\begin{proof}
  $1$ als neutrales Element ist enthalten, weil $1^{2} = 1$.
  Wenn $a^{2}, b^{2} \in (\mathbb{F}_{p}^{\times})^{2}$, dann auch $(ab)^{2} \in (\mathbb{F}_{p}^{\times})^{2}$, weil $a^{2} \cdot b^{2} = (ab)^{2}$.

  Seien $a, b \in \mathbb{F}_{p}^{times}$.
  \begin{equation}
    q(a \cdot b) = (a \cdot b)^{2} = a^{2} \cdot b^{2} = q(a) \cdot q(b)
  \end{equation}
\end{proof}

\subsection{Teil 2}

\begin{proof}
  \begin{equation}
    [x] \in \ker q \Leftrightarrow [x^{2}] = [1] \Leftrightarrow \exists n \in \mathbb{N}_{0} : x^{2} = np + 1 \Leftrightarrow p \mid (x^{2} - 1) \Leftrightarrow x \in \{ 1, p - 1 \} = \{ 1, -1 \}
  \end{equation}

  Zuerst sieht man, dass $|\mathbb{F}_{p}^{\times}| = p - 1$.
  Sei $n \in [1, \frac{p - 1}{2}]$.
  \begin{equation}
    q([n]) = [n]^{2} = [n^{2}]
  \end{equation}
  \begin{equation}
    q([p - n]) = [p - n]^{2} = [(p - n)^{2}] = [p^{2} - 2pn + n^{2}] = [n^{2}]
  \end{equation}
  $q$ bildet also $[n]$ und $[p - n]$ auf dasselbe ab.
  Somit ist
  \begin{equation}
    |(\mathbb{F}_{p}^{\times})^{2}| = \frac{|\mathbb{F}_{p}^{\times}|}{2} = \frac{p - 1}{2}
  \end{equation}

  Da $\mathbb{F}_{p} \setminus \mathbb{F}_{p}^{\times} = \{ 0 \}$ und $0 \not\in (\mathbb{F}_{p}^{\times})^{2}$ und $0^{2} = 0$, ist $|\mathbb{F}_{p}^{2}| = |(\mathbb{F}_{p}^{\times})^{2}| + 1 = \frac{p + 1}{2}$.
\end{proof}

\subsection{Teil 3}

\begin{proof}


  Seien $a, b \in \mathbb{F}_{p}^{\times}$.
  \begin{equation}
    r(a \cdot b) = (a \cdot b)^{\frac{p - 1}{2}} = a^{\frac{p - 1}{2}} \cdot b^{\frac{p - 1}{2}} = r(a) \cdot r(b)
  \end{equation}
  Also ist $r$ ein Gruppenhomomorphismus.
\end{proof}

\subsection{Teil 4}

\begin{proof}
  Die linke Gleichung ist ja direkt die Definition des Kerns.
  Man betrachte also die rechte Gleichung.

  $\subset$: Sei $x \in \ker r$.
  \begin{align*}
    r(x) = 1 \Rightarrow & x^{\frac{p - 1}{2}} = 1 = x^{p - 1} = x^{\frac{p - 1}{2} + \frac{p - 1}{2}} = x^{\frac{p - 1}{2}} \cdot x^{\frac{p - 1}{2}} = (x^{\frac{p - 1}{2}})^{2} \Rightarrow x^{\frac{p - 1}{2}} \in (\mathbb{F}_{p}^{\times})^{2}
  \end{align*}

  $\supset$: Sei $x^{2} \in (\mathbb{F}_{p}^{\times})^{2}$.
  \begin{equation}
    r(x) = (x^{2})^{\frac{p - 1}{2}} = x^{\frac{p - 1}{2} \cdot 2} = x^{p - 1} = 1 \Rightarrow x \in \ker r
  \end{equation}
\end{proof}

\subsection{Teil 5}

\begin{proof}
  $\Rightarrow$: Sei $-1 \in \mathbb{F}_{p}^{2}$.
  Es gibt also ein $x \in \mathbb{F}_{p}$ mit $[x^{2}] = [-1] = [p - 1]$.
  Somit ist $p \equiv x^{2} + 1 \pmod $

  $\Leftarrow$: Sei $p \equiv 1 \pmod 4$.
  Dann ist $p = 4m + 1$ für ein $m \in \mathbb{N}_{0}$.
  Es ist $[-1] = [p - 1] = [4m + 1 - 1] = [4m]$.
  Es ist also zu zeigen, dass es ein $x \in \mathbb{F}_{p}$ gibt, mit $x^{2} = 4m$.
\end{proof}

\subsection{Teil 6}

$X^{2} + 1$ ist genau dann irreduzibel, wenn es keine Nullstellen in $\mathbb{F}_{p}[X]$ hat, also wenn $-1 \not\in (\mathbb{F}_{p}^{\times})^{2}$.
Nach Teil 5 ist dies wiederum genau dann der Fall, wenn $p \not\equiv 1 \pmod 4$.

\subsection{Teil 7}

\begin{proof}
  Sei $q = (n!)^{2} + 1 = 1 + \prod_{i = 1}^{n} i^{2}$.
  Wenn $p \le n$, teilt $p$ $\prod_{i = 1}^{n} i^{2}$, aber dann kann $p$ $q$ nicht teilen, weil $p$ $1$ nicht teilt.
  Also muss $p > n$ sein.

  Man betrachte nun $\mathbb{F}_{p}$.
  Da $p$ $(n!)^{2} + 1$ teilt, ist $[(n!)^{2} + 1] = [0]$ und $[(n!)^{2}] = [-1]$.
  Es gibt also ein Element $[(n!)^{2}] \in \mathbb{F}_{p}^{2}$, dessen Klasse $[-1]$ ist.
  Dann ist $p \equiv 1 \pmod 4$ nach Teil 5.
  Also ist $p$ der Form $4m + 1$ für ein $m \in \mathbb{N}$.

  Es gilt natürlich auch $p \le (n!)^{2} + 1$, da $p$ ja sonst $(n!)^{2} + 1$ nicht teilen könnte.
  Induktiv sieht man dann, dass es unendlich viele Primzahlen dieser Form geben muss.
  Für $n = 1$ ist $q = 2$ und es gibt kein $p$, das die Bedingung $p > 2$ erfüllt.
  Für $n = 2$ ist $q = 5$ und $p = 5 = 4 + 1 \in [1, q]$.
  Angenommen es gibt ein $n \in \mathbb{N}$ so, dass es $k$ Primzahlen der Form $4m + 1$ in $[1, n]$ gibt, so gibt es $k + 1$ solcher Primzahlen in $[1, (n!)^{2} + 1]$.
  Damit ist es gezeigt.
\end{proof}

\end{document}