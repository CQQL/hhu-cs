\documentclass[10pt,a4paper]{article}
\usepackage[utf8]{inputenc}
\usepackage[german]{babel}
\usepackage{mathrsfs}
\usepackage{amsmath}
\usepackage{amsfonts}
\usepackage{amssymb}
\usepackage{amsthm}
\usepackage[left=2cm,right=2cm,top=2cm,bottom=2cm]{geometry}

\DeclareMathOperator{\Ker}{Ker}

\begin{document}

\section{Übung 1}

\begin{proof}
  \begin{equation}
    e_{G} \in H_{i} \forall i \in I \Rightarrow e_{G} \in \cap_{i \in I} H_{i}
  \end{equation}

  Seien $a, b \in \cap_{i \in I} H_{i}$.
  \begin{equation}
    ab^{-1} \in H_{i} \forall i \in I \Rightarrow ab^{-1} \in \cap_{i \in I} H_{i}
  \end{equation}
\end{proof}

\section{Übung 2}

\begin{proof}
  $\Rightarrow$:
  Sei $\mathbb{Z} / p \mathbb{Z}$ einfach.
  Wenn $p$ nicht prim wäre, hätte $p$ einen Teiler $k \not\in \{ 1, p \}$.
  $\{ k \cdot n \mid 0 \le n \le \frac{k}{p} \}$ wäre dann eine weitere Untergruppe und $\mathbb{Z} / p \mathbb{Z}$ wäre nicht einfach.

  $\Leftarrow$:
  Sei $p$ eine Primzahl und $A := \mathbb{Z} / p \mathbb{Z}$.
  Dann ist $|A| = p$.
  Sei $H$ eine Untergruppe von $A$.
  Nach dem Satz von Lagrange ist $|A| = |H| \cdot (A : H)$.
  Weil $p$ prim ist, musst also $|H|$ $1$ oder $p$ sein.
  Wenn $|H| = p$ ist, ist $H = A$.
  Wenn $|H| = 1$ ist, musst $H = \{ e_{A} \}$ sein, weil andere Elemente alleine keine Untergruppe bilden können (weil dann ja $e_{A}$ nicht enthalten sein kann).
  Also ist $A$ einfach.
\end{proof}

\section{Übung 3}

\subsection{Teil 1}

\begin{proof}
  Offensichtlich ist $Id$ in beiden enthalten.

  $H$:
  \begin{align*}
    (123)(123) = (213)(213) & = (123) \in H\\
    (123)(213) = (213)(123) & = (213) \in H
  \end{align*}

  $A$:
  \begin{align*}
    (123)(123) = (231)(231) = (312)(312) & = (123) \in A\\
    (123)(231) = (231)(123) = (231)(312) & = (231) \in A\\
    (123)(312) = (312)(123) = (312)(231) & = (312) \in A
  \end{align*}

  Also sind in beiden die Inversen enthalten und sie sind abgeschlossen.
\end{proof}

\subsection{Teil 2}

\begin{proof}
  Da $G$ 6 Elemente hat, muss $N_{G}(H)$ als Untergruppe von $G$ nach dem Satz von Lagrange 1, 2, 3 oder 6 Elemente haben.
  Da $H$ wiederum eine Untergruppe von $N_{G}(H)$ ist, muss die Anzahl der Elemente von $N_{G}(H)$ ein Vielfaches von 2 sein.
  In der Vorlesung haben wir gesehen, dass $H$ nicht Normalteiler von $G$ ist, sodass nur 2 übrig bleibt und $N_{G}(H) = H$ folgt.

  Wir betrachten $(132) \in S_{3}$.
  \begin{equation}
    (132)^{-1} = (132)
  \end{equation}
  \begin{equation}
    (132)(123)(132) = (123) \in A_{3}
  \end{equation}
  \begin{equation}
    (132)(231)(132) = (213)(132) = (312) \in A_{3}
  \end{equation}
  \begin{equation}
    (132)(312)(132) = (321)(132) = (231) \in A_{3}
  \end{equation}
  Da $A_{3} \subset N_{G}(A_{3})$ und $N_{G}(A_{3}) \subset G$, muss $N_{G}(A_{3})$ mindestens 3 Elemente enthalten und gleichzeitig entweder 1, 2, 3 oder 6.
  Es enthält aber noch ein weiteres Element $(132)$, das nicht in $A_{3}$ enthalten ist.
  Somit muss $N_{G}(A_{3})$ 6 Elemente haben und $N_{G}(A_{3}) = G = S_{3}$ sein.
\end{proof}

\subsection{Teil 3}

\begin{proof}

\end{proof}

\subsection{Teil 4}

Wie aus der Vorlesung bekannt, ist $n \mathbb{Z} \triangleleft \mathbb{Z}$.

\begin{proof}
  $\{ \pm 1 \}$ ist eine Gruppe mit neutralem Element $1$ und der Verknüpfung Multiplikation $\cdot$.

  Definiere $f : \mathbb{Z} \rightarrow \{ \pm 1 \}$ durch
  \begin{equation}
    f(z) := (-1)^{z}
  \end{equation}
  \begin{equation}
    f(0) = (-1)^{0} = 1
  \end{equation}
  und für $a, b \in \mathbb{Z}$
  \begin{equation}
    f(a + b) = (-1)^{a + b} = (-1)^{a}(-1)^{b} = f(a)f(b)
  \end{equation}
  ist $f$ ein Gruppenhomomorphismus.

  Sei $a' = 2a \in 2 \mathbb{Z}$.
  \begin{equation}
    f(a') = (-1)^{a'} = (-1)^{2a} = 1^{a} = 1
  \end{equation}
  Sei $a \in \Ker(f)$.
  \begin{equation}
    1 = f(a) = (-1)^{a} \Leftrightarrow a \in 2 \mathbb{Z}
  \end{equation}
  Also ist $\Ker(f) = 2 \mathbb{Z}$.

  \begin{equation}
    f(0) = 1, f(1) = -1
  \end{equation}
  Somit ist $f$ surjektiv.

  Damit ist die Abbildung $\bar{f} : \mathbb{Z} / 2 \mathbb{Z} \rightarrow \{ \pm 1 \}$ definiert durch
  \begin{equation}
    \bar{f}([n]) = f(n)
  \end{equation}
  nach Satz 1.2.10 ein Gruppenisomorphismus.
\end{proof}

\begin{proof}
  Definiere $f : \mathbb{Z} \rightarrow A^{3}$ durch
  \begin{equation}
    f(n) := \sigma^{n}
  \end{equation}
  \begin{equation}
    f(0) := \sigma^{0} = (123)
  \end{equation}
  Seien $a, b \in \mathbb{Z}$
  \begin{equation}
    f(a + b) = \sigma^{a + b} = \sigma^{a} \sigma^{b} = f(a)f(b)
  \end{equation}
  Also ist $f$ ein Gruppenhomomorphismus.

  Diese Abbildung kann jedoch kein Isomorphismus sein, da $|\mathbb{Z} / 3 \mathbb{Z}| = 3$, aber $|\bar{f}(\mathbb{Z} / 3 \mathbb{Z})| = 2$.
  Außerdem ist $\bar{f}$ nicht wohldefiniert, weil $[0] = [3]$, aber $\bar{f}([0]) \ne \bar{f}([3])$.
\end{proof}

\section{Übung 4}

\subsection{Teil 1}

\begin{proof}
  Das neutrale Element ist offensichtlich enthalten.

  Ebenfalls ist für jedes Element ein inverses enthalten.
  \begin{equation}
    1 = 1 \cdot 1 = -1 \cdot -1 = K \cdot -K = J \cdot -J = I \cdot -I = -K \cdot K = -J \cdot J = -I \cdot I
  \end{equation}

  Für die Abgeschlossenheit werde ich nur die Produkte von $I, J, K$ betrachten, weil Produkte mit $1$ auf jeden Fall drin sind und etwaige Vorzeichen davor gezogen werden können.
  \begin{equation}
    I \cdot  I = -1
  \end{equation}
  \begin{equation}
    I \cdot  J = K
  \end{equation}
  \begin{equation}
    I \cdot K = -J
  \end{equation}
  \begin{equation}
    J \cdot I = -K
  \end{equation}
  \begin{equation}
    J \cdot J = -1
  \end{equation}
  \begin{equation}
    J \cdot K = I
  \end{equation}
  \begin{equation}
    K \cdot I = J
  \end{equation}
  \begin{equation}
    K \cdot J = -I
  \end{equation}
  \begin{equation}
    K \cdot K = -1
  \end{equation}
\end{proof}

\subsection{Teil 2}

Nein, weil
\begin{equation}
  I \cdot J = K \ne -K = J \cdot I
\end{equation}

\subsection{Teil 3}

Es gibt natürlich die triviale Untergruppe und $\mathbb{H}$.
Dazu gibt es genau eine 2-elementige Untergruppe $\{ 1, -1 \}$, weil die anderen Matrizen ihre Negativen als Inverse erfordern.
Die 4-Elementigen sind jeweils durch $\{ 1, -1 \}$ und eine weitere Matrix mit ihrem Negativen gegeben: $\{ 1, -1, I, -I \}, \{ 1, -1, J, -J \}, \{ 1, -1, K, -K \}$.
Nach dem Satz von Lagrange kann es keine Untergruppen anderer Ordnung geben, da die Teiler von 8 $1, 2, 4, 8$ sind.

\end{document}