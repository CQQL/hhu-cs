\documentclass[10pt,a4paper]{article}
\usepackage[utf8]{inputenc}
\usepackage[german]{babel}
\usepackage{mathrsfs}
\usepackage{amsmath}
\usepackage{amsfonts}
\usepackage{amssymb}
\usepackage{amsthm}
\usepackage[left=2cm,right=2cm,top=2cm,bottom=2cm]{geometry}

\DeclareMathOperator{\id}{Id}
\newtheorem{lemma}{Lemma}
\newtheorem{corolarry}{Korollar}

\begin{document}

\section{Übung 1}

Wenn eine Matrix $m$ in $Z(\mathbb{H})$ ist, ist auch ihr Negatives $-m$ enthalten, weil $mp = pm\ \forall p \in \mathbb{H} \Rightarrow (-m)p = -mp = -pm = p(-m)\ \forall p \in \mathbb{H}$.
Durch Einsetzung von $-m$ für $m$ erhält man die andere Richtung und dies ist eine gdw. Beziehung.
Und wenn eine Matrix $m$ mit einer Matrix $p$ kommutiert, so kommutiert sie auch mit $-p$:
\begin{equation}
  mp = pm \Rightarrow m(-p) = -(mp) = -(pm) = (-p)m
\end{equation}
Wenn man $p$ durch $-p$ ersetzt, erhält man die andere Richtung und dies ist ebenfalls eine gdw. Beziehung.

Es ist auf jeden Fall $\id$ in $Z(\mathbb{H})$ und somit auch $-\id$.
Nach den obigen Argumenten bleibt also nur zu überprüfen, ob $I$, $J$ und $K$ mit $I$, $J$ und $K$ kommutieren.
Dabei kam auf Blatt 1 heraus
\begin{equation}
  I \cdot  I = -1
\end{equation}
\begin{equation}
  I \cdot  J = K
\end{equation}
\begin{equation}
  I \cdot K = -J
\end{equation}
\begin{equation}
  J \cdot I = -K
\end{equation}
\begin{equation}
  J \cdot J = -1
\end{equation}
\begin{equation}
  J \cdot K = I
\end{equation}
\begin{equation}
  K \cdot I = J
\end{equation}
\begin{equation}
  K \cdot J = -I
\end{equation}
\begin{equation}
  K \cdot K = -1
\end{equation}
Wie man sieht, kommutieren diese jedoch alle nur jeweils mit sich selbst.
Also kommutieren ihre Negativen auch nicht mit allen Elementen von $\mathbb{H}$.

Es bleibt übrig
\begin{equation}
  Z(\mathbb{H}) \{ 1, -1 \}
\end{equation}

\section{Übung 2}

\begin{proof}
  Wir haben in der Vorlesung gesehen, dass $g^{|\langle g \rangle|} = e_{G}$.
  Weiterhin ist $\langle g \rangle$ die kleinste Untergruppe, die $g$ enthält, und somit eine Untergruppe von $G$.
  Nach Korollar 1.1.28 ist $|G|$ ein Vielfaches von $|\langle g \rangle|$, also $|G| = h \cdot |\langle g \rangle|$ für ein $h \in \mathbb{N}$.
  \begin{equation}
    g^{|G|} = (g^{|\langle g \rangle|})^{h} = e_{G}^{h} = e_{G}
  \end{equation}
\end{proof}

\section{Übung 3}

\section{Übung 4}

\subsection{Teil 1}

\begin{equation}
  Z(S_{1}) = S_{1}
\end{equation}

\begin{equation}
  Z(S_{2}) = S_{2}
\end{equation}

\subsection{Teil 2}

\begin{equation}
  \sigma \tau_{i, k}(i) = \sigma(k) \ne j \textit{ weil $k \ne i$}
\end{equation}
\begin{equation}
  \tau_{i, k} \sigma(i) = \tau_{i, k}(j) = j
\end{equation}

\subsection{Teil 3}

\begin{proof}
  Sei $\sigma \ne \id \in S_{n}$.
  Dann gibt es $i, j \in [1, n]$ mit $\sigma(i) = j$ und $i \ne j$, weil sonst $\sigma = \id$ wäre.
  Also sind die Bedingungen von Teil 2 erfüllt und $\sigma$ kommutiert nicht mit einem solchen $\tau_{i, k}$.
  Somit ist $Z(S_{n}) = \{ \id \}$, weil $\id$ per Definition mit allem kommutiert.
\end{proof}

\section{Übung 5}

\subsection{Teil 1}

\begin{lemma}
  Seien $a = (a_{i, j}), b = (b_{i, j})$ zwei obere $n \times n$ Dreiecksmatrizen.
  Dann ist die Diagonale von $c := ab$ gegeben durch
  \begin{equation}
    c_{i, i} = a_{i, i} \cdot b_{i, i}
  \end{equation}
\end{lemma}

\begin{proof}
  \begin{align*}
    c_{i, i} & = \sum_{k = 1}^{n} a_{i, k} b_{k, i}\\
    & = \sum_{k = i}^{n} a_{i, k} b_{k, i}\textit{ denn $a_{i, k} = 0$ für $k < i$}\\
    & = \sum_{k = i}^{i} a_{i, k} b_{k, i}\textit{ denn $b_{i, k} = 0$ für $k > i$}\\
    & = a_{i, i} b_{i, i}
  \end{align*}
\end{proof}

\begin{corollary}
  Sei $a = (a_{i, j})$ eine invertierbare, obere $n \times n$ Dreiecksmatrix und $a' = (a'_{i, j})$ ihr Inverses.
  Dann ist $a'_{i, i} = a_{i, i}^{-1}$ für $i \in [1, n]$.
\end{corollary}

\begin{proof}
  Es ist $c := a \cdot a' = 1$.
  Nach dem vorherigen Lemma gilt
  \begin{equation}
    c_{i, i} = a_{i, i} \cdot a'_{i, i} = 1 \Leftrightarrow a'_{i, i} = \frac{1}{a_{i, i}} = a_{i, i}^{-1}
  \end{equation}
\end{proof}

\begin{proof}
  Indizes sind hier immer zwischen 1 und $n$, sofern nicht anders angegeben.
  Sei $b = (b_{i, j}) \in B$ und $u = (u_{i, j}) \in U$.
  Sei $b' = (b'_{i, j}) \in B$ das Inverse von $B$.
  Da die oberen Dreiecksmatrizen eine Untergruppe bilden, ist $b'$ ebenfalls eine obere Dreiecksmatrix.
  Definiere $q := bub'$.
  Aus demselben Grund wie bei $b'$ ist auch $q$ eine obere Dreiecksmatrix.
  Zu zeigen bleibt, dass $q_{i, i} = 1$ für alle $i \in [1, n]$.
  Durch zweimaliges Anwenden des obigen Lemmas erhält man
  \begin{equation}
    q_{i, i} = b_{i, i} \cdot u_{i, i} \cdot b'_{i, i} = b_{i, i} \cdot b'_{i, i} = b_{i, i} \cdot b_{i, i}^{-1} = 1
  \end{equation}
  Also ist $bub' \in U$ und $bUb' \subset U$, also $U \triangleleft B$.

  Definiere $diag : B \rightarrow T$ durch die Abbildung einer oberen Dreiecksmatrix auf ihre Hauptdiagonale.
  Seien $a, b \in B$.
  Nach obigem Lemma gilt
  \begin{equation}
    diag(ab) = diag(a) \cdot diag(b)
  \end{equation}
  $T$ ist eine Gruppe mit der Einheitsmatrix als neutralem Element und der Matrixmultiplikation als Verknüpfung.
  Somit ist $diag$ ein Gruppenhomomorphismus.
  $diag$ ist auch surjektiv, da $T$ eine Untergruppe von $B$ ist und $diag$ Elemente von $T$ auf sich selbst abbildet.

  Der Kern von $diag$ sind genau die Matrizen, die nur $1$ auf der Diagonalen haben, also $U$.

  Mit Korollar 1.2.11 folgt $B / U \simeq T$.
\end{proof}

\subsection{Teil 2}

\subsection{Teil 3}

\subsection{Teil 4}

\end{document}