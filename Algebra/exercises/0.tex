\documentclass[10pt,a4paper]{article}
\usepackage[utf8]{inputenc}
\usepackage[german]{babel}
\usepackage{mathrsfs}
\usepackage{amsmath}
\usepackage{amsfonts}
\usepackage{amssymb}
\usepackage{amsthm}
\usepackage[left=2cm,right=2cm,top=2cm,bottom=2cm]{geometry}

\begin{document}

\section{Übung 1}

\subsection{Teil 1}

\begin{proof}
  \begin{equation}
    f(e_{H}) = f(e_{G}) = e_{G'} \Rightarrow e_{G'} \in f(H)
  \end{equation}
  Seien $a', b' \in f(H)$.
  Es existieren $a, b \in H$ mit $f(a) = a'$ und $f(b) = b'$.
  \begin{equation}
    ab = f(a)f(b) = f(ab) \in f(H)
  \end{equation}

  Sei $a' \in f(H)$ und $a \in H$ mit $f(a) = a'$.
  \begin{equation}
    a'^{-1} = f(a)^{-1} = f(a^{-1}) \in f(H)
  \end{equation}
\end{proof}

\subsection{Teil 2}

\begin{proof}
  Was ist mit der Abbildung $f(h) = e_{G'}$?
  Dort lässt sich $f^{-1}(x)$ nicht gut ein Wert zuweisen.
\end{proof}

\section{Übung 2}

\begin{proof}
  Definiere $Int^{-1}_{g}$ durch
  \begin{equation}
    Int^{-1}_{g}(h) = g^{-1}hg
  \end{equation}
  Dann ist
  \begin{equation}
    Int^{-1}_{g}(Int_{g}(h)) = Int^{-1}_{g}(ghg^{-1}) = g^{1}ghg^{-1}g = h
  \end{equation}
  und
  \begin{equation}
    Int_{g}(Int^{-1}_{g}(h)) = Int_{g}(g^{-1}hg) = gg^{1}hgg^{-1} = h
  \end{equation}
\end{proof}

\section{Übung 3}

\subsection{Teil 1}

\begin{proof}
  \begin{enumerate}
    \item{Reflexiv} Wähle $h = e_{h} \in H$
    \item{Transitiv} Seien $a, b, c \in G$ mit $a \sim b$ und $b \sim c$, also $b = ah_{1}$ und $c = bh_{2}$.
      Dann ist
      \begin{equation}
        c = bh_{2} = ah_{1}h_{2}
      \end{equation}
      mit $h_{1}h_{2} \in H$ und $a \sim c$.
    \item{Symmetrisch} Seien $a, b \in G$ mit $a \sim b$, also $b = ah$ für ein $h \in H$.
      \begin{equation}
        b = ah \Leftrightarrow a = bh^{-1}
      \end{equation}
      mit $h^{-1} \in H$ und $b \sim a$.
  \end{enumerate}
\end{proof}

\subsection{Teil 2}

\begin{equation}
  \{ g' \mid g \sim g' \} = \{ gh \in G \mid h \in H \}
\end{equation}

\begin{proof}
  $\subset$:
  Sei $g' \in \{ g' \mid g \sim g' \}$.
  \begin{equation}
    \exists h \in H : g' = gh \Rightarrow g' \in \{ gh \in G \mid h \in H \}
  \end{equation}

  $\supset$:
  Sei $gh \in \{ gh \in G \mid h \in H \}$.
  \begin{equation}
    g = ghe_{H} \Rightarrow gh \sim g \Rightarrow g \sim gh \Rightarrow gh \in \{ g' \mid g \sim g' \}
  \end{equation}
\end{proof}

\section{Übung 4}

\begin{proof}
  Sei $H$ eine Untergruppe von $G$.

  $\Rightarrow$:
  Sei $H$ Normalteiler.
  Dann gilt für jedes $g \in G$ und $h \in H$
  \begin{equation}
    ghg^{-1} = h \Leftrightarrow gh = hg
  \end{equation}
  also $gH = Hg$.

  $\Leftarrow$:
  Es gelte $gH = Hg$ für alle $g \in G$.
  Sei $h \in H$.
  Dann ist $gh \in gH$, aber auch $gh \in Hg$.
\end{proof}

\section{Übung 5}

\subsection{Teil 1}

\begin{proof}
  \begin{equation}
    n \cdot 0 = 0 \in n \mathbb{Z}
  \end{equation}
  Seien $a, b \in n \mathbb{Z}$.
  Dann ist $a = n \cdot a'$ und $b = n \cdot b'$.
  \begin{equation}
    a + b = (n \cdot a') + (n \cdot b') = n \cdot (a' + b') \in n \mathbb{Z}
  \end{equation}
\end{proof}

\subsection{Teil 2}

\begin{proof}
  Sei $H$ eine Untergruppe von $(\mathbb{Z}, +)$.

\end{proof}

\subsection{Teil 3}

\subsection{Teil 4}

\end{document}