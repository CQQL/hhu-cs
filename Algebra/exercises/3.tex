\documentclass[10pt,a4paper]{article}
\usepackage[utf8]{inputenc}
\usepackage[german]{babel}
\usepackage{mathrsfs}
\usepackage{amsmath}
\usepackage{amsfonts}
\usepackage{amssymb}
\usepackage{amsthm}
\usepackage[left=2cm,right=2cm,top=2cm,bottom=2cm]{geometry}

\DeclareMathOperator{\Id}{Id}
\DeclareMathOperator{\sig}{sig}

\begin{document}

\section{Übung 1}

\subsection{Teil 1}

\begin{proof}
  \begin{equation}
    \Id \in V_{4}
  \end{equation}
  \begin{equation}
    \sig(\Id) = \sig(2143) = \sig(3412) = \sig(4321) = 1 \Rightarrow V_{4} \subset A_{4}
  \end{equation}
  \begin{equation}
    \Id^{-1} = \Id
  \end{equation}
  \begin{equation}
    (2143)(2143) = \Id
  \end{equation}
  \begin{equation}
    (2143)(3412) = (4321)
  \end{equation}
  \begin{equation}
    (2143)(4321) = (3412)
  \end{equation}
  \begin{equation}
    (3412)(2143) = (4321)
  \end{equation}
  \begin{equation}
    (3412)(3412) = \Id
  \end{equation}
  \begin{equation}
    (3412)(4321) = (2143)
  \end{equation}
  \begin{equation}
    (4321)(2143) = (3412)
  \end{equation}
  \begin{equation}
    (4321)(3412) = (2143)
  \end{equation}
  \begin{equation}
    (4321)(4321) = \Id
  \end{equation}
  Es ist also auch abgeschlossen und alle Elemente sind ihre eigenen Inversen.
  Damit ist $V_{4}$ eine Untergruppe von $A_{4}$.

  Sei $a \in A_{4}$ und $s \in S_{4}$.
  \begin{equation}
    \sig(sas^{1}) = \sig(s)\sig(a)\sig(s^{-1}) = \sig(s)\sig(s^{-1}) = \sig(ss^{-1}) = \sig(\Id) = 1
  \end{equation}
  Also $sas^{-1} \in A_{4}$ und $A_{4} \triangleleft S_{4}$.
\end{proof}

\subsection{Teil 2}

\subsection{Teil 3}

\begin{proof}
  \begin{equation}
    \Id \in H
  \end{equation}
  \begin{equation}
    (2314)(2314) = (2314)
  \end{equation}
  \begin{equation}
    (2314)(3124) = (1234)
  \end{equation}
  \begin{equation}
    (3124)(2314) = (1234)
  \end{equation}
  \begin{equation}
    (3124)(3124) = (3124)
  \end{equation}
  \begin{equation}
    \sig(\Id) = \sig(2314) = \sig(3124) = 1 \Rightarrow H \subset A_{4}
  \end{equation}
  Also ist $H$ eine Untergruppe von $A_{4}$.

  Es ist also gezeigt, dass $A_{4}$ eine Gruppe, $H$ eine Untergruppe von $A_{4}$ und $V_{4}$ ein Normalteiler von $A_{4}$ ist.
  Außerdem ist $H \cap V_{4} = \{ e_{A_{4}} \}$.
  Es bleibt zu zeigen, dass $A_{4} = V_{4}H$.
  Dann liefert Satz 1.7.5 den Isomorphismus.

  \begin{equation}
    \Id \circ \Id = \Id
  \end{equation}
  \begin{equation}
    \Id \circ (2314) = (2314)
  \end{equation}
  \begin{equation}
    \Id \circ (3124) = (3124)
  \end{equation}
  \begin{equation}
    (2143) \circ \Id = (2143)
  \end{equation}
  \begin{equation}
    (2143)(2314) = (1423)
  \end{equation}
  \begin{equation}
    (2143)(3124) = (4213)
  \end{equation}
  \begin{equation}
    (3412) \circ \Id = (3412)
  \end{equation}
  \begin{equation}
    (3412)(2314) = (4132)
  \end{equation}
  \begin{equation}
    (3412)(3124) = (1342)
  \end{equation}
  \begin{equation}
    (4321) \circ \Id = (4321)
  \end{equation}
  \begin{equation}
    (4321)(2314) = (3241)
  \end{equation}
  \begin{equation}
    (4321)(3124) = (2431)
  \end{equation}
  Da hier 12 verschiedene Ergebnisse auftauchen und $A_{4}$ eine Gruppe ist, ist $A_{4} = V_{4}H$ und es ist gezeigt.
\end{proof}

\section{Übung 2}

\subsection{Teil 1}

\begin{proof}
  Sei $n = |G|$.
  Da $G$ zyklisch ist, ist $G$ isomorph zu $\mathbb{Z} / n \mathbb{Z}$.
  $H = \mathbb{Z} / (n / p) \mathbb{Z}$ ist eine Untergruppe der Ordnung $p$ davon, weil $p$ $n$ teilt.
  Sei $g$ das Urbild von $n / p$ unter dem Isomorphismus $f$ zwischen $G$ und $\mathbb{Z} / n \mathbb{Z}$.
  Dann ist $ord(f(g)) = p$, also $ord(g) = p$.
\end{proof}

\subsection{Teil 2.a}

\subsection{Teil 2.b}

\begin{proof}
  Sei $n = |G|$.

  Sei $n = 1$.
  Dann muss $p = 1$ sein und es gibt $e_{G} \in G$ mit $ord(e_{G}) = 1$, was von $p$ geteilt wird.

  Sei $n > 1$ und die Behauptung für kleinere $n$ bereits gezeigt.
\end{proof}

\subsection{Teil 2.c}

\begin{proof}
  Wie in Teil 2.b gesehen, gibt es ein $g \in G$ mit $p | ord(g)$.
  Die endliche Untergruppe, und somit Gruppe, $\langle g \rangle$ ist also zyklisch und $p$ ist ein Primteiler ihrer Ordnung.
  Also kann man Teil 1 wiederum auf $\langle g \rangle$ anwenden.
\end{proof}

\section{Übung 3}

\subsection{Teil 1}

\begin{proof}
  Seien $A, B \in H$.
  Zu zeigen ist
  \begin{equation}
    \Phi_{AB} = \Phi_{A} \circ \Phi_{B}
  \end{equation}
  Sei $n \in N$.
  \begin{equation}
    \Phi_{AB}(n) = (AB)n = A(Bn) = A(\Phi_{B}(n)) = \Phi_{A} \circ \Phi_{B}(n)
  \end{equation}
\end{proof}

\subsection{Teil 2}

\begin{proof}
  Sei $0 \in N$ der Nullvektor und $I \in H$ die Einheitsmatrix.
  Dann ist $(0, I) \in G$ das neutrale Element von $G$.
  Sei $y \in X$.
  \begin{equation}
    (0, I) \cdot y = 0 + Iy = y
  \end{equation}

  Seien $(a, A), (b, B) \in G$ und $y \in X$.
  \begin{align*}
    (a, A) \cdot ((b, B) \cdot y) & = (a, A) \cdot (b + By)\\
    & = a + A(b + By)\\
    & = a + Ab + ABy\\
    & = (a + Ab, AB)y\\
    & = ((a, A)(b, B)) \cdot y
  \end{align*}
\end{proof}

\subsection{Teil 3}

Die Operation ist transitiv.
\begin{proof}
  Seien $x, y \in X$ und $(y, 0) \in G$ ($0$ ist die Nullmatrix).
  \begin{equation}
    (y, 0) \cdot x = y + 0x = y
  \end{equation}
\end{proof}

Die Operation ist treu.
\begin{proof}
  Sei $(a, A) \in G$ und $(a, A) \cdot x = x$ für alle $x \in X$.o
  Sei $0 \in N$ der Nullvektor.
  Dann gilt insbesondere
  \begin{equation}
    (a, A) \cdot 0 = a + A(0) = a + 0 = a = 0 \Rightarrow a = 0
  \end{equation}
  Somit gilt
  \begin{equation}
    (0, A) \cdot x = Ax = x \forall x \in X
  \end{equation}
  Insbesondere auch für die Einheitsvektoren und deshalb ist $A = I$.
  Also ist
  \begin{equation}
    (a, A) = (0, I)
  \end{equation}
  und die Operation ist treu.
\end{proof}

\subsection{Teil 4}

Sei $y \in X$.
\begin{align*}
  G_{y} & = \{ g \in G \mid g \cdot y = y \}\\
  & = \{ (a, A) \in G \mid a + Ay = y \}\\
  & = \{ (a, A) \in G \mid a = y - Ay \}\\
  & = \{ (y - Ay, A) \in G \}
\end{align*}

\section{Übung 4}

\end{document}