\documentclass[10pt,a4paper]{article}
\usepackage[utf8]{inputenc}
\usepackage[german]{babel}
\usepackage{mathrsfs}
\usepackage{amsmath}
\usepackage{amsfonts}
\usepackage{amssymb}
\usepackage{amsthm}
\usepackage[left=2cm,right=2cm,top=2cm,bottom=2cm]{geometry}

\DeclareMathOperator{\id}{id}
\DeclareMathOperator{\im}{Im}
\DeclareMathOperator{\ker}{Ker}
\DeclareMathOperator{\ord}{ord}
\DeclareMathOperator{\Aut}{Aut}

\begin{document}

\section{Aufgabe 1}

\subsection{Teil 1}

Da $12 = 2^{2} \cdot 3$ ist $k_{2} \in \{ 1, 3 \}$ und $k_{3} \in \{ 1, 4 \}$ nach dem zweiten Sylowsatz.

\subsection{Teil 2}

Da $1$ die höchste Potenz von $3$ ist, die $12$ teilt, haben $3$-Sylowuntergruppen Ordnung $3$.
\begin{equation}
  |H_{1} \cap H_{2}| = 1
\end{equation}
Wenn die Ordnung des Schnitts größer als $1$ wäre, wären $H_{1} = H_{2}$, weil sie zyklisch sind und von ihren nicht-neutralen Elementen erzeugt werden (Korollar 1.4.6).

\subsection{Teil 3}

$2$-Sylowuntergruppen haben Ordnung $4$.
Sei $q = |H_{1} \cap H_{2}|$.
$q$ kann nicht $4$ sein, weil $H_{1} \ne H_{2}$.
$q$ kann nicht $3$ sein, weil der Schnitt eine Untergruppe ist, dessen Ordnung nach dem Satz von Lagrange $q$ teilen muss.
\begin{equation}
  q \in \{ 1, 2 \}
\end{equation}

\subsection{Teil 4}

\begin{proof}
  Der Schnitt einer $3$- und einer $4$-Sylowuntergruppe ist immer $\{ e \}$, weil $ggT(3, 4) = 1$ (Satz von Lagrange).
  Also zähle man einfach die Elemente.
  $4$ $3$-Sylowuntergruppen haben $4 \cdot 2$ unterschiedliche Elemente, die nur in jeweils einer Untergruppe enthalten sind.
  $3$ $2$-Sylowuntergruppen haben mindestens $3 \cdot 2$ unterschiedliche Elemente, die nur in jeweils einer Untergruppe enthalten sind.
  Man hätte also mindestens $8 + 6 + 1 = 15$ Elemente, aber $|G| = 12$.
\end{proof}

\section{Aufgabe 2}

\begin{proof}
  $\Rightarrow$: Sei $a$ Nullteiler.
  Angenommen $a$ sei invertierbar mit dem Inversen $a^{-1}$.
  Sei $b \in R$ mit $ab = 0$.
  Durch Linksmultiplikation mit $a^{-1}$ erhält man $b = 0$ und somit wäre $a$ nicht Nullteiler.
  Dies ist ein Widerspruch und $a$ ist nicht invertierbar.

  $\Leftarrow$: Sei $b \in R$ mit $ab = 0$.
  Durch Linksmultiplikation mit $a^{-1}$ erhält man $b = 0$ und somit ist $a$ nicht Nullteiler.
\end{proof}

\section{Aufgabe 3}

\begin{equation}
  \ker f =
\end{equation}

\begin{equation}
  \im f =
\end{equation}

\section{Aufgabe 4}

\subsection{Teil 1}

\begin{proof}
  $\subset$: Sei $[k] \in (\mathbb{Z} / n \mathbb{Z})^{\times}$.
  Also $kb = 1 + an$ für $a \in \mathbb{N}_{0}$, $b \in \mathbb{N}$.
  Deshalb ist $ggT(kb, an) = 1$.
  Aber dann ist $kb$ auch teilerfremd zu den Teilern von $an$, also auch zu $n$, und $n$ wiederum zu $k$.

  $\supset$: Sei $[k] \in \{ [x] \in \mathbb{Z} / n \mathbb{Z} \mid ggT(n, x) = 1 \}$.
  Nach dem erweiterten euklidischen Algorithmus gibt es $a, b \in \mathbb{N}_{0}$ mit $ak + bn = 1$.
  Modulo $n$ ist $[ak + bn] = [ak] = [a][k] = 1$, also $[k]^{-1} = [a]$ und $[k]$ ist invertierbar.
\end{proof}

\subsection{Teil 2}

\begin{proof}
  $\mathbb{Z} / n \mathbb{Z}$ ist zyklisch und $1$ ist ein Erzeuger.
  Da $\varphi$ ein Isomorphismus ist, ist $\langle \varphi(1) \rangle$ ebenfalls zyklisch und hat dieselbe Ordnung wie $\langle 1 \rangle$.
  Also ist $\ord(\varphi(1)) = n$.
\end{proof}

$\varphi(1)$ muss ein Erzeuger von $\mathbb{Z} / n \mathbb{Z}$ sein.
Das heißt, es gibt ein $b \in \mathbb{Z} / n \mathbb{Z}$ mit $\varphi(1)b = c$ für $c \in \mathbb{Z} / n \mathbb{Z}$, insbesondere für $[1]$.
Also $\varphi(1) \in (\mathbb{Z} / n \mathbb{Z})^{\times}$.

\subsection{Teil 3}

\begin{proof}
  Sei $[x], [y] \in (\mathbb{Z} / n \mathbb{Z})^{\times}$ und $[z] \in \mathbb{Z} / n \mathbb{Z}$.
  \begin{equation}
    (\varPhi([x]) \circ  \varPhi([y]))([z]) = \varPhi([x])([y][z]) = [x][y][z] = [xy][z] = \varPhi([xy])([z])
  \end{equation}
  Also ist $\varPhi$ ein Gruppenhomomorphismus.

  Sei $[x] \in \ker(\varPhi)$.
  Das heißt $\varPhi([x]) = \id$ und $\varPhi([x])([z]) = [z]$ für alle $[z] \in \mathbb{Z} / n \mathbb{Z}$.
  Insbesondere ist $\varPhi([x])([1]) = [x][1] = [x] = [1]$.
  Somit ist $\varPhi$ injektiv.

  Es bleibt zu zeigen, dass $\varPhi$ surjektiv ist.
  Sei $\varphi \in \Aut(\mathbb{Z} / n \mathbb{Z})$.
  Sei $f = \varPhi(\varphi([1]))$.
  Dies ist definiert wie in Teil 2 gesehen.
  Es bleibt zu zeigen, dass $f([z]) = \varphi([z])$ für alle $[z] \in \mathbb{Z} / n \mathbb{Z}$.
  Sei $[z] \in \mathbb{Z} / n \mathbb{Z}$.
  \begin{equation}
    f([z]) = \varphi([1])[z] = \sum_{i = 1}^{z} \varphi([1]) = \varphi(\sum_{i = 1}^{z} [1]) = \varphi([z])
  \end{equation}
\end{proof}

\subsection{Teil 4}

$8 = 2^{3}$ und $29$ ist eine Primzahl, also sind $8$ und $29$ teilerfremd, $ggT(8, 29) = 1$ und $[8] \in (\mathbb{Z} / n \mathbb{Z})^{\times}$.
Man bestimme $[8]^{-1}$ mit dem erweiterten euklidischen Algorithmus.
\begin{equation}
  29 = 3 \cdot 8 + 5
\end{equation}
\begin{equation}
  8 = 1 \cdot 5 + 3
\end{equation}
\begin{equation}
  5 = 1 \cdot 3 + 2
\end{equation}
\begin{equation}
  3 = 1 \cdot 2 + 1
\end{equation}
\begin{equation}
  2 = 2 \cdot 1 + 0
\end{equation}

\begin{align*}
  1 & = 3 - 2 = (8 - 5) - (5 - 3) = 8 - 2 \cdot 5 + 3\\
  & = 8 - 2 \cdot (29 - 3 \cdot 8) + (8 - 5)\\
  & = -2 \cdot 29 + 8 \cdot 8 - 5\\
  & = -2 \cdot 29  + 8 \cdot 8 - 29 + 3 \cdot 8\\
  & = 11 \cdot 8 - 3 \cdot 29
\end{align*}
\begin{equation}
  [8]^{-1} = [11]
\end{equation}

Gesucht ist $[b] \in \mathbb{Z} / n \mathbb{Z}$ mit
\begin{equation}
  [8][b] = [9]
\end{equation}
Linksmultiplikation mit $[11]$ liefert
\begin{equation}
  [b] = [11][9] = [99] = [12]
\end{equation}
Also $x \in [12]$, z.B. $12, 41, -17$.

\end{document}