\documentclass[10pt,a4paper]{article}
\usepackage[utf8]{inputenc}
\usepackage[german]{babel}
\usepackage{mathrsfs}
\usepackage{amsmath}
\usepackage{amsfonts}
\usepackage{amssymb}
\usepackage{amsthm}
\usepackage[left=2cm,right=2cm,top=2cm,bottom=2cm]{geometry}

\DeclareMathOperator{\id}{Id}

\begin{document}

\section{Übung 1}

\subsection{Teil 1}

\begin{align*}
  & x^{4} + 2x^{2} - 2 = 0\\
  \Leftrightarrow \, & z^{2} + 2z - 2 = 0\\
  \Leftrightarrow \, & z = -1 \pm \sqrt{1 - (-2)} = -1 \pm \sqrt{3}\\
  \Leftrightarrow \, & x = \sqrt{z}\\
  \Leftrightarrow \, & x \in \{ \sqrt{\sqrt{3} - 1}, -\sqrt{\sqrt{3} - 1}, i\sqrt{\sqrt{3} + 1}, -i\sqrt{\sqrt{3} + 1} \}
\end{align*}
\begin{equation}
  \alpha = \pm\sqrt{\sqrt{3} - 1}
\end{equation}
\begin{equation}
  \beta = \pm i\sqrt{\sqrt{3} + 1}
\end{equation}
\begin{equation}
  x = i\sqrt{2}
\end{equation}
\begin{equation}
  y = \sqrt{3}
\end{equation}

\subsection{Teil 2}

\begin{proof}
  Da $K = \mathbb{Q}(\alpha, x)$, kann man teilen und
  \begin{equation}
    \beta = \frac{x}{\alpha} \in K
  \end{equation}
  $K$ ist minimal, weil $i \not\in K(\alpha)$ und somit $\beta \not\in K(\alpha)$.
  Deshalb ist $K$ der Zerfallungskörper.

  \begin{equation}
    [K : \mathbb{Q}] = 3
  \end{equation}
\end{proof}

\subsection{Teil 3}

\begin{proof}
  Sei $\sigma$ die komplexe Konjugation.
  Dann ist $\sigma(a) = a$ für alle $a \in \mathbb{Q}$.
  Also ist $\sigma \in Gal(K / \mathbb{Q})$.

  Sei $f \in Gal(K / \mathbb{Q})$ mit $f(\alpha) = \beta$.
  \begin{equation}
    f(\pm \alpha) = \pm \beta
  \end{equation}
  \begin{equation}
    f(\pm \beta) = \pm \alpha
  \end{equation}
  \begin{equation}
    f \circ \sigma(\pm \alpha) = \pm \beta
  \end{equation}
  \begin{equation}
    f \circ \sigma(\pm \beta) = f(\mp \beta) = \mp \alpha
  \end{equation}
\end{proof}

\subsection{Teil 4}

\begin{proof}
  Man betrachte die Permutation $\alpha, -\alpha, \beta, -\beta$.
  Sei $f \in Gal(K / \mathbb{Q})$ mit $f(\alpha) = -\alpha$.
  Dann ist $f(-\alpha) = -f(\alpha) = \alpha$.
  Somit kann kein $f$ diese Permutation induzieren.
\end{proof}

\subsection{Teil 5}

\section{Übung 2}

\begin{equation}
  L = \mathbb{Q}(\sqrt[4]{5}, i)
\end{equation}

\subsection{Teil 1}

Sei $G = \{ \id, \sigma \}$ wobei $\sigma$ die komplexe Konjugation ist.
Dann ist $K = L^{G}$.

\begin{proof}
  Sei $x \in K$.
  Dann ist $\id(x) = x, \sigma(x) = x$, also $x \in L^{G}$.

  Sei $x = a + b\sqrt[4]{5} + ci \in L^{G}$.
  Dabei ist $\sigma(x) = x$, also $c = 0$.
  Daher ist $x \in K$.
\end{proof}

Somit ist $Gal(L / K) = G$.

\subsection{Teil 2}

\subsection{Teil 3}

\end{document}