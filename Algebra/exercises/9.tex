\documentclass[10pt,a4paper]{article}
\usepackage[utf8]{inputenc}
\usepackage[german]{babel}
\usepackage{mathrsfs}
\usepackage{amsmath}
\usepackage{amsfonts}
\usepackage{amssymb}
\usepackage{amsthm}
\usepackage[left=2cm,right=2cm,top=2cm,bottom=2cm]{geometry}

\begin{document}

\section{Übung 1}

\subsection{Teil 1.a}

Ist irreduzibel in $\mathbb{Q}[X]$ nach dem Kriterium von Eisenstein, weil $7$ $7$ und $21$ teilt, aber nicht $1$ und $7^{2}$ $21$ nicht teilt.

\subsection{Teil 1.b}



\subsection{Teil 1.c}

\subsection{Teil 2}

\section{Übung 2}

\subsection{Teil 1}

\begin{equation}
  X - (\sqrt[3]{5} + i\sqrt{2})
\end{equation}

\subsection{Teil 2}

\begin{equation}
  X^{2} - 2\sqrt[3]{5}X - 2
\end{equation}
\begin{equation}
    (\sqrt[3]{5} + i\sqrt{2})^{2} - 2\sqrt[3]{5}(\sqrt[3]{5} + i\sqrt{2}) + 2 + \sqrt[3]{5} & = \sqrt[3]{5}^{2} + 2i\sqrt{2}\sqrt[3]{5} - 2 - 2\sqrt[3]{5}(\sqrt[3]{5} + i\sqrt{2}) + 2 + \sqrt[3]{5} = 0
\end{equation}

\subsection{Teil 3}

\begin{equation}
  X^{3} - \sqrt[3]{5}X^{2}
\end{equation}
\begin{align*}
  a^{3} - \sqrt[3]{5}a^{2} & = (5 + 3\sqrt[3]{5}^{2}i\sqrt{2} - 6\sqrt[3]{5} - 2i\sqrt{2}) - \sqrt[3]{5}(5 + 2i\sqrt{2}\sqrt[3]{5} - 2)
\end{align*}

\subsection{Teil 4}

\subsection{Teil 5}

\begin{equation}

\end{equation}

\section{Übung 3}

\subsection{Teil 1}

\begin{proof}
  Dass $\mathcal{H}(\mathbb{C})$ nicht trivial und kommutativ ist, ist klar.
  Es bleibt zu zeigen, dass es auch nullteilerfrei ist.
  Seien $f, g \in \mathcal{H}(\mathbb{C})$ mit $fg = 0$ und $f \ne 0$.
  ...
\end{proof}

\subsection{Teil 2}

\begin{proof}
  Da die Bedingungen immer schwächer werden, bilden die $I_{k}$ eine aufsteigende Kette.
  Wenn $\mathcal{H}(\mathbb{C})$ noethersch wäre, gäbe es ein $n \in \mathbb{N}$, sodass $I_{m} = I_{n}$ für alle $m \ge n$.
  Man betrachte die Funktion $f(z) = \cos(2^{z - n}\pi z)$.
\end{proof}

\subsection{Teil 3}

\subsection{Teil 4}

\subsection{Teil 5}

\end{document}