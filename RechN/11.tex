\documentclass[10pt,a4paper]{article}
\usepackage[utf8]{inputenc}
\usepackage[german]{babel}
\usepackage{mathrsfs}
\usepackage{amsmath}
\usepackage{amsfonts}
\usepackage{amssymb}
\usepackage{amsthm}
\usepackage[left=2cm,right=2cm,top=2cm,bottom=2cm]{geometry}

\begin{document}

\section{Aufgabe 11.1}

Ja, mit den beiden Primzahlen kann sie die beiden Schlüssel wiederherstellen und
es wird auch derselbe Schlüssel sein, der im Zertifikat signiert ist.

\section{Aufgabe 11.2}

Bei einem Playback/Replay-Angriff wird eine korrekt verschlüsselte und signierte
Nachricht von einem Angreifer aufgezeichnet und dann beliebig oft
abgespielt. Man kann zwar den Inhalt der Nachricht nicht beeinflussen, aber
ggf. Aktionen mehrfach auslösen.

Man kann es verhindern, indem man z.B. eine Sequenznummer in die Nachrichten
einbaut, sodass wiederholte Pakete erkannt werden können.

\section{Aufgabe 11.3}

Es geht so. Man kann als Angreifer ja signierte Nachrichten mit dem öffentlichen
Schlüssel entschlüsseln. Dann könnte man solange zufällige Bitfolgen mit dem
öffentlichen Schlüssel entschlüsseln, bis man eine findet, die zu einer Uhrzeit
in der näheren Zukunft führt. Und dann muss man nur noch zur richtigen Zeit
zuschlagen, also die Kommunikation beginnen, um die gefundene Zeit als Nonce zu
erzwingen.

\section{Aufgabe 11.4}

Der symmetrische Schlüssel wird in der Kommunikation zwischen den Mailservern
verwendet. Sie können z.B. einen mit dem Diffie-Hellman-Schlüsselaustausch
erzeugen.

\section{Aufgabe 11.5}

Bei einem Man-in-the-middle-Angriff empfängt der Angreifer die Nachrichten und
leitet sie dann an den jeweils richtigen Empfänger weiter. So kann er den
kompletten Verkehr mitlesen und manipulieren. Schützen können sich die Aktoren
davor mit Verschlüsselung und Signaturen, aber nur mit Zertifikaten.

Wenn man keine Zertifikate einsetzt oder Zertifikaten vertraut, die nicht von
vertrauenswürdigen Stellen ausgestellt worden sind, kann trotzdem ein Angriff
stattfinden. Der Angreifer C schickt A ein Zertifikat, das seinen Public-Key mit
der Identität von B verbindet, und B ein Zertifikat, das seinen Public-Key mit
der Identität von A verbindet.

\section{Aufgabe 11.6}

Vor Replay-Angriffen.

\section{Aufgabe 11.7}

Ein zustandsloser Paketfilter trifft seine Entscheidungen nur auf Basis des
aktuell betrachteten Pakets. Ein zustandsbasierter Filter kann sich
darüberhinaus auch Dinge über die Zeit hinweg merken und so etwa Verbindungen
nachvollziehen.

Z.B. kann ein zustandsbasierter Filter nur UDP-Pakete in sein Netz lassen, wenn
die Zieladresse auch in letzter Zeit UDP-Pakete an die Quelle gesendet hat.

\section{Aufgabe 11.8}

Wenn Herr Zert integer ist, ist nichts einzuwenden. Aber als Ganove könnte er
sich die ausgegebenen privaten Schlüssel merken und meine Identität annehmen.

Sicherer wäre es, wenn ich nach erfolgreicher Authentifizierung einen
selbstgenerierten öffentlichen Schlüssel abgeben könnte. Dieser reicht aus, um
das Zertifikat zu erstellen und kann nicht benutzt werden, um sich als mich
auszugeben.

\section{Aufgabe 11.9}

A könnte sein Zertifikat von B signieren lassen und so Alices Certificate Chain
um einen Schritt verlängern. Bob würde sich dann an der Chain entlanghangeln bis
zu B und dem Zertifikat dann vertrauen.

\section{Aufgabe 11.10}

\subsection{Teil 1}

Weil ein MAC nur die Integrität des Pakets überprüft und dass es von einem
bestimmten, bekannten Sender kommt. Man kann damit aber nicht die Identität des
Senders feststellen. Ein Angreifer könnte also genauso wie ein richtiger Router
einfach mit den Routern einen Key ausmachen und dann mit MAC mit ihnen
kommunizieren.

\subsection{Teil 2}

Ihre MAC-Adresse. Die IP-Adresse ist nicht beständig genug.

\subsection{Teil 3}

Indem jeder Host eine eigene Sequenznummer speichert und mitsendet. Man
akzeptiert Broadcast-Pakete dann nur, wenn die Sequenznummer größer als die
letzte von diesem Host empfangene ist.

\section{Aufgabe 11.11}

In einem Web of Trust veröffentlichen die Teilnehmer ihre öffentlichen Schlüssel
und stellen sich dann gegenseitig Zertifikate aus, die auch wiederum
veröffentlicht werden. Um dann mit jemandem zu kommunizieren, lädt man sich
seinen öffentlichen Schlüssel runter und guckt dann, ob es eine oder mehrere
Wege im Web of Trust von sich selbst zu ihm gibt. Wenn dies der Fall ist, hat
man entweder den Schlüssel selbst signiert oder er würde von jemandem signiert,
dem man vertraut, bzw. von jemandem dem man vertraut, weil jemand, dem man
vertraut, ihm vertraut und so weiter.

\section{Aufgabe 11.12}

Nein. Ein NAT-Router ist am ehesten eine Firewall die alle eingehenden
Verbindungen ablehnt. Eine richtige Firewall kann darüber hinaus auch ausgehende
Verbindungen kontrollieren, um zum Beispiel Trojanern das nach Hause
telefonieren unmöglich zu machen.

\section{Aufgabe 11.13}

\subsection{Teil 1}

Der Arbeitsplatzrechner muss eine Verbindung nach Hause offen halten, über die
der Mitarbeiter dann von zuhause Pakete in die Firma senden kann.

\subsection{Teil 2}

OpenVPN?

\subsection{Teil 3}

Der Administrator könnte nur Pakete von und zu ganz bestimmten Hosts zulassen,
über die die Mitarbeiter nicht die Kontrolle erlangen können.

\section{Aufgabe 11.14}

\subsection{Teil 1}

Am einfachsten wäre es wohl, einen weiteren Proxy zu entwickeln, der bei
Verbindungsaufbau hardcodiert die Verbindung über den Uni-Proxy zu dem
MySQL-Server aufbaut. Diesen lässt man dann lokal laufen und verbindet sich mit
seinem MySQL-Client dazu.

\subsection{Teil 2}

Es ist eigentlich als Erweiterung für HTTP entwickelt worden. Mit ihr kann man
Proxies auch für HTTPS-Verkehr benutzen. Da der Proxy nicht in den Traffic
reinschauen kann, bleibt wohl nur, die Portnummern in den Requests
einzuschränken, z.B. 80 und 443 für HTTP.

\end{document}