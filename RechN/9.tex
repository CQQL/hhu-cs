\documentclass[10pt,a4paper]{article}
\usepackage[utf8]{inputenc}
\usepackage[german]{babel}
\usepackage{mathrsfs}
\usepackage{amsmath}
\usepackage{amsfonts}
\usepackage{amssymb}
\usepackage{amsthm}
\usepackage[left=2cm,right=2cm,top=2cm,bottom=2cm]{geometry}

\begin{document}

\section{Aufgabe 9.1}

Er wird belastet, weil der gesamte Verkehr bei allen angeschlossenen Teilnehmern
ankommt und zumindest die ersten zwei Schichten/Header müssen immer verarbeitet
werden, um festzustellen, ob das Paket relevant ist, also ob man selbst das Ziel
ist oder es ein Broadcasepaket ist.

\section{Aufgabe 9.2}

\subsection{Teil 1}

Nein, weil 100\%tige Fehlerkorrektur unmöglich ist.

\subsection{Teil 2}

Nein, weil die Sicherungsschicht nur einen einzelnen Link sichert. Aber bei
einer Folge von Links kann z.B. auch einer ausfallen, sodass ein Paket verloren
geht, was erst von TCP erkannt wird.

Man könnte die Funktionalität auf der Sicherungsschicht theoretisch weglassen,
aber das würde die Performance von TCP stark verschlechtern, weil jeder kleine
Fehler von TCP gehandhabt werden muss, was wiederum denkt, dass das Netz
überlastet ist.

\section{Aufgabe 9.3}

Wenn man bei CSMA/CD eine Kollision bemerkt, legt man dieses Signal auf die
Leitung, damit alle sofort das Senden einstellen und schneller mit der
Kollisionsvermeidung begonnen werden kann.

\section{Aufgabe 9.4}

\subsection{Teil 1}

Wenn eine Kollision erkannt wird, wartet man eine zufällige Zeitspanne bis man
erneut versucht, zu senden. Dabei wächst die Menge der möglichen Zeitspannen
exponentiell.

Dieses Verfahren wird allerdings für jedes Paket wieder von vorne begonnen, weil
man davon ausgeht, dass die Auslastung des Netzwerks stark fluktuiert.

\subsection{Teil 2}

Die Wartezeit steigt exponentiell, um sich schon nach wenigen Iterationen an
starke Auslastung anzupassen.

\section{Aufgabe 9.5}

IP-Adressen braucht man für die globale Adressierung und Domain-Namen, damit
Menschen sie sich besser merken können.

\section{Aufgabe 9.6}

\subsection{Teil 1}

Die Wahrscheinlichkeit, dass $A$ in einem Schlitz erfolgreich sendet, ist
\begin{equation}
  P(A) = p \cdot (1 - p)^{2}
\end{equation}
Die Wahrscheinlichkeit für das Gegenteil ist
\begin{equation}
  P(\lnot A) = 1 - P(A) = 1 - p(1 - p)^{2}
\end{equation}
Die Antwort ist dann
\begin{equation}
  P(\lnot A)^{4} \cdot P(A) = \left( 1 - p(1 - p)^{2} \right)^{4} \cdot p (1 - p)^{2}
\end{equation}

\subsection{Teil 2}

Die Wahrscheinlichkeit, dass irgendein Knoten in irgendeinem Schlitz erfolgreich
ist, ist
\begin{equation}
  P(A) + P(B) + P(C) = 3 \cdot p(1 - p)^{2}
\end{equation}

\subsection{Teil 3}

Die Wahrscheinlichkeit, dass kein Knoten erfolgreich ist, ist dann
\begin{equation}
  1 - (P(A) + P(B) + P(C)) = 1 - 3 \cdot p(1 - p)^{2}
\end{equation}

Die gesuchte Wahrscheinlichkeit ist also
\begin{equation}
  \left( 1 - 3 \cdot p(1 - p)^{2} \right)^{3} \cdot 3 \cdot p(1 - p)^{2}
\end{equation}

\subsection{Teil 4 + 5}

Wie in Teil 2 bestimmt, ist die Wahrscheinlichkeit, dass irgendein Knoten
erfolgreich ist, gegeben durch
\begin{equation}
  P(X) = 3 \cdot p(1 - p)^{2}
\end{equation}
Die Ableitung davon ist
\begin{equation}
  P'(X) = 3 \cdot \left( (1 - p)^{2} + p \cdot (-2)(1 - p) \right) = 3 \cdot (1 - p) \cdot (1 - 3p) = 9 \cdot (1 - p) \cdot (\frac{1}{3} - p)
\end{equation}
Dies hat die Nullstellen $\frac{1}{3}$ und $1$. Da kein Paket durchkommt, wenn
jeder immer sendet, kann $1$ schon kein Maximum sein. $0$ ist auch kein Maximum,
weil dann nie jemand sendet. Also ist das Maximum bei $\frac{1}{3}$. Dann ist
die optimale Effizienz
\begin{equation}
  3 \cdot \frac{1}{3}(1 - \frac{1}{3})^{2} = \left( \frac{2}{3} \right)^{2} = \frac{4}{9} \simeq 44\%
\end{equation}

\section{Aufgabe 9.7}

\subsection{Teil 1}

\begin{tabular}{l|r}
Schnittstelle & IP-Adresse\\
\hline
A & 111.111.111.1\\
B & 111.111.111.2\\
C & 122.122.122.1\\
D & 122.122.122.2\\
E & 133.133.133.1\\
F & 133.133.133.2\\
Router 1 - links & 111.111.111.3\\
Router 1 - rechts & 122.122.122.3\\
Router 2 - links & 122.122.122.4\\
Router 2 - rechts & 133.133.133.3\\
\end{tabular}

\subsection{Teil 2}

Die MAC-Adressen der Computer seinen jeweils ihr Buchstabe. Die Adressen der
Router-Schnittstellen seien von links nach rechts G, H, I, J.

\subsection{Teil 3}

\begin{enumerate}
\item A sieht, dass F in einem anderen Netzwerk ist und schickt das Paket ans
  Gateway
\item A löst die IP-Adresse vom Gateway mit seiner lokalen ARP-Tabelle in eine
  MAC-Adresse auf
\item A schickt das Paket in seinem lokalen Paket an die MAC-Adresse von F
\item Router 1 empfängt das Paket und dekodiert es bis zur Ziel-IP
\item Router 1 schlägt die Ziel-IP in seiner Routing-Tabelle nach und sieht,
  dass er über Router 2 weiterleiten muss
\item Router 1 löst die IP von Router 2 mit seiner ARP-Tabelle in eine
  MAC-Adresse auf und sendet das Paket an diese
\item Router 2 empfängt das Paket und dekodiert es bis zur Ziel-IP
\item Router 2 schlägt die IP in seiner Routing-Tabelle nach und sieht, dass er
  am richtigen Netz angeschlossen ist
\item Router 2 löst die Ziel-IP in eine MAC-Adresse auf
\item Router 2 schickt das Paket an F
\end{enumerate}

\subsection{Teil 4}

Genauso wie davor nur wird jetzt immer noch ein ARP-Request für die IP-Adresse
im versandt.

\subsection{Teil 5}

So wie in Teil 4, nur Sendet der Switch erstmal an alle bis seine
Switching-Tabelle die entsprechenden Einträge enthält.

\section{Aufgabe 9.8}

\subsection{Teil 1}

\begin{equation}
  \frac{800}{2 \cdot 10^{8}} + 4 \cdot (20 \cdot 10^{-7}) = 4 \cdot 10^{-6} + 80 \cdot 10^{-7} = 12 \cdot 10^{-6}s
\end{equation}

\subsection{Teil 2}

A wird also nach $t_{1} = 12 \cdot 10^{-6}s$ merken, dass B ebenfalls sendet und
den exponentiellen Backoff einleiten und dasselbe gilt für B. B wartet daraufhin
noch $512 \cdot 10^{-6}s$. A braucht dann $150 \cdot 10^{-6}s$, um sein Paket
komplett zu versenden. B sendet dann nicht mehr, weil ja jeder Sender vorher
horcht, ob die Leitung bereits belegt ist.

Das Paket wurde also nach $1512 \cdot 10^{-6}s$ komplett an B übertragen.

\subsection{Teil 3}

Die Ausbreitungsverzögerung über die gesamte Strecke bleibt gleich, aber das
Paket muss jetzt fünf mal übertragen werden. Es dauert somit
\begin{equation}
  (12 + 5 \cdot 150) \cdot 10^{-6}s = 762 \cdot 10^{-6}s
\end{equation}

\end{document}