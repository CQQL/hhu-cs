\documentclass[10pt,a4paper]{article}
\usepackage[utf8]{inputenc}
\usepackage[german]{babel}
\usepackage{mathrsfs}
\usepackage{amsmath}
\usepackage{amsfonts}
\usepackage{amssymb}
\usepackage{amsthm}
\usepackage[left=2cm,right=2cm,top=2cm,bottom=2cm]{geometry}
\usepackage{listings}

\begin{document}

\section{Aufgabe 3.1}

UDP wobei der Server ein eigenes Mini-ACK sendet. Also so wie TCP nur ohne
Handshake.

\section{Aufgabe 3.2}

Synchronisation von notwendigen Informationen bevor die eigentliche
Kommunikation beginnt.

\section{Aufgabe 3.3}

Ja, MX.

\section{Aufgabe 3.4}

Ja, wenn er welche hat, die Alice noch nicht hat und Alice diese
anfragt. Vielleicht weil er nett ist und sonst um auch weiterhin Daten von Alice
zu bekommen.

\section{Aufgabe 3.5}

Entweder kriegt er ihn direkt vom Seed, der nur hochlädt, oder er hat Glück beim
``optimistic unchoking'' eines anderen Peers, der ihm dann einen Datenblock
schickt.

\section{Aufgabe 3.6}

\section{Aufgabe 3.7}

$n + 1$, $n$ für die $n$ Client-Verbindungen plus einen Serversocket. Das liegt
daran, dass ein TCP-Socket einer bestimmten Verbindung zugeordnet ist, während
ein UDP-Socket einfach jedes Datenpaket annimmt ohne weiter zu unterteilen.

\section{Aufgabe 3.8}

{
  \setlength{\parindent}{0cm}

  Die Konvention scheint zu sein, dass niemand rekursives DNS unterstützt.

  dns-anfrager.hhu.de an dns.hhu.de
  \begin{lstlisting}
    www.cs.berkeley.edu. IN A
  \end{lstlisting}

  dns.hhu.de an dns.global.net (IP bekannt, weil es ein Root-Server ist)
  \begin{lstlisting}
    www.cs.berkeley.edu. IN A
  \end{lstlisting}

  dns.global.net an dns.hhu.de
  \begin{lstlisting}
    ;; AUTHORITY
    edu. IN NS dns.edu.

    ;; ADDITIONAL
    dns.edu. IN A 155.4.4.4
  \end{lstlisting}

  dns.hhu.de an dns.edu
  \begin{lstlisting}
    www.cs.berkeley.edu. IN A
  \end{lstlisting}

  dns.edu an dns.hhu.de
  \begin{lstlisting}
    ;; AUTHORITY
    www.cs.berkeley.edu. IN NS dns.berkeley.edu

    ;; ADDITIONAL
    dns.berkeley.edu IN A 128.32.40.2
  \end{lstlisting}

  dns.hhu.de an dns.berkeley.edu
  \begin{lstlisting}
    www.cs.berkeley.edu. IN A
  \end{lstlisting}

  dns.berkeley.edu an dns.hhu.de
  \begin{lstlisting}
    ;; ANSWER
    www.cs.berkeley.edu. IN A 128.32.64.42
  \end{lstlisting}

  dns.hhu.de an dns-anfrager.hhu.de
  \begin{lstlisting}
    ;; ANSWER
    www.cs.berkeley.edu. IN A 128.32.64.42
  \end{lstlisting}
}

\section{Aufgabe 3.9}

In der Praxis scheint niemand rekursive DNS-Anfragen zu unterstützten. Die
Anzahl der verschickten Pakete ändert sich dadurch nicht, aber die Arbeit wird
auf die Clients verteilt. Es sorgt also für weniger Auslastung der DNS-Server.

\section{Aufgabe 3.10}

``A'' ist eine IPv4-Adresse. ``AAAA'' ist eine IPv6-Adresse. ``NS'' enthält den
Hostnamen des autoritativen Nameservers.

\section{Aufgabe 3.11}

Der DNS-Server, der Dynamic DNS anbietet, liefert nur Antworten mit kurzen TTLs
aus, damit Clients immer die aktuelle Adresse erfahren, sobald sie sich
ändert. Die Seite, die die dynamische IP hat, ist dafür verantwortlich, dem
Dynamic-DNS-Anbieter immer die aktuelle IP mitzuteilen, wenn sie sich ändert,
z.B. nach einer Zwangstrennung oder einem Routerneustart.

\section{Aufgabe 3.12}

Da eine Kante in einem Overlaynetzwerk virtuell ist, ist die Anzahl der Kanten
unabhängig von der Anzahl der Router. Es gibt also $N$ Knoten und
$\sum_{i = 1}^{N} i = \frac{N \cdot (N + 1)}{2}$ Kanten.

\section{Aufgabe 3.13}

\subsection{Teil 1}

Jeder Peer sendet Anfragen an alle Peers weiter, die er kennt außer dem
Quellpeer. Dabei ignoriert er aber jede Anfrage, die er schonmal gesehen
hat. Dazu werden Anfragen wohl entweder eine ID haben oder nur für eine
bestimmte Zeit ignoriert.

\subsection{Teil 2}

Um die Anzahl der Weiterleitungen zu kontrollieren, beschränkt man die
Reichweite der Anfragen, also wieviele Kanten im Netzwerk sie nehmen darf, bis
die Weiterleitung aufhört.

Dies kann umgesetzt werden, indem man jeder Anfrage eine Zahl mitgibt, die
angibt, wieviele Sprünge die Anfrage noch machen darf. Bei jeder Weiterleitung
wird diese Zahl dann um 1 reduziert.

\subsection{Teil 3}

Man erreicht die maximale Anzahl an Weiterleitungen, wenn jeder Peer $N$
Nachbarn hat und die Nachbar-Mengen aller Peers paarweise disjunkt sind.  Dann
sendet jeder Peer die Anfrage $N$ mal weiter und das passiert $K$ mal. Die
Anzahl der Anfragenachrichten ist also durch $N^{K}$ beschränkt.

\section{Aufgabe 3.14}

\subsection{Teil 1}

Wenn der Peer immernoch mehr als 4 Verbindungen offen hat, würde er nichts
machhen. Wenn seine Verbindungszahl allerdings auf $3$ gesunken ist, würde er
versuchen, neue Verbindungen aufzubauen mittels einer Methode seiner
Wahl. Z.B. würde er erneut eine Ping-Nachricht verschicken, wie auch beim
Eintreten in das Netzwerk.

\subsection{Teil 2}

Die Nachbarn werden merken, dass die Verbindung getrennt wurde, wenn sie
Versuchen Kontakt aufzunehmen und mehrfach Timeouts haben. Dann würden sie wohl
ihrerseits ebenfalls den Socket schließen und sich verhalten wie in Teil 1.

\end{document}