\documentclass[10pt,a4paper]{article}
\usepackage[utf8]{inputenc}
\usepackage[german]{babel}
\usepackage{mathrsfs}
\usepackage{amsmath}
\usepackage{amsfonts}
\usepackage{amssymb}
\usepackage{amsthm}
\usepackage[left=2cm,right=2cm,top=2cm,bottom=2cm]{geometry}

\begin{document}

\section{Frage 1}

% BEGIN RECEIVE ORGTBL updates
\begin{tabular}{ll}
Komponente & Änderungen \\
\hline
Webbrowser & Muss die geupdateten Netzwerkbibliotheken verwenden \\
Client-DNS-Bibliothek & Muss AAAA-Records verstehen \\
Client-IP-Stack & Muss IPv6 verwenden \\
Heimrouter & Muss IPv6-Pakete routen können \\
ISP-Geräte & Wenn das auch Router sind, müssen die auch IPv6 verstehen \\
DNS-Server des ISPs & Muss AAAA-Records ausliefern, wenn die DNS-Server des Webseiten-Betreibers per IPv6 erreichbar sind \\
Internet-Backbone-Router & Muss IPv6-Pakete routen können \\
DNS-Server des Webseiten-Betreibers & Muss den AAAA-Record der Webseite ausliefern \\
Server-TCP/IP-Stack & Muss IPv6 verwenden \\
Webserver & Muss den aktualisierten IP-Stack verwenden \\
\end{tabular}
% END RECEIVE ORGTBL updates

\section{Frage 2}

Der Webseitenbetreiber muss die IPv6-Adresse seines Hosts auf seinem DNS-Server
hinterlegen und den Webserver auf der IPv6-Adresse horchen lassen.

\section{Frage 3}

Ja, wenn man die IPv4-Erreichbarkeit nicht erhält. Dies ist bei Linux jedoch
kein Problem, weil die neuen Funktionen automatisch IP4 und 6 kompatibel sind.

Dann muss der Client 6in4-Tunneling verwenden.

\section{Frage 4}

Die gesuchten Varianten heißen getaddrinfo, getnameinfo und inet\_pton.

\section{Frage 5}

% BEGIN RECEIVE ORGTBL addresses
\begin{tabular}{lllll}
IPv6-Adresse & Gültig? & Generische Schreibweise & Typ & Global? \\
\hline
fe80::2512:0121:1 & Ja & fe80::2512:121:1 & Link-Local Unicast & Nein \\
2001:6f8:1377:abcd:0000:0000:0001:0002 & Ja & 2001:6f8:1377:abcd::1:2 & Global Unicast & Ja \\
2013:123::56::7 & Nein &  &  &  \\
::0:1 & Ja & ::1 & Loopback & Nein \\
ff12::42 & Ja & ff12::42 & Multicast & Nein (Scope 2 ist Link-Local) \\
ff72::0x3 & Nein &  &  &  \\
0::0.0.0.0 & Ja & ::0 & Unspecified & ? \\
200a:a:b:c:d:e:f & Nein &  &  &  \\
2ac2:::09:3 & Nein &  &  &  \\
ff15::aaaa & Ja & ff15::aaaa & Multicast & Nein (Scope 5 ist Site-Local) \\
ff00:180bac::1 & Nein &  &  &  \\
ff11:1:2:3:4:5:6:7:8 & Nein &  &  &  \\
\end{tabular}
% END RECEIVE ORGTBL addresses

\section{Frage 6}

Bei statischer Konfiguration werden alle Einstellungen auf dem Host direkt
vorgegeben z.B. in einer Konfigurationsdatei.

Bei DHCP werden Adress- und Routingkonfiguration von einem DHCP-Server
empfangen. Der Client initialisiert die Kommunikation, indem er alle DHCP-Server
per Multicast anspricht. Diese schicken ihm daraufhin Konfigurationen an seine
selbsterstellte Link-Local-Address, von denen er eine akzeptiert. Das
Akzeptieren findet ebenfalls per Multicast statt, damit die abgelehnten
DHCP-Server Bescheid wissen.

Bei Stateless Address Autoconfiguration mit Router Advertisements bestimmt der
Host seine Link-Local Address ebenfalls selbst und empfängt dann jedoch alle
weiteren relevanten Einstellungen in Paketen (Router Advertisements), die die
Router periodisch versenden. Um den Prozess zu beschleunigen, können Hosts
jedoch diese Advertisements auch explizit anfragen.

\end{document}

#+ORGTBL: SEND updates orgtbl-to-latex :splice nil :skip 0
| Komponente                          | Änderungen                                                                                          |
|-------------------------------------+-----------------------------------------------------------------------------------------------------|
| Webbrowser                          | Muss die geupdateten Netzwerkbibliotheken verwenden                                                 |
| Client-DNS-Bibliothek               | Muss AAAA-Records verstehen                                                                         |
| Client-IP-Stack                     | Muss IPv6 verwenden                                                                                 |
| Heimrouter                          | Muss IPv6-Pakete routen können                                                                      |
| ISP-Geräte                          | Wenn das auch Router sind, müssen die auch IPv6 verstehen                                           |
| DNS-Server des ISPs                 | Muss AAAA-Records ausliefern, wenn die DNS-Server des Webseiten-Betreibers per IPv6 erreichbar sind |
| Internet-Backbone-Router            | Muss IPv6-Pakete routen können                                                                      |
| DNS-Server des Webseiten-Betreibers | Muss den AAAA-Record der Webseite ausliefern                                                        |
| Server-TCP/IP-Stack                 | Muss IPv6 verwenden                                                                                 |
| Webserver                           | Muss den aktualisierten IP-Stack verwenden                                                          |

#+ORGTBL: SEND addresses orgtbl-to-latex :splice nil :skip 0
| IPv6-Adresse                           | Gültig? | Generische Schreibweise | Typ                | Global?                       |
|----------------------------------------+---------+-------------------------+--------------------+-------------------------------|
| fe80::2512:0121:1                      | Ja      | fe80::2512:121:1        | Link-Local Unicast | Nein                          |
| 2001:6f8:1377:abcd:0000:0000:0001:0002 | Ja      | 2001:6f8:1377:abcd::1:2 | Global Unicast     | Ja                            |
| 2013:123::56::7                        | Nein    |                         |                    |                               |
| ::0:1                                  | Ja      | ::1                     | Loopback           | Nein                          |
| ff12::42                               | Ja      | ff12::42                | Multicast          | Nein (Scope 2 ist Link-Local) |
| ff72::0x3                              | Nein    |                         |                    |                               |
| 0::0.0.0.0                             | Ja      | ::0                     | Unspecified        | ?                             |
| 200a:a:b:c:d:e:f                       | Nein    |                         |                    |                               |
| 2ac2:::09:3                            | Nein    |                         |                    |                               |
| ff15::aaaa                             | Ja      | ff15::aaaa              | Multicast          | Nein (Scope 5 ist Site-Local) |
| ff00:180bac::1                         | Nein    |                         |                    |                               |
| ff11:1:2:3:4:5:6:7:8                   | Nein    |                         |                    |                               |
