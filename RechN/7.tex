\documentclass[10pt,a4paper]{article}
\usepackage[utf8]{inputenc}
\usepackage[german]{babel}
\usepackage{mathrsfs}
\usepackage{amsmath}
\usepackage{amsfonts}
\usepackage{amssymb}
\usepackage{amsthm}
\usepackage[left=2cm,right=2cm,top=2cm,bottom=2cm]{geometry}
\usepackage{minted}

\begin{document}

\section{Aufgabe 7.1}

Dies ist besser, weil im dem wohl wahrscheinlichten Fall für den Ausfall einer,
nämlich die Zerstörung eines Links oder Absturz des Routers, dieser ja nicht
mehr melden kann, dass er abgestürzt ist bzw. keine Verbindung mehr hat.

\section{Aufgabe 7.2}

Areas gruppieren das Netzwerk, sodass jeder Router nur die Topologie seiner
Gruppe im Auge behalten muss. Das hält die Link-State-Datenbank klein und sorgt
auch dafür, dass Router diese nicht die ganze Zeit updaten müssen, weil sich in
sehr großen Netzwerken immer irgendwo etwas ändert.

\section{Aufgabe 7.3}

Bei iBGP sendet der Router den original NEXT-HOP und nicht seine eigene IP
weiter.

\section{Aufgabe 7.4}

Mit der Zeit synchronisieren sich die Router und alle $x$ Sekunden gibt es
Traffic-Bursts.

\section{Aufgabe 7.5}

\subsection{Teil 1}

Die Tabellen enthalten im Moment ca. 530000 Einträge. Durch Aggregation könnte
diese Zahl um 44.9\% verringert werden. (Quelle:
http://www.cidr-report.org/as2.0/\#Gains)

\subsection{Teil 2}

\section{Aufgabe 7.6}

\subsection{Direkt nach dem Ausfall}

\begin{minted}{shell}
Tabelle A: A(lokal) - 0, B() - u, C(ac) - 1, D(ac) - 3, E(ac) - 2
Tabelle B: A() - u, B(lokal) - 0, C(be) - 2, D(bd) - 1, E(be) - 1
\end{minted}

\subsection{Runde 1}

\begin{minted}{shell}
Von A nach C: A - 0, B - u, C - u, D - u, E - u
Von B nach D: A - u, B - 0, C - 2, D - u, E - 1
Von B nach E: A - u, B - 0, C - u, D - 1, E - u
\end{minted}

\begin{minted}{shell}
Tabelle C: A(ac) - 1, B(ce) - 2, C(lokal) - 0, D(ce) - 2, E(ce) - 1
Tabelle D: A(de) - 3, B(bd) - 1, C(de) - 2, D(lokal) - 0, E(de) - 1
Tabelle E: A(ce) - 2, B(be) - 1, C(ce) - 1, D(de) - 1, E(lokal) - 0
\end{minted}

\subsection{Runde 2}

\begin{minted}{shell}
Von C nach A: A - u, B - 2, C - 0, D - 2, E - 1
Von C nach E: A - 1, B - u, C - 0, D - u, E - u
Von D nach B: A - 3, B - u, C - 2, D - 0, E - 1
Von D nach E: A - u, B - 1, C - u, D - 0, E - u
Von E nach B: A - 2, B - u, C - 1, D - 1, E - 0
Von E nach C: A - u, B - 1, C - u, D - 1, E - 0
Von E nach D: A - 2, B - 1, C - 1, D - u, E - 0
\end{minted}

\begin{minted}{shell}
Tabelle A: A(lokal) - 0, B(ac) - 3, C(ac) - 1, D(ac) - 3, E(ac) - 2
Tabelle B: A(be) - 3, B(lokal) - 0, C(be) - 2, D(bd) - 1, E(be) - 1
\end{minted}

\subsection{Runde 3}

\begin{minted}{shell}
Von A nach C: A - 0, B - u, C - u, D - u, E - u
Von B nach D: A - 3, B - 0, C - 2, D - u, E - 1
Von B nach E: A - u, B - 0, C - u, D - 1, E - u
\end{minted}

\begin{minted}{shell}
Keine Updates
\end{minted}

\section{Aufgabe 7.7}

Ich kann hier kein Beispiel konstruieren, weil das Netzwerk zu klein ist und
deshalb die Kosten der einzelnen Pfade zu ähnlich.

\section{Aufgabe 7.8}

\subsection{Teil 1}

\begin{tabular}{c|c}
  An & Annonciert\\\hline
  AS3320 & AS4711, AS31337\\
  AS1299 & AS4711, AS31337\\
  AS15169 & alle\\
  AS31337 & AS3320, AS1299, AS15169, AS4711
\end{tabular}

\subsection{Teil 2}

Erstmal nichts, aber wenn sich die Netzwerktopologie innerhalb von AS4711
ändert, kriegen es die Router nicht mit und routen vielleicht über
nicht-optimale Routen oder ins Leere.

\end{document}