\documentclass[10pt,a4paper]{article}
\usepackage[utf8]{inputenc}
\usepackage[german]{babel}
\usepackage{mathrsfs}
\usepackage{amsmath}
\usepackage{amsfonts}
\usepackage{amssymb}
\usepackage{amsthm}
\usepackage[left=2cm,right=2cm,top=2cm,bottom=2cm]{geometry}

\begin{document}

\section{Aufgabe 4.1}

UDP ist geeignet, wenn der Verlust einzelner Datenpakete irrelevant ist, weil
z.B. immer nur die aktuellsten Daten interessant sind. Ein solcher
Anwendungsfall könnte ein Thermometer sein, dass seine Messungen per UDP
versendet. Wenn eine der Messungen verloren geht, wartet man einfach auf die
nächste, weil die besser ist als die verloren gegangene.

\section{Aufgabe 4.2}

Ja, wenn sie ein individuelles Zuverlässigkeitsprotokoll auf UDP implementiert
oder über ein eigenes, internes Netzwerk läuft, über dass man volle Kontrolle
hat, sodass solche Garantien auf einer niedrigeren Schicht gemacht werden
können.

\section{Aufgabe 4.3}

\subsection{Teil 1}

Nein, für jeden Client wird ein eigener Socket erstellt.

\subsection{Teil 2}

Das IP-Protokoll enthält die Adresse des Senders. Darüber hinaus können auch
noch höher liegende Protokolle wie TLS zum Einsatz kommen, um sicherzustellen,
dass niemand im Namen eines anderen mit Host C kommuniziert.

\subsection{Teil 3}

Durch IP und Port von Absender und Empfänger.

\section{Aufgabe 4.4}

\subsection{Teil 1}

Um auf ein bestimmtes Paket zu verweisen und den Paketen eine Reihenfolge zu
geben.

\subsection{Teil 2}

Um zu bestätigen, dass ein Paket angekommen ist.

\subsection{Teil 3}

Um zu reagieren, falls entweder ein Paket oder ein ACK verloren gegangen ist.

\subsection{Teil 4}

Um zu überprüfen, ob ein Paket beschädigt wurde.

\subsection{Teil 5}

Um die Auslastung der Leitung zu erhöhen.

\section{Aufgabe 4.5}

Nein, denn die Prüfsumme ist stets kürzer als die Daten, die sie prüfen
soll. Deshalb müssen mehrere Daten auf dieselbe Prüfsumme abgebildet
werden. Wenn die Bits also besonders unglücklich kippen, wird daraus ein zur
Prüfsumme passendes, aber falsches Datenpaket entstehen.

\section{Aufgabe 4.6}

\section{Aufgabe 4.7}

Selber Empfänger wie bei RDT 2.2.

\section{Aufgabe 4.8}

\section{Aufgabe 4.9}

Alle Werte in dieser Aufgabe sind modulo $1024$ zu genießen.

\subsection{Teil 1}

Mögliche Werte für $base$ sind $k, k - 1, k - 2, k - 3$. Die möglichen
Sequenznummernbereiche, sind alle möglichen $base$ bis $base + 3$.

Wenn der Empfänger $k$ erwartet, kann der Sender noch kein ACK für $k$ bekommen
haben. Dadurch ist $base$ nach oben durch $k$ beschränkt. Durch die Fenstergröße
von $4$ kann der Sender aber auch nicht mehr als $3$ Pakete vor $k$ verschickt
haben, auf deren Bestätigung er noch wartet. Die beschränkt $k$ nach unten durch
$k - 3$.

Die Sequenznummernbereiche errechnen sich direkt aus den möglichen Werten für
$base$.

\subsection{Teil 2}

Damit der Empfänger $k$ erwarten kann, muss er vorher $k - 1$ empfangen
haben. Damit aber der Sender $k - 1$ verschickt haben kann, muss er bereits ein
ACK für $(k - 1) - 4$ empfangen haben. Also für das letzte Paket vor dem
Fenster, in dem $k - 1$ das letzte Paket ist.

Somit sind die ACKs $k - 4, k - 3, k - 2, k - 1$ möglich.

\section{Aufgabe 4.10}

Die maximale Datenrate wird erreicht, wenn der Sender immer ein Paket lossendet,
wenn möglich und keine Pakete oder ACKs verloren gehen. Wenn ein ACK schneller
zurückkommt als der Sender braucht, um ein komplettes Fenster zu senden, kann er
so ununterbrochen senden.

Die Zeit, um ein Paket mit zu vernachlässigender Größe über die Leitung zu
versenden, ist
\begin{equation}
  t = \frac{s}{v} = \frac{2 \cdot 10^{7} m}{2 \cdot 10^{8} \frac{m}{s}} = 10^{-1}s
\end{equation}

Ein Paket von Größe $p$ mit Datenrate $d$ zu versenden und ein ACK zu empfangen,
dauert dann
\begin{equation}
  T = (\frac{p}{d} + t) + t = \frac{p}{d} + 2t
\end{equation}
Es dauert also $T$, das erste Paket des Fenster zu verschicken und das
entsprechende ACK zu empfangen.

Wir erhalten die maximale Datenrate $d_{max}$, wenn wir in Zeit $T$ das gesamte
Fenster verschicken.
\begin{equation}
  d_{max} = \frac{N \cdot p}{T} = \frac{N \cdot p}{\frac{p}{d_{max}} + 2 \cdot \frac{s}{v}}
\end{equation}
Es folgen ein paar Umformungen
\begin{align*}
  d_{max} & = \frac{N \cdot p}{\frac{p}{d_{max}} + 2 \cdot \frac{s}{v}}\\
  \Leftrightarrow \quad d_{max} \left( \frac{p}{d_{max}} + 2 \cdot \frac{s}{v} \right) & = N \cdot p\\
  \Leftrightarrow \quad p + 2 d_{max} \cdot \frac{s}{v} & = N \cdot p\\
  \Leftrightarrow \quad 2 d_{max} \cdot \frac{s}{v} & = (N - 1) \cdot p\\
  \Leftrightarrow \quad d_{max} & = \frac{1}{2} \frac{v}{s} (N - 1) \cdot p\\
\end{align*}

In diesem Fall ist dann
\begin{equation}
  d_{max} = \frac{1}{2} \frac{2 \cdot 10^{8} \frac{m}{s}}{2 \cdot 10^{7}m} \cdot 19 \cdot 10^{3} bit = \frac{19}{2} \cdot 10^{4} \frac{bit}{s}
\end{equation}

Man kann die Datenrate erhöhen, indem man an den Variablen entsprechend
rumschraubt.

\end{document}