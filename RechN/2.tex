\documentclass[10pt,a4paper]{article}
\usepackage[utf8]{inputenc}
\usepackage[german]{babel}
\usepackage{mathrsfs}
\usepackage{amsmath}
\usepackage{amsfonts}
\usepackage{amssymb}
\usepackage{amsthm}
\usepackage[left=2cm,right=2cm,top=2cm,bottom=2cm]{geometry}

\begin{document}

\section{Aufgabe 2.1}

\section{Aufgabe 2.2}

\section{Aufgabe 2.3}

\section{Aufgabe 2.4}

\section{Aufgabe 2.5}

\section{Aufgabe 2.6}

\section{Aufgabe 2.7}

\section{Aufgabe 2.8}

\subsection{Teil 1}

Falsch, weil bei HTTP jede Resource einzeln angefragt werden muss und der Server
nicht selbstständig welche senden kann.

\subsection{Teil 2}

Falsch, weil die Anfragen an denselben Host gerichtet sind. Wenn es
unterschiedliche Hosts wären, wäre es nicht möglich.

\subsection{Teil 3}

Falsch, weil bei non-persistent connections mehrere Verbindungen aufgebaut
werden und ein Paket zu genau einer Verbindung gehört, da es über den Header
identifiziert wird, den es nur einmal gibt.

\subsection{Teil 4}

Falsch, wenn beide Optionen die Leitung voll auslasten. 1.1 kann die Objekte
über dieselbe Leitung herunterladen, während 1.0 drei Verbindungen aufbauen
muss. Dies könnte nur schneller sein, wenn die drei Verbindungen parallel laufen
würden. Die Annahme über die Geschwindigkeit schließt dies jedoch aus.

\section{Aufgabe 2.9}

\subsection{Teil 1}

Beide können es signalisieren, indem sie den Connection-Header mit dem Wert
``close'' senden. Dies bedeutet, dass die Verbindung nicht persistent ist und
nach dem Abschluss der Anfrage geschlossen werden soll.

\subsection{Teil 2}

Keine. Um HTTP verschlüsselt zu übertragen, muss man auf HTTPS (RFC2818)
zurückgreifen. Dort wird nur gesagt, dass man doch HTTP über eine TLS-Verbindung
senden soll.

\section{Aufgabe 2.10}

Unabhängig davon, ob rekursives oder iterative DNS-Auflösung verwendet wird,
beträgt die zur Namensauflösung benötigte Zeit
\begin{equation}
  t_{DNS} = \sum_{i = 1}^{n} RTT_{i}
\end{equation}
Es ist nur eine $RTT$ pro Server notwendig, weil DNS über UDP läuft.

Da die Webseite sehr klein ist, zählt dort auch nur die Ausbreitungsverzögerung
und die Gesamtdauer ist
\begin{equation}
  t_{DNS} + 2 \cdot RTT_{0}
\end{equation}
Hier sind es 2 $RTT$, weil man erst die TCP-Verbindung aufbauen muss und dann in
einem zweiten Paket die Daten übertragen werden.

\section{Aufgabe 2.11}

\subsection{Teil 1}

\begin{equation}
  t_{DNS} + 5 \cdot (2 \cdot RTT_{0})
\end{equation}

\subsection{Teil 2}

Zuerst baut man eine TCP-Verbindung auf und dann werden nacheinander die 5
Objekte geladen.
\begin{equation}
  t_{DNS} + RTT_{0} + 5 \cdot RTT_{0}
\end{equation}

\subsection{Teil 3}

Das dauert genausolange wie ein einzelnes Objekt anzufragen, weil die einzelnen
Pakete keine Übertragungsverzögerung haben und deshalb alle gleichzeitig
versendet werden.
\begin{equation}
  t_{DNS} + RTT_{0} + RTT_{0}
\end{equation}

\subsection{Teil 4}

Da die 4 Objekte gleichzeitig geladen werden, dauert das Übertragen der 4
weiteren Dateien nur so lange, wie eine einzelne zu laden.
\begin{equation}
  t_{DNS} + (2 \cdot RTT_{0}) + (2 \cdot RTT_{0})
\end{equation}

\section{Aufgabe 2.12}

Wenn Upload und Download über eine Internetverbindung nicht gleichzeitig möglich
sind, führt das Herunterladen durch die anderen offensichtlich zu einer
Verlangsamung.

Wenn das jedoch möglich ist, findet immernoch eine Verlangsamung statt, aber sie
ist geringer. Die anderen Teilnehmer werden zwischendurch Kontrollinformationen
wie Statusinformationen oder Anfragen für weitere Dateiteile an uns senden, was
Empfangskapazität benötigt.

\section{Aufgabe 2.13}

\subsection{Teil 1}

Der Server sendet die Datei mit derselben, maximalen Geschwindigkeit an alle
Clients. Dann dauert die Übertragung der Datei an einen Client
\begin{equation}
  t_{client} = \frac{F}{\frac{d_{S}}{N}} = \frac{NF}{d_{S}}
\end{equation}
Da alle Übertragungen parallel stattfinden, ist dies auch die Verteilungszeit.

\subsection{Teil 2}

Der Server sendet die Datei gleichzeitig an alle Clients mit maximal möglicher
Datenrate, aber mindestens $d_{min}$. Dann dauert die Übertragung der Datei an
Client $i$
\begin{equation}
  t_{i} = \frac{F}{d_{i}}
\end{equation}
wobei $d_{i}$ die Datenrate zu Client $i$ ist. Da alle Übertragungen
gleichzeitig stattfinden, ist die Verteilungszeit das Maximum der Einzelzeiten.
\begin{equation}
  \max_{i} t_{i} = \max_{i} \frac{F}{d_{i}} = \frac{F}{d_{min}}
\end{equation}

\end{document}