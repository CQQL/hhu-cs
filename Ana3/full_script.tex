\documentclass[10pt,a4paper]{report}
\usepackage[utf8]{inputenc}
\usepackage[german]{babel}
\usepackage{mathrsfs}
\usepackage{amsmath}
\usepackage{amsfonts}
\usepackage{amssymb}
\usepackage{amsthm}
\usepackage{enumitem}
\usepackage{chngcntr}
\usepackage[left=2cm,right=2cm,top=2cm,bottom=2cm]{geometry}
\title{Analysis III, Wintersemester 2013-2014, Prof. Dr. Singhof}
\author{Marten Lienen}

\newtheorem{satz}{Satz}
\newtheorem*{lemma}{Lemma}
\newtheorem*{defi}{Definition}
\newtheorem*{remark}{Bemerkung}
\newtheorem*{example}{Beispiel}

\counterwithout{section}{chapter}
\renewcommand{\thesection}{§\arabic{section}}
\counterwithin{satz}{section}
\renewcommand{\thesatz}{\arabic{satz}}

\newcommand{\dotcup}{\ensuremath{\mathaccent\cdot\cup}}

\begin{document}

\maketitle

\chapter{Maß- und Integrationstheorie}

\section{Quader und Figuren}

\section{$\sigma$-Algebren und Maße}

\section{Das Lebesgue-Maß}

\section{Messbare Funktionen}

\section{Integrationstheorie}

\section{Vertauschbarkeit des Integrals mit Grenzprozessen}

\section{Der Satz von Fubini}

\section{Die Transformationsformel}

\section{Die Räume $L^{p}$}

\chapter{Vektoranalysis}

\section{Untermannigfaltigkeiten des $\mathbb{R}^{n}$}

\section{Zusammenhängende metrische Räume}

\begin{defi}
  Ein metrischer Raum $X$ heißt \underline{zusammenhängend}, wenn die einzigen Teilmengen von $X$, die sowohl offen als auch abgeschlossen in $X$ sind, $\emptyset$ und $X$ sind.
\end{defi}

\begin{example}
  $X = [0, 1] \cup [2, 3]$ ist nicht zusammenhängend, denn $[0, 1]$ und $[2, 3]$ sind offen und abgeschlossen in $X$.
\end{example}

\begin{remark}
  Genau dann ist $X$ zusammenhängend, wenn gilt:
  Sind $A, B$ offene Teilmengen von $X$ mit $X = A \dotcup B$, so ist $A \ne \emptyset$ oder $B \ne \emptyset$.
\end{remark}

\begin{satz}
  Sei $X$ eine Teilmenge von $\mathbb{R}$, die mehr als einen Punkt enthält.
  Genau dann ist $X$ zusammenhängend, wenn $X$ ein (offenes/abgeschlossenes/halboffenes/eigentliches/uneigentliches) Intervall ist.
\end{satz}

\begin{proof}
  \begin{itemize}
  \item Angenommen, $X$ sei kein Intervall.
    Dann gibt es $x_{1}, x_{2}, \xi \in \mathbb{R}$ mit $x_{1}, x_{2} \in X$ und $x_{1} < \xi < x_{2}$.
    Sei $A := \{ x \in X \mid x < \xi \}$ und $B := \{ x \in X \mid x > \xi \}$.
    Dann ist $A \dotcup B = X$ und $A$ und $B$ sind offen in $X$ nach §10 Satz 3.
  \item Angenommen $X$ sei nicht zusammenhängend.
    Dann gibt es offene Teilmengen $A, B$ von $X$ mit $A \dotcup B = X$ und $A \ne \emptyset \ne B$.
    Sei $a \in A$, $b \in B$.
    O.b.d.A. sei $a < b$.
    Wäre $X$ ein Intervall, so wäre $[a, b] \subseteq X$.
    $M := \{ x \in A \mid x < b \}$.
    Wegen $a \in M$ ist $M \ne \emptyset$.
    $M$ ist nach oben beschränkt.
    Sei $c := \sup M \in [a, b] \subseteq X$.
    Da $A = X \setminus B$ abgeschlossen in $X$, folgt $c \in A$, also $c \notin B$.
  \end{itemize}
\end{proof}


\begin{proof}
  Beweisskizee als Hilfestellung für Übung:
  \begin{enumerate}[label={\alph*)}]
  \item Sei $M$ zusammenhängend.
    Sei $x_{0} \in M$.
    Sei $A := \{ x \in M \mid \textit{es gibt einen Weg $w : [0, 1] \rightarrow M$ mit $w(0) = x_0$ und $w(1) = x$} \}$.
    Zu zeigen: $A = M$.
    Dann ist $A \ne \emptyset$, denn $x_{0} \in A$.
    \begin{description}
    \item[$A$ ist offen in $M$] Sei $a \in A$.
      Nach der Folgerung aus Satz 1 von §10 besitzt $a$ eine Umgebung $U$ in $M$, die homöomorph zu $\mathbb{R}^{n}$ ist.
      Ist $x \in U$, so gibt es einen Weg $w$ in $U$ mit $w(0) = a$, $w(1) = x$.
      Es gibt einen Weg $v$ von $x_{0}$ nach $a$.
      Deshalb gibt es auch einen Weg von $x_{0}$ nach $x$.
      \item[$A$ ist abgeschlossen in $M$] Weil $M$ zusammenhängend ist, folgt $A = M$.
    \end{description}
  \item Ähnlich
  \item folgt direkt aus b)
  \end{enumerate}
\end{proof}

\section{Kompakte metrische Räume}

\begin{defi}
  Ist $X$ eine Menge und ist $\{ A_{i} \mid i \in \Lambda \}$ eine Menge von Teilmengen von $X$ so heißt $\{ A_{i} \mid i \in \Lambda \}$ eine \underline{Überdeckung} von $X$, wenn $X = \cup_{i \in \Lambda} A_{i}$.
\end{defi}

\begin{defi}
  Sei $X$ ein metrischer Raum und $\{ A_{i} \mid i \in \Lambda \}$ eine Überdeckung von X.
  Dann heißt $\{ A_{i} \mid i \in \Lambda \}$ eine \underline{offene Überdeckung} von $X$, wenn alle $A_{i}$ offen in $X$ sind.
\end{defi}

\begin{defi}
  Ein metrischer Raum heißt \underline{kompakt}, wenn jede offene Überdeckung von $X$ eine endliche Teilüberdeckung besitzt, wenn also gilt:
  Ist $\{ A_{i} \mid i \in \Lambda \}$ eine offene Überdeckung von $X$, so gibt es $n \in \mathbb{N}$ und $i_{1}, \dots, i_{n} \in \Lambda$ mit $X = A_{i_{1}} \cup \dots \cup A_{i_{n}}$.
\end{defi}

\begin{remark}
  Eine Teilmenge $A$ eines metrischen Raumes $X$ ist genau dann kompakt, wenn gilt:
  Sind $A_{i} (i \in \Lambda)$ offene Teilmengen von $X$ mit $A \subseteq \cup_{i \in \Lambda} A_{i}$, so gibt es ein $n \in \mathbb{N}$ und $i_{1}, \dots, i_{n} \in \Lambda$ mit $A \subseteq A_{i_{1}} \cup \dots \cup A_{i_{n}}$.
\end{remark}

\begin{satz}
  Seien $X, Y$ metrische Räume und sei $f : X \rightarrow Y$ stetig.
  Ist K eine kompakte Teilmenge von $X$, so ist $f(K)$ kompakt.
\end{satz}

\begin{proof}
  Seien $A_{i} (i \in \Lambda))$ offene Teilmengen von $Y$ mit $f(K) \subseteq \cup_{i \in \Lambda} A_{i}$.
  Weil $f$ stetig ist, ist $f^{-1}(A_{i})$ offen in $X$ $\forall i \in \Lambda$.
  Es ist $K \subseteq f^{-1}(\cup_{i \in \Lambda}) A_{i} = \cup_{i \in \Lambda} f^{-1}(A_{i})$.
  Es gibt also ein $n \in \mathbb{N}$ und $i_{1}, \dots, i_{n} \in \Lambda$ mit $K \subseteq f^{-1}(A_{i_{1}}) \cup \dots \cup f^{-1}(A_{i_{n}})$.
  Daraus folgt
  \begin{equation}
    f(K) \subseteq f(f^{-1}(A_{i_{1}}) \cup \dots \cup f^{-1}(A_{i_{n}})) = f(f^{-1}(A_{i_{1}})) \cup \dots \cup f(f^{-1}(A_{i_{n}})) = A_{i_{1}} \cup \dots \cup A_{i_{n}}
  \end{equation}
\end{proof}

\begin{satz}
  Jede abgeschlossene Teilmenge A eines kompakten metrischen Raumes $X$ ist kompakt.
\end{satz}

\begin{proof}
  Seien $A_{i} (i \in \Lambda))$ offene Teilmengen von $X$ mit $A \subseteq \cup_{i \in \Lambda} A_{i}$.
  Die $A_{i}$ zusammen mit $X \setminus A$ bilden eine offene Überdeckung von $X$.
  Weil $X$ kompakt ist, gibt es eine endliche Teilüberdeckung von $X$, d.h. es gibt ein $n \in \mathbb{N}$ und $i_{1}, \dots, i_{n} \in \Lambda$ mit $X = A_{i_{i}} \cup \dots \cup A_{i_{n}} \cup (X \setminus A) \Rightarrow A \setminus A_{i_{1}} \cup \dots \cup A_{i_{n}}$.
\end{proof}

\begin{satz}
  Jede kompakte Teilmenge $A$ eines metrischen Raumes $X$ ist abgeschlossen in $X$.
\end{satz}

\begin{proof}
  Zeige $X \setminus A$ ist offen in $X$.
  Sei $x \in X \setminus A$.
  Wir wollen zeigen: $X \setminus A$ ist Umgebung von $x$ in $X$.
  Ist $y \in A$, so ist $y \ne x$; deswegen gibt es offene Teilmengen $U_{y} und V_{y}$ von $X$ mit $x \in U_{y}$, $y \in V_{y}$ und $U_{y} \cap V_{y} = \emptyset$.
  Dann ist $A \subseteq \cup_{y \in A} V_{y}$.
  Weil $A$ kompakt ist, gibt es ein $n \in \mathbb{N}$ und Punkte $y_{1}, \dots, y_{n} \in A$ mit $A \subseteq V_{y_{1}} \cup \dots \cup V_{y_{n}} = V$.
  Sei $U := U_{y_{1}} \cap \dots \cap U_{Y-n}$.
  Dann ist $U$ eine offene Umgebung von $x$ mit $U \cap V = \emptyset$, also $U \subseteq X \setminus A$.
\end{proof}

\begin{satz}
  Seien $X, Y$ metrische Räume; $X$ sei kompakt und $f : X \rightarrow Y$ sei stetig und bijektiv.
  Dann ist $f^{-1} : Y \rightarrow X$ stetig.
  Deswegen ist $f$ ein Homöomorphismus.
\end{satz}

\begin{proof}
  Wir zeigen: Ist $A$ abgeschlossen in $X$, so ist $f(A) = (f^{-1})^{-1}(A)$ abgeschlossen in $Y$.
  Nach Satz 2 ist $A$ abgeschlossen $\Rightarrow$ nach Satz 1 ist $f(A)$ kompakt $\Rightarrow$ nach Satz 3 $f(A)$ ist abgeschlossen in $Y$.
\end{proof}

\begin{defi}
  Sei $X$ ein metrischer Raum, $(x_{n})$ eine Folge in $X$ und $a \in X$.
  Dann heißt $a$ ein Häufungspunkt von $(x_{n})$, wenn es eine Teilfolge von $(x_{n})$ gibt, die gegen $a$ konvergiert; äquivalent dazu: Wenn es für jede Umgebung $U$ von $a$ in $X$ unendlich viele $n \in \mathbb{N}$ gibt mit $x_{n} \in U$.
\end{defi}

\begin{satz}
  Sei $X$ ein metrischer Raum.
  Äquivalent sind:
  \begin{enumerate}[label={\alph*)}]
  \item $X$ ist kompakt
  \item Jede Folge in $X$ besitzt eine Häufungspunkt in $X$
  \item $X$ ist vollständig und für jedes $\varepsilon > 0$ gibt es ein $n \in \mathbb{N}$ und $x_{n}, \dots, x_{n} \in X$ mit $X = \cup_{i = 1}^{n} B_{\varepsilon}(x_{i})$
  \end{enumerate}
\end{satz}

\begin{proof}
  $a) \Rightarrow b)$: Sei $X$ kompakt und $(x_{n})$ eine Folge in $X$.
  Sei $F_{n}$ der Abschluss der Menge $\{ x_{n}, x_{n + 1}, x_{n + 2}, \dots \}$ in $X$.
  Wir werden zeigen, dass $\cap_{n = 1}^{\infty} F_{n} \ne \emptyset$.
  (Ein Element von $\cap_{n = 1}^{\infty} F_{n}$ ist ein Häufungspunkt von $(x_{n})$)

  Angenommen, es sei $\cap_{n = 1}^{\infty} F_{n} = \emptyset$.
  Sei $A_{n} := X \setminus F_{n}$.
  Dann ist $A_{n}$ offen in $X$ und $\cap_{n = 1}^{\infty} A_{n} = \cap_{n} (X \setminus F_{n}) = X \setminus \cap_{n} F_{n} = X$.
  Deswegen bilden die $A_{n}$ mit $n \in \mathbb{N}$ eine offene Überdeckung von $X$.
  Weil $X$ kompakt ist, gibt es $n_{n}, \dots, n_{k} \in \mathbb{N}$ mit $X = A_{n_{1}} \cup \dots \cup A_{n_{k}}$.
  Für $n \ge m$ ist $F_{n} \subseteq F_{m}$, also $A_{n} \supseteq A_{m}$.
  Ist $n_{0} = \max \{ n_{1}, \dots, n_{k} \}$, so ist also $X = A_{n_{0}} \Rightarrow F_{n_{0}} = \emptyset$, Widerspruch, da $x_{n_{0}} \in F_{n_{0}}$.

  $b) \Rightarrow c)$: Sei $(x_{n})$ eine Cauchy-Folge in $X$.
  Dann besitzt $(x_{n})$ einen Häufungspunkt $a$, d.h. eine Teilfolge von $(x_{n})$ konvergiert gegen $a$.
  Nach Aufgabe 40 konvergiert $(x_{n})$ gegen $a$.
  Deswegen ist $X$ vollständig.

  Sei $\varepsilon > 0$.
  Angenommen $X$ ist nicht die Vereinigung von endlich vielen Kugeln von Radius $\varepsilon$.
  Dann definiert man induktiv eine Folge $(x_{n})$ in $X$, so dass gilt: Ist $n \ne m$, so ist $d(x_{n}, x_{m}) \ge \varepsilon$.
  Dann kann $(x_{n})$ keinen Häufungspunkt besitzen, Widerspruch.

  $c) \Rightarrow a)$: Sei $\{ A_{i} \mid i \in \Lambda \}$ offene Überdeckung von $X$.
  Angenommen, es gäbe keine endliche Teilüberdeckung.
  Wir werden induktiv eine Folge $(B_{n})$ von Kugeln vom Radius $\frac{1}{2^{n}}$ definieren, von denen jede nicht durch endlich viele $A_{i}$ überdeckt wird:
  \begin{description}
  \item[$n = 0$] Nach Vorraussetzung wird $X$ von endlich vielen Kugeln vom Radius $1$ überdeckt.
    Von diesen kann eine nicht von endlich vielen $A_{i}$ überdeckt werden; nenne sie $B_{0}$.
  \item[$n - 1 \rightarrow n$] Sei bereits $B_{n - 1}$ konstruiert.
    Weil $X$ von endlich vielen Kugeln vom Radius $\frac{1}{2^{n}}$ überdeckt wird, gibt es unter diesen eine, die nicht von endlich vielen der $A_{i}$ überdeckt wird und nicht-leeren Schnitt mit $B_{n - 1}$ hat.
    $B_{n}$ habe den Mittelpunkt $x_{n}$.
    Wegen $B_{n} \cap B_{n - 1} \ne \emptyset$ ist $d(x_{n}, x_{n - 1}) \le \frac{1}{2^{n}} + \frac{1}{2^{n - 1}} \le \frac{1}{2^{n - 2}}$.
    Ist also $n \le p < q$, so $d(x_{p}, x_{q}) \le d(x_{p}, x_{p - 1}) + \dots + d(x_{q - 1}, x_{q}) \le \frac{1}{2^{p - 2}} \le \frac{1}{2^{n - 2}}$.
  \end{description}
  Deswegen ist $(x_{n})$ eine Cauchy-Folge, konvergiert also gegen ein $a \in X$.
  Es gibt ein $i_{0} \in \Lambda$ mit $a \in A_{i_{0}}$.
  Da $A_{i_{0}}$ offen ist, existiert $\varepsilon > 0$ mit $B_{\varepsilon}(a) \subseteq A_{i_{0}}$.
  Für großes $n$ ist $x_{n} \in B_{\frac{\varepsilon}{2}}(a)$ und $B_{n} \subseteq B_{\varepsilon}(a)$.
  Daher ist $B_{n}$ für großes $n$ in $A_{i_{0}}$ enthalten, Widerspruch.
\end{proof}

\begin{satz}[Heine-Borel]
  Für eine Teilmenge $X$ von $\mathbb{R}^{n}$ sind äquivalent:
  \begin{enumerate}[label={\alph*)}]
  \item $X$ ist kompakt
  \item $X$ ist beschränkt und abgeschlossen in $\mathbb{R}^{n}$
  \end{enumerate}
\end{satz}

\begin{proof}
  $a) \Rightarrow b)$: Ist $X$ kompakt, so ist $X$ abgeschlossen in $\mathbb{R}^{n}$ nach Satz 3.
  Nach Satz 5 wird $X$ durch endlich viele Kugeln vom Radius $1$ überdeckt, ist also beschränkt.

  $b) \Rightarrow a)$: Weise Bedingung $c)$ vom Satz 5 nach:
  \begin{itemize}
  \item $X$ ist vollständig: Sei $(x_{m})$ eine Cauchy-Folge in $X$.
    Weil $\mathbb{R}^{n}$ vollständig ist, konvergiert $(x_{m})$ gegen ein $a \in \mathbb{R}^{n}$.
    Weil $X$ abgeschlossen in $\mathbb{R}^{n}$ ist, ist $a \in X$.
  \item Weil $X$ beschränkt ist, wird $X$ für jedes $\varepsilon > 0$ durch endlich viele $B_{\varepsilon}(x_{i})$ überdeckt.
  \end{itemize}
\end{proof}

\begin{defi}
  Ein metrischer Raum $X$ heißt \underline{lokalkompakt}, wenn jeder Punkt $a \in X$ eine kompakte Umgebung in $X$ besitzt.
\end{defi}

\begin{example}
  \begin{itemize}
  \item $\mathbb{R}^{n}$ ist lokalkompakt, aber nicht kompakt
  \item Jede Untermannigfaltigkeit des $\mathbb{R}^{n}$ ist lokalkompakt
  \end{itemize}
\end{example}

\section{Tangentialräume und Orientierungen}

\begin{defi}
  Sei $M$ eine $n$-dimensionale Untermannigfaltigkeit von $\mathbb{R}^{n}$ und $a \in M$.
  Ein Element $v \in \mathbb{R}^{n}$ heißt \underline{Tangentialvektor} an $M$ im Punkt $a$, wenn es ein offenes Intervall $I$ in $\mathbb{R}$ mit $0 \in I$ und eine $C^{1}$-Abbildung $\psi : I \rightarrow \mathbb{R}^{n}$ gibt mit:
  \begin{itemize}
    \item $\psi(I) \in M$
    \item $\psi(0) = a$
    \item $\psi'(0) = v$
  \end{itemize}
  Mit $T_{a}(M)$ bezeichnet man die Menge aller Tangentialvektoren an $M$ im Punkt $a$ und nennt $T_{a}(M)$ den \underline{Tangentialraum} an $M$ in $a$.
\end{defi}

\begin{satz}
  Sei $M$ eine $n$-dimensionaler linearer Teilraum von $\mathbb{R}^{n}$.
  \begin{enumerate}[label={\alph*)}]
    \item $T_{a}(M)$ ist en $n$-dimensionaler linearer Teilraum von $\mathbb{R}^{n}$
    \item Sei $\varphi : W \rightarrow V$ eine Karte von $M$ und $a \in V$. Sei $b \in W$ mit $\varphi(b) = a$. Dann ist $T_{a}(M) = Bild(D\varphi(b)) = \{ D\varphi(b) \cdot u \mid u \in \mathbb{R}^{n} \}$
    \item Sei $U$ eine offene Umgebung von $a$ in $\mathbb{R}^{n}$ und sei $g : U \rightarrow \mathbb{R}^{N - n}$ eine Submersion mit $M \cap U = \{ x \in U \mid g(x) = 0 \}$.
      Dann ist:
      \begin{equation}
        T_{a}(M) = Kern(Dg(a)) = \{ v \in \mathbb{R}^{n} \mid Dg(a) \cdot v = 0 \}
      \end{equation}
  \end{enumerate}
\end{satz}

\begin{proof}
  Analysis II, §16, Satz 5
\end{proof}

\begin{example}
  \begin{equation}
    M = S^{N - 1}
  \end{equation}
  Sei $U = \{ x \in \mathbb{R}^{N} \mid x \ne 0 \}$ und $g : U \rightarrow \mathbb{R}$ gegeben durch $g(x) = x_{1}^{2} + \dots + x_{N}^{2} - 1$.
  Dann ist $S^{N - 1} = \{ x \in U \mid g(x) = 0 \}$.
  \begin{equation}
    Dg(x) = 2x^{T}
  \end{equation}
  Nach Satz 1c) ist $T_{a}(S^{N - 1}) = \{ v \in \mathbb{R}^{n} \mid 2a^{T}v = 0 \} = a^{T}$.
\end{example}

\begin{example}
  Sei $M$ ein $n$-dimensionaler affiner Teilraum von $\mathbb{R}^{N}$, d.h. es gibt ein $x_{0} \in \mathbb{R}^{N}$ und einen $n$-dimensionalen linearen Teilraum $E$ von $\mathbb{R}^{n}$ mit $M = x_{0} + E$.
  Dann ist $M$ eine $n$-dimensionale Untermannigfaltigkeit von $\mathbb{R}^{N}$: Sei $h : \mathbb{R}^{n} \rightarrow E$ ein linearer Isomorphismus.
  Sei $\varphi(y) := x_{0} + h(y)$, $\varphi : \mathbb{R}^{n} \rightarrow \mathbb{R}^{N}$.
  Für $y \in \mathbb{R}^{n}$ ist $D\varphi(y) \cdot u = h(u)$.
  Deswegen ist $\varphi$ eine Karte von $M$ mit $\varphi(\mathbb{R}^{n}) = M$.
  Sei $a \in M$ und $b \in \mathbb{R}^{n}$ mit $\varphi(b) = a$.
  Nach Satz 1b) ist $T_{a}(M) = Bild(D\varphi(b)) = Bild(h) = E$.
\end{example}

\begin{example}
  Sei $M$ wie im vorigen Beispiel und sei $U$ offen in $M$.
  Dann ist $U$ eine $n$-dimensionale Untermannigfaltigkeit von $\mathbb{R}^{N}$ und für $a \in U$ ist $T_{a}(u) = E$.
\end{example}

\begin{defi}
  Sei $M$ eine $n$-dimensionale Untermannigfaltigkeit von $\mathbb{R}^{n}$ und $a \in M$.
  Ein Element $v \in \mathbb{R}^{N}$ heißt \underline{Normalenvektor} an $M$ in $a$, wenn $\langle v \mid w \rangle = 0 \forall w \in T_{a}(M)$.
  Die Menge aller Normalenvektoren an $M$ in $a$ wird mit $N_{a}(M)$ bezeichnet und heißt der \underline{Normalenraum} an $M$ in $a$.
  \begin{equation}
    N_{a}(M) = T_{a}(M)^{T}
  \end{equation}
  Dies ist ein $(N - n)$-dimensionaler Teilraum vom $\mathbb{R}^{n}$.
\end{defi}

\begin{example}
  \begin{equation}
    N_{a}(S^{n - 1}) = \mathbb{R} \cdot a
  \end{equation}
\end{example}

\begin{defi}
  Sei $M$ eine Hyperfläche in $\mathbb{R}^{N}$.
  Ein \underline{Einheitsnormalenfeld} auf $M$ ist eine stetige Abbildung $\nu : M \rightarrow \mathbb{R}^{N}$ mit $\nu(a) \in N_{a}(M)$ und $||\nu(a)||_{2} = 1 \forall a \in M$.
\end{defi}

\begin{example}
  Auf $S^{N - 1}$ gibt es zwei Normalfelder: $\nu_{+}$ und $\nu_{-}$
  \begin{equation}
    \nu_{+}(a) := a, \nu_{-}(a) := -a
  \end{equation}
\end{example}

\begin{defi}
  Seien $U, V$ offen in $\mathbb{R}^{n}$ und $\varphi : U \rightarrow V$ ein Diffeomorphismus.
  $\varphi$ heißt \underline{orientierungserhalten}, wenn $\det(D\varphi(x)) > 0$.
\end{defi}

\begin{defi}
  Sei $M$ eine $n$-dimensionale Untermannigfaltigkeit von $\mathbb{R}^{N}$ mit $n \ge 1$.
  \begin{enumerate}[label={\alph*)}]
  \item Zwei Karten $\varphi_{1} : W_{1} \rightarrow V_{1}$ und $\varphi_{2} : W_{2} \rightarrow V_{2}$ von $M$ heißen \underline{gleichorientiert}, wenn die Parametertransformation $\tau(\varphi_{1}, \varphi_{2})$ orientierungserhalten ist.
  \item Ein Atlas $\mathscr{A}$ von $M$ heißt \underline{orientiert}, wenn je zwei seiner Karten gleich orientiert sind.
  \item $M$ heißt orientierbar, wenn $M$ einen orientierten Atlas besitzt.
  \item Zwei Atlanten $\mathscr{A}$ und $\mathscr{B}$ von $M$ heißen \underline{äquivalent}, wenn jede Karte von $\mathscr{A}$ mit jeder Karte von $\mathscr{B}$ gleichorientiert ist.
  \item Eine Äquivalenzklasse $\sigma$ orientierter Atlanten von $M$ heißt eine \underline{Orientierung} von $M$. Man nennt dann $(M, \sigma)$ eine \underline{orientierte Untermannigfaltigkeit}.
  \end{enumerate}
\end{defi}

\begin{remark}
  Meist sagt man: ``Sei $M$ eine orientierte Manigfaltigkeit.'' statt ``Sei $(M, \sigma)$ eine orientierte Manigfaltigkeit''.
\end{remark}

\begin{example}
  \begin{itemize}
  \item Sei $M$ eine $n$-dimensionale Untermannigfaltigkeit von $\mathbb{R}^{N}$.
    Es gebe eine Karte $\varphi$ von $M$, so dass $M$ das Bild von $\varphi$ ist.
    Dann ist $\{ \varphi \}$ ein orientierter Atlas von $M$; daher ist $M$ orientierbar.
  \item Ist insbesondere $U$ offen in $\mathbb{R}^{n}$, so ist $U$ eine $n$-dimensionale Untermannigfaltigkeit von $\mathbb{R}^{n}$ mit Atlas $\{ id_{U} \}$.
    Er definiert eine Orientierung von $U$, die sogenannte \underline{kanonische Orientierung}.
  \end{itemize}
\end{example}

\begin{remark}
  Sei $M$ eine $n$-dimensionale Untermannigfaltigkeit von $\mathbb{R}^{N}$ mit $n \ge 1$ und $M$ orientierbar.
  \begin{enumerate}[label={\alph*)}]
  \item Es gibt einen orientierten Atlas $\mathscr{A}$ von $M$, so dass alle Karten von $\mathscr{A}$ den Definitionsbereich $\mathbb{R}^{n}$ haben.
  \item Sei $\mathscr{A}$ ein orientierter Atlas wie in a).
    Ist $\varphi : \mathbb{R}^{n} \rightarrow V$ eine Karte in $\mathscr{A}$, so definieren wir $\tilde{\varphi} : \mathbb{R}^{n} \rightarrow V$ durch $\tilde{\varphi}(x_{1}, \dots, x_{n}) := \varphi(-x_{1}, x_{2}, \dots, x_{n})$.
    Dann ist $\tilde{\varphi}$ eine Karte von $M$ und $\varphi$ und $\tilde{\varphi}$ sind nicht gleichorientiert.
    Sei $\tilde{\mathscr{A}}$ die Menge aller $\tilde{\varphi}$ mit $\varphi \in \mathscr{A}$.
    Dann ist $\tilde{\mathscr{A}}$ ein orientierter Atlas von $M$ und $\mathscr{A}$ und $\tilde{\mathscr{A}}$ sind nicht äquivalent.
    Deswegen besitzt $M$ mindestens zwei verschiedene Orientiertungen (falls $M \ne \emptyset$).
  \item Ist $M$ orientierbar und zusammenhängend, so besitzt $M$ genau zwei Orientiertungen.
  \end{enumerate}
\end{remark}

\begin{satz}
  Sei $M$ eine Hyperfläche in $\mathbb{R}^{N}$.
  Dann sind äquivalent:
  \begin{enumerate}[label={\alph*)}]
  \item $M$ ist orientierbar
  \item Es gibt ein Einheitsnormalenfeld auf $M$
  \end{enumerate}
\end{satz}

\begin{proof}
  $a) \Rightarrow b)$: Sei $\mathscr{A}$ ein orientierter Atlas von $M$.
  Sei $a \in M$.
  Wähle eine Karte $\varphi : W \rightarrow V$ in $\mathscr{A}$ mit $a \in V$ und sei $b \in W$ mit $\varphi(b) = a$.
  Die lineare Abbildung $D\varphi(b)$ ist injektiv.
  Ist $e_{1}, \dots, e_{n}$ die Standardbasis von $\mathbb{R}^{n}$, so ist $D\varphi(b)e_{1}, \dots, D\varphi(b)e_{n}$ eine Basis von $T_{a}(M)$ (nach Satz 1).
  Sei $\nu(a)$ dasjenige der beiden Elemente vom Normalenraum $N_{a}(M)$ mit Norm $1$, für das die Matrix mit den Spalten $D\varphi(b)e_{1}, \dots, D\varphi(b)e_{n}, \nu(a)$ positive Determinante hat.
  Dann ist $\nu$ ein Einheitsnormalenfeld auf $M$.
\end{proof}

\begin{example}
  \begin{itemize}
  \item $S^{n - 1}$ ist orientierbar, weil es Einheitsnormalenfelder auf $S^{n - 1}$ gibt
  \item Das Möbiusband ist nicht orientierbar
  \end{itemize}
\end{example}

\begin{remark}
  Der Beweis von Satz 2 liefert für Hyperflächen eine Bijektion von der Menge der Orientierungen auf die Menge der Einheitsnormalenfelder.
\end{remark}

\begin{satz}
  Sei $M$ eine $n$-dimensionale orientierte Untermannigfaltigkeit von $\mathbb{R}^{N}$, $n \ge 2$.
  Sei $X$ eine abgeschlossene $n$-dimensionale Untermannigfaltigkeit mit Rand von $M$.
  Dann ist die $(n - 1)$-dimensionale Untermannigfaltigkeit $\partial X$ von $\mathbb{R}^{N}$ orientierbar.
\end{satz}

\begin{proof}
  Sei $\mathscr{A}$ ein Atlas von $M$, der zur gegebenen Orientierung gehört und folgende Eigenschaften hat:
  \begin{itemize}
  \item Ist $\varphi : W \rightarrow V$ eine Karte aus $\mathscr{A}$ mit $V \cap \partial X \ne \emptyset$, so ist $\varphi$ randadaptiert, d.h. $\varphi(\mathbb{R}_{-}^{n} \cap W) = X \cap V$, $\varphi(\partial \mathbb{R}_{-}^{n} \cap W) = \partial X \cap V$, $\mathbb{R}_{-}^{n} := \{ x \in \mathbb{R}^{n} \mid x_{1} \le 0 \}$.
  \item Wenn man eine Karte $\varphi : W \rightarrow V$ hat, so setzt man $W_{0} := \{ x \in \mathbb{R}^{n - 1} \mid (0, x) \in W \}$, $V_{0} := V \cap \partial X$.
    $\varphi_{0} : W_{0} \rightarrow V_{o}$ sei gegeben durch $\varphi_{0}(x) := \varphi(0, x)$.
  \end{itemize}
  Dann bilden die $\varphi_{0}$ eine orientierten Atlas von $\partial X$.
\end{proof}

\begin{defi}
  Ist $\sigma$ eine Orientierung von $M$, so liefert der Beweis von Satz 3 eine Orientierung von $\partial X$, welche die von $\sigma$ \underline{induzierte Orientierung} von $\partial X$ heißt.
\end{defi}

\section{Glatte Zerlegung der Eins}

\begin{defi}
  Sei $X$ ein metrischer Raum, $f : X \rightarrow \mathbb{R}$ eine Funktion.
  Mit $Supp(f)$ bezeichnet man den Abschluss der Menge $\{ x \in X \mid f(x) \ne 0 \}$ in $X$ und nennt $Supp(f)$ den \underline{Träger} von $f$.
\end{defi}

\begin{remark}
  Ziel: gegeben eine Untermannigfaltigkeit $M$ von $\mathbb{R}^{N}$, ein Atlas $\mathscr{A}$ von $M$ und eine $C^{\infty}$-Abbildung $f : M \rightarrow \mathbb{R}$ (Funktion).
  Wir wollen $f$ als Summe von $C^{\infty}$-Funktionen $f_{\alpha}$ schreiben, so dass gilt: Für jedes $\alpha$ gibt es eine Karte $\varphi : W \rightarrow V$ in $\mathscr{A}$ mit $Supp(f_{\alpha}) \subseteq V$.
\end{remark}

\begin{lemma}
  Definiert man $f : \mathbb{R} \rightarrow \mathbb{R}$ durch
  \begin{equation}
    f(x) := \begin{cases}
      0 & \textit{wenn $x \le 0$}\\
      \exp(-\frac{1}{x^{2}}) & \textit{wenn $x > 0$}
    \end{cases}
  \end{equation}
  so ist $f$ von der Klsse $C^{\infty}$.
\end{lemma}

\begin{proof}
  \begin{equation}
    \lim_{x \rightarrow 0} f(x) = \lim_{t \rightarrow \infty} exp(t) = 0
  \end{equation}
  deswegen ist $f$ stetig in $0$.
  Für $x > 0$ ist $f'(x) = \frac{2}{x^{3}}\exp(-\frac{1}{x^{2}})$, allgemeiner $f^{(n)}(x) = P_{n}(\frac{1}{x})\exp(-\frac{1}{x^{2}})$, wobei jedes $P_{n}$ ein Polynom ist.
  Also $\lim_{x \rightarrow 0} f^{(n)}(x) = 0$.
  Deswegen ist $f$ von der Klasse $C^{\infty}$.
\end{proof}

\begin{satz}
  Es gibt eine $C^{\infty}$-Funktion $g : \mathbb{R}^{n} \rightarrow \mathbb{R}$ mit
  \begin{itemize}
  \item $g(x) > 0$ für $x \in ]-1, 1[^{n}$
  \item $Supp(g) = [-1, 1]^{n}$
  \end{itemize}
\end{satz}

\begin{proof}
  Sei $f$ wie im Lemma.
  Definiere $f_{1} : \mathbb{R} \rightarrow \mathbb{R}$ durch $f_{1}(x) := f(1 + x)f(1 - x)$.
  Dann ist $f_{1} \ge 0$ und $Supp(f_{1}) = [-1, 1]$.
  Sei $g(x_{1}, \dots, x_{n}) = f_{1}(x_{1}) \cdot \dots \cdot f_{n}(x_{n})$.
\end{proof}

\begin{satz}
  Sei $U$ offen in $\mathbb{R}^{n}$ und $X$ eine kompakte Teilmenge von $\mathbb{R}^{n}$.
  Seien $A_{1}, \dots, A_{m}$ offene Teilmengen von $U$ mit $X \subseteq A_{1} \cup \dots \cup A_{m}$.
  Dann gibt es $C^{\infty}$-Funktionen $g_{1}, \dots, g_{m} : U \rightarrow \mathbb{R}$ mit
  \begin{itemize}
  \item $g_{j}(x) \ge 0\ \forall x \in U, \forall j$
  \item $\sum_{j = 1}^{m} g_{j}(x) \le 1\ \forall x \in U$
  \item $\forall j$ ist $Supp(g_{j})$ kompakt und enthalten in $A_{j}$
  \item $\forall x \in X$ ist $\sum_{j = 1}^{m} g_{j}(x) = 1$
  \end{itemize}
  Man nennt die Menge $g_{1}, \dots, g_{m}$ eine der Überdeckung $\{ a_{1}, \dots, A_{m} \}$ \underline{untergeordnete Zerlegung der Eins} auf $X$.
\end{satz}

\begin{proof}
  Nach Satz 1 gibt es eine $C^{\infty}$-Funktion $g : \mathbb{R}^{n} \rightarrow \mathbb{R}$ mit $g(x) > 0\ \forall x \in ]-1, 1[^{n}$ und $Supp(g) = [-1, 1]^{n}$.
  $\forall x \in X$ wähle ein $k_{x} \in \{ 1, \dots, m \}$ mit $x \in A_{k_{x}}$, ein $r_{x} > 0$ mit $B_{r_{x}}(x) \subseteq A_{k_{x}}$ und einen Diffeomorphismus $\varphi_{x}$ von $B_{r_{x}}(x)$ auf $\mathbb{R}^{n}$ mit $\varphi_{x}(x) = 0$.
  Definiere $f_{x} : U \rightarrow \mathbb{R}$
  \begin{equation}
    f_{x}(y) = \begin{cases}
      g(\varphi_{x}(y)) & \textit{wenn $y \in B_{r_{x}}(x)$}\\
      0 & \textit{sonst}
    \end{cases}
  \end{equation}
  Dann ist $f_{x} \ge 0$, glatt, $C^{\infty}$ und $f_{x}(x) > 0$.
  $Supp(f_{x})$ ist kompakt und $\subseteq A_{k_{x}}$.
  Sei $C_{x} := \varphi_{x}^{-1}(]-1, 1[^{n})$.
  Dann ist $C_{x}$ offen mit $x \in C_{x}$.
  Also bilden die $C_{x}$ mit $x \in X$ eine offene Überdeckung von $X$.
  Weil $X$ kompakt ist, gibt es eine endliche Teilüberdeckung $(x_{1}, \dots, x_{N}\textit{ mit }X \subseteq C_{x_{1}} \cup \dots C_{x_{N}})$.
  Für $j = 1, \dots, m$ sei $h_{j} : U \rightarrow \mathbb{R}$ definiert durch
  \begin{equation}
    h_{j} = \sum_{i \in [1, \dots, N]\\k_{x_{i}} = j} f_{x_{i}}
  \end{equation}
  Dann ist $h_{j}$ glatt; es ist
  \begin{itemize}
  \item $h_{j} \ge 0$ auf U
  \item $Supp(h_{j})$ ist kompakt und $\subseteq A_{j}$
  \item Für $x \in X$ ist $\sum_{j = 1}^{m} h_{j}(x) > 0$
  \end{itemize}
  Sei $h := \sum_{j = 1}^{m} h_{j} : U \rightarrow \mathbb{R}$.
  Dann ist $h$ aus $C^{\infty}$.
  Sei $K := Supp(h)$.
  Dann ist $K$ kompakt und $X \subseteq K$.
  $\partial K$ ist eine kompakte Teilmenge von $U \setminus X$.
  Wende das bisher Bewiesene an auf
  \begin{itemize}
  \item $U \setminus X$ statt $U$
  \item $\partial K$ statt $X$
  \end{itemize}
  Man erhält (statt $h$) eine glatte Funktion $\tilde{h} : U \setminus X \rightarrow \mathbb{R}$ mit
  \begin{itemize}
  \item $\tilde{h} \ge 0$ auf $U \setminus X$
  \item $\tilde{K} := Supp(\tilde{h})$ ist kompakt und enthalten in $U \setminus X$
  \item Für $x \in \partial K$ ist $\tilde{h}(x) > 0$
  \end{itemize}
  Wir können $\tilde{h}$ auf ganz $U$ fortsetzen zu einer glatten Funktion durch $\tilde{h}(x) := 0\ \forall x \in X$.
  Definiere $g_{j} : U \rightarrow \mathbb{R}$ durch
  \begin{equation}
    g_{j}(x) := \begin{cases}
      \frac{h_{j}(x)}{h(x) + \tilde{h}(x)} & \textit{falls $h(x) + \tilde{h}(x) \ne 0$}\\
      0 & \textit{sonst}
    \end{cases}
  \end{equation}
  Dann ist $g_{j}$ glatt, $g_{j} \ge 0$, $\sum_{j = 1}^{m} g_{j}(x) = \frac{h(x)}{h(x) + \tilde{h}(x)} \le 1$, falls $h(x) + \tilde{h}(x) \ne 0$.
  $Supp(g_{j}) = Supp(h_{j})$ ist kompakt und $\subseteq A_{j}$.
  Für $x \in X$ ist $h(x) > 0$ und $\tilde{h}(x) = 0$, also $\sum_{j = 1}^{m} g_{j}(x) = \frac{h(x)}{h(x)} = 1$.
\end{proof}

\section{Alternierende Multilinearformen}



\end{document}