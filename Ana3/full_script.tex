\documentclass[10pt,a4paper]{report}
\usepackage[utf8]{inputenc}
\usepackage[german]{babel}
\usepackage{mathrsfs}
\usepackage{amsmath}
\usepackage{amsfonts}
\usepackage{amssymb}
\usepackage{amsthm}
\usepackage[left=2cm,right=2cm,top=2cm,bottom=2cm]{geometry}
\title{Analysis III, Wintersemester 2013-2014, Prof. Dr. Singhof}
\author{Marten Lienen}

\newtheorem{defi*}{Definition}
\newtheorem{satz}{Satz}
\newtheorem{example*}{Beispiel}

\begin{document}

\maketitle

\chapter{Maß- und Integrationstheorie}

\section{Quader und Figuren}

\section{$\sigma$-Algebren und Maße}

\section{Das Lebesgue-Maß}

\section{Messbare Funktionen}

\section{Integrationstheorie}

\section{Vertauschbarkeit des Integrals mit Grenzprozessen}

\section{Der Satz von Fubini}

\section{Die Transformationsformel}

\section{Die Räume $L^{p}$}

\chapter{Vektoranalysis}

\section{Untermannigfaltigkeiten des $\mathbb{R}^{n}$}

\section{Zusammenhängende metrische Räume}

\section{Kompakte metrische Räume}

\section{Tangentialräume und Orientierungen}

\begin{defi*}
  Sei $M$ eine $n$-dimensionale Untermannigfaltigkeit von $\mathbb{R}^{n}$ und $a \in M$.
  Ein Element $v \in \mathbb{R}^{n}$ heißt \underline{Tangentialvektor} an $M$ im Punkt $a$, wenn es ein offenes Intervall $I$ in $\mathbb{R}$ mit $0 \in I$ und eine $C^{1}$-Abbildung $\psi : I \rightarrow \mathbb{R}^{n}$ gibt mit:
  \begin{enumerate}
    \item $\psi(I) \in M$
    \item $\psi(0) = a$
    \item $\psi'(0) = 0$
  \end{enumerate}
  Mit $T_{a}(M)$ bezeichnet man die Menge aller Tangentialvektoren an $M$ im Punkt $a$ und nennt $T_{a}(M)$ den \underline{Tangentialraum} an $M$ in $a$.
\end{defi*}

\begin{satz}
  Sei $M$ eine $n$-dimensionaler linearer Teilraum von $\mathbb{R}^{n}$.
  \begin{enumerate}
    \item $T_{a}(M)$ ist en $n$-dimensionaler linearer Teilraum von $\mathbb{R}^{n}$
    \item Sei $\varphi : W \rightarrow V$ eine Karte von $M$ und $a \in V$. Sei $b \in W$ mit $\varphi(b) = a$. Dann ist $T_{a}(M) = Bild(D\varphi(b)) = \{ D\varphi(b) \cdot u \mid u \in \mathbb{R}^{n} \}$
    \item Sei $U$ eine offene Umgebung von $a$ in $\mathbb{R}^{n}$ und sei $g : U \rightarrow \mathbb{R}^{N - n}$ eine Submersion mit $M \cap U = \{ x \in U \mid g(x) = 0 \}$.
      Dann ist:
      \begin{equation}
        T_{a}(M) = Kern(Dg(a)) = \{ v \in \mathbb{R}^{n} \mid Dg(a) \cdot = 0 \}
      \end{equation}
  \end{enumerate}
\end{satz}

\begin{proof}
  Analysis II, §16, Satz 5
\end{proof}

\begin{example*}
  \begin{equation}
    M = S^{N - 1}
  \end{equation}
  Sei $U = \{ x \in \mathbb{R}^{N} \mid x \ne 0 \}$ und $g : U \rightarrow \mathbb{R}$ gegeben durch $g(x) = x_{1}^{2} + \dots + x_{N}^{2} - 1$.
  Dann ist $S^{N - 1} = \{ x \in U \mid g(x) = 0 \}$.
  \begin{equation}
    Dg(x) = 2x^{T}
  \end{equation}
  Nach Satz 1c) ist $T_{a}(S^{N - 1}) = \{ v \in \mathbb{R}^{n} \mid 2a^{T}v = 0 \} = a^{T}$.
\end{example*}

\end{document}