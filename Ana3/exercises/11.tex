\documentclass[10pt,a4paper]{article}
\usepackage[utf8]{inputenc}
\usepackage[german]{babel}
\usepackage{mathrsfs}
\usepackage{amsmath}
\usepackage{amsfonts}
\usepackage{amssymb}
\usepackage{amsthm}
\usepackage[left=2cm,right=2cm,top=2cm,bottom=2cm]{geometry}

\begin{document}

\section{Aufgabe 45}

\subsection{Teil a}

\subsection{Teil b}

\subsection{Teil c}

\section{Aufgabe 46}

\section{Aufgabe 47}

\section{Aufgabe 48}

\begin{proof}
  Sei $f_{n} \in (f_{n})$.
  Dann ist $||f_{n}|| = 1$, weil $f_{n}(\frac{\pi}{2^{n + 1}}) = 1$, $0 < \frac{\pi}{2^{n + 1}} < 2\pi$ und der Wertebereich des Sinus $[-1, 1]$ ist.

  Angenommen es gäbe eine Cauchy-Teilfolge $(f_{i_{k}})$ für $k \in \mathbb{N}$.
  Dann gäbe es ein $N \in \mathbb{N}$, sodass $||f_{i_{n}} - f_{i_{m}}|| < 1$ für alle $m, n > N$.
  O.b.d.A. sei $n > m$.
  Da jedoch $f_{i_{m}}(\frac{\pi}{2^{i_{m} + 1}}) = 1$ und $f_{i_{n}}(\frac{\pi}{2^{i_{m} + 1}}) = \sin(2^{i_{n} - i_{m} - 1}\pi) = 0$, weil $i_{n} - i_{m} - 1 \in \mathbb{N}_{0}$, ist $||f_{i_{m}} - f_{i_{n}}|| \ge 1$.
  Somit gibt es keine Cauchy-Teilfolge, also auch keine konvergente Teilfolge.

  Wegen $||f_{n}|| = 1$, ist $(f_{n})$ eine Folge in $B$, die keinen Häufungspunkte besitzt.
  Nach Satz 5 ist $B$ somit nicht kompakt.
\end{proof}

\section{Aufgabe 49}

\begin{proof}
  Sei $Q = \{ A_{i_{k}} \mid k \in \{ 1, 2, \dots, n \}, i_{k} \in \Lambda \}$ eine endliche Teilüberdeckung von $X$.
  Diese existiert, weil $X$ kompakt ist.
  Alle Elemente von $Q$ sind offene Teilmengen von $X$.
\end{proof}

\end{document}