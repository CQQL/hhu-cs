\documentclass[10pt,a4paper]{article}
\usepackage[utf8]{inputenc}
\usepackage[german]{babel}
\usepackage{mathrsfs}
\usepackage{amsmath}
\usepackage{amsfonts}
\usepackage{amssymb}
\usepackage{amsthm}
\usepackage[left=2cm,right=2cm,top=2cm,bottom=2cm]{geometry}

\begin{document}

\section{Aufgabe 45}

\subsection{Teil a}

\begin{proof}
  Sei $a \in Z$.
  Dann ist $Z$ eine offene Umgebung von $a$.
  Definiere $g : Z \rightarrow \mathbb{R}$ durch
  \begin{equation}
    g(x, y, z) = x^{2} + y^{2} - 1
  \end{equation}
  $g$ ist eine Submersion:
  \begin{equation}
    Rang Dg(x, y, z) = Rang \begin{pmatrix}
      2x & 2y & 0
    \end{pmatrix} = 1 = 3 - 2
  \end{equation}
  Dann ist
  \begin{equation}
    Z \cap Z = \{ x \in Z \mid x_{1}^{2} + x_{2}^{2} = 1 \} = \{ x \in Z \mid g(x_{1}, x_{2}, x_{3}) = 0 \}
  \end{equation}
  Und somit ist $Z$ eine 2-dimensionale Untermannigfaltigkeit des $\mathbb{R}^{3}$.
\end{proof}

\subsection{Teil b}

Sei $a \in Z$ und $g$ aus Teil a.
\begin{align*}
  T_{a}(Z) & = \{ v \in \mathbb{R}^{3} \mid Dg(a) \cdot v = 0 \}\\
  & = \{ v \in \mathbb{R}^{3} \mid 2a_{1} \cdot v_{1} + 2a_{2} \cdot v_{2} = 0 \}\\
  & = \{ v \in \mathbb{R}^{3} \mid a_{1} \cdot v_{1} + a_{2} \cdot v_{2} = 0 \}\\
  & = \begin{pmatrix}
    a_{1}\\a_{2}\\0
  \end{pmatrix}^{T}
\end{align*}

\subsection{Teil c}

Definiere $\nu : Z \rightarrow \mathbb{R}^{3}$ durch
\begin{equation}
  \nu(x) = \begin{pmatrix}
    x_{1}\\x_{2}\\0
  \end{pmatrix}
\end{equation}
Sei $a \in Z$.
\begin{equation}
  ||\nu(a)||_{2} = \sqrt{a_{1}^{2} + a_{2}^{2} + 0^{2}} = \sqrt{a_{1}^{2} + a_{2}^{2}} = 1
\end{equation}
Sei $b \in T_{a}(Z)$.
\begin{equation}
  \langle \nu(a) | b \rangle = a_{1}b_{1} + a_{2}b_{2} = 0
\end{equation}
Also ist $\nu$ ein Einheitsnormalenfeld.

\section{Aufgabe 46}

\section{Aufgabe 47}

\begin{proof}
  Sei $Q = \{ A_{i} \mid i \in \Lambda \}$ eine offene Überdeckung von $A \cup B$.
  Dann ist $Q$ ebenfalls eine offene Überdeckung von $A$ und von $B$.
  Da $A$ und $B$ kompakt sind, gibt es eine endliche Teilüberdeckung $Q_{A}$ von $A$ und eine weitere endliche Teilüberdeckung $Q_{B}$ von $B$.
  Die Vereinigung $Q_{A} \cup Q_{B}$ ist immernoch endlich und überdeckt $A \cup B$.
  Somit ist $A \cup B$ kompakt.
\end{proof}

\section{Aufgabe 48}

\begin{proof}
  Sei $f_{n} \in (f_{n})$.
  Dann ist $||f_{n}|| = 1$, weil $f_{n}(\frac{\pi}{2^{n + 1}}) = 1$, $0 < \frac{\pi}{2^{n + 1}} < 2\pi$ und der Wertebereich des Sinus $[-1, 1]$ ist.

  Angenommen es gäbe eine Cauchy-Teilfolge $(f_{i_{k}})$ für $k \in \mathbb{N}$.
  Dann gäbe es ein $N \in \mathbb{N}$, sodass $||f_{i_{n}} - f_{i_{m}}|| < 1$ für alle $m, n > N$.
  O.b.d.A. sei $n > m$.
  Da jedoch $f_{i_{m}}(\frac{\pi}{2^{i_{m} + 1}}) = 1$ und $f_{i_{n}}(\frac{\pi}{2^{i_{m} + 1}}) = \sin(2^{i_{n} - i_{m} - 1}\pi) = 0$, weil $i_{n} - i_{m} - 1 \in \mathbb{N}_{0}$, ist $||f_{i_{m}} - f_{i_{n}}|| \ge 1$.
  Somit gibt es keine Cauchy-Teilfolge, also auch keine konvergente Teilfolge.

  Wegen $||f_{n}|| = 1$, ist $(f_{n})$ eine Folge in $B$, die keinen Häufungspunkte besitzt.
  Nach Satz 5 ist $B$ somit nicht kompakt.
\end{proof}

\section{Aufgabe 49}

\begin{proof}
  Angenommen es gäbe für jedes $\varepsilon > 0$ mindestens ein $B_{\varepsilon} \subset X$, sodass $B_{\varepsilon} \notin A_{i}\ \forall i \in \Lambda$.
  Definiere eine Folge $(x_{n})$ durch: $x_{k} \in X$ mit $B_{\frac{1}{k}}(x_{k}) \not\subset A_{i}\ \forall i \in \Lambda$.
  Weil $X$ kompakt ist, gibt es eine konvergente Teilfolge $(x_{i_{n}})$, die gegen den Häufungspunkt $x$ konvergiert.

  Sei $\{ A_{g_{n}} \}$ eine endliche Teilüberdeckung von $X$.
  Dann ist gibt es ein $A \in \{ A_{g_{n}} \}$, das $x$ enthält, weil $\{ A_{g_{n}} \}$ eine Überdeckung ist.
  Weil $A$ offen ist, gibt es ein $\delta$ für das $B_{\delta}(x) \subset A$ ist.
  Da $x$ der Grenzwert der Folge $(x_{i_{n}})$ ist, gibt es ein $N \in \mathbb{N}$, sodass $||x_{i_{n}} - x|| < \frac{\delta}{2}$ gilt für alle $n \ge N$.
  Außerdem gibt es $m \ge N$, sodass $\frac{1}{m} < \frac{\delta}{4}$.
  Sei $b \in B_{\frac{1}{m}}(x_{m})$.
  \begin{equation}
    ||b - x|| \le ||b - x_{m}|| + ||x_{m} - x|| < \frac{\delta}{4} + \frac{\delta}{2} = \frac{3}{4}\delta
  \end{equation}
  Also ist $B_{\frac{1}{m}}(x_{m}) \subset B_{\delta}(x) \subset A$.
  Dies ist jedoch ein Widerspruch dazu, dass es kein $A_{i}, i \in \Lambda$ gibt mit $B_{\frac{1}{m}}(x_{m}) \subset A_{i}$.

  Also gibt es für mindestens ein $\varepsilon > 0$ kein $B_{\varepsilon} \subset X$, sodass $B_{\varepsilon} \notin A_{i}\ \forall i \in \Lambda$.
\end{proof}

\end{document}