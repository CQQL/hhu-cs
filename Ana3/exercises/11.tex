\documentclass[10pt,a4paper]{article}
\usepackage[utf8]{inputenc}
\usepackage[german]{babel}
\usepackage{mathrsfs}
\usepackage{amsmath}
\usepackage{amsfonts}
\usepackage{amssymb}
\usepackage{amsthm}
\usepackage[left=2cm,right=2cm,top=2cm,bottom=2cm]{geometry}

\begin{document}

\section{Aufgabe 45}

\subsection{Teil a}

\subsection{Teil b}

\subsection{Teil c}

\section{Aufgabe 46}

\section{Aufgabe 47}

\section{Aufgabe 48}

\section{Aufgabe 49}

\begin{proof}
  Sei $Q = \{ A_{i_{k}} \mid k \in \{ 1, 2, \dots, n \}, i_{k} \in \Lambda \}$ eine endliche Teilüberdeckung von $X$.
  Diese existiert, weil $X$ kompakt ist.
  Alle Elemente von $Q$ sind offene Teilmengen von $X$.
\end{proof}

\end{document}