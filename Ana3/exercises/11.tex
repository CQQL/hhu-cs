\documentclass[10pt,a4paper]{article}
\usepackage[utf8]{inputenc}
\usepackage[german]{babel}
\usepackage{mathrsfs}
\usepackage{amsmath}
\usepackage{amsfonts}
\usepackage{amssymb}
\usepackage{amsthm}
\usepackage[left=2cm,right=2cm,top=2cm,bottom=2cm]{geometry}

\begin{document}

\section{Aufgabe 45}

\subsection{Teil a}

\begin{proof}
  Sei $a = (a_{1}, a_{2}, a_{3}) \in Z$.
  Sei $V_{i} = \{ x \in Z \mid x_{1} \ne i \}$ für $i \in \{ 1, -1 \}$.
  Sei $W = ]0, 2\pi[ \times \mathbb{R}$.
  Dann ist $V_{i}$ offen in $M$ und $W$ offen in $\mathbb{R}^{2}$.
  Definiere $\varphi_{i} : W \rightarrow \mathbb{R}^{3}$ durch
  \begin{equation}
    \varphi_{i}(x, y) = \begin{pmatrix}
      i \cdot \cos(x)\\
      \sin(x)\\
      y
    \end{pmatrix}
  \end{equation}
  \begin{equation}
    D\varphi_{i}(x, y) = \begin{pmatrix}
      -i \cdot \sin(x) & 0\\
      \cos(x) & 0\\
      0 & 1\\
    \end{pmatrix}
  \end{equation}
  Dann ist der Rang von $D\varphi = 2$ und $\varphi$ stetig.

\end{proof}

\subsection{Teil b}

\subsection{Teil c}

\section{Aufgabe 46}

\section{Aufgabe 47}

\begin{proof}
  Sei $Q = \{ A_{i} \mid i \in \Lambda \}$ eine offene Überdeckung von $A \cup B$.
  Dann ist $Q$ ebenfalls eine offene Überdeckung von $A$ und von $B$.
  Da $A$ und $B$ kompakt sind, gibt es eine endliche Teilüberdeckung $Q_{A}$ von $A$ und eine weitere endliche Teilüberdeckung $Q_{B}$ von $B$.
  Die Vereinigung $Q_{A} \cup Q_{B}$ ist immernoch endlich und überdeckt $A \cup B$.
  Somit ist $A \cup B$ kompakt.
\end{proof}

\section{Aufgabe 48}

\section{Aufgabe 49}

\begin{proof}
  Sei $Q = \{ A_{i_{k}} \mid k \in \{ 1, 2, \dots, n \}, i_{k} \in \Lambda \}$ eine endliche Teilüberdeckung von $X$.
  Diese existiert, weil $X$ kompakt ist.
  Alle Elemente von $Q$ sind offene Teilmengen von $X$.

  Angenommen es gäbe für jedes $\varepsilon > 0$ eine Kugel $Q$ vom Radius $\varepsilon$ in $X$, die nicht in irgendeinem
\end{proof}

\end{document}