\documentclass[10pt,a4paper]{article}
\usepackage[utf8]{inputenc}
\usepackage[german]{babel}
\usepackage{mathrsfs}
\usepackage{amsmath}
\usepackage{amsfonts}
\usepackage{amssymb}
\usepackage{amsthm}
\usepackage[left=2cm,right=2cm,top=2cm,bottom=2cm]{geometry}

\begin{document}

\section{Aufgabe 5}

\subsection{Teil a}
\begin{equation}
  \{ \mathscr{P}(Y) \mid Y \in \mathscr{P}(X) \}
\end{equation}
\begin{equation}
  \{ \emptyset \}
\end{equation}
\begin{equation}
  \{ \emptyset, 1 \}
\end{equation}
\begin{equation}
  \{ \emptyset, 2 \}
\end{equation}
\begin{equation}
  \{ \emptyset, 3 \}
\end{equation}
\begin{equation}
  \{ \emptyset, 1, 2 \}
\end{equation}
\begin{equation}
  \{ \emptyset, 1, 3 \}
\end{equation}
\begin{equation}
  \{ \emptyset, 2, 3 \}
\end{equation}
\begin{equation}
  \{ \emptyset, 1, 2, 3 \}
\end{equation}
\begin{equation}
  8
\end{equation}

\subsection{Teil b}
$X$ ist die einzige $\sigma$-Algebra in $X$, weil keine echte Teilmenge von $X$ eine $\sigma$-Algebra erzeugen kann, die $X$ enthält, was per Definition gefordert ist.

\subsection{Teil c}
\begin{equation}
  4
\end{equation}

\subsection{Teil d}
\begin{equation}
  1
\end{equation}

\subsection{Teil e}
Ich wähle die $\sigma$-Algebra $X$.
Dann definieren wir $\mu_{1, a, b}$ durch
\begin{equation}
  \mu_{1, a, b}(x) = 
  \begin{cases}
    & 1 \textit{wenn $x \ne a$ und $x \ne b$}\\
    & 0 \textit{sonst}
  \end{cases}
\end{equation}
und $\mu_{1/2, a, b}$ durch
\begin{equation}
  \mu_{1/2, a, b}(x) =
  \begin{cases}
    & \frac{1}{2} \textit{wenn $x = a$ oder $x = b$}\\
    & 0 \textit{sonst}
  \end{cases}
\end{equation}
für alle $\{a, b\} \subset \{1, 2, 3\}$.
Dies sind insgesamt $6$.

\section{Aufgabe 6}

\subsection{Teil a}
\begin{proof}
  \begin{equation}
    |\emptyset| = 0 \Rightarrow \emptyset \in \mathscr{R}
  \end{equation}
  
  Seien $A, B \in \mathscr{R}$.
  Da $A$ und $B$ endlich sind, sind $k = |A|, n = |B| \in \mathbb{N}$ und $|A \cup B| \le k + n \Rightarrow A \cup B \in \mathscr{R}$.
  Aus demselben Grund ist $|A \setminus B| \le k \Rightarrow A \setminus B \in \mathscr{R}$.
\end{proof}

\subsection{Teil b}
Für endliche Mengen $X$.
\begin{proof}
  Wenn $X$ nicht endlich wäre, so wäre $X \not\subset \mathscr{R}$, sodass $\mathscr{R}$ keine $\sigma$-Algebra sein kann.
  
  Wenn $X$ jedoch endlich ist, ist $\mathscr{R} = \mathscr{P}(X)$, was eine $\sigma$-Algebra ist.
\end{proof}

\subsection{Teil c}
\begin{proof}
  \begin{equation}
    |\emptyset| = 0 \Rightarrow \mu(\emptyset) = 0
  \end{equation}
  Sei $A \in \mathscr{R}$.
  Dann ist $A$ endlich und $|A| \in \mathbb{N}_{0}$, also $\mu(A) \ge 0$.
  
  Seien $A_{1}, A_{2}, \dots \in \mathscr{R}$ paarweise disjunkt.
  \begin{align*}
    \mu(\bigcup_{n = 1}^{\infty} A_{n}) & = |\bigcup_{n = 1}^{\infty} A_{n}|\\
    & = \sum_{n = 1}^{\infty} |A_{n}|\\
    & = \sum_{n = 1}^{\infty} \mu(A_{n})
  \end{align*}
\end{proof}

\section{Aufgabe 7}
\begin{proof}
  \begin{equation}
    x \notin \emptyset \forall x \in X \Rightarrow \epsilon_{x}(\emptyset) = 0
  \end{equation}
  \begin{equation}
    \epsilon_{x}(A) \in \{ 0, 1 \} \forall A \in \mathscr{R} \Rightarrow \epsilon_{x}(A) \ge 0 \forall A \in \mathscr{R}
  \end{equation}
  
  Seien $A_{1}, A_{2}, \dots \in \mathscr{R}$ paarweise disjunkt.
  Dann ist entweder $x$ in genau einer der Teilmengen von $X$ oder in keiner, also $\epsilon_{x}(\cup_{n = 1}^{\infty} A_{n}) = 1$ oder $\epsilon_{x}(\cup_{n = 1}^{\infty} A_{n}) = 0$.
  In beiden Fällen gilt
  \begin{equation}
    \epsilon_{x}(\bigcup_{n = 1}^{\infty} A_{n}) = \sum_{n = 1}^{\infty} \epsilon_{x}(A_{n})
  \end{equation}
  weil die einzelnen Terme der Summe alle 0 oder alle 0 bis auf genau einer sind.
  Dieser ist dann 1.
\end{proof}

\section{Aufgabe 8}
\begin{proof}
  Angenommen $\mu$ sei ein solches Maß.
  Da $\mu(X) = 1$ und $\mu$ nur die Werte 0 und 1 annimmt, kann man $X$ in 2 gleichgroße offene Mengen aufteilen, wobei man verbleibende einelementige Teilmengen ignorieren kann, weil diese sowieso ein Volumen von 0 haben.
  Wegen der $\sigma$-Additivität muss $\mu$ für genau eine der beiden Teilmengen den Wert $1$ und für die andere den Wert $0$ annehmen.
  Diese beiden Teilmengen kann man dann mit derselben Begründung wiederum aufteilen und das immer so weiter.
  Sei $A_{1}$ diejenige Hälfte von $X$, für die $\mu$ den Wert 1 annimmt, und $A_{i}$ die Hälfte von $A_{i - 1}$, für die $\mu$ den Wert 1 annimmt.
  Da wir uns auf einem Ring befinden, können wir den Schnitt $A$ aller dieser Mengen bilden.
  \begin{equation}
    A = \bigcap_{n = 1}^{\infty} A_{n}
  \end{equation}
  Dann ist $\mu(A) = 1$, weil $\mu(\cap_{k = 1}^{n} A_{k}) = \mu(A_{k}) = 1$ und somit $\lim_{n \rightarrow \infty} \mu(\cap_{k = 1}^{n} A_{k}) = \lim_{n \rightarrow \infty} 1 = 1$.

  $A$ kann nicht leer sein, weil $A_{i} \in A_{i - 1}$ und $A_{i} \neq \emptyset$.
  
  Es gibt also mindestens ein Element $a \in A$, als ist $a \in A_{k}\ \forall k$.

  Angenommen es gäbe ein zweites Element $b \neq a \in A$.
  Dann müsste $b$ ebenfalls in jedem $A_{k}$ enthalten sein.
  Aber jedes $A_{j} = ]a_{j}, b_{j}[$.
  Und da wir immer halbieren ist $|a_{j} - b_{j}| = \frac{1}{2^{n}}$.
  Dies konvergiert jedoch gegen $0$, sodass die $A_{k}$ beliebig klein werden und es ein $N$ gibt, sodass $|a_{k} - b_{k}| < |a - b|$ für alle $k \ge N$.
  Da $a \in A_{N}$, ist $b \notin A_{N} \Rightarrow b \notin A$.

  Also ist $A = \{ a \}$ und $\mu(A) = 0$, Widerspruch.
\end{proof}

\end{document}