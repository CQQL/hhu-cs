\documentclass[10pt,a4paper]{article}
\usepackage[utf8]{inputenc}
\usepackage[german]{babel}
\usepackage{mathrsfs}
\usepackage{amsmath}
\usepackage{amsfonts}
\usepackage{amssymb}
\usepackage{amsthm}
\usepackage[left=2cm,right=2cm,top=2cm,bottom=2cm]{geometry}

\begin{document}

\section{Aufgabe 5}

\subsection{Teil a}

\subsection{Teil b}

\subsection{Teil c}

\subsection{Teil d}

\subsection{Teil e}

\section{Aufgabe 6}

\subsection{Teil a}
\begin{proof}
  \begin{equation}
    |\emptyset| = 0 \Rightarrow \emptyset \in \mathscr{R}
  \end{equation}
  
  Seien $A, B \in \mathscr{R}$.
  Da $A$ und $B$ endlich sind, sind $k = |A|, n = |B| \in \mathbb{N}$ und $|A \cup B| \le k + n \Rightarrow A \cup B \in \mathscr{R}$.
  Aus demselben Grund ist $|A \setminus B| \le k \Rightarrow A \setminus B \in \mathscr{R}$.
\end{proof}

\subsection{Teil b}
Borrellmengen??

\subsection{Teil c}
\begin{proof}
  \begin{equation}
    |\emptyset| = 0 \Rightarrow \mu(\emptyset) = 0
  \end{equation}
  Sei $A \in \mathscr{R}$.
  Dann ist $A$ endlich und $|A| \in \mathbb{N}_{0}$, also $\mu(A) \ge 0$.
  
  Seien $A_{1}, A_{2}, \dots \in \mathscr{R}$ disjunkt.
  \begin{align*}
    \mu(\bigcup_{n = 1}^{\infty} A_{n}) & = |\bigcup_{n = 1}^{\infty} A_{n}|\\
    & = \sum_{n = 1}^{\infty} |A_{n}|\\
    & = \sum_{n = 1}^{\infty} \mu(A_{n})
  \end{align*}
\end{proof}

\section{Aufgabe 7}

\section{Aufgabe 8}

\end{document}