\documentclass[10pt,a4paper]{article}
\usepackage[utf8]{inputenc}
\usepackage[german]{babel}
\usepackage{mathrsfs}
\usepackage{amsmath}
\usepackage{amsfonts}
\usepackage{amssymb}
\usepackage{amsthm}
\usepackage[left=2cm,right=2cm,top=2cm,bottom=2cm]{geometry}

\begin{document}

\section{Aufgabe 50}

\subsection{Teil a}

\begin{proof}
  Sei $a \in M$ und $U = \mathbb{R}^{3}$.
  Dann ist $U$ eine offene Umgebung von $a$.
  Definiere $g : U \rightarrow \mathbb{R}^{1}$ durch
  \begin{equation}
    g(x, y, z) = y^{2}(y - z)^{2} - x = y^{4} - 2zy^{3} + z^{2}y^{2} - x
  \end{equation}
  Dann ist
  \begin{equation}
    Dg(x, y, z) = \begin{pmatrix}
      -1 & 4y^{3} - 6zy^{2} + 2z^{2}y & -2y^{3} + 2y^{2}z
    \end{pmatrix}
  \end{equation}
  und der Rang von $Dg$ ist $1$ für alle $(x, y, z) \in U$ und $g$ ist eine Submersion.
  \begin{equation}
    M \cap U = M = \{ (x, y, z) \in \mathbb{R}^{3} \mid y^{2}(y - z)^{2} = x \} = \{ (x, y, z) \in \mathbb{R}^{3} \mid g(x, y, z) = 0 \}
  \end{equation}
\end{proof}

\subsection{Teil b}

\begin{equation}
  T_{x, y, z}(M) = Kern(Dg(x, y, z)) = \langle \begin{pmatrix}
    4y^{3} - 6zy^{2} + 2z^{2}y\\1\\0
  \end{pmatrix},
  \begin{pmatrix}
    -2y^{3} + 2y^{2}z\\0\\1
  \end{pmatrix}\rangle
\end{equation}

\begin{equation}
  N_{x, y, z}(M) = T_{x, y, z}(M)^{\perp} = \langle
  \begin{pmatrix}
    1\\-4y^{3} + 6zy^{2} - 2z^{2}y\\2y^{3} - 2y^{2}z
  \end{pmatrix}
  \rangle
\end{equation}

\subsection{Teil c}

\subsection{Teil d}

\subsection{Teil e}

\subsection{Teil f}

\section{Aufgabe 51}

\section{Aufgabe 52}

\begin{proof}
  Weil $X$ lokalkompakt ist, existiert eine kompakte Umgebung $B$ von $a$ in $X$.
  Dann gibt es eine Kugel $B_{\varepsilon}(x) \subset U$, weil $U$ eine Umgebung von $x$ ist, und eine Kugel $B_{\delta}(x) \subset B$, weil $B$ auch eine Umgebung von $x$ ist
  Sei $\varphi = \min \{ \varepsilon, \delta \}$.
  Dann ist die abgeschlossene Kugel $\bar{B}_{\frac{\varphi}{2}}(x)$ sowohl in $B$ als auch in $U$ enthalten.
  Satz 2 besagt dann, dass $\bar{B}_{\frac{\varphi}{2}}(x)$ kompakt ist.
  Offensichtlich ist $\bar{B}_{\frac{\varphi}{2}}(x)$ auch eine Umgebung von $x$.
\end{proof}

\section{Aufgabe 53}

\subsection{Teil a}

\begin{proof}
  Sei $x \in p(A)$ und $y \in A$ mit $p(y) = x$.
  Dann gibt es eine Kugel $B_{\varepsilon}(y) \subset A$.
  \begin{equation}
    p(B_{\varepsilon}(y)) = ]x - \varepsilon, x + \varepsilon[
  \end{equation}
  Also ist $]x - \varepsilon, x + \varepsilon[$ eine offene Umgebung von $x$ in $p(A)$ und $p(A)$ ist offen.
\end{proof}

\subsection{Teil b}

\begin{proof}
  Sei $x \in p(A)$ und $y \in A$ mit $p(y) = x$.
  Weil $A$ lokalkompakt ist, gibt es eine kompakte Umgebung $B$ von $y$ in $A$.
  Weil $p$ stetig ist, ist $p(B)$ kompakt nach Satz 1.
  Weil $y \in B$, ist $x \in p(B)$ und $p(B)$ ist eine kompakte Umgebung von $x$.
  Deshalb ist $p(A)$ lokalkompakt.
\end{proof}

\section{Aufgabe 54}

\begin{equation}
  \mathbb{R}_{< 0}
\end{equation}
Das Komplement $\mathbb{R}_{\ge 0}$ ist nicht lokalkompakt, weil es keine Umgebung und somit auch keine kompakte Umgebung von $0$ enthält.

\end{document}