\documentclass[10pt,a4paper]{article}
\usepackage[utf8]{inputenc}
\usepackage[german]{babel}
\usepackage{mathrsfs}
\usepackage{amsmath}
\usepackage{amsfonts}
\usepackage{amssymb}
\usepackage{amsthm}
\usepackage[left=2cm,right=2cm,top=2cm,bottom=2cm]{geometry}

\begin{document}

\section{Aufgabe 32}
Sei $S_{r}^{n - 1}$ die $n - 1$-dimensionale Sphäre im $\mathbb{R}^{n}$ mit Radius $r$.
Dann ist $\lambda^{n}(S_{r}^{n - 1}) = 0$.
\begin{proof}
  Sei $n - 1 = 0$.
  Dann ist $S_{r}^{n - 1} = \{ \pm r \}$ und $\lambda^{1}(S_{r}^{n - 1}) = 0$.

  Sei $n - 1 > 0$ und die Behauptung wahr für alle $m < n - 1$.
  Sei $E_{x} = \{ y \in \mathbb{R}^{n - 1} \mid (y, x) \in S_{r}^{n - 1} \}$ mit $x \in \mathbb{R}$.
  Dann gilt
  \begin{equation}
     y \in E_{x} \Leftrightarrow \sum_{i = 1}^{n - 1} y_{i}^{2} + x^{2} = r^{2} \Leftrightarrow \sum_{i = 1}^{n - 1} y_{i} = r^{2} - x^{2} \Leftrightarrow y \in S_{\sqrt{r^{2} - x^{2}}}^{n - 2}
  \end{equation}
  Also ist $E_{x} = S_{\sqrt{r^{2} - x^{2}}}^{n - 2}$.
  Das bedeutet
  \begin{equation}
    \lambda^{n}(S_{r}^{n - 1}) = \int_{\mathbb{R}} \lambda^{n - 1}(E_{x})\ d\lambda^{1}(x) = \int_{\mathbb{R}} \lambda^{n - 1}(S_{\sqrt{r^{2} - x^{2}}}^{n - 2})\ d\lambda^{1}(x) = \int_{\mathbb{R}} 0\ d\lambda^{1}(x) = 0
  \end{equation}
\end{proof}
Die ursprüngliche Behauptung ergibt sich als Spezialfall für $r = 1$.

\section{Aufgabe 33}
\begin{proof}
  $\mathscr{L}^{1}(\mathbb{R}) \not\subset \mathscr{L}^{2}(\mathbb{R})$: Definiere $f: \mathbb{R} \rightarrow \mathbb{R}$ durch
  \begin{equation}
    f(x) =
    \begin{cases}
      \frac{1}{\sqrt{x}} & \textit{wenn $0 < x \le 1$}\\
      0 & \textit{sonst}
    \end{cases}
  \end{equation}
  \begin{equation}
    ||f||_{1} = \int_{\mathbb{R}} |f|\ d\lambda = \int_{0}^{1} \frac{1}{\sqrt{x}}\ dx = 2\sqrt{x} \mid_{0}^{1} = 2
  \end{equation}
  \begin{equation}
    ||f||_{2} = \left( \int_{\mathbb{R}} |f|^{2}\ d\lambda \right)^{\frac{1}{2}} = \sqrt{\int_{0}^{1} \frac{1}{x}\ d\lambda} = \sqrt{\log(1) - \log(0)} = \infty
  \end{equation}

  $\mathscr{L}^{2}(\mathbb{R}) \not\subset \mathscr{L}^{1}(\mathbb{R})$: Definiere $f: \mathbb{R} \rightarrow \mathbb{R}$ durch
  \begin{equation}
    f(x) =
    \begin{cases}
      \frac{1}{x} & \textit{wenn $x > 1$}\\
      0 & \textit{0}
    \end{cases}
  \end{equation}
  \begin{equation}
    ||f||_{1} = \int_{\mathbb{R}} |f|\ d\lambda = \int_{1}^{\infty} \frac{1}{x}\ dx = \int_{1}^{\infty} \log(x)\ dx = \infty
  \end{equation}
  \begin{equation}
    ||f||_{2} = \left( \int_{\mathbb{R}} |f|^{2}\ d\lambda \right)^{\frac{1}{2}} = \sqrt{\int_{1}^{\infty} \frac{1}{x^{2}}\ d\lambda} = \sqrt{1} = 1
  \end{equation}
\end{proof}

\section{Aufgabe 34}

\subsection{Teil a}
\begin{proof}
  Statt $\varphi$ betrachte man die Einschränkung von $\varphi$ auf $U =\ ]0, \infty[ \times ]0, 2\pi[ \times \mathbb{R}$.
  Dann bildet $\varphi$ bijektiv auf $V = \mathbb{R}^{3} \setminus \{ x \in \mathbb{R}^{3} \mid x_{1} \ge 0\ \land\ x_{2} = 0 \}$ ab.
  Und statt $f$ betrachte man die Einschränkung von $f$ auf $V$.
  Weil man die Abbildungen nur auf einer Nullmenge abgeändert hat, ändert dies nichts an den Integralen.
  Dann sind $U, V$ offen in $\mathbb{R}^{3}$ und $\varphi$ und $f$ nach der Transformationsformel integrierbar.
  Deshalb ist dann auch $(r, t, z) \rightarrow |\det D\varphi| \cdot (f \circ \varphi) = f(\varphi(r, t, z)) \cdot r$ integrierbar.
  Weil das alles integrierbar ist, kann man auch den Satz von Fubini anwenden und es ergibt sich
  \begin{align*}
    \int_{\mathbb{R}^{3}} f(x)\ dx & = \int_{V} f(x)\ dx\\
    & = \int_{U} f(\varphi(x)) \cdot r\ dx\\
    & = \int_{]0, \infty[ \times ]0, 2\pi[ \times \mathbb{R}} f(\varphi(x)) \cdot r\ dx\\
    & = \int_{\mathbb{R}^{3}} f(\varphi(x)) \cdot r \cdot \chi_{]0, \infty[}(x_{1}) \cdot \chi_{]0, 2\pi[}(x_{2})\ dx\\
    & = \int_{\mathbb{R}} \int_{\mathbb{R}} \int_{\mathbb{R}} f(\varphi(x_{1}, x_{2}, x_{3})) \cdot r \cdot \chi_{]0, \infty[}(x_{1}) \cdot \chi_{]0, 2\pi[}(x_{2})\ dx_{3}\ dx_{2}\ dx_{1}\\
    & = \int_{\mathbb{R}} \int_{\mathbb{R}} \int_{\mathbb{R}} f(\varphi(x_{1}, x_{2}, x_{3})) \cdot r \cdot \chi_{]0, \infty[}(x_{1}) \cdot \chi_{]0, 2\pi[}(x_{2})\ dx_{1}\ dx_{2}\ dx_{3}\\
    & = \int_{-\infty}^{\infty} \int_{0}^{2\pi} \int_{0}^{\infty} f(\varphi(x_{1}, x_{2}, x_{3})) \cdot r\ dx_{1}\ dx_{2}\ dx_{3}\\
    & = \int_{-\infty}^{\infty} \int_{0}^{2\pi} \int_{0}^{\infty} f(\varphi(r, t, z)) \cdot r\ dr\ dt\ dz\\
  \end{align*}
\end{proof}

\subsection{Teil b}
\begin{proof}
  Wie oben schränkt man $\varphi$ und $f$ ein auf
  \begin{equation}
    U =\ ]0, \infty[ \times ]0, \pi[ \times ]0, 2\pi[
  \end{equation}
  und
  \begin{equation}
    V = \mathbb{R}^{3} \setminus \{ x \in \mathbb{R}^{3} \mid x_{1} = 0\ \land\ x_{2} = 0 \}
  \end{equation}
\end{proof}

\section{Aufgabe 35}
Durch Einsetzen von $t = 0$ folgt aus der linken Gleichung
\begin{equation}
  F(0)^{2} = G(0)^{2}
\end{equation}
Daraus folgt $F(0) = G(0)$, weil $F(0) = -G(0)$ bedeuten würde, dass $2F(0)G(0) \le 0 < \frac{\pi}{4}$, was der rechten Gleichung widerspräche.
Also
\begin{equation}
  2F(0)^{2} = 2G(0)^{2} = \frac{\pi}{4} \Leftrightarrow F(0) = G(0) = \sqrt{\frac{\pi}{8}} \Rightarrow \int_{0}^{\infty} \cos(x^{2})\ dx = \int_{0}^{\infty} \sin(x^{2})\ dx = \sqrt{\frac{\pi}{8}}
\end{equation}

\end{document}