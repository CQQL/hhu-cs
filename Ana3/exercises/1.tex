\documentclass[10pt,a4paper]{article}
\usepackage[utf8]{inputenc}
\usepackage[german]{babel}
\usepackage{mathrsfs}
\usepackage{amsmath}
\usepackage{amsfonts}
\usepackage{amssymb}
\usepackage{amsthm}
\usepackage[left=2cm,right=2cm,top=2cm,bottom=2cm]{geometry}

\begin{document}

\section{Aufgabe 1}
\subsection{Teil a}
\subsection{Teil b}
\subsection{Teil c}

\section{Aufgabe 2}
\begin{proof}
  Wenn $P \cap Q = \emptyset$, ist es wahr, weil $\emptyset$ ein Quader ist.

  Sei also $P \cap Q \ne \emptyset$.
  Sei $P = [a, b[$ und $Q = [c, d[$, wobei die jeder der Vektoren $a, b, c, d$ $n$ nummerierte Komponenten $a_{i}, \dots$ hat.
  Seien $e$ und $f$ Vektoren mit $n$ Komponenten, wobei $e_{i} = \max(a_{i}, c_{i})$ und $f_{i} = \min(b_{i}, d_{i})$.
  Dann ist $P \cap Q = [e, f[$ und somit ein Quader:
  
  Sei $x \in P \cap Q$.
  Dann ist $x \in P$ und $x \in Q$, also $a_{i} \le x_{i} < b_{i}$ und $c_{i} \le x_{i} < d_{i}$, also $\max(a_{i}, c_{i}) \le x_{i} < \min(b_{i}, d_{i})$, also $x \in [e, f[$.
  
  Sei $x \in [e, f[$.
  Dann ist $\max(a_{i}, c_{i}) \le x_{i} < \min(b_{i}, d_{i})$, also $a_{i} \le x_{i} < b_{i}$ und $c_{i} \le x_{i} < d_{i}$, also $x \in P$ und $x \in Q$, also $x \in P \cap Q$.
\end{proof}

\section{Aufgabe 3}
\begin{proof}
  
\end{proof}

\section{Aufgabe 4}

\subsection{Teil a}
\begin{proof}
  Sei $r \in R$ und $A = \{ x \in X \mid r(x) = 1 \}$.
  Dann ist $r = \chi_{A}$ und $A \mapsto \chi_{A}$ ist surjektiv.
  
  Seien $A, B \in \mathscr{P}(X)$, so dass $\chi_{A} = \chi_{B}$.
  Also gilt für alle $x \in X$
  \begin{equation}
    (\chi_{A}(x) = 1 \Leftrightarrow \chi_{B}(x) = 1) \Leftrightarrow (x \in A \Leftrightarrow x \in B) \Leftrightarrow A = B
  \end{equation}
  Also ist $A \mapsto \chi_{A}$ auch injektiv und somit bijektiv.
\end{proof}

\subsection{Teil b}
\begin{proof}
  Zu zeigen ist, dass $(\mathscr{P}(X), \Delta)$ eine kommutative Gruppe ist, $(\mathscr{P}(X), \cap)$ eine kommutative Halbgruppe mit neutralem Element ist und das Distributivgesetz gilt.
  
  $\emptyset$ ist das neutrale Element bezüglich $\Delta$:
  \begin{equation}
    A \Delta \emptyset = (A \setminus \emptyset) \cup (\emptyset \setminus A) = A \cup \emptyset = A
  \end{equation}
  $A^{-1} = A$ ist das inverse Element bezüglich $\Delta$:
  \begin{equation}
    A \Delta A^{-1} = (A \setminus A) \cup (A \setminus A) = \emptyset
  \end{equation}
  $\Delta$ ist assoziativ:
  \begin{align*}
    (A \Delta B) \Delta C & = (((A \setminus B) \cup (B \setminus A)) \setminus C) \cup (C \setminus ((A \setminus B) \cup (B \setminus A)))\\
    & = A\\
    & = (A \setminus ((B \setminus C) \cup (C \setminus B))) \cup (((B \setminus C) \cup (C \setminus B)) \setminus A) = A \Delta (B \Delta C)
  \end{align*}
  $\Delta$ ist kommutativ:
  \begin{equation}
    A \Delta B = (A \setminus B) \cup (B \setminus A) = (B \setminus A) \cup (A \setminus B) = B \Delta A
  \end{equation}
  
  $\cap$ ist assoziativ:
  \begin{equation}
    (A \cap B) \cap C = \{ x \in X \mid x \in A \land x \in B \land x \in C \} = A \cap (B \cap C)
  \end{equation}
  $X$ ist das neutrale Element bezüglich $\cap$:
  \begin{equation}
    A \cap X = A
  \end{equation}
  $\cap$ ist kommutativ:
  \begin{equation}
    A \cap B = \{ x \in X \mid x \in A \land x \in B \} = B \cap A
  \end{equation}
  
  \begin{align*}
    A \cap (B \Delta C) & = A \cap ((B \setminus C) \cup (C \setminus B))\\
    & = \\
    & = ((A \setminus B) \cup (B \setminus A)) \cap ((A \setminus@ C) \cup (C \setminus@ A))\\
    & = (A \Delta B) \cap (A \Delta C)
  \end{align*}
\end{proof}

\subsection{Teil c}
\begin{proof}
  $\Leftarrow$: Sei $\mathscr{R}$ ein Unterring von $(\mathscr{P}(X), \Delta, \cap)$.
  Dann enthält $\mathscr{R}$ die leere Menge als neutrales Element bezüglich $\Delta$.
  Man kann $\cup$ definieren durch:
  \begin{equation}
    A \cup B = (((A \setminus B) \cup (B \setminus A)) \setminus (A \cap B)) \cup ((A \cap B) \setminus ((A \setminus B) \cup (B \setminus A))) = (A \Delta B) \Delta (A \cap B)
  \end{equation}
  und $\setminus$ durch:
  \begin{equation}
    A \setminus B = A \Delta (A \cap B)
  \end{equation}

  $\Rightarrow$: Sei $\mathscr{R}$ ein Ring von Teilmengen von $X$ im Sinne der Definition aus der Vorlesung.
  Dann ist $\Delta$ bereits definiert und $\cap$ definiert man durch:
  \begin{equation}
    A \cap B = (A \cup B) \setminus ((A \setminus B) \cup (B \setminus A))
  \end{equation}
\end{proof}

\end{document}