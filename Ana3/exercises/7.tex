\documentclass[10pt,a4paper]{article}
\usepackage[utf8]{inputenc}
\usepackage[german]{babel}
\usepackage{mathrsfs}
\usepackage{amsmath}
\usepackage{amsfonts}
\usepackage{amssymb}
\usepackage{amsthm}
\usepackage[left=2cm,right=2cm,top=2cm,bottom=2cm]{geometry}

\begin{document}

\section{Aufgabe 26}

\section{Aufgabe 27}

\section{Aufgabe 28}
In beiden Teilen betrachte man die Funktion $\chi_{B} \cdot f$, die nur auf $B$ den Wert $f(x, y)$ annimmt, und sonst überall $0$ ist.
Diese Funktion ist auch beide male integrierbar, weil $f$ auf $B$ beschränkt ist und $\mu(B)$ endlich ist.

\subsection{Teil a}
\begin{align*}
  \int_{\mathbb{R}^{2}} \chi_{B} \cdot f\ d\lambda^{2} & = \int_{-\infty}^{\infty} \int_{-\infty}^{\infty} \chi_{B} \cdot f\ d\lambda(y)\ d\lambda(x)\\
  & = \int_{0}^{1} \int_{0}^{1 - x} f(x, y)\ dy\ dx\\
  & = \int_{0}^{1} \int_{0}^{1 - x} x^{2} + y^{2}\ dy\ dx\\
  & = \int_{0}^{1} \left[ x^{2}y + \frac{1}{3}y^{3} \right]_{0}^{1 - x}\ dx\\
  & = \int_{0}^{1} x^{2} - x^{3} + \frac{1}{3}(1 - x)^{3}\ dx\\
  & = \int_{0}^{1} x^{2} - x^{3} + \frac{1}{3}(1 - 3x + 3x^{2} - x^{3})\ dx\\
  & = \int_{0}^{1} x^{2} - x^{3} + \frac{1}{3} - x + x^{2} - \frac{x^{3}}{3}\ dx\\
  & = \int_{0}^{1} \frac{1}{3} - x + 2x^{2} - \frac{4}{3}x^{3}\ dx\\
  & = \left[ \frac{x}{3} - \frac{1}{2}x^{2} + \frac{2}{3}x^{3} - \frac{1}{3}x^{4} \right]_{0}^{1}\\
  & = \frac{1}{3} - \frac{1}{2} + \frac{2}{3} - \frac{1}{3} = \frac{1}{6}
\end{align*}

\subsection{Teil b}
$f$ nimmt auf dem Kreisring mit dem Radius $r$ nur einen Wert an:
\begin{equation}
  f(x, \sqrt{r^{2} - x^{2}}) = x^{2} + \sqrt{r^{2} - x^{2}}^{2} = x^{2} + r^{2} - x^{2} = r^{2}
\end{equation}

\begin{align*}
  \int_{\mathbb{R}^{2}} \chi_{B} \cdot f\ d\lambda^{2} & = \int_{-\infty}^{\infty} \int_{-\infty}^{\infty} \chi_{B} \cdot f\ d\lambda(y)\ d\lambda(x)\\
  & = \int_{-1}^{1} \int_{-\sqrt{1 - x^{2}}}^{\sqrt{1 - x^{2}}} f(x, y)\ dy\ dx\\
  & = \int_{-1}^{1} \int_{-\sqrt{1 - x^{2}}}^{\sqrt{1 - x^{2}}} x^{2} + y^{2}\ dy\ dx\\
  & = \int_{-1}^{1} \left[ x^{2}y + \frac{1}{3}y^{3} \right]_{-\sqrt{1 - x^{2}}}^{\sqrt{1 - x^{2}}}\ dx\\
  & = \int_{-1}^{1} x^{2} \cdot \sqrt{1 - x^{2}} + \frac{1}{3}(1 - x^{2})^{\frac{3}{2}} + x^{2}\sqrt{1 - x^{2}} + \frac{1}{3}(1 - x^{2})^{\frac{3}{2}}\ dx\\
  & = \int_{-1}^{1} 2x^{2} \cdot \sqrt{1 - x^{2}} + \frac{2}{3}(1 - x^{2})^{\frac{3}{2}}\ dx\\
  & = \int_{-1}^{1} \sqrt{1 - x^{2}} \cdot (2x^{2} + \frac{2}{3}(1 - x^{2}))\ dx\\
  & = \frac{2}{3} \int_{-1}^{1} \sqrt{1 - x^{2}} \cdot (2x^{2} + 1)\ dx\\
  & = \frac{2}{3} \int_{-1}^{1} \sqrt{1 - x^{2}} \cdot (2x^{2} + 1)\ dx\\
\end{align*}

\section{Aufgabe 29}

\section{Aufgabe 30}

\section{Aufgabe 31}

\end{document}