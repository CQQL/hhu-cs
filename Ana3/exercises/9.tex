\documentclass[10pt,a4paper]{article}
\usepackage[utf8]{inputenc}
\usepackage[german]{babel}
\usepackage{mathrsfs}
\usepackage{amsmath}
\usepackage{amsfonts}
\usepackage{amssymb}
\usepackage{amsthm}
\usepackage[left=2cm,right=2cm,top=2cm,bottom=2cm]{geometry}

\begin{document}

\section{Aufgabe 36}

\subsection{Teil a}

\subsection{Teil b}

\subsection{Teil c}

\section{Aufgabe 37}

\section{Aufgabe 38}
\begin{proof}
  Sei $f \in \mathscr{L}^{p}(X)$.
  Dann ist
  \begin{equation}
    ||f||_{p} = \left( \int_{X} |f|^{p}\ d\mu \right)^{\frac{1}{p}} < \infty
  \end{equation}
  Also ist auch
  \begin{equation}
    \int_{X} |f|^{p}\ d\mu < \infty
  \end{equation}
  Wenn $|f(x)|^{p} < 1$, ist $|f(x)|^{p} < |f(x)|^{q} < 1$, und wenn $|f(x)|^{p} \ge 1$, ist $1 \le |f(x)|^{q} < |f(x)|^{p}$.
  Also können wir $|f|^{q}$ an den Stellen, wo es kleiner als $1$ ist, durch $1$ und an den restlichen Stellen durch $|f|^{p}$ abschätzen.
  \begin{equation}
    \int_{X} |f|^{q}\ d\mu \le 1 \cdot \mu(X) + \int_{X} |f|^{p}\ d\mu < \infty \Rightarrow f \in \mathscr{L}^{q}(X)
  \end{equation}
\end{proof}

\section{Aufgabe 39}

\subsection{Teil a}
\begin{equation}
  A = \{ (\frac{1}{2^{n}}, 0, \dots, 0) \in \mathbb{R}^{N} \mid n \in \mathbb{N} \}
\end{equation}
Dann gibt es um den Punkt $(\frac{1}{2^{n}}, 0, \dots)$ eine offene Kugel mit dem Radius $\frac{1}{2^{n + 1000}}$.

\subsection{Teil b}
\begin{equation}
  A = \{ (q, 0, \dots, 0) \in \mathbb{R}^{N} \mid q \in \mathbb{Q} \}
\end{equation}

\subsection{Teil c}
\begin{proof}
  Sei $d$ die euklidische Metrik auf $\mathbb{R}^N$.
  Sei $A$ eine diskrete Teilmenge von $\mathbb{R}^{N}$.
  Sei $p \in A$.
  Dann gibt es eine offene Kugel um $p$ mit Radius $r$, in der sich kein weiterer Punkt aus $A$ befindet.
  Sei $q_{p}$ ein Punkt aus $B_{\frac{r}{2}}(p)$ mit nur rationalen Koordinaten.
  Dann ist $d(q_{p}, p) < d(q_{p}, s)$ für alle $s \in A$ und $q_{p}$ kann in keiner weiteren solchen Kugel mit halbiertem Radius eines weiteren Punktes aus $A$ sein.
  Also ist die Abbildung $p \rightarrow q_{p}$ injektiv auf $\mathbb{Q}^{N}$, sodass $A$ höchstens abzählbar sein kann.
\end{proof}

\section{Aufgabe 40}
\begin{proof}
  Sei $a$ der Grenzwert der konvergenten Teilfolge $(a_{i_{n}})_{n}$.

\end{proof}

\end{document}