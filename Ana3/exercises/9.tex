\documentclass[10pt,a4paper]{article}
\usepackage[utf8]{inputenc}
\usepackage[german]{babel}
\usepackage{mathrsfs}
\usepackage{amsmath}
\usepackage{amsfonts}
\usepackage{amssymb}
\usepackage{amsthm}
\usepackage[left=2cm,right=2cm,top=2cm,bottom=2cm]{geometry}

\begin{document}

\section{Aufgabe 36}

\subsection{Teil a}
\begin{proof}
  Da $A$ beschränkt ist, ist $|x_{j}| < m_{j}$ für ein $m_{j} \in \mathbb{R}$ und alle $x \in A$.
  \begin{align*}
    \int_{\mathbb{R}^{n}} x_{j} \chi_{A}(x)\ d\lambda = \int_{A} x_{j}\ d\mu < \int_{A} m_{j}\ d\lambda = \lambda^{n}(A) \cdot m_{j} \le m_{j} \cdot \prod_{i = 1}^{n} m_{i} < \infty
  \end{align*}
\end{proof}

\subsection{Teil b}
\begin{align*}
  S(\varphi(A)) & = \frac{1}{\lambda^{n}(\varphi(A))} \int_{\varphi(A)} x\ dx\\
  & = \frac{1}{\lambda^{n}(a + TA)} \int_{\varphi(A)} x\ dx\\
  & = \frac{1}{|\det T| \cdot \lambda^{n}(A)} \int_{\varphi(A)} x\ dx\\
  & = \frac{1}{|\det T| \cdot \lambda^{n}(A)} \int_{\varphi^{-1}(\varphi(A))} \varphi(x) \cdot |\det D\varphi(x)|\ dx\\
  & = \frac{|\det D\varphi(x)|}{|\det T| \cdot \lambda^{n}(A)} \int_{\varphi^{-1}(\varphi(A))} \varphi(x)\ dx\\
  & = \frac{|\det T|}{|\det T| \cdot \lambda^{n}(A)} \int_{\varphi^{-1}(\varphi(A))} \varphi(x)\ dx\\
  & = \frac{1}{\lambda^{n}(A)} \int_{A} \varphi(x)\ dx\\
  & = \frac{1}{\lambda^{n}(A)} \int_{A} a\ dx + \frac{1}{\lambda^{n}(A)} \int_{A} Tx\ dx\\
  & = \frac{1}{\lambda^{n}(A)} \cdot a \cdot \lambda^{n}(A) + T \cdot \frac{1}{\lambda^{n}(A)} \int_{A} x\ dx\\
  & = a + T \cdot \frac{1}{\lambda^{n}(A)} \int_{A} x\ dx\\
  & = \varphi \left(\frac{1}{\lambda^{n}(A)} \int_{A} x\ dx \right) = \varphi(S)
\end{align*}

\subsection{Teil c}
Wie in Teil b gezeigt, kann man die Körper beliebig drehen und verschieben, sodass wir eine Kugel mit Mittelpunkt $0$, eine Halbkugel, die die obere Hälfte dieser ganzen Kugel ist, und einen Volltorus mit Mittelpunkt $0$ betrachten.
\begin{equation}
  A = \textit{Kugel mit Radius $r$}
\end{equation}
\begin{align*}
  S(A) & = \frac{1}{\lambda^{3}(A)} \int_{A} x\ dx = \frac{1}{\lambda^{3}(A)} \cdot 0 = 0 = (0, 0, 0)
\end{align*}

\begin{equation}
  B = \textit{Halbkugel $(x, y, z)$ mit $z > 0$ und Grundfläche mit Radius $r$}
\end{equation}
Dann sind die $x$ und $y$-Koordinaten des Schwerpunkts $0$ aus Symmetriegründen.
Die $z$-Koordinate ist
\begin{align*}
  \frac{1}{\lambda^{3}(B)} \cdot \int_{A} x_{3}\ dx & = \frac{1}{\lambda^{3}(B)} \cdot \int_{]0, r[} x_{3} \cdot \int_{B_{\sqrt{r^{2} - x_{3}^{2}}}}\ dx\ dx_{3}\\
  & = \frac{1}{\lambda^{3}(B)} \cdot \int_{]0, r[} x_{3} \cdot (\pi \cdot (r^{2} - x_{3}^{2}))\ dx_{3}\\
  & = \frac{\pi}{\lambda^{3}(B)} \cdot \int_{]0, r[} x_{3} \cdot (r^{2} - x_{3}^{2})\ dx_{3}\\
  & = \frac{\pi}{\lambda^{3}(B)} \cdot \left( r^{2} \cdot \int_{]0, r[} x_{3}\ dx_{3} -  \int_{]0, r[} x_{3}^{3}\ dx_{3} \right)\\
  & = \frac{\pi}{\lambda^{3}(B)} \cdot \left( \frac{r^{4}}{2} - \frac{r^{4}}{4} \right)\\
  & = \frac{\pi}{\lambda^{3}(B)} \cdot \frac{r^{4}}{4}\\
  & = \frac{\pi}{\frac{1}{2} \cdot \frac{4}{3} \pi r^{3}} \cdot \frac{r^{4}}{4}\\
  & = \frac{1}{\frac{1}{2} \cdot \frac{4}{3}} \cdot \frac{r}{4} = \frac{3}{8}r\\
\end{align*}
Also ist $S(B) = (0, 0, \frac{3}{8}r)$.

Aus Symmetriegründen ist der Schwerpunkt eines Volltorus $0$.

\section{Aufgabe 37}


\section{Aufgabe 38}
\begin{proof}
  Sei $f \in \mathscr{L}^{p}(X)$.
  Dann ist
  \begin{equation}
    ||f||_{p} = \left( \int_{X} |f|^{p}\ d\mu \right)^{\frac{1}{p}} < \infty
  \end{equation}
  Also ist auch
  \begin{equation}
    \int_{X} |f|^{p}\ d\mu < \infty
  \end{equation}
  Wenn $|f(x)|^{p} < 1$, ist $|f(x)|^{p} < |f(x)|^{q} < 1$, und wenn $|f(x)|^{p} \ge 1$, ist $1 \le |f(x)|^{q} < |f(x)|^{p}$.
  Also können wir $|f|^{q}$ an den Stellen, wo es kleiner als $1$ ist, durch $1$ und an den restlichen Stellen durch $|f|^{p}$ abschätzen.
  \begin{equation}
    \int_{X} |f|^{q}\ d\mu \le 1 \cdot \mu(X) + \int_{X} |f|^{p}\ d\mu < \infty \Rightarrow f \in \mathscr{L}^{q}(X)
  \end{equation}
\end{proof}

\section{Aufgabe 39}

\subsection{Teil a}
\begin{equation}
  A = \{ (\frac{1}{2^{n}}, 0, \dots, 0) \in \mathbb{R}^{N} \mid n \in \mathbb{N} \}
\end{equation}
Dann gibt es um den Punkt $(\frac{1}{2^{n}}, 0, \dots)$ eine offene Kugel mit dem Radius $\frac{1}{2^{n + 1000}}$.

\subsection{Teil b}
\begin{equation}
  A = \{ (q, 0, \dots, 0) \in \mathbb{R}^{N} \mid q \in \mathbb{Q} \}
\end{equation}

\subsection{Teil c}
\begin{proof}
  Sei $d$ die euklidische Metrik auf $\mathbb{R}^N$.
  Sei $A$ eine diskrete Teilmenge von $\mathbb{R}^{N}$.
  Sei $p \in A$.
  Dann gibt es eine offene Kugel um $p$ mit Radius $r$, in der sich kein weiterer Punkt aus $A$ befindet.
  Sei $q_{p}$ ein Punkt aus $B_{\frac{r}{2}}(p)$ mit nur rationalen Koordinaten.
  Dann ist $d(q_{p}, p) < d(q_{p}, s)$ für alle $s \in A$ und $q_{p}$ kann in keiner weiteren solchen Kugel mit halbiertem Radius eines weiteren Punktes aus $A$ sein.
  Also ist die Abbildung $p \rightarrow q_{p}$ injektiv auf $\mathbb{Q}^{N}$, sodass $A$ höchstens abzählbar sein kann.
\end{proof}

\section{Aufgabe 40}
\begin{proof}
  Sei $a$ der Grenzwert der konvergenten Teilfolge $(a_{i_{n}})_{n}$.
  Sei $\varepsilon > 0$.
  Dann gibt es ein $N \in \mathbb{N}$, sodass $|a_{m} - a_{n}| < \frac{\varepsilon}{2}$ für alle $m, n > N$.
  Es gibt ebenfalls ein $M \in \mathbb{N}$, sodass $|a_{i_{m}} - a| < \frac{\varepsilon}{2}$ für alle $m > M$.

  Sei $Q = \max(M, N)$.
  Dann gilt für alle $m, n > Q$
  \begin{equation}
    |a_{m} - a| = |a_{m} + a_{i_{n}} - a_{i_{n}} - a| \le |a_{m} - a_{i_{n}}| + |a_{i_{n}} - a| < \frac{\varepsilon}{2} + \frac{\varepsilon}{2} = \varepsilon
  \end{equation}
  Also konvergiert $(a_n)$ gegen $a$.
\end{proof}

\end{document}