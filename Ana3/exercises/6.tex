\documentclass[10pt,a4paper]{article}
\usepackage[utf8]{inputenc}
\usepackage[german]{babel}
\usepackage{mathrsfs}
\usepackage{amsmath}
\usepackage{amsfonts}
\usepackage{amssymb}
\usepackage{amsthm}
\usepackage[left=2cm,right=2cm,top=2cm,bottom=2cm]{geometry}

\begin{document}

\section{Aufgabe 22}

\subsection{Teil a}

\subsubsection{Frage 1}
Ja, weil es eine Treppenfunktion im Riemannschen Sinn ist.

\subsubsection{Frage 2}
Ja, weil es Riemann-integrierbar ist.
\begin{equation}
  \int_{0}^{1} f(x)\ dx = 0
\end{equation}

\subsection{Teil b}

\subsubsection{Frage 1}
Nein, weil es keine Folgen von Ober- und Untertreppenfunktionen von $f$ gibt, die gegeneinander konvergieren.

\subsubsection{Frage 2}
Ja, weil $f$ nur 2 Werte annimmt.
\begin{equation}
  \int_{0}^{1} f(x)\ dx = 0
\end{equation}
weil der Wert 1 nur auf einer Nullmenge angenommen wird.

\subsection{Teil c}

\subsubsection{Frage 1}
Ja, weil das Infimum der Obertreppenfunktionen gleich der Nullfunktion ist.

\subsubsection{Frage 2}
Ja, weil es Riemann-integrierbar ist.
\begin{equation}
  \int_{0}^{1} f(x)\ dx = 0
\end{equation}

\subsection{Teil d}

\subsubsection{Frage 1}
Nein, weil es nicht beschränkt auf dem kompakten Intervall $[0, 1]$ ist.

\subsubsection{Frage 2}
Ja, weil die Nichtnullwerte nur auf einer Nullmenge angenommen werden.
\begin{equation}
  \int_{0}^{1} f(x)\ dx = 0
\end{equation}

\subsection{Teil e}

\subsubsection{Frage 1}
Nein, weil es nicht beschränkt auf dem kompakten Intervall $[0, 1]$ ist.

\subsubsection{Frage 2}
Ja, weil $\int |f| < \infty$:
\begin{equation}
  \int_{0}^{1} |f(x)|\ dx = \int_{0}^{1} f(x)\ dx = \sum_{n = 1}^{\infty} \frac{1}{2^{n}} \cdot 2^{\frac{n}{2}} = \sum_{n = 1}^{\infty} \left( \frac{1}{2} \right)^{n} = \frac{1}{1 - \frac{1}{2}} - 1 = 1
\end{equation}

\subsection{Teil f}

\subsubsection{Frage 1}
Nein, weil es nicht beschränkt auf dem kompakten Intervall $[0, 1]$ ist.

\subsubsection{Frage 2}
Nein, weil $\int_{0}^{1} |f(x)|\ dx = \infty$.
\begin{equation}
  \int_{0}^{1} |f(x)|\ dx = \sum_{n = 1}^{\infty} |- \frac{1}{2^{n}} \cdot \frac{(-2)^{n}}{n}| = \sum_{n = 1}^{\infty} \frac{1}{n} = \infty
\end{equation}
Man darf allerdings $\int_{0}^{1} f(x)\ dx = \log 2$ schreiben, weil das uneigentliche Lebesgue-Integral existiert:
\begin{equation}
  \lim_{n \rightarrow \infty} \int_{\frac{1}{2^{n}}}^{1} f(x)\ dx = \lim_{n \rightarrow \infty} \sum_{k = 1}^{n} - \frac{1}{2^{k}} \cdot \frac{(-2)^{k}}{k} = \lim_{n \rightarrow \infty} \sum_{k = 1}^{n} \frac{(-1)^{k + 1}}{k}
\end{equation}
Dies konvergiert nach dem Leibnizkriterium.
\begin{equation}
  \int_{0}^{1} f(x)\ dx = \sum_{n = 1}^{\infty} \frac{(-1)^{n + 1}}{n} = 0 + \sum_{n = 1}^{\infty} \frac{(-1)^{n + 1} \cdot (n - 1)! \cdot \frac{1}{1^{n}}}{n!} \cdot 1^{n} = \sum_{n = 0}^{\infty} \frac{\log^{(n)}(1)}{n!} \cdot (2 - 1)^{n} = \log(2)
\end{equation}
Das Ergebnis ist die Taylorentwicklung der Logarithmus an der Stelle 1 angewandt auf $2$.

\section{Aufgabe 23}
\begin{proof}
  \begin{align*}
    f_{\alpha}\textit{ ist Lebesgue-integrierbar} & \Leftrightarrow |f_{\alpha}| \textit{ ist integrierbar}\\
    & \Leftrightarrow \lim_{b \rightarrow \infty} |\int_{1}^{b} x^{\alpha}\ dx| < \infty\\
    & \Leftrightarrow \lim_{b \rightarrow \infty} |\left[ \frac{x^{\alpha + 1}}{\alpha + 1} \right]_{1}^{b}| < \infty\\
    & \Leftrightarrow \lim_{b \rightarrow \infty} |\frac{b^{\alpha + 1}}{\alpha + 1} - \frac{1}{\alpha + 1}| < \infty\\
    & \Leftrightarrow \lim_{b \rightarrow \infty} |\frac{b^{\alpha + 1}}{\alpha + 1}| < \infty\\
    & \Leftrightarrow \lim_{b \rightarrow \infty} \frac{|b^{\alpha + 1}|}{|\alpha + 1|} < \infty\\
    & \Leftrightarrow \lim_{b \rightarrow \infty} |b^{\alpha + 1}| < \infty\\
    & \Leftrightarrow \lim_{b \rightarrow \infty} b^{\alpha + 1} < \infty\\
    & \alpha + 1 < 0\\
    & \alpha < -1\\
  \end{align*}
\end{proof}

\section{Aufgabe 24}

\subsection{Teil a}
Wir wenden Satz 5 an.
Sei $p(t, x) = e^{-tx} \cdot \frac{\sin x}{x}$.
Weil $|\frac{\sin x}{x}| \le 1$ ist, ist $g(x) = e^{-tx}$ eine integrierbare Majorante von $p$.
Deshalb ist $p$ für jedes $t$ nach $t$ integrierbar nach dem Satz von der majorisierten Konvergenz.
Die zweite Bedingung ist ebenfalls erfüllt, weil $\frac{\partial p}{\partial t}(t, x) = -e^{-tx} \cdot \sin(x)$.
Die dritte Bedingung wird auch von der Majoranten $g$ erfüllt.
Also gilt
\begin{align*}
  f'(t) & = \int_{0}^{\infty} -e^{-tx} \cdot \sin(x)\ dx\\
  & = \int_{0}^{\infty} -e^{-tx} \cdot \frac{1}{2i}(e^{ix} - e^{-ix})\ dx\\
  & = -\frac{1}{2i} \int_{0}^{\infty} e^{-tx} \cdot(e^{ix} - e^{-ix})\ dx\\
  & = -\frac{1}{2i} \int_{0}^{\infty} e^{(i - t)x} - e^{(-i - t)x}\ dx\\
  & = -\frac{1}{2i} \left[ \frac{1}{i - t} e^{(i - t)x} + \frac{1}{i + t} e^{(-i - t)x} \right]_{0}^{\infty}\\
  & = \frac{1}{2i} \left( \frac{1}{i - t} + \frac{1}{i + t} \right)\\
  & = \frac{1}{2i} \frac{2i}{-1 - t^{2}} \\
  & = \frac{1}{-1 - t^{2}} \\
  & = -\frac{1}{1 + t^{2}} \\
\end{align*}
Daraus folgt
\begin{equation}
  f(t) = c - \arctan(t)
\end{equation}

\subsection{Teil b}
\begin{equation}
  \lim_{t \rightarrow \infty} f(t) = \lim_{t \rightarrow \infty} c - \arctan(t) = \lim_{t \rightarrow \infty} \int_{0}^{\infty} e^{-tx} \cdot \frac{\sin x}{x}\ dx = 0 \Rightarrow c = \lim_{t \rightarrow \infty} \arctan(t) = \frac{
\pi}{2}  
\end{equation}

\subsection{Teil c}
\begin{equation}
  f(0) = \int_{0}^{\infty} \frac{\sin x}{x}\ dx = \frac{\pi}{2} - \arctan(0) = \frac{\pi}{2}
\end{equation}

\section{Aufgabe 25}
\begin{proof}
  Sei $f : \mathbb{R} \rightarrow \mathbb{R}$ eine Lebesgue-integrierbare Funktion.
  Sei $B$ die Menge aller $x$ für die $\lim_{n \rightarrow \infty} |f(x + n)| > 0$.
  Angenommen $\mu(B) > 0$.
  Seien $D_{m} \subseteq B$ Teilmengen mit $\lim_{n \rightarrow \infty} |f(x + n)| > \frac{1}{m}$ für alle $x \in D_{m}$.
  Dann ist $\cup_{m \in \mathbb{N}} D_{m} = B$ und somit wegen der Monotonie eines Maßes $\lim_{m \rightarrow \infty} \mu(D_{m}) = \mu(B)$.
  Es gibt also ein $M$, sodass $\mu(B) - \mu(D_{m}) < \epsilon$ für alle $m \ge M$ und jedes $\epsilon > 0$.
  Sei $D_{m}$ eine solche Menge, sodass $\mu(B) - \mu(D_{m}) < \frac{\mu(B)}{2} \Leftrightarrow \mu(D_{m}) > \frac{\mu(B)}{2} > 0$.

  Wegen $\lim_{n \rightarrow \infty} |f(x + n)| > a$ für alle $x \in D_{m}$, gibt es ein $N$, sodass $|f(x + n)| > m - \epsilon$ für jedes $\epsilon > 0$ und alle $n \ge N$.
  Sei $N \in \mathbb{N}$, sodass $|f(x + n)| > \frac{m}{2}$ für alle $n \ge N$.
  
  Dann gibt es eine nicht-negative Treppenfunktion $g$, die den Wert $\frac{m}{2}$ annimmt auf einer Menge $D$ und sonst den Wert 0.
  Wenn $\mu(D_{m}) = \infty$, sei $D = D_{m}$.
  Ist $\mu(D_{m})$ jedoch endlich, kann man $D_{m}$ nach der selben Argumentation, wie auch $D_{m}$ selbst konstruiert wurde, mit einer wachsenden Kugel schneiden, bis man $\bar{D}_{m} = D_{m} \cap \bar{B}_{r}$ erhält, sodass $\mu(\bar{D}_{m}) > 0$ ist.
  Dabei ist $\bar{B}_{r}$ die 1-dimensionale Kugel mit Radius $r$.
  Dann sind die Mengen $\bar{D}_{m, p} = \{ p \cdot 2 \lceil r \rceil + N + x \mid x \in \bar{D}_{m} \}$ disjunkt für alle $p \in \mathbb{N}$.
  Wir setzen nun $D = \cup_{p \in \mathbb{N}} \bar{D}_{m, p}$.
  Da $D \subset \{ n \cdot x \mid n \ge N\ \land\ x \in D \}$, sind die Grenzwerte $\lim_{n \rightarrow \infty} |f(x + n)|$ für alle $x \in D$ ebenfalls durch $\frac{m}{2}$ nach unten beschränkt.
  Somit ergibt sich $\mu(D) \ge \sum_{p = 1}^{\infty} \frac{m}{2} \cdot \mu(\bar{D}_{m,p}) = \infty$.
  
  Dann ist $g \le |f|$.
  Daraus folgt
  \begin{equation}
    \int |f| \ge \int g = \mu(D) \cdot \frac{m}{2} = \infty \cdot \frac{m}{2} = \infty
  \end{equation}
  
  Dies widerspricht jedoch der Lebesgue-Integrierbarkeit von $f$.
  Also ist $\mu(B) = 0$.
\end{proof}

\end{document}