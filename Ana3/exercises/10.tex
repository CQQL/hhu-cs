\documentclass[10pt,a4paper]{article}
\usepackage[utf8]{inputenc}
\usepackage[german]{babel}
\usepackage{mathrsfs}
\usepackage{amsmath}
\usepackage{amsfonts}
\usepackage{amssymb}
\usepackage{amsthm}
\usepackage[left=2cm,right=2cm,top=2cm,bottom=2cm]{geometry}

\begin{document}

\section{Aufgabe 41}

\subsection{Teil a}
\begin{proof}

\end{proof}

\subsection{Teil b}
\begin{proof}

\end{proof}

\section{Aufgabe 42}

\subsection{Teil a}
\begin{proof}
  Sei $a = (a_{m}, a_{n}) \in M \times N$.
  Dann gibt es eine Umgebung $U_{m}$ von $a_{m}$ und $U_{n}$ von $a_{n}$ in $\mathbb{R}^{m}$ bzw. $\mathbb{R}^{n}$ mit jeweils einer Submersion $g_{m} : U_{m} \rightarrow \mathbb{R}^{m - k}$ und $g_{n} : U_{n} \rightarrow \mathbb{R}^{n - h}$ der Klasse $C^{\infty}$, sodass
  \begin{equation}
    M \cap U_{m} = \{ x \in U_{m} \mid g_{m}(x) = 0 \}
  \end{equation}
  bzw.
  \begin{equation}
    N \cap U_{n} = \{ x \in U_{n} \mid g_{n}(x) = 0 \}
  \end{equation}

  Dann ist $U = U_{m} \times U_{n}$ eine offene Umgebung von $a$ in $\mathbb{R}^{m + n}$.
  Sei $g : U \rightarrow \mathbb{R}^{(m + n) - (k + h)}$ definiert durch
  \begin{equation}
    g(x_{m}, x_{n}) = g_{m}(x_{m}) \times g_{n}(x_{n})
  \end{equation}
  Dann ist $g$ aus $C^{\infty}$.
  Außerdem ist
  \begin{equation}
    Dg(x_{m}, x_{n}) = \begin{pmatrix}
      Dg_{m}(x_{m}) & 0\\
      0 & Dg_{n}(x_{n})
    \end{pmatrix}
  \end{equation}
  Dann ist $Rg Dg(x_{m}, x_{n}) = Rg Dg_{m}(x_{m}) + Rg Dg_{n}(x_{n}) = (m + n) - (k + h)$.
  \begin{align*}
    (M \times N) \cap U & = (M \cap U_{m}) \times (N \cap U_{n})\\
    & = \{ x \in U_{m} \mid g_{m}(x) = 0 \} \times \{ x \in U_{n} \mid g_{n}(x) = 0 \}\\
    & = \{ x \in U \mid g_{m}(x_{m}) = 0\ \land\ g_{n}(x_{n}) = 0 \}\\
    & = \{ x \in U \mid g(x) = 0 \}
  \end{align*}
  Also ist $M \times N$ eine $k + h$-dimensionale Untermannigfaltigkeit des $\mathbb{R}^{m + n}$.
\end{proof}

\subsection{Teil b}
\begin{proof}
  Sei $a = (a_{m}, a_{n}) \in M \times N$.
  Es existiert eine offene Umgebung $V_{m}$ von $a_{m}$ in $M$, eine offene Teilmenge $W_{m}$ in $\mathbb{R}^{k}$ und eine Immersion $\varphi_{m} : W_{m} \rightarrow \mathbb{R}^{m}$ der Klasse $C^{\infty}$, die $W_{m}$ homöomorph auf $V_{m}$ abbildet.
  Es existiert eine offene Umgebung $V_{n}$ von $a_{n}$ in $N$, eine offene Teilmenge $W_{n}$ in $\mathbb{R}^{h}$ und eine Immersion $\varphi_{n} : W_{n} \rightarrow \mathbb{R}^{n}$ der Klasse $C^{\infty}$, die $W_{n}$ homöomorph auf $V_{n}$ abbildet.
  Dann ist $V = V_{m} \times V_{n}$ offene Umgebung von $a$ in $M \times N$ und $W = W_{m} \times W_{n}$ offen in $\mathbb{R}^{k + h}$.
  Sei $\varphi : W \rightarrow \mathbb{R}^{m + n}$ definiert durch
  \begin{equation}
    \varphi(x_{m}, x_{n}) = \varphi_{m}(x_{m}) \times \varphi_{n}(x_{n})
  \end{equation}
  Dann ist $\varphi$ ebenfalls von der Klasse $C^{\infty}$ und homöomorph.
  Dabei ist
  \begin{equation}
    \varphi(W) = \varphi(W_{m} \times W_{n}) = \varphi_{m}(W_{m}) \times \varphi(W_{n}) = V_{m} \times V_{n} = V
  \end{equation}
  Außerdem ist
  \begin{equation}
    D\varphi(x_{m}, x_{n}) = \begin{pmatrix}
      D\varphi_{m}(x_{m}) & 0\\
      0 & D\varphi_{n}(x_{n})
    \end{pmatrix}
  \end{equation}
  Also ist $Rg D\varphi(x_{m}, x_{n}) = Rg D\varphi_{m}(x_{m}) + Rg D\varphi_{n}(x_{n}) = k + h$ und $\varphi$ ist eine Immersion.

  Insgesamt ist $\varphi$ also eine Karte von $M \times N$.
\end{proof}

\subsection{Teil c}
\begin{proof}

\end{proof}

\section{Aufgabe 43}

\subsection{Teil a}
\begin{proof}
  $\Leftarrow$: Wegzusammenhängende Mengen sind auch zusammenhängend.

  $\Rightarrow$: Sei $M$ zusammenhängend und $x_{0} \in M$.
  Sei $A = \{ x \in M \mid \textit{Es gibt einen Weg von $x_0$ nach $x$} \}$.
  Dann ist $A$ nicht leer, weil es mindestens $x_{0}$ enthält.
  Sei $x \in A$ und $w_{x}$ der Weg von $x_{0}$ nach $x$.
  Es gibt eine Karte $\varphi : W \rightarrow \mathbb{R}^{N}$ von $M$ um $x$ und eine offene Umgebung $V$ von $x$ in $M$, wobei $W \subset \mathbb{R}^{n}$ und $\varphi(W) = V$.
  Sei $y \in V$.
  Sei $w_{y} : [0, 1] \rightarrow M$ definiert durch
  \begin{equation}
    w_{y}(t) = \varphi(\varphi^{-1}(x) + t \cdot (\varphi^{-1}(y) - \varphi^{-1}(x)))
  \end{equation}
  Dann ist $w_{y}$ stetig und $w_{y}(0) = x$ und $w_{y}(1) = y$.
  Also ist $w_{y}$ ein Weg von $x$ nach $y$.
  Definiere $w : [0, 1] \rightarrow M$ durch
  \begin{equation}
    w(t) = \begin{cases}
      w(2t) & \textit{wenn $0 \le t < \frac{1}{2}$}\\
      w(2t - 1) & \textit{wenn $\frac{1}{2} \le t \le 1$}
    \end{cases}
  \end{equation}
  Dann ist $w$ ebenfalls stetig und ein Weg von $x_{0}$ nach $y$.
  Das bedeutet, dass $y \in A$, also $V \subset A$.
  Also ist $A$ offen in $M$.

  Sei $x \in A^{C}$.
  Es gibt also keinen Weg von $x_{0}$ nach $x$.
  Dann gibt es wie im ersten Teil eine Karte mit einer offenen Umgebung $V$ von $x$ in $M$.
  Sei $y \ne x \in V$.
  Angenommen es gäbe einen Weg von $x_{0}$ nach $y$.
  Dann könnte man auf dieselbe Weise wie im ersten Teil einen Weg von $y$ nach $x$ und somit von $x_{0}$ nach $x$ erstellen.
  Dies widerspricht also der Annahme und es gibt keinen Weg von $x_{0}$ nach $y$.
  Es folgt, dass auch $A^{C}$ offen in $M$ ist.
  Dies bedeutet wiederum, dass $A$ in $M$ auch abgeschlossen ist.

  Da $A$ sowohl offen, abgeschlossen als auch nicht-leer ist, gilt $A = M$, weil $M$ zusammenhängend ist, und $M$ ist wegzusammenhängend.
\end{proof}

\subsection{Teil b}

\subsection{Teil c}
\begin{proof}
  Da $M$ eine $n$-dimensionale Untermannigfaltigkeit des $\mathbb{R}^{N}$ ist, gelten die Bedingungen an einer Untermannigfaltigkeit natürlich für jeden Punkt von $M$.
  Wenn man eine Zusammenhangskomponente von $M$ betrachtet, gelten diese Bedingungen dann auch für alle Punkte dieser Komponente, weil sie offen ist und somit die geforderte offene Umgebung auch vollständig in der Komponente enthalten ist.
  Das macht die Zusammenhangskomponente ebenfalls zu einer $n$-dimensionalen Untermannigfaltigkeit des $\mathbb{R}^{N}$.
\end{proof}

\section{Aufgabe 44}

\subsection{Teil a}
\begin{proof}
  Seien $x, y \in Y$ und o.b.d.A. $x < y$.
  Sei $w : [0, 1] \rightarrow Y$ definiert durch
  \begin{equation}
    w(t) = \begin{pmatrix}
      x + t \cdot (y - x)\\
      \sin(\frac{1}{x + t \cdot (y - x)})
    \end{pmatrix}
  \end{equation}
  Dann ist $w$ stetig und $w(0) = x$ und $w(1) = y$.
\end{proof}

\subsection{Teil b}
\begin{proof}
  Weil $Y$ wegzusammenhängend ist, ist es auch zusammenhängend und keine Teilmenge von $X$ außer $\emptyset$ und $Y$, die nicht den Nullpunkt enthält, ist gleichzeitig offen und abgeschlossen in $Y$.
  Deshalb kann keine echte Teilmenge von $Y$ offen und abgeschlossen in $X$ sein.
  Sei $M$ eine beliebige Teilmenge von $X$, die den Nullpunkt enthält und nicht $X$ ist.
  Da $M \ne X$, ist $M \cap Y \ne Y$.
  Angenomment $M$ sei offen.
  Dann gäbe es einen Rand von $M$ in $Y$, der nicht zu $M$ gehört.
  Somit wäre $M$ nicht abgeschlossen.
  Angenommen $M$ sei abgeschlossen.
  Dann könnte aus demselben Grund das Komplement nicht abgeschlossen sein $M$ nicht offen.
  Angenommen $M$ sei nur der Nullpunkt.
  Dann ist $M$ abgeschlossen, jedoch nicht offen, weil es nach einem ähnlichen Argument wie in Teil c in jeder Umgebung von $0$ einen Punkt aus $X$ gäbe, der nicht zu $M$ gehört.
  $Y$ kann auch nicht gleichzeitig offen und abgeschlossen sein, da sein Komplement $\{ (0, 0) \}$ es nicht ist.

  Somit sind alle Mengen außer $\emptyset$ und $X$ ausgeschlossen.
\end{proof}

\subsection{Teil c}
\begin{proof}
  Angenommen es gäbe einen Weg $w$ von $(0, 0)$ nach $x \in X \setminus \{ (0, 0) \}$.
  Dann gäbe es für jedes $\varepsilon > 0$ ein $T$, sodass $w(t) \in B_{\varepsilon}(0)$ enthalten ist für alle $0 \le t < T$.
  Sei $\varepsilon = \frac{1}{2}$ und $T$ entsprechend.
  Die Folge $\left( \frac{2}{(1 + 4n) \cdot \pi} \right)_{n}$ konvergiert gegen 0.
  Folglich gibt es ein $n$, sodass $\frac{2}{(1 + 4n) \cdot \pi} < T$.
  Aber
  \begin{equation}
    \sin\left( \frac{2}{(1 + 4n) \cdot \pi} \right) = \sin\left( \frac{1 + 4n}{2} \cdot \pi \right) = 1
  \end{equation}
  jedoch
  \begin{equation}
    |(\frac{2}{(1 + 4n) \cdot \pi}, w(\frac{2}{(1 + 4n) \cdot \pi})) - 0| \ge 1 > \frac{1}{2}
  \end{equation}
  Also liegt dieser Punkt nicht in der offenen Kugel mit Radius $\frac{1}{2}$ um den Nullpunkt und $w$ existiert nicht.
\end{proof}

\end{document}