\documentclass[10pt,a4paper]{article}
\usepackage[utf8]{inputenc}
\usepackage[german]{babel}
\usepackage{mathrsfs}
\usepackage{amsmath}
\usepackage{amsfonts}
\usepackage{amssymb}
\usepackage{amsthm}
\usepackage[left=2cm,right=2cm,top=2cm,bottom=2cm]{geometry}

\begin{document}

\section{Aufgabe 17}

\section{Aufgabe 18}

\subsection{Teil a}

\subsection{Teil b}

\subsection{Teil c}
\begin{proof}
  $\Rightarrow$: Sei $lim_{n \rightarrow \infty} a_{n} = a$.
  Nach Definition gibt es für jedes $\epsilon$ ein $n$, sodass $|a_{k} - a| < \frac{\epsilon}{2} \Leftrightarrow -\frac{\epsilon}{2} < a_{k} - a < \frac{\epsilon}{2}$ für alle $k \ge n$.
  Dann ist
  \begin{align*}
    & a - \epsilon < \inf \{ a_{k} \mid k \ge n \} \le \sup \{ a_{k} \mid k \ge n \} < a + \epsilon\\
    \Leftrightarrow & -\epsilon < \inf \{ a_{k} \mid k \ge n \} - a \le \sup \{ a_{k} \mid k \ge n \} - a < \epsilon\\
    \Leftrightarrow & -\epsilon < \inf \{ a_{k} \mid k \ge n \} - a \le \sup \{ a_{k} \mid k \ge n \} - a < \epsilon\\
    \Leftrightarrow & |\inf \{ a_{k} \mid k \ge n \} - a| < \epsilon\ \land\ |\sup \{ a_{k} \mid k \ge n \} - a| < \epsilon\\
    \Leftrightarrow & \liminf_{n \rightarrow \infty} a_{n} = a = \limsup_{n \rightarrow \infty} a_{n}
  \end{align*}
  Die andere Richtung ist einfach rückwärts.
\end{proof}

\subsection{Teil d}

\section{Aufgabe 19}

\section{Aufgabe 20}

\section{Aufgabe 21}
\begin{equation}
  f_{n}(x) = 
  \begin{cases}
    & 1\textit{ wenn $n \le x \le n + 1$}\\
    & 0\textit{ sonst}
  \end{cases}
\end{equation}
\begin{equation}
  g_{n}(x) = 
  \begin{cases}
    & 2\textit{ wenn $n \le x \le n + 1$}\\
    & 0\textit{ sonst}
  \end{cases}
\end{equation}
Dann konvergieren bei Folgen gegen die $0$-Funktion, weil es für jedes $x$ ein $N$ gibt, sodass $f_{n}(x) = g_{n}(x) = 0$ für alle $n \ge N$.
Allerdings ist $\lim_{n \rightarrow \infty} \int f_{n} d \lambda^{1} = 1$ und $\lim_{n \rightarrow \infty} \int g_{n} d \lambda^{1} = 2$.

\end{document}