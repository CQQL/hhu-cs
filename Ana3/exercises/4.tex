\documentclass[10pt,a4paper]{article}
\usepackage[utf8]{inputenc}
\usepackage[german]{babel}
\usepackage{mathrsfs}
\usepackage{amsmath}
\usepackage{amsfonts}
\usepackage{amssymb}
\usepackage{amsthm}
\usepackage[left=2cm,right=2cm,top=2cm,bottom=2cm]{geometry}

\begin{document}

\section{Aufgabe 13}

\subsection{Teil a}

\subsection{Teil b}

\subsection{Teil c}

\section{Aufgabe 14}
Wir werden die Menge mithilfe einer Matrix $A$ in den 4-dimensionalen Einheitsquader verformen, sodass $\lambda^{4}(P) = \frac{1}{|\det(A)|}$ ist.
Dabei soll $A$ die 4 aufspannenden Vektoren auf die 4 Einheitsvektoren abbilden.
Dafür bestimmen wir $A$ über $A^{-1}$, dass die 4 Vektoren aus Spalten hat.

\begin{equation}
  A^{-1} = 
  \begin{pmatrix}
    1 & 1 & 1 & 5\\
    2 & 2 & 2 & -1\\
    0 & 3 & 3 & 0\\
    0 & 0 & 4 & 0
  \end{pmatrix}
\end{equation}

Dann ist $A$ nach dem Gauß-Algorithmus
\begin{equation}
  A = 
  \begin{pmatrix}
    12 & 60 & -44 & 0\\
    0 & 0 & 44 & -33\\
    0 & 0 & 0 & -33\\
    24 & -12 & 0 & 0
  \end{pmatrix}
\end{equation}

Damit ergibt sich
\begin{equation}
  \lambda^{4}(P) = \frac{1}{|\det(A)|} = \frac{1}{|-\frac{1}{132}|} = 132
\end{equation}

\section{Aufgabe 15}

\subsection{Teil a}

\subsection{Teil b}

\section{Aufgabe 16}

\end{document}