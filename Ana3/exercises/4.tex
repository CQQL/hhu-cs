\documentclass[10pt,a4paper]{article}
\usepackage[utf8]{inputenc}
\usepackage[german]{babel}
\usepackage{mathrsfs}
\usepackage{amsmath}
\usepackage{amsfonts}
\usepackage{amssymb}
\usepackage{amsthm}
\usepackage[left=2cm,right=2cm,top=2cm,bottom=2cm]{geometry}

\begin{document}

\section{Aufgabe 13}

\subsection{Teil a}
\begin{equation}
  A = \bigcup_{x \in \mathbb{Q}} \{ (x, y) \in \mathbb{R}^{2} \}
\end{equation}
Die einzelnen Mengen sind affine Hyperebenen des $\mathbb{R}^{2}$, weil sie als Verschiebung der $1$-dimensionalen y-Achse dargestellt werden können.
Da $\mathbb{Q}$ abzählbar ist, ist $A$ eine abzählbare Vereinigung von Borel-Mengen (und somit selbst eine), sodass die $\sigma$-Additivität von $\lambda$ gilt und $\lambda^{2}(A) = 0$ ist, weil das Volumen einer affinen Hyperebene $0$.

\subsection{Teil b}


\subsection{Teil c}

\section{Aufgabe 14}
Wir werden die Menge mithilfe einer Matrix $A$ in den 4-dimensionalen Einheitsquader verformen, sodass $\lambda^{4}(P) = \frac{1}{|\det(A)|}$ ist.
Dabei soll $A$ die 4 aufspannenden Vektoren auf die 4 Einheitsvektoren abbilden.
Dafür bestimmen wir $A$ über $A^{-1}$, dass die 4 Vektoren aus Spalten hat.

\begin{equation}
  A^{-1} = 
  \begin{pmatrix}
    1 & 1 & 1 & 5\\
    2 & 2 & 2 & -1\\
    0 & 3 & 3 & 0\\
    0 & 0 & 4 & 0
  \end{pmatrix}
\end{equation}

Dann ist $A$ nach dem Gauß-Algorithmus
\begin{equation}
  A = 
  \begin{pmatrix}
    12 & 60 & -44 & 0\\
    0 & 0 & 44 & -33\\
    0 & 0 & 0 & -33\\
    24 & -12 & 0 & 0
  \end{pmatrix}
\end{equation}

Damit ergibt sich
\begin{equation}
  \lambda^{4}(P) = \frac{1}{|\det(A)|} = \frac{1}{|-\frac{1}{132}|} = 132
\end{equation}

\section{Aufgabe 15}

\subsection{Teil a}

\subsection{Teil b}

\section{Aufgabe 16}
\begin{proof}
  $b \Leftrightarrow e$: Sei $x \in X$ und $\alpha \in \mathbb{R}$.
  Wenn $x \notin \{ x \in X \mid f(x) \ge \alpha \}$, dann $f(x) < \alpha$ und $x \in \{ x \in X \mid f(x) < \alpha \}$.
  Wenn $x \notin \{ x \in X \mid f(x) < \alpha \}$, dann $f(x) \ge \alpha$ und $x \in \{ x \in X \mid f(x) \ge \alpha \}$.
  Also ist $X$ die disjunkte Vereinigung von $\{ x \in X \mid f(x) \ge \alpha \}$ und $\{ x \in X \mid f(x) < \alpha \}$.
  Und da $\sigma$-Algebren bezüglich Komplementbildung abgeschlossen sind, ist die jeweils andere Menge auch in der $\sigma$-Algebra enthalten, wenn die eine Menge enthalten ist.

  $c \Leftrightarrow d$: Analog zu $b \Leftrightarrow e$.

  $d \Rightarrow e$: Sei $\alpha \in \mathbb{R}$, sodass $\{ x \in X \mid f(x) \ge \alpha \} \in \mathscr{A}$.
  Sei $(a_{n})$ eine Folge in $\mathbb{R}$, die gegen $\alpha$ konvergiert und $a_{n} < \alpha$ für alle $n$.
  Dann kann man die endliche Vereinigung aller entsprechenden Mengen bilden, die wegen der Abgeschlossenheit einer $\sigma$-Algebra auch wieder in dieser enthalten ist.
  \begin{equation}
    \bigcup_{n = 1}^{\infty} \{ x \in X \mid f(x) \le a_{n} \} = \{ x \in X \mid f(x) < \alpha \} \in \mathscr{A}
  \end{equation}

  $e \Rightarrow d$: Sei $\alpha \in \mathbb{R}$, sodass $\{ x \in X \mid f(x) < \alpha \} \in \mathscr{A}$.
  Sei $(a_{n})$ eine Folge in $\mathbb{R}$, die gegen $\alpha$ konvergiert und $a_{n} > \alpha$ für alle $n$.
  Da $\sigma$-Algebren gegenüber endlicher Vereinigung und Komplementbildung abgeschlossen sind, gilt folgendes
  \begin{align*}
    \left( \bigcup_{n = 1}^{\infty} \{ x \in X \mid f(x) < a_{n} \}^{C} \right)^{C} & = \left( \bigcup_{n = 1}^{\infty} \{ x \in X \mid f(x) \ge a_{n} \} \right)^{C}\\
    & = \left( \{ x \in X \mid f(x) > \alpha \} \right)^{C}\\
    & = \{ x \in X \mid f(x) \le \alpha \} \in \mathscr{A}
  \end{align*}
\end{proof}

\end{document}